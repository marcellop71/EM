%% =========================================================================
\section{The Character Sum Reduction}
\label{sec:character}
%% =========================================================================

Sections~\ref{sec:walk}--\ref{sec:bootstrap} established the picture: for the first
missing prime~$q$, the walk lives in $(\ZZ/q\ZZ)^\times$ forever, the
multipliers generate the full group, yet the walk avoids~$-1$
permanently past the sieve gap.  The death channel
$f(n) = -\walkZ{q}{n}^{-1}$ is dodged at every step.  The Single Hit
Theorem (\S\ref{sec:single-hit}) reduces MC to producing one hit
on~$-1$ past the sieve gap at each prime.

The question is not ``does the walk equidistribute?''---that is far
stronger than needed.  The question is: \emph{can the walk really
avoid one class out of~$q{-}1$ forever?}  This section develops the
harmonic-analytic tools to show it cannot: character sums detect the
anomaly that permanent avoidance would create, and bounding them rules
it out.

\paragraph{Why character sums?}
Permanent avoidance of~$-1$ means the hit count
$|\{n < N : \walkZ{q}{n} = -1\}|$ stays at zero past the sieve gap.
Character orthogonality decomposes this count into a uniform share
$N/(q{-}1)$ plus correction terms built from non-trivial character
sums $S_\chi(N) = \sum_{n<N}\chi(\walkZ{q}{n})$.  The uniform share
grows linearly; for the hit count to remain zero, the correction terms
must cancel this growth.  That requires at least one non-trivial
character sum to be $\Omega(N/(q{-}1))$---a sustained asymmetry in
the character spectrum.  The hypothesis CCSB (all $|S_\chi(N)| =
o(N)$) rules this out: it forces the hit count to be eventually
positive, contradicting permanent avoidance.

\medskip\noindent\textbf{What is new vs.\ what is infrastructure.}\enspace
The CCSB~$\Rightarrow$~MC reduction (Definition~\ref{thm:ccsb-mc}),
the Fourier bridge (Theorem~\ref{thm:fourier-step}), the
Decorrelation--PED chain (\S\ref{sec:dec-chain}), and the telescoping
no-go results (\S\ref{sec:telescope}) are original contributions of
this formalization.  The large sieve infrastructure
(\S\ref{sec:large-sieve}, Appendix~\ref{sec:sve})---including the weak ALS,
Gauss sum inversion, van der Corput, and Parseval---formalizes known
results; its purpose is to identify the precise \emph{transfer gap}
between classical tools and the EM orbit, which is itself a
contribution (see \S\ref{sec:large-sieve}).

The formalization develops this Fourier-analytic reduction because
character sums provide the cleanest interface between the
death-channel avoidance problem and the toolkit of analytic number
theory: the Bombieri--Vinogradov theorem, the large sieve inequality,
and Gauss sum inversion all produce character sum bounds, and the
formalization shows exactly how each connects to ruling out permanent
avoidance.

\subsection{Character Sums and Permanent Avoidance}

For a Dirichlet character $\chi : (\ZZ/q\ZZ)^\times \to \mathbb{C}^\times$,
the \emph{walk character sum} is
\[
  S_\chi(N) \;=\; \sum_{n < N} \chi\bigl(\walkZ{q}{n}\bigr).
\]

\begin{definition}[\defname{ComplexCharSumBound} (\abbr{CCSB})]
For every missing prime~$q$, every non-trivial character~$\chi$, and
every $\varepsilon > 0$, there exists $N_0$ such that for all
$N \geq N_0$:
\[
  \|S_\chi(N)\| \;\leq\; \varepsilon \cdot N.
\]
In other words, the walk character sums are $o(N)$---they grow
strictly slower than linearly.  The contrapositive: if the walk
permanently avoids~$-1$, some non-trivial character sum must be
$\Omega(N)$ (to cancel the uniform share in the Fourier bridge).
CCSB rules this out.
\end{definition}

\begin{theorem}[\lean{EM/EquidistSelfCorrecting.lean}{complex\_csb\_mc'}]
\label{thm:ccsb-mc}
$\mathrm{CCSB} \;\Longrightarrow\; \MC$.
\end{theorem}

This is a single-hypothesis reduction with no additional parameters.
The proof composes three bridges:

\begin{enumerate}[nosep]
\item \textbf{Fourier inversion} (\lean{EM/EquidistFourier.lean}{complex\_csb\_implies\_hit\_count\_lb\_proved}):
  CCSB implies that the hit count at~$-1$ satisfies
  $|\{n < N : \walkZ{q}{n} = -1\}| = N/(q{-}1) + o(N)$,
  which is eventually positive.  This directly contradicts permanent
  avoidance past the sieve gap (the character orthogonality formula,
  Theorem~\ref{thm:fourier-step}, gives this for any target~$t$;
  specializing to $t = -1$ is what matters).

\item \textbf{Cofinal hitting gives HH}
  (\lean{EM/EquidistFourier.lean}{walk\_equidist\_mc}):
  the eventually-positive hit count at~$-1$ gives cofinal hitting,
  hence HH\@.  (In fact, the Fourier bridge yields equidistribution
  across all classes, but only the $t = -1$ case is needed.  SE is a
  side effect: a walk visiting~$-1$ cannot be confined to a proper
  coset.)

\item \textbf{DH implies MC} (Theorem~\ref{thm:dh-mc}): via the
  inductive bootstrap.
\end{enumerate}

\subsection{The Fourier Bridge}

The Fourier bridge is the single most important proved result after
DH~$\Rightarrow$~MC itself: it converts character sum bounds into hit
count lower bounds, and hence into MC.

\begin{theorem}[\lean{EM/EquidistFourier.lean}{walk\_hit\_count\_fourier\_step}]
\label{thm:fourier-step}
For any target $t \in (\ZZ/q\ZZ)^\times$:
\[
  \bigl|\{n < N : \walkZ{q}{n} = t\}\bigr|
  \;=\; \frac{1}{q-1}\sum_{\chi} \overline{\chi(t)}\, S_\chi(N),
\]
where the sum is over all Dirichlet characters mod~$q$.
\end{theorem}

This is a standard Fourier inversion formula on the finite group
$(\ZZ/q\ZZ)^\times$, stated for all~$t$.  Specializing to $t = -1$
and splitting the trivial character $\chi_0$ (with $\chi_0(a) = 1$
for all~$a$) from the non-trivial ones:
\[
  \bigl|\{n < N : \walkZ{q}{n} = -1\}\bigr|
  \;=\; \underbrace{\frac{N}{q-1}}_{\text{uniform share}}
  \;+\; \underbrace{\frac{1}{q-1}\sum_{\chi \neq \chi_0}
    \overline{\chi(-1)}\, S_\chi(N)}_{\text{correction}}.
\]
The first term grows linearly.  The correction term is bounded by
$\tfrac{1}{q-1}\sum_{\chi \neq \chi_0} |S_\chi(N)|$, since
$|\overline{\chi(-1)}| = 1$.  If CCSB holds---all
$|S_\chi(N)| = o(N)$---the correction is $o(N)$, and the hit count
is $N/(q{-}1) + o(N)$, which is eventually positive.

\paragraph{The contrapositive.}
Section~\ref{sec:bootstrap} showed that for the first missing
prime~$q$, the hit count at~$-1$ is exactly~0 past the sieve gap.
By the Fourier bridge, this forces the correction terms to
cancel the entire main term $N/(q{-}1)$.  Since there are only
$q{-}2$ non-trivial characters, at least one must satisfy
$|S_\chi(N)| = \Omega(N/(q{-}1)^2)$---a sustained linear-scale
bias in the character spectrum.  CCSB ($= o(N)$) rules this out.
The Fourier bridge thus converts the geometric statement ``the walk
avoids~$-1$ permanently'' into the spectral statement ``some character
sum has sustained bias,'' and CCSB negates the latter.

Returning to our running example ($q = 41$): the walk must accumulate
a deficit of $\sim N/40$ hits at~$-1$ compared to uniform, which
requires sustained character sum bias across the 39~non-trivial
characters.  Proving CCSB for $q = 41$ would rule out this bias and
establish that 41~eventually appears.

\subsection{The Decorrelation--PED--CCSB Chain}
\label{sec:dec-chain}

CCSB rules out permanent avoidance of~$-1$.  But the walk is built
from \emph{multipliers}: $\walkZ{q}{n{+}1} = \walkZ{q}{n} \cdot
\multZ{q}{n}$.  Since $\chi$ is a group homomorphism,
$\chi(\walkZ{q}{n}) = \chi(\walkZ{q}{0}) \cdot \prod_{k<n}
\chi(\multZ{q}{k})$: the walk character sum is a sum of
\emph{partial products} of the multiplier characters.  The question
becomes: what properties of the multiplier sequence would prevent
the walk from permanently dodging the death channel?

This subsection formalizes a chain of progressively weaker hypotheses
about the multipliers, each implying the next via proved bridges.
The goal is to decompose CCSB into sharper conditions on the multiplier
sequence, and to identify exactly where the irreducible difficulty
lies.

\begin{definition}[\defname{PositiveEscapeDensity} (\abbr{PED})]
For every missing prime~$q$ and non-trivial~$\chi$, there exist
$\delta > 0$ and $N_0$ such that for $N \geq N_0$:
$|\{k < N : \chi(\multZ{q}{k}) \neq 1\}| \geq \delta N$.
\end{definition}

The name ``escape'' comes from the SubgroupEscape perspective: the
kernel $\ker(\chi)$ is a proper subgroup of $(\ZZ/q\ZZ)^\times$, and
$\chi(\multZ{q}{k}) \neq 1$ means the $k$-th multiplier ``escapes''
from $\ker(\chi)$.  PED asks that a positive fraction of multipliers
escape \emph{every} proper subgroup, not just occasionally but with
positive density.  The connection to the death channel: since the
death channel is a single class and multipliers that escape
$\ker(\chi)$ ``rotate'' the walk character value $\chi(\walkZ{q}{n})$
by a non-trivial amount, enough escapes should prevent the systematic
avoidance needed for permanent death-channel dodging.  PED is a weak
condition---it says nothing about cancellation, only that the
multipliers are not asymptotically trapped in any subgroup.

\begin{definition}[\defname{DecorrelationHypothesis}]
For every missing prime~$q$ and non-trivial~$\chi$, the multiplier
character sums are $o(N)$:
$\|\sum_{n<N} \chi(\multZ{q}{n})\| \leq \varepsilon N$ for large~$N$.
\end{definition}

Decorrelation is stronger than PED: it asks not merely that many
multipliers escape $\ker(\chi)$, but that they do so with enough
balance that the character values cancel.  If the multipliers were
independent random elements of $(\ZZ/q\ZZ)^\times$, the sum would be
$O(\sqrt{N})$ by the law of large numbers---far smaller than
$\varepsilon N$.  Decorrelation asks for the much weaker $o(N)$.

\begin{definition}[\defname{NoLongRuns}$(L)$]
For every missing prime~$q$ and non-trivial~$\chi$, past some
threshold, no $L$ consecutive multipliers all lie in $\ker(\chi)$.
\end{definition}

NoLongRuns is a qualitative cousin of PED: if multipliers never stay
inside $\ker(\chi)$ for $L$~steps in a row, then at least $1/(2L)$ of
them escape.  This condition is easier to verify in practice because it
only requires checking short blocks.

\begin{definition}[\defname{BlockRotationEstimate} (\abbr{BRE})]
If the escape count is $\geq \delta N$, then the walk character sums
are $o(N)$.  This encapsulates the Cauchy--Schwarz / van der Corput
step in harmonic analysis.
\end{definition}

BRE is the bridge between the multiplier-level conditions
(PED/Decorrelation) and the walk-level condition (CCSB).  It says:
given that multipliers escape with positive density, the walk
character sums must cancel.  The intuition is that each escape event
``rotates'' the walk character value $\chi(\walkZ{q}{n})$ by a
non-trivial amount, and sufficiently many such rotations produce
cancellation in the sum.  BRE is the sole unproved bridge in the PED
route.

The hypotheses above---PED, Decorrelation, NoLongRuns---all treat
the multiplier sequence as a single stream, ignoring what the walk is
doing at the moment.  But permanent death-channel avoidance is a
statement about the \emph{coupling} between walk position and
multiplier: the death channel $-\walkZ{q}{n}^{-1}$ is a function of
walk position, so avoiding it means the multiplier distribution at
each walk position is biased away from one class.  A more refined
hypothesis should address this coupling head-on.

\begin{definition}[\defname{ConditionalMultiplierEquidist} (\abbr{CME})]
\label{def:cme}%
For every missing prime~$q$, non-trivial~$\chi$, $\varepsilon > 0$,
there exists~$N_0$ such that for $N \geq N_0$ and every walk
position~$c \in (\ZZ/q\ZZ)^\times$:
$\|\sum_{\substack{n < N \\ \walkZ{q}{n} = c}} \chi(\multZ{q}{n})\|
\leq \varepsilon N$.
\end{definition}

CME says the multiplier distribution is the same at every walk
position.  Since the death channel is a function of walk position,
CME implies the death channel has no special status---the multiplier
is no more likely to avoid the forbidden class than any other.
Formally, CME is strictly stronger than Decorrelation: it bounds the
fiber sums
$\sum_{\substack{n<N \\ \walkZ{q}{n} = c}} \chi(\multZ{q}{n}) = o(N)$
separately for each position~$c$, not just the global sum.
Since the global sum is the sum of the fiber sums, CME implies
Decorrelation by the triangle inequality
(\lean{EM/LargeSieveSpectral.lean}{cme\_implies\_dec}).
The significance of CME is that it also implies CCSB
\emph{directly}, bypassing PED and BRE entirely: the fiber
decomposition and the telescoping identity together convert
conditional multiplier cancellation into walk character sum
cancellation, for \emph{all} character orders, without needing the
intermediate PED $\to$ BRE $\to$ CCSB chain.

\begin{theorem}[\lean{EM/EquidistSelfCorrecting.lean}{decorrelation\_implies\_ped}]
Decorrelation $\Rightarrow$ PED.
\end{theorem}

\begin{proof}[Proof sketch]
Contrapositive.  If few multipliers escape $\ker(\chi)$---say fewer
than $\delta N$---then most contribute $\chi(m(n)) = 1$ to the sum.
The at most $\delta N$ exceptions contribute values of norm~$\leq 1$.
By the reverse triangle inequality, $|\sum \chi(m(n))| \geq N - 2\delta N$,
which is $\geq \varepsilon N$ for $\delta$ small enough.  This
contradicts Decorrelation.
\end{proof}

\begin{theorem}[\lean{EM/EquidistSelfCorrecting.lean}{noLongRuns\_implies\_ped}]
NoLongRuns$(L) \Rightarrow$ PED with $\delta = 1/(2L)$.
\end{theorem}

\begin{proof}[Proof sketch]
Partition $\{0, \ldots, N{-}1\}$ into blocks of length~$L$.  Each block
contains at least one escape (by assumption), so the total escape count
is $\geq N/(2L)$.
\end{proof}

\begin{theorem}[\lean{EM/EquidistSelfCorrecting.lean}{block\_rotation\_implies\_ped\_csb}]
BRE $\Rightarrow$ PEDImpliesComplexCSB.
\end{theorem}

The PED route, with all proved arrows:
\[
\mathrm{Dec} \;\xrightarrow{\text{proved}}\; \mathrm{PED}
\;\xleftarrow{\text{proved}}\; \mathrm{NoLongRuns}(L)
\;\xrightarrow{\text{BRE, open}}\;
\mathrm{CCSB}
\;\xrightarrow{\text{proved}}\;
\mathrm{MC}.
\]
The sole open bridge in this route is BRE: converting positive escape
density into walk character sum cancellation.

However, the PED route is not the only path.  CME implies CCSB
\emph{directly}, bypassing PED and BRE entirely
(\lean{EM/LargeSieveSpectral.lean}{cme\_implies\_ccsb}):
\[
\mathrm{CME}
\;\xrightarrow{\text{proved}}\;
\mathrm{CCSB}
\;\xrightarrow{\text{proved}}\;
\mathrm{MC}.
\]
This bypass is significant: the $d \geq 3$ barrier
(Remark~\ref{rem:bre-impossible}) blocks the PED $\to$ CCSB
factorization for characters of order $\geq 3$, but CME $\to$ CCSB
works for \emph{all} character orders, using only the telescoping
identity and fiber decomposition.

\subsection{Vanishing Conditional Bias}
\label{sec:vcb}

CME is a strong hypothesis: it asks that \emph{every} fiber character
sum $F(c, \chi) = \sum_{\substack{n < N \\ w(n) = c}} \chi(m(n))$ be
$o(N)$---in other words, that $\mu = 0$ in the proportionality
$F(c, \chi) \approx \mu \cdot V_N(c)$.  But the telescoping route to
CCSB does not need~$\mu = 0$; it needs only that $\mu \neq 1$.  This
observation motivates a strictly weaker hypothesis.

\begin{definition}[\defname{VanishingConditionalBias} (\abbr{VCB})]
\label{def:vcb}%
For every missing prime~$q$, non-trivial~$\chi$, and $\varepsilon > 0$,
there exists~$N_0$ such that for $N \geq N_0$ there is a complex
number~$\mu = \mu(N, \chi)$ with $|\mu| \leq 1$ satisfying, for all
$c \in (\ZZ/q\ZZ)^\times$:
\[
  \left\|\sum_{\substack{n < N \\ w(n) = c}} \chi(m(n))
    \;-\; \mu \cdot V_N(c)\right\|
  \;\leq\; \varepsilon \cdot N,
\]
where $V_N(c) = |\{n < N : w(n) = c\}|$ is the visit count.
\end{definition}

In words: the factoring channel treats all walk positions
proportionally (the formal content of the factoring channel analogy
from \S\ref{sec:intro})---the character statistics of multipliers are
the same at every position, up to a common constant~$\mu$ (which may vary
with~$N$ and~$\chi$ but must be common across all positions~$c$).
Combined with PED (enough multipliers escape), the proportionality
constant cannot equal~1; the telescope then forces the walk character
sum to be $o(N)$, ruling out the sustained bias needed for permanent
avoidance.

\begin{proposition}[\lean{EM/LargeSieveSpectral.lean\#L2007}{cme\_implies\_vcb}]
$\mathrm{CME} \Rightarrow \mathrm{VCB}$ with $\mu = 0$.
Conversely, $\mathrm{VCB} + \mathrm{Dec} \Rightarrow \mathrm{CME}$.
Thus CME decomposes as $\mathrm{CME} = \mathrm{VCB} + \mathrm{Dec}$.
VCB alone is strictly weaker than CME: it permits the fiber sums to
be $\Theta(N)$, provided they are proportional to the visit counts.
\end{proposition}

\begin{theorem}[\lean{EM/LargeSieveSpectral.lean\#L2256}{vcb\_ped\_implies\_ccsb}]
\label{thm:vcb-ped-ccsb}
$\mathrm{VCB} + \mathrm{PED} \;\Longrightarrow\; \mathrm{CCSB}$.
\end{theorem}

\begin{proof}[Proof sketch]
From VCB, the telescope identity gives
$(\mu - 1) \cdot S_N = O(1) + O(\varepsilon N (q{-}1))$.
If $|1 - \mu|$ is bounded away from zero, then
$S_N = O(\varepsilon N / |1{-}\mu|) = o(N)$.

To show $|1 - \mu| \geq c_0 > 0$, PED provides $\delta > 0$ such
that at least $\delta N$ multipliers escape $\ker(\chi)$.  For each
escaping multiplier, $\chi(m(n))$ is a non-trivial $d$-th root of
unity, satisfying $\mathrm{Re}(\chi(m(n))) \leq 1 - \eta_0^2/2$
where $\eta_0 = \min_{\zeta \in \mu_d \setminus \{1\}} |\zeta - 1| > 0$
(\lean{EM/LargeSieveSpectral.lean\#L2024}{norm\_sub\_one\_sq\_eq},
\lean{EM/LargeSieveSpectral.lean\#L2037}{unit\_norm\_re\_le\_of\_dist}).
This real-part defect accumulates over $\delta N$ escaping steps,
forcing $|\sum \chi(m(n)) - N| \geq c_0 N$ for
$c_0 = \delta \eta_0^2 / 2$.  Combined with VCB's control
$|\sum \chi(m(n)) - \mu N| \leq C \varepsilon N$, the reverse
triangle inequality gives $|1 - \mu| \geq c_0/2$ for small
$\varepsilon$.
\end{proof}

\begin{theorem}[\lean{EM/LargeSieveSpectral.lean\#L2366}{vcb\_ped\_implies\_mc}]
$\mathrm{VCB} + \mathrm{PED} \;\Longrightarrow\; \MC$.
\end{theorem}

This decomposition splits the monolithic CME hypothesis into two
independently attackable pieces:
\begin{itemize}[nosep]
\item \textbf{VCB} (``the factoring channel preserves proportionality''):
  the character distribution of multipliers is the same at every walk
  position, up to a common rate.  This captures the ``factorization
  independence'' intuition---that the factoring operation destroys the
  correlation between walk position and multiplier residue.
\item \textbf{PED} (``enough multipliers escape every character kernel''):
  a positive fraction of multiplier residues fall outside any proper
  subgroup.
\end{itemize}
VCB is a recent contribution of the formalization.

\subsection{Walk Telescoping Identities}
\label{sec:telescope}

The hypotheses above attack CCSB by decomposing it into conditions on
the multiplier sequence.  Before proceeding to the large sieve route,
we pause to examine what the walk recurrence
$\chi(w(n{+}1)) = \chi(w(n)) \cdot \chi(m(n))$ itself forces.  This
multiplicative structure is the \emph{only} algebraic relation
connecting walk and multiplier character values, and it constrains
any possible proof of CCSB\@.  The following identities make these
constraints explicit.  They are formalized not because they solve
CCSB, but because they reveal the structural landscape: they show
which proof strategies are compatible with the walk's algebra and
which are ruled out.

\begin{theorem}[\lean{EM/EquidistSelfCorrecting.lean}{walk\_telescope\_identity}]
\label{thm:telescope}
For any $\chi$ and $N$:
\[
  \sum_{n < N} \chi(w(n))\cdot\bigl(\chi(m(n)) - 1\bigr)
  \;=\; \chi(w(N)) - \chi(w(0)).
\]
\end{theorem}

This identity follows immediately from the walk recurrence
$\chi(w(n{+}1)) = \chi(w(n)) \cdot \chi(m(n))$: writing
$\chi(w(n)) \cdot (\chi(m(n)) - 1) = \chi(w(n{+}1)) - \chi(w(n))$,
the sum telescopes to $\chi(w(N)) - \chi(w(0))$.

\begin{theorem}[\lean{EM/EquidistSelfCorrecting.lean}{walk\_telescope\_norm\_bound}]
The telescoping sum has norm $\leq 2$ (triangle inequality on
unit-norm terms).
\end{theorem}

The $\leq 2$ bound looks innocent, but it constrains how the walk
can respond to the death channel.  Each summand
$\chi(w(n))\cdot(\chi(m(n)) - 1)$ is non-zero exactly when
$\chi(m(n)) \neq 1$---i.e., when the multiplier escapes
$\ker(\chi)$, ``rotating'' the character value.  The $\leq 2$ bound
means the net rotation over all $N$ steps is negligible, tightly
coupling the walk character sum $S_N = \sum_{n<N} \chi(w(n))$ to the
multiplier character sum $M_N = \sum_{n<N} \chi(m(n))$.  Splitting
the product in Theorem~\ref{thm:telescope} gives
$S_N \cdot \overline{M_N/N} - S_N = O(1)$ (after normalization).

\begin{theorem}[\lean{EM/EquidistSelfCorrecting.lean}{walk\_shift\_one\_correlation}]
\label{thm:shift-one}
$\sum_{n < N} \chi(w(n)) \cdot \overline{\chi(w(n+1))}
  = \overline{\sum_{n < N} \chi(m(n))}$.
\end{theorem}

This identity says that the lag-1 autocorrelation of the walk
character equals the conjugate of the multiplier character sum.  It is
a \emph{no-go result} for the van der Corput method with $H = 1$:
VdC bounds $|S_N|^2$ in terms of autocorrelations, but at lag $h = 1$,
the autocorrelation is exactly $|M_N|$---the multiplier character
sum---which need not be small.  So VdC with a single shift gives only
$|S_N| \leq O(\sqrt{N \cdot |M_N|})$, which is $O(N)$ in the worst
case, not the $o(N)$ that CCSB requires.  This means any proof of CCSB
must either (i)~use higher-order correlations (HOD,
Appendix~\ref{sec:sve}) or (ii)~establish multiplier decorrelation
first.

\subsection{The Large Sieve Route}
\label{sec:large-sieve}

The preceding subsections attacked the question ``can the walk avoid
$-1$ permanently?'' for a single modulus~$q$.  A different strategy
works with \emph{many moduli simultaneously}: classical analytic
number theory produces character sum estimates averaged over all
moduli $q \leq Q$, and such averaged results are often easier to
prove than pointwise ones.  The large sieve inequality and the
Bombieri--Vinogradov theorem are the two central tools for this.

The formalization develops this multi-modular route across three files
(\code{LargeSieve.lean}, \code{Large\-Sieve\-Harmonic.lean},
\code{Large\-Sieve\-Analytic.lean}) totaling ${\sim}5{,}870$~lines,
connecting classical analytic number theory to MC via a multi-modular
character sum bound.

\paragraph{Why formalize the large sieve?}
The analytic large sieve inequality and the Bombieri--Vinogradov theorem
are among the most powerful tools in analytic number theory for
controlling the distribution of primes in arithmetic progressions.
If these tools could be applied to the EM walk, MC would follow.
We formalize the connection---not the deep theorems themselves (which
are known results, stated as open Props)---for two reasons:
\begin{enumerate}[nosep]
\item To identify \emph{precisely} what transfer hypothesis is needed to
  apply each classical result to the specific EM orbit, and
\item To verify that six apparently independent routes (BV, ArithLS, ALS,
  PrimeArithLS, LoD, sieve transfer) all reduce to the \emph{same}
  orbit-specificity gap.
\end{enumerate}
This diagnosis is itself a mathematical contribution: it shows that the
difficulty of MC is not a failure of existing analytic tools but a
fundamental obstacle in applying ensemble-averaged results to a single
deterministic orbit.

\begin{definition}[\defname{MultiModularCSB} (\abbr{MMCSB})]
There exists $Q_0$ such that for all $q \geq Q_0$ prime, every
non-trivial character~$\chi$ mod~$q$, and every $\varepsilon > 0$,
there exists $N_0$ such that for $N \geq N_0$:
$\|S_\chi(N)\| \leq \varepsilon N$.
\end{definition}

MultiModularCSB is weaker than CCSB in that it allows finitely many
exceptional primes below~$Q_0$.  This weakening is crucial because
averaged results like BV naturally produce bounds that fail for
finitely many moduli.

\begin{theorem}[\lean{EM/LargeSieve.lean\#L303}{mmcsb\_implies\_mc}]
\label{thm:mmcsb-mc}
$\mathrm{MultiModularCSB} \;\Longrightarrow\; \MC$.
\end{theorem}

The proof composes the per-prime Fourier bridge
(Theorem~\ref{thm:fourier-step}) with the inductive bootstrap: for
$q \geq Q_0$, MMCSB gives hit count $\sim N/(q{-}1) > 0$ at~$-1$,
contradicting permanent avoidance; for the finitely many primes
$q < Q_0$, the bootstrap (Section~\ref{sec:bootstrap}) handles
them---once MC holds for all primes below~$q$, the sieve gap and
one-prime gap (Theorem~\ref{thm:one-prime-gap}) reduce MC($q$) to a
single hitting event.

Three parallel routes to MultiModularCSB are formalized:

\paragraph{Bombieri--Vinogradov route.}
BV says primes are equidistributed among arithmetic progressions ``on
average over moduli'': for most $q \leq Q = \sqrt{x}/(\log x)^A$, the
count of primes $\leq x$ in any progression $a \bmod q$ is close to
the expected $\pi(x)/\phi(q)$.  If this applies to the EM multipliers,
the death channel---a single progression---gets its fair share of
multipliers, breaking the avoidance.

\begin{theorem}[\lean{EM/LargeSieve.lean\#L362}{bv\_chain\_mc}]
$\mathrm{BV} + \mathrm{BVImpliesMMCSB} \;\Longrightarrow\; \MC$.
\end{theorem}
The transfer hypothesis BVImpliesMMCSB is a \textbf{genuine frontier}:
it requires transferring the averaged equidistribution statement of BV
(valid for primes in generic progressions) to the specific EM walk orbit.
The EM sequence is not a generic sample of primes---it is a deterministic
sequence defined by iterated factorization---so its multipliers could
exhibit special correlations that BV's averaged estimate cannot detect.

The route was decomposed into two stages:
\[
\mathrm{BV}
\xrightarrow{\text{sieve transfer}}
\mathrm{EMMultCSB}
\xrightarrow{\text{walk bridge}}
\mathrm{MMCSB}
\xrightarrow{\text{proved}}
\mathrm{MC},
\]
separating the number-theoretic content (BV~$\Rightarrow$~EMMultCSB,
where EMMultCSB bounds the \emph{multiplier} character sums)
from the dynamical content (EMMultCSB~$\Rightarrow$~MMCSB, converting
multiplier bounds to \emph{walk} bounds).  However, the walk bridge
\textbf{MultCSBImpliesMMCSB is false in general}
(\lean{EM/LargeSieve.lean\#L1226}{MultCSBImpliesMMCSB}):
the walk character sum $\sum \chi(w(n))$ is a \emph{partial product}
$\prod_{k<n} \chi(m(k))$ of the multiplier characters, and partial
products of equidistributed unit complex numbers need not cancel---they
perform a random walk on the unit circle whose norm grows as~$\sqrt{N}$,
not as~$o(N)$.  The telescope identity (Theorem~\ref{thm:shift-one})
makes this obstruction precise: the $h\!=\!1$ autocorrelation equals
the multiplier character sum, so van der Corput with a single shift
gives only~$O(N)$, not~$o(N)$.
This is why the CME bypass (fiber decomposition $+$ telescoping,
Definition~\ref{def:cme}) is essential: it goes directly from
conditional multiplier equidistribution to CCSB without ever
requiring the walk bridge.

\paragraph{Additional sieve routes.}
Two further routes are formalized---the arithmetic large sieve
(ArithLS~$\Rightarrow$~MC, a dead end) and the analytic
large sieve (ALS~$\Rightarrow$~PrimeArithLS~$\Rightarrow$~MC), where
the ALS-to-PrimeArithLS bridge via Gauss sum inversion is fully proved
across eight internal lemmas.  In both cases, the genuine open content
is the same \emph{orbit-specificity transfer}: applying averaged
results to one deterministic orbit.  The full details, including the
ALS definition, weak ALS proof, Gauss sum inversion theorem, and a
spectral energy route (SVE, van der Corput, HOD, CME) with its
complete hypothesis hierarchy, appear in Appendix~\ref{app:routes}.

