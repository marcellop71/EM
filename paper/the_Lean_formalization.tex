%% =========================================================================
\section{The Lean Formalization}
\label{sec:lean}
%% =========================================================================

Having developed the mathematical reduction and identified the
structural obstacles, we describe the Lean~4 formalization that
certifies these results.

\paragraph{Codebase Structure}

The formalization uses Lean~4 with Mathlib~v4.27.0 across 42~files
totaling ${\sim}29{,}600$~lines.  The dependency chain is linear, with
three leaf modules:

\begin{center}
\small
\begin{tabular}{llr}
\toprule
\textbf{File} & \textbf{Content} & \textbf{Lines} \\
\midrule
\code{Euclid.lean} & Constructive Euclid's theorem & 422 \\
\code{MullinDefs.lean} & \code{seq}, \code{prod}, \code{aux}, identities & 527 \\
\code{MullinConjectures.lean} & MC, Conjecture A (FALSE), HH & 490 \\
\code{MullinDWH.lean} & DivisorWalkHypothesis (leaf) & 547 \\
\code{MullinResidueWalk.lean} & WalkCoverage, residue walk, concrete MC & 603 \\
\code{MullinGroupCore.lean} & walkZ, multZ, confinement, SE & 422 \\
\code{MullinGroupEscape.lean} & Escape lemmas, 8-element argument & 630 \\
\code{MullinGroupSEInstances.lean} & 29 concrete SE instances ($q \leq 157$) & 364 \\
\code{MullinGroupPumping.lean} & Gordon sequenceability (leaf) & 343 \\
\code{MullinGroupQR.lean} & QR obstruction ($\leq 1.6\%$) (leaf) & 684 \\
\code{MullinCRT.lean} & CRT multiplier invariance, walk recurrence & 160 \\
\code{MullinDepartureGraph.lean} & Departure graph, infinite recurrence, safe prime lattice & 641 \\
\code{RotorRouter.lean} & Scheduled walk coverage (standalone) & 455 \\
\code{MullinRotorBridge.lean} & EMPR + SE $\Rightarrow$ MC bridge & 87 \\
\code{EquidistPreamble.lean} & PE $\Rightarrow$ MC, bootstrapping & 234 \\
\code{EquidistSieve.lean} & Sieve, WHP $\Leftrightarrow$ HH, forbidden multiplier, M\"obius death & 755 \\
\code{EquidistSelfAvoidance.lean} & Self-avoidance, periodicity & 450 \\
\code{EquidistCharPRE.lean} & Character non-vanishing, PRE $\Leftrightarrow$ SE & 807 \\
\code{EquidistBootstrap.lean} & Inductive bootstrap, DH $\Rightarrow$ MC, first passage & 721 \\
\code{EquidistThreshold.lean} & Sieve gap, one-prime gap, cofinal pair avoidance & 321 \\
\code{EquidistOrbitAnalysis.lean} & Cofinal orbits, quotient walk, sieve, selectability & 1429 \\
\code{EquidistFourier.lean} & Character sums, Fourier bridge & 1317 \\
\code{EquidistSelfCorrecting.lean} & Decorrelation, BRE, telescoping, kernel (\S31--\S37, \S72) & 1163 \\
\code{EquidistSieveTransfer.lean} & Sieve transfer, coprimality, neg-inv involution (\S38--\S78) & 1457 \\
\code{CMEVariants.lean} & CME weaker variants (CME\_d, CME\_avg, CME\_subseq, CME\_target) & 113 \\
\code{SieveDefinedDynamics.lean} & Abstract SDDS framework, factoring rules & 195 \\
\code{SDDSBridge.lean} & SDDS $\leftrightarrow$ EM bridge, orbit/walk/mult correspondence & 311 \\
\code{SDDSReduction.lean} & StrongSME $\Rightarrow$ MC via sieve-map equidist.\ & 126 \\
\code{ExcursionIndependence.lean} & Walk excursions, char.\ sum independence, avoidance discrepancy & 461 \\
\code{CRTFiberIndependence.lean} & SE $\Rightarrow$ NAO, CRT independence, death channel & 296 \\
\code{WeakMullin.lean} & Missing primes, Weak Mullin, reciprocal divergence, EMBV $\Rightarrow$ MC & 363 \\
\code{CMEDecomposition.lean} & CME $=$ A $+$ B decomposition, surjection lemma (\S90--\S92) & 199 \\
\code{LargeSieve.lean} & BV, ALS, ArithLS, MMCSB, sieve bridge (\S41--\S52, \S79) & 1795 \\
\code{LargeSieveHarmonic.lean} & Parseval, Gauss sums, DFT, kernel (\S53--\S55) & 899 \\
\code{LargeSieveAnalytic.lean} & Gauss inversion, WeakALS, GCT, dead ends (\S56--\S65, \S81--\S82) & 1683 \\
\code{LargeSieveSpectral.lean} & Walk energy, HOD, VdC, CME, VCB, SVE, transition matrix (\S66--\S86) & 2693 \\
\midrule
\code{IKCh1.lean} & Iwaniec--Kowalski~\cite{IwaniecKowalski2004} Ch.\,1: arithmetic functions & 437 \\
\code{IKCh2.lean} & Iwaniec--Kowalski Ch.\,2: summation formulas & 270 \\
\code{IKCh3.lean} & Iwaniec--Kowalski Ch.\,3: combinatorial sieve & 557 \\
\code{IKCh4.lean} & Iwaniec--Kowalski Ch.\,4: summation formulas & 593 \\
\code{IKCh5.lean} & Iwaniec--Kowalski Ch.\,5: Kloosterman sums & 877 \\
\code{IKCh7.lean} & Iwaniec--Kowalski Ch.\,7: bilinear forms, duality, Gram, MLS, sieve & 2722 \\
\bottomrule
\end{tabular}
\end{center}

\subsection{Axiom Usage: What's Constructive}

The core definitions (\code{seq}, \code{prod}, \code{aux}) and
their basic properties (\code{seq\_isPrime}, \code{seq\_injective})
are \textbf{fully constructive}: they use only \code{propext} and
\code{Quot.sound} (no \code{Classical.choice}, no
\code{Decidable} instances beyond~$\NN$).  Euclid's theorem
itself (\code{Euclid.lean}) is constructive.

Classical reasoning enters at the reduction level:
\begin{itemize}[nosep]
\item The $\HH \Rightarrow \MC$ proof uses well-founded induction
  (strong induction on $\NN$), which in Lean~4 is constructive but
  relies on \code{Classical.choice} for the
  cofinal-implies-hit argument.
\item Character theory (orthogonality, Fourier inversion) is
  inherently classical via \code{open Classical}.
\item All open hypotheses are stated as \code{def \ldots : Prop},
  never as \code{sorry}'d theorems.  The type-checker guarantees
  that no proof obligation is silently assumed.
\end{itemize}

\subsection{Mathlib Dependencies}

The formalization draws on several Mathlib libraries:
\begin{itemize}[nosep]
\item \textbf{Group theory}: \code{Subgroup}, \code{QuotientGroup},
  \code{orderOf}, cyclic group structure, maximal subgroups
  (\code{Subgroup.IsCoatom}).
\item \textbf{Number theory}: \code{Nat.minFac}, Legendre symbols,
  quadratic reciprocity, \code{ZMod}, Dirichlet characters, Gauss sums.
\item \textbf{Character theory}: \code{DirichletCharacter.Orthogonality},
  roots of unity in algebraically closed fields, character bounds,
  \code{MulChar.sum\_eq\_zero\_of\_ne\_one}.
\item \textbf{Analysis}: \code{norm\_sum\_le}, complex norms,
  \code{IsOfFinOrder.norm\_eq\_one}, Fourier analysis on $\ZZ/n\ZZ$
  (\code{ZMod.dft}, discrete Fourier transform).
\item \textbf{Dirichlet's theorem}: \code{Nat.infinite\_setOf\_prime\_and\_eq\_mod}
  (primes in arithmetic progressions, via $L$-series).
\item \textbf{Harmonic analysis}: Parseval's theorem for finite abelian
  groups, trigonometric exponentials, geometric series identities.
\end{itemize}

\begin{center}
\begin{tabular}{lr}
\toprule
Lines of Lean code & ${\sim}29{,}600$ \\
Files & 42 \\
Theorems/lemmas & ${\sim}1{,}000$ \\
Definitions & ${\sim}500$ \\
\code{sorry} occurrences & \textbf{0} \\
Open hypotheses (stated as \code{def}) & ${\sim}35$ \\
Mathlib version & v4.27.0 \\
\bottomrule
\end{tabular}
\end{center}

