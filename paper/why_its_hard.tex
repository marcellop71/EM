%% =========================================================================
\section{Why It's Hard}
\label{sec:hard}
%% =========================================================================

\noindent\textbf{Original contribution.}\enspace
The selectability analysis, oracle barrier, and CCSB-as-frontier
argument below are new.  They explain \emph{why} the remaining
hypothesis resists both computation and existing analytic tools.

\subsection{The Selectability Perspective}
\label{sec:selectability}

The Euclid construction guarantees fresh primes at every step: every
prime factor of~$\Prod(n)+1$ is new (coprime to the running product).
The difficulty of MC is not the \emph{existence} of new primes but
whether the $\minFac$ rule eventually \emph{selects} each one.  The
formalization makes this contrast precise.

\begin{theorem}[\lean{EM/EquidistOrbitAnalysis.lean}{divisor\_not\_yet\_in\_seq}]
\label{thm:divisor-fresh}
If $p \mid \Prod(n)+1$, then $\seq(m) \neq p$ for all $m \leq n$.
\end{theorem}

\begin{proof}
Any $\seq(m)$ with $m \leq n$ divides $\Prod(n)$.  A number $\geq 2$
cannot divide both~$a$ and~$a+1$.
\end{proof}

\begin{theorem}[\lean{EM/EquidistOrbitAnalysis.lean}{passed\_over\_persists}]
If $p \mid \Prod(n)+1$ but $\seq(n\!+\!1) \neq p$ (the $\minFac$ rule
chose a smaller prime), then $\seq(m) \neq p$ for all $m \leq n+1$.
The prime survives to potentially divide future Euclid numbers.
\end{theorem}

\begin{theorem}[\lean{EM/EquidistOrbitAnalysis.lean}{selectability\_extinguished}]
\label{thm:extinct}
Once $\seq(m) = p$, we have $p \mid \Prod(n)$ for all $n \geq m$, so
$p \nmid \Prod(n)+1$ ever again.  Selectability is a one-shot resource.
\end{theorem}

\begin{definition}[\defname{InfinitelySelectable}]\label{def:inf-sel}
A prime~$p$ is \emph{infinitely selectable} if $p \mid \Prod(n)+1$ for
cofinally many~$n$: $\forall\, N,\, \exists\, n \geq N,\,
p \mid \Prod(n)+1$.
\end{definition}

By Theorem~\ref{thm:extinct}, MC$(p)$ and
Infinitely\-Selectable$(p)$ are mutually exclusive
(\lean{EM/EquidistOrbitAnalysis.lean}%
{mc\_implies\_not\_infinitely\_selectable}):
a prime that enters the sequence can never be selectable again.

\begin{theorem}[\lean{EM/EquidistOrbitAnalysis.lean}{dh\_implies\_infinitely\_selectable}]
\label{thm:dh-inf-sel}
Under DH, every prime that never appears in the sequence (with SE
satisfied) is infinitely selectable.
\end{theorem}

\paragraph{The random-factor variant is easy.}
Consider a variant of the Euclid--Mullin construction where, instead of
the smallest prime factor, one picks a \emph{random} prime factor
of~$\Prod(n)+1$ at each step.  In this variant, MC follows from DH
alone: whenever $p \mid \Prod(n)+1$, simply choose~$p$.  Under DH with
SE, this happens infinitely often (Theorem~\ref{thm:dh-inf-sel}), so
$p$ eventually gets picked.

The argument is even simpler probabilistically.  For any target
prime~$p$, the residue $r_n = \Prod(n) \bmod p$ performs a
multiplicative walk on $(\ZZ/p\ZZ)^\times$.  In the random-factor
variant, each multiplier is a random element of the group; once the
multipliers generate the full group (which PRE guarantees), the walk
is a genuine random walk with full support.  By classical
equidistribution on finite groups, $r_n$ converges to uniform,
so $r_n = -1$ (i.e., $p \mid \Prod(n)+1$) occurs with
probability $\to 1/(p\!-\!1)$---infinitely often with
probability~1.

\paragraph{The lpf variant is hard.}
The actual Euclid--Mullin sequence uses
$\seq(n{+}1) = \minFac(\Prod(n){+}1)$, a \emph{deterministic}
function of the walk position.  This creates the exact correlation
identified by the oracle analysis (\S\ref{sec:oracle}): the
multiplier at step~$n$ depends on the full value of~$\Prod(n)+1$,
coupling walk position to multiplier.  The random-factor variant
breaks this coupling by choosing multipliers independently of
position; the $\minFac$ rule preserves it.

The difficulty of Mullin's Conjecture is entirely in the minimality of
the prime selection, not in the Euclidean construction itself.  The
inductive bootstrap (Section~\ref{sec:bootstrap}) bridges this gap:
$\MC({<}\,p)$ ensures all primes below~$p$ are already in the sequence,
hence divide~$\Prod(n)$, hence cannot divide~$\Prod(n)+1$.  Past a
computable stage, $p$ is the \emph{smallest} available factor whenever
it divides the Euclid number---reducing the $\minFac$ variant to the
``any-factor'' variant for the tail of the sequence.

\subsection{The Marginal/Joint Barrier}
\label{sec:oracle}

The verified reductions (TailSE, CofinalEscape, QuotientDH) exhaust
what can be proved about the \emph{marginal} distribution of multiplier
residues.

\begin{theorem}[\lean{EM/EquidistOrbitAnalysis.lean}{emfe\_iff\_tail\_se\_at}]
$\mathrm{EuclidMinFacEscape}(q) \Leftrightarrow \mathrm{TailSE}(q)$.
\end{theorem}

Even perfect per-position equidistribution of multipliers is consistent
with HH failure.  DH is a \emph{joint} statement---the
(position, multiplier) pair must hit the \emph{death curve}
$\multZ{q}{n} = -\walkZ{q}{n}^{-1}$---and no marginal statement can
force this.

\paragraph{The orbit chain gap.}
The cofinal orbit analysis picks one cofinal multiplier~$s_x$ per
walk position, producing a cycle $x_0 \to x_1 \to \cdots \to x_0$
in $(\ZZ/q\ZZ)^\times$.  Even when the cofinal multipliers generate
the full group, the cycle size~$k$ can be less than~$|(\ZZ/q\ZZ)^\times|$.
Example: in $\ZZ/6\ZZ$, the cycle $0 \to 1 \to 0$ has
$\langle 1, 5 \rangle = \ZZ/6\ZZ$ but misses~3.

Closing this gap requires showing that at each cofinally visited
position, \emph{multiple} multiplier classes appear---expanding the
cycle until it covers $-1$.  This is the ``specific-orbit problem'':
transferring generic equidistribution of $\minFac$ residues to the
particular EM orbit.

\subsection{The BRE Impossibility for $d \geq 3$}

\begin{remark}\label{rem:bre-impossible}
Positive escape density (PED) alone does \emph{not} imply
CCSB for characters of order $d \geq 3$.

\emph{Counterexample}: a walk on $\ZZ/3\ZZ$ that alternates between
only two of the three $d$-th roots of unity (phase-aligned escapes)
achieves positive escape density yet has walk sum
$\approx N/2 \cdot (1 + \omega) \neq o(N)$.

For $d = 2$ this degeneracy vanishes: the only non-trivial rotation
is~$-1$, so escape frequency \emph{is} the rotation distribution.
The order-2 BRE from NoLongRuns$(L)$ is proved in the formalization
(\lean{EM/EquidistSelfCorrecting.lean}{order2\_noLongRuns\_mc}).
But for $d \geq 3$, PED constrains how often the walk rotates without
constraining the \emph{distribution} among $d-1$ non-identity
rotations.  The $\mathrm{PED} \Rightarrow \mathrm{BRE} \Rightarrow
\mathrm{CCSB}$ factorization is invalid for $d \geq 3$.

This barrier is specific to the PED route.  The CME $\to$ CCSB
reduction (\lean{EM/LargeSieveSpectral.lean}{cme\_implies\_ccsb})
bypasses PED and BRE entirely, working for all character orders $d$
via the telescoping identity.  The $d \geq 3$ problem is therefore
not a barrier for the \emph{reduction}---only for the particular
factorization through PED.
\end{remark}

\paragraph{The Van der Corput Barrier}

The van der Corput inequality (Theorem~\ref{thm:vdc}, now fully proved
in the formalization) converts character sum bounds into autocorrelation
bounds.  Theorem~\ref{thm:shift-one} gives $R_1 = o(N)$ under the
Decorrelation Hypothesis.  VdC with $H = 1$
yields $|S_N|^2 \leq \frac{N+1}{2}(N + 2|R_1|) = N^2/2 + o(N^2)$,
hence $|S_N| \leq N/\sqrt{2}$.  This is non-trivial but \emph{not}
$o(N)$.  To get $o(N)$, one needs higher-order correlations $R_h = o(N)$
for $h \geq 2$, which requires HigherOrderDecorrelation
(Theorem~\ref{thm:hod-mc}).
The telescoping identity $\sum_n \chi(w(n))(\chi(m(n))-1) = O(1)$
is a precise structural constraint.

\paragraph{The Walk Bridge Falsity}
\label{sec:walk-bridge-false}

The BV route decomposes into two stages: sieve transfer
(BV~$\Rightarrow$~EMMultCSB, bounding \emph{multiplier} character sums)
and the walk bridge (EMMultCSB~$\Rightarrow$~MMCSB, converting multiplier
bounds to \emph{walk} bounds).  The walk bridge
\textbf{MultCSBImpliesMMCSB}
(\lean{EM/LargeSieve.lean\#L1243}{MultCSBImpliesMMCSB})
is stated as an open \code{Prop} and is \textbf{false in general}.

The obstruction is structural: the walk character sum is a
\emph{partial product} $\chi(w(n)) = \prod_{k<n} \chi(m(k))$
of the multiplier characters.  Even when the individual factors
$\chi(m(k))$ are equidistributed on the unit circle (so their
\emph{sum} cancels), their \emph{partial products} perform a
multiplicative walk whose norm grows as~$\sqrt{N}$,
not~$o(N)$.  Cancellation of sums does not imply cancellation
of cumulative products.

This negative result explains why the CME bypass
(Definition~\ref{def:cme}) is essential.  CME uses fiber
decomposition and telescoping to go directly from conditional
multiplier equidistribution to CCSB, circumventing the walk bridge
entirely.

\subsection{The Factorization Independence Heuristic}
\label{sec:hash-heuristic}

The preceding barriers explain why MC is hard to \emph{prove}.  This
subsection explains why it should be \emph{true}---and why the
formalization's sole remaining hypothesis (CME) is the precise
mathematical content of a natural intuition about factorization.

\paragraph{The information bottleneck.}
The information bottleneck of \S\ref{sec:sufficient-conditions}---$O(\log q)$
bits visible to the walk versus ${\sim}2^n$ bits determining the
multiplier---means the mutual information vanishes exponentially.
This is directly analogous to the pseudorandomness of iterated hash
functions: if~$H$ is a cryptographic hash, the sequence
$x, H(x), H(H(x)), \ldots$ is deterministic but statistically
indistinguishable from random, because the hash destroys recoverable
correlations.  For the EM sequence, integer factorization plays the
role of the hash function.  Both processes are fully deterministic
(given the seed, every term is uniquely determined), one-way
(computing forward is trivial; extracting structure from the output
is computationally hard), and \emph{de facto} uncorrelated
(consecutive terms pass every reasonable test for independence).

The analogy is heuristic, not rigorous.  SHA-256 is \emph{designed}
for pseudorandomness; $\minFac$ is not designed for anything.  We do
not claim that computational hardness of factoring implies MC---what
MC requires is a number-theoretic statement (CME), not a
complexity-theoretic one.  But the analogy explains the structure of
the problem: proving pseudorandomness of a deterministic process
requires showing that \emph{no exploitable structure exists}, which is
harder than finding structure.  This is why the conjecture resists
proof despite overwhelming heuristic evidence.

\paragraph{Negative analogies: why the selection rule matters.}
The factorization independence heuristic explains why the EM sequence
($\minFac$ variant) should contain all primes, while related sequences
do not.

\begin{itemize}[nosep]
\item The \textbf{Sylvester sequence} $s(n{+}1) = s(0) \cdots s(n) + 1$
  has density-zero prime divisors~\cite{Odoni1985}.  Its terms grow
  doubly exponentially, giving each prime only $O(1)$ chances to appear
  as a factor.  The ``hash chain'' runs too fast.

\item \textbf{Fermat numbers} $F_n = 2^{2^n} + 1$ have the even stronger
  property that $\sum 1/p$ converges over their prime divisors.  Again,
  doubly exponential growth limits opportunities.

\item The \textbf{second EM sequence} ($\mathrm{maxFac}$ variant)
  provably omits infinitely many primes~\cite{CoxVdP1968,
  PollackTrevino2014}.  Here the ``hash function'' ($\mathrm{maxFac}$
  instead of $\minFac$) produces large multipliers, causing the product
  to grow rapidly---analogous to using a hash function that amplifies
  rather than compresses.

\item The \textbf{first EM sequence} ($\minFac$ variant) keeps the
  product growing slowly---heuristically as $\exp(n^2/2)$
  (see~\S\ref{sec:lean}).  This is analogous to a hash function that
  compresses, giving exponentially many iterations within any fixed
  modulus.  The conjecture is that this compression ensures coverage.
\end{itemize}

\paragraph{What the formalization adds.}
The formalization identifies three equivalent formulations of the
factoring channel's decorrelation:
\begin{enumerate}[nosep]
\item \textbf{Conditional Multiplier Equidistribution (CME)}:
  the distribution of $\minFac(\Prod(n)+1) \bmod q$, conditioned on
  $\Prod(n) \equiv c \pmod{q}$, is asymptotically independent of~$c$
  (Definition~\ref{def:cme}).
\item \textbf{Decorrelation Hypothesis}: the character sum
  $\sum \chi(\multZ{q}{n}) = o(N)$ for nontrivial~$\chi$---the
  multiplier residues have cancelling character sums, as independent
  random variables would.
\item \textbf{ComplexCharSumBound (CCSB)}: the walk character sums
  $\sum \chi(\walkZ{q}{n}) = o(N)$ for nontrivial~$\chi$---ruling out
  permanent avoidance of any residue class.
\end{enumerate}
These are related by proved implications
($\mathrm{CME} \Rightarrow \mathrm{Dec} \Rightarrow \mathrm{PED}$)
and direct bypass ($\mathrm{CME} \Rightarrow \mathrm{CCSB}$).
Each implies MC through the verified reduction chain.  The irreducible
mathematical content is: does the factoring operation destroy enough
correlation for character sums to cancel?

\subsection{Dead Ends as a Roadmap}

Over a hundred potential approaches
were explored and found to be dead ends (catalogued in full in the
project's \code{dead\_ends.md}).  Each elimination is informative:
it narrows the space of viable strategies.  A representative selection,
grouped by the type of obstruction:

\smallskip
\begin{center}
\renewcommand{\arraystretch}{1.15}
\small
\begin{tabular}{@{}p{3.6cm}p{8.8cm}@{}}
\toprule
\textbf{Dead end} & \textbf{Why it fails} \\
\midrule
\multicolumn{2}{@{}l}{\textsc{Ensemble-to-orbit transfer}:
  \emph{tool applies to generic sequences, not the specific EM orbit}} \\[2pt]
BV for EM subsequence
  & BV applies to all primes in APs, not to a greedy subsequence. \\
Furstenberg / ergodic theory
  & Standard ergodic methods assume classical multiplicativity;
    the EM sequence is recursive and non-multiplicative. \\
Diaconis--Shahshahani lemma
  & Requires i.i.d.\ random steps; inapplicable to the
    deterministic EM walk. \\
\midrule
\multicolumn{2}{@{}l}{\textsc{Independence / linearity violated}:
  \emph{tool requires additive or independent structure the walk lacks}} \\[2pt]
Large sieve for partial products
  & The large sieve handles linear sums, not multiplicative walks. \\
Abel summation
  & Converts multiplier decorrelation to walk-sum bounds, but the
    summation weights \emph{amplify} rather than cancel. \\
Self-avoidance $\Rightarrow$ CCSB
  & Self-avoidance (no repeated $\hat{\ZZ}$ positions) is invisible
    to characters, which see only residues. \\
\midrule
\multicolumn{2}{@{}l}{\textsc{Wrong algebraic structure}:
  \emph{the group or decomposition has no room for the desired bound}} \\[2pt]
Non-abelian / representation
  & $(\ZZ/q\ZZ)^\times$ is cyclic; all irreps are 1-d characters.
    No higher-dimensional structure to exploit. \\
CRT product group
  & Reformulating on $\prod_{q \leq Q}(\ZZ/q\ZZ)^\times$ makes
    the problem harder: the product group is exponentially large. \\
NoLongRuns + PED $\Rightarrow$ BRE ($d \!\geq\! 3$)
  & Variable block lengths align adversarially with character phases. \\
DPED $\Rightarrow$ CCSB ($d \!\geq\! 3$)
  & Alternating $\omega,\omega^2$ rotations satisfy DPED yet produce
    $\Theta(N)$ walk sums.  All PED-to-CCSB intermediates ruled out. \\
\midrule
\multicolumn{2}{@{}l}{\textsc{Reduces to single-modulus CCSB}:
  \emph{no genuine simplification}} \\[2pt]
Multi-modular approaches
  & All variants (BV + threshold, CRT, death coupling) collapse
    to single-modulus CCSB. \\
Death set coupling across moduli
  & Death sets $\{m : \minFac(m) \equiv -c^{-1}\}$ vary per step;
    no uniform coupling bound exists. \\
Spectral gap (deterministic walk)
  & Spectral gap theory applies to probability measures on groups
    (convergence of random sampling); the EM walk is a single
    deterministic path. \\
Information-theoretic bounds
  & Category error: entropy/mutual-information tools assume a
    random variable; the EM sequence is deterministic with zero
    entropy. \\
\bottomrule
\end{tabular}
\renewcommand{\arraystretch}{1.0}
\end{center}

\paragraph{The Four-Way Blocker.}
The majority of the 105~dead ends reduce to a single meta-obstacle:
every known technique for proving equidistribution of sequences on
finite groups requires at least \emph{one} of (1)~independence of
steps, (2)~multiplicativity of the generating function,
(3)~algebraic-geometric structure (parameter families, monodromy),
or (4)~ergodic stationarity.  The EM walk satisfies \emph{none} of
these: the steps are deterministic and mutually dependent, the walk
is not a multiplicative function, the multipliers have no known
algebraic-geometric parametrization, and the non-autonomous dynamics
(a different multiplier at each step) rule out stationarity.  This
explains why classical tools---random-walk mixing, Hal\'asz's theorem,
Katz monodromy, Birkhoff averages---all fail.

\paragraph{The telescope exhausts the algebra.}
The telescoping identity $\chi(w(n{+}1)) = \chi(w(n)) \cdot
\chi(m(n))$ is the \emph{complete} algebraic content of the walk.
Only two decomposition strategies for the character sum $S_N =
\sum_{n<N} \chi(w(n))$ exist: grouping by \emph{value} (fiber
decomposition $\to$ CME) and grouping by \emph{lag} (autocorrelation
$\to$ HOD).  Every other rearrangement---Abel summation, M\"obius
inversion, Dirichlet series, block decomposition---either reduces to
one of these two or fails outright (Abel gives $O(N^2)$ remainder in
the wrong direction; M\"obius/Dirichlet require multiplicativity).
There is no third algebraic route to CCSB\@.

\smallskip\noindent
The pattern: every approach that avoids the specific EM orbit's
joint distribution either reduces to CCSB or fails.  Since CME
implies CCSB (proved), and CME decomposes as VCB~$+$~Dec
(\S\ref{sec:vcb}), the sharpest targets are now VCB~$+$~PED or CME:
conditional equidistribution of multipliers given walk position, or
the weaker proportionality condition combined with escape density.
CME is strictly weaker than CCSB and is the \emph{irreducible
analytic content}.

\paragraph{The Mathematical Landscape}

We need to prove one of these equivalent statements for every missing
prime~$q$:
\begin{itemize}[nosep]
\item \textbf{DH}: If the multipliers generate $(\ZZ/q\ZZ)^\times$,
  the walk $\walkZ{q}{n} = \Prod(n) \bmod q$ hits $-1$ cofinally.
\item \textbf{CCSB}: For every non-trivial character
  $\chi \colon (\ZZ/q\ZZ)^\times \to \mathbb{C}^\times$, the sum
  $\sum_{n < N} \chi(\walkZ{q}{n}) = o(N)$.
\item \textbf{$d{=}2$ special case} (as a stepping stone): For every
  quadratic character~$\chi$, the $\pm 1$-valued walk character sum is
  $o(N)$.
\end{itemize}
The formalization has conclusively shown that every tool requiring
independence, classical multiplicativity, or ensemble averaging fails.
So we need ideas that exploit what the EM walk specifically has.

\paragraph{Structural Features of the EM Walk}

The dead ends above show what does not work.  Complementarily, the EM
walk has four structural features---all proved or formalized---that no
dead-end approach has successfully exploited.  Any proof of MC will
almost certainly use at least one.

\medskip\noindent\textbf{Feature~1: Super-exponential growth.}\enspace
$\Prod(n) \geq 2^n$
(\lean{EM/LargeSieve.lean}{prod\_lower\_bound\_for\_sieve}).  The
Euclid numbers $\Prod(n)+1$ grow absurdly fast.  This means the
\emph{sieve level}---the threshold below which all prime factors have
been excluded---grows super-exponentially.  By step~$n$, the Euclid
number $\Prod(n)+1$ is coprime to each of $\seq(0), \ldots, \seq(n)$,
a growing set of distinct primes.  The pool of ``available'' small
primes as factors of $\Prod(n)+1$ shrinks, but the size of
$\Prod(n)+1$ grows so fast that it must have enormous prime factors
most of the time.

\medskip\noindent\textbf{Feature~2: Mutual coprimality of Euclid
numbers.}\enspace For $m > n$, $\Prod(m)$ is divisible by
$\seq(n+1)$, which divides $\Prod(n)+1$.  So
$\Prod(m)+1 \equiv 1 \pmod{\seq(n+1)}$: successive Euclid numbers
live in different residue classes modulo earlier sequence terms.  This
coprimality structure means the Euclid numbers cannot all ``avoid'' a
residue class in a coordinated way---their residues are forced apart
by the construction.

\medskip\noindent\textbf{Feature~3: The multiplier is the smallest
prime factor.}\enspace  This is the key constraint that everyone
mentions but nobody has quantified.  If $\walkZ{q}{n} \neq -1$ (so
that $q$ does not divide $\Prod(n)+1$ as the smallest factor), then
the multiplier $\multZ{q}{n} = \minFac(\Prod(n)+1)$ satisfies
$\multZ{q}{n} \leq (\Prod(n)+1)^{1/2}$.  For a number of size
${\sim}2^n$, this smallest factor could be as small as~$3$ or as
large as~${\sim}2^{n/2}$.  The $\minFac$ rule creates a deterministic
coupling between walk position and multiplier: the multiplier at
step~$n$ depends on the full value of~$\Prod(n)+1$, not just its
residue.

\medskip\noindent\textbf{Feature~4: Self-correcting feedback.}\enspace
If the walk concentrates on certain residues mod~$q$---say
$\walkZ{q}{n} \equiv a \pmod{q}$ for many~$n$---then
$\Prod(n)+1 \equiv a+1 \pmod{q}$ for many~$n$.  The smallest prime
factor of numbers $\equiv a+1 \pmod{q}$ depends on~$a+1$, creating a
feedback loop: concentration in one residue class biases the
multiplier distribution, which in turn pushes the walk away from that
class.  This self-correcting mechanism has been formalized
(\code{EquidistSelfCorrecting.lean}), but all paths from it lead to
\textsc{SieveTransfer}---the open hypothesis that generic
$\minFac$~equidistribution transfers to the specific EM orbit.
We return to this feature in our assessment of whether DH is true
(\S\ref{sec:open}).

\medskip
These four features---growth, coprimality, the $\minFac$ selection
rule, and self-correcting feedback---are the raw material that any
successful approach must engage with.  The dead ends above fail
precisely because they treat the walk generically (as a random walk,
or as an arbitrary multiplicative walk) rather than exploiting the
specific arithmetic of the EM construction.

