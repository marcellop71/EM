\begin{abstract}
The Euclid--Mullin sequence is defined by $a(0)=2$,
$a(n{+}1) = \text{lpf}(a(0)\cdots a(n)+1)$, where $\text{lpf}$ is
the least prime factor.
Mullin's Conjecture (MC, 1963) asserts that every prime eventually
appears.  We present a Lean~4 formalization
(${\sim}26{,}900$~lines, 35~files, \textbf{zero sorry}) that reduces MC to a
single open hypothesis.

An \emph{inductive bootstrap} yields the primary reduction: the
\textbf{Single Hit Theorem} shows that MC follows if, for each
prime~$q$, the multiplicative walk on $(\ZZ/q\ZZ)^\times$ hits~$-1$
at least once past a computable bound.  The algebraic precondition
(SubgroupEscape) is free: for any prime $p \geq 5$, some odd prime
$r < p$ escapes every proper subgroup of $(\ZZ/p\ZZ)^\times$, proved
using only modular arithmetic.  A separate \emph{Fourier bridge}
gives a parallel reduction: MC follows whenever certain walk character
sums are $o(N)$.

Multiple reduction routes---algebraic, character-analytic,
sieve-theoretic---all converge on the same
\emph{orbit-specificity gap}: transferring generic equidistribution
to one deterministic orbit.  The sharpest sufficient condition is
Conditional Multiplier Equidistribution~(CME)---the statement that
the factoring operation in the EM construction destroys correlation
between consecutive multiplier residues---proved to imply the
Complex Character Sum Bound~(CCSB) for all character orders,
bypassing the $d \geq 3$ barrier.  CME decomposes as
$\mathrm{CME} = \mathrm{VCB} + \mathrm{Dec}$, where
Vanishing Conditional Bias~(VCB) is a strictly weaker hypothesis
permitting fiber sums to be proportional to visit counts rather than
small.  We prove $\mathrm{VCB} + \mathrm{PED} \Rightarrow
\mathrm{CCSB}$, giving nine independent routes to MC.
Over a hundred dead ends are
documented, precisely delineating the boundary of current methods.
\end{abstract}

