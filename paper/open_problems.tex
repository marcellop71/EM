%% =========================================================================
\section{Open Problems}
\label{sec:open}
%% =========================================================================

The formalization certifies the reductions above and pinpoints the
frontier.  This section identifies the open problems whose resolution
would close Mullin's Conjecture.

\subsection{CCSB as the Precise Frontier}

The formalization identifies \textbf{ComplexCharSumBound} as the
irreducible analytic content.  The question:

\begin{center}
\emph{Are the walk character sums $\sum_{n<N} \chi(\walkZ{q}{n})$ bounded
$o(N)$ for every non-trivial $\chi$?}
\end{center}

CCSB is a single hypothesis that implies MC with no additional
conditions.  As shown in Section~\ref{sec:character}, CCSB rules out
permanent avoidance of~$-1$: the Fourier bridge forces the hit count
at~$-1$ to be $N/(q{-}1) + o(N)$, contradicting the zero hits that
the first missing prime's walk must achieve past the sieve gap.

The walk telescoping identities (Section~\ref{sec:character}) provide
precise structural constraints.  The identity
$\sum_n \chi(w(n))(\chi(m(n))-1) = O(1)$ means that the walk sum
$S_N$ and the multiplier sum $M_N = \sum_n \chi(m(n))$ satisfy
$S_N \approx S_N + (M_N - S_N) = M_N + O(1)$ only in the crude sense;
the telescoping does \emph{not} separate them.

\subsection{Connection to Bombieri--Vinogradov}

A Bombieri--Vinogradov type result for EM walk residues would give:
\[
\sum_{\substack{q \leq Q \\ q\text{ prime}}}
\max_{a}
\Bigl| |\{n \leq N : w(n) \equiv a \pmod{q}\}|
  - \frac{N}{q-1} \Bigr|
\;\ll\; \frac{NQ}{(\log N)^A}.
\]
For non-exceptional primes, the hit count at~$-1$ would be
$\sim N/(q{-}1) > 0$, contradicting permanent avoidance; the finitely
many exceptional primes below~$Q_0$ are handled by the inductive
bootstrap (\S\ref{sec:bootstrap}), closing the conjecture.

The difficulty is that BV applies to the set of \emph{all} primes,
not to a specific subsequence.  The EM walk is deterministic and
self-referential: the walk at step~$n$ depends on the factorization
of~$\Prod(n)+1$, which depends on all previous walk values.  Standard
BV does not apply.

\subsection{Connection to Chebotarev}

The \textbf{EffectiveKummerEscape} hypothesis asserts: for each
prime~$\ell$, there exists~$B$ such that for $q \geq B$ with
$\ell \mid q\!-\!1$, some multiplier among the first~$B$ escapes the
$\ell$-th power kernel.  This is a Chebotarev-type statement for the
Kummer extension $\mathbb{Q}(\zeta_\ell, 3^{1/\ell}, \ldots,
53^{1/\ell})$: the Frobenius at~$q$ determines which multiplier
primes are $\ell$-th power residues.

An effective Chebotarev density theorem for this fixed number field
would give EKE for all but finitely many~$q$ (effectively bounded).
Combined with finite verification for the remaining~$q$, this would
prove PRE and hence SE unconditionally---but SE is
\emph{already} proved unconditionally via the elementary PRE.
The Chebotarev approach would give a stronger \emph{effective}
bound on how quickly SE kicks in, refining the density argument of
\S\ref{sec:bootstrap}.

\subsection{The Sieve-Theoretic Approach}

\textbf{Mertens\-Escape}: for any prime~$q$ and proper
subgroup~$H$, infinitely many primes outside~$H$ exist
(Dirichlet content).
\textbf{Sieve\-Amplification}: Mertens escape should force
eventual $\minFac(\Prod(n){+}1)$ escape from~$H$, via
super-exponential growth and mutual coprimality of
successive Euclid numbers.

The formally verified chain:
$\mathrm{MertensEscape} + \mathrm{SieveAmplification}
\xrightarrow{\text{proved}}
\mathrm{TailSE} \xrightarrow{\text{proved}}
\mathrm{CofinalEscape} \xrightarrow{\text{proved}}
\mathrm{QuotientDH}$.

The formalization articulates a richer sieve infrastructure in two parallel
routes:

\paragraph{Cumulative route.}
\begin{gather*}
\mathrm{PDE}
\xrightarrow{\text{Alladi}}
\mathrm{GLPFE}
\xrightarrow{\mathrm{SieveTransfer}}
\mathrm{SieveEquidist}
\xrightarrow{\text{open}}
\mathrm{NoLongRuns} \\
\xrightarrow{\text{proved}}
\mathrm{PED}
\xrightarrow{\text{open}}
\mathrm{CCSB}
\xrightarrow{\text{proved}}
\mathrm{MC}.
\end{gather*}
Here PDE is PrimeDensityEquipartition (PNT in arithmetic progressions,
a known theorem not yet in Mathlib), and GLPFE is GenericLPFEquidist
(Alladi's theorem~\cite{Alladi1977} on $\minFac$ distribution of generic integers,
also known but not formalized).  Both ends of the chain---from PDE to
GLPFE via Alladi, and from CCSB to MC via Fourier inversion---are
formally proved.

\paragraph{Window route.}
\[
\mathrm{StrongSieveEquidist}
\xrightarrow{\text{proved}}
\mathrm{NoLongRunsAt}
\xrightarrow{\text{proved}}
\mathrm{PEDAt}
\xrightarrow{\text{open}}
\mathrm{CCSB}
\xrightarrow{\text{proved}}
\mathrm{MC}.
\]
StrongSieveEquidist asserts that EM multipliers are equidistributed
within sliding windows; NoLongRunsAt and PEDAt are per-prime variants
proved by pigeonhole and block-counting respectively
(\lean{EM/EquidistSieveTransfer.lean\#L185}{strongSieveEquidist\_noLongRunsAt},
\lean{EM/EquidistSieveTransfer.lean\#L254}{noLongRunsAt\_ped}).

\paragraph{The genuine frontier.}
\textbf{SieveTransfer} is the critical open hypothesis: does the
equidistribution of $\minFac$ residues for generic integers transfer
to the specific EM orbit?  Everything above SieveTransfer is known
mathematics; everything below it is proved.  SieveTransfer is where
``known but not formalized'' meets ``genuinely open.''

The difficulty: for \emph{generic} $q$-rough integers,
$\minFac$ residues are equidistributed (by CRT + Mertens)---the
death channel gets its fair share of multipliers.  But the EM
Euclid numbers are not generic: each is determined by the entire
walk history.  Transferring the generic equidistribution to this
specific orbit is the open step.

\paragraph{Sieve-to-harmonic convergence.}
The sieve hierarchy (\S36--\S39 of \code{EquidistSelfCorrecting.lean})
and the harmonic hierarchy (\S30--\S35) converge: both produce
DecorrelationHypothesis as output.  The full chain
\[
\mathrm{SieveEquidist}
\xrightarrow{\text{proved}}
\mathrm{Dec}
\xrightarrow{\text{proved}}
\mathrm{PED}
\xrightarrow[\text{sole gap}]{\text{open}}
\mathrm{CCSB}
\xrightarrow{\text{proved}}
\mathrm{MC}
\]
is formalized, with the first two arrows machine-verified
(\lean{EM/LargeSieve.lean\#L1634}{sieve\_equidist\_implies\_decorrelation},
\lean{EM/EquidistSelfCorrecting.lean\#L139}{decorrelation\_implies\_ped}).
The sieve route achieves SieveEquidist~$\Rightarrow$~Dec via a
counting-to-character-sum bridge: SieveEquidistribution produces
\code{EMMultCharSumBound} with $Q_0 = 0$, meaning multiplier character
sums cancel for \emph{all} primes~$q$, which is exactly
DecorrelationHypothesis.  The sole remaining gap on this route is
\textbf{PEDImpliesComplexCSB}
(\lean{EM/EquidistSelfCorrecting.lean\#L109}{PEDImpliesComplexCSB}):
does positive escape density for all primes imply walk character sum
cancellation?  Any proof of SieveEquidistribution (e.g., from PNT
in APs $+$ Alladi's theorem) would immediately yield Dec and PED
for free, isolating this single bridge as the only open step.

\subsection{What Would Close the Conjecture}

The cleanest paths to MC:
\begin{enumerate}
\item \textbf{Prove CME} (sharpest target): show that the multiplier
  character sum $\sum_{\substack{n < N \\ w(n) = c}} \chi(m(n))$ is
  $o(N)$ for each walk position~$c$.  CME is strictly weaker than
  CCSB, and $\mathrm{CME} \Rightarrow \mathrm{CCSB}$ is proved
  (\lean{EM/LargeSieveSpectral.lean}{cme\_implies\_ccsb}).
  CME asks only about the \emph{conditional} distribution of
  multipliers given walk state---it does not require controlling the
  walk character sum itself.  This bypasses the $d \geq 3$ barrier
  entirely.

\item \textbf{Prove CCSB directly}: show that no non-trivial
  character sum can sustain the $\Omega(N)$ bias that permanent
  avoidance of~$-1$ would require.  The self-correcting sieve
  (concentration of EM primes in a residue class is exponentially
  self-limiting) is the strongest heuristic argument.

\item \textbf{Prove a BV-type estimate}: even an averaged bound
  over~$q$ would suffice---for $q \geq Q_0$ it rules out permanent
  avoidance, and the finitely many $q < Q_0$ are handled by the
  inductive bootstrap (\S\ref{sec:bootstrap}).

\item \textbf{Close the orbit chain gap}: show that at each cofinally
  visited walk position, at least two distinct multiplier classes
  appear.  This would force the orbit chain to expand to the full group.

\item \textbf{Prove DH directly}: show that a multiplicative walk
  on a cyclic group with a generating set of multipliers must hit
  every element cofinally.  This is a combinatorial question about
  deterministic walks.

\item \textbf{Prove SieveTransfer}: show that the EM orbit's $\minFac$
  distribution matches that of generic integers, at least on average.
  The cumulative sieve route (\S38) reduces this to known number theory
  (PNT in APs + Alladi's theorem); closing SieveTransfer gives CME
  (and hence CCSB) via the conditional multiplier equidistribution
  framework.

\item \textbf{Prove BVImpliesMMCSB or PrimeArithLSImpliesMMCSB}: the
  large sieve route (\S41--\S65) reduces MC to a transfer hypothesis.
  Given Bombieri--Vinogradov or the analytic large sieve, the remaining
  open step is showing that the EM orbit's multipliers receive their
  fair share of each residue class---in particular, that the death
  channel is not systematically avoided.  This is the same
  orbit-specificity gap as SieveTransfer, approached from a different
  mathematical toolkit.
\end{enumerate}

\subsection{Does the Walk Hit $-1$?}

The irreducible open question is not whether DH holds (cofinal
hitting is a convenient sufficient condition) but whether the walk
hits~$-1$ \emph{at least once} past the sieve gap.  The Single Hit
Theorem (Theorem~\ref{thm:single-hit}) shows that a single hit at
each prime suffices for MC; DH, CCSB, and CME are strategies for
producing that hit.

\paragraph{Evidence for.}
Two independent lines of evidence suggest the walk does hit~$-1$.
(1)~\emph{Self-correcting feedback}: the formalized sieve analysis
(\code{EquidistSelfCorrecting.lean}) shows that concentration of EM
primes in a residue class is exponentially self-limiting---a walk
biased toward missing $-1$ automatically biases the multiplier
distribution toward correcting that miss.
(2)~\emph{Analogy with Artin's conjecture}: Artin's conjecture
(that every non-square integer is a primitive root for infinitely
many primes) has the same orbit-specificity structure and is believed
true; Hooley~\cite{Hooley1967} proved it conditional on GRH.  The
walk-hitting question is the analogous statement for the EM walk
and would follow from an analogous uniformity hypothesis.

\paragraph{Evidence against.}
Two features of the EM sequence give pause.
(1)~\emph{Cox--van der Poorten}~\cite{CoxVdP1968}: the ``largest factor'' variant of
the Euclid--Mullin sequence provably misses primes.  The EM
sequence's completeness is not a soft consequence of the Euclid
construction but depends sensitively on the $\minFac$ selection
rule.  This fragility means heuristic arguments
(``it should work because Euclid numbers have many factors'') are
not reliable.
(2)~\emph{The $d \geq 3$ barrier}: the formalization proves that
the most natural route from multiplier escape density to walk
character sum cancellation (PED~$\Rightarrow$~BRE~$\Rightarrow$~CCSB) is
\emph{impossible} for character orders $d \geq 3$.  This is not
evidence against the hit, but it shows that producing one
requires mechanisms beyond the simplest equidistribution
framework.  The CME bypass sidesteps this barrier, but the barrier's
existence means any proof must be genuinely subtle.

\paragraph{Assessment.}
We believe the walk does hit~$-1$ at every prime, primarily because
the self-correcting sieve mechanism provides a concrete dynamical
reason (not merely a probabilistic heuristic) for the walk to
eventually hit~$-1$.  The strongest form of this belief: CME should
hold because the EM multipliers, conditioned on walk position, have
no arithmetic reason to avoid the death channel systematically.
But we acknowledge that no existing technique can prove this, and
the $d \geq 3$ barrier shows that the proof, when found, will need
to exploit the specific structure of the EM walk in ways that
current analytic number theory does not.

%% =========================================================================
\section{Summary of Verified Results}
\label{sec:summary}
%% =========================================================================

\renewcommand{\arraystretch}{1.25}
\footnotesize
\begin{longtable}{@{}p{5.5cm}l>{\raggedright\arraybackslash}p{4.5cm}@{}}
\toprule
\textbf{Result} & \textbf{Status} & \textbf{Lean identifier} \\
\midrule
\endfirsthead
\toprule
\textbf{Result} & \textbf{Status} & \textbf{Lean identifier} \\
\midrule
\endhead
\midrule
\multicolumn{3}{r}{\emph{continued on next page}} \\
\endfoot
\bottomrule
\endlastfoot
\multicolumn{3}{l}{\emph{Sequence foundations}} \\
\quad Every $\seq(n)$ is prime & Proved & \code{seq\_isPrime} \\
\quad No prime repeats & Proved & \code{seq\_injective} \\
\midrule
\multicolumn{3}{l}{\emph{Main reductions to MC}} \\
\quad \textbf{SHH $\Rightarrow$ MC} (Single Hit Theorem) & Proved & \code{single\_hit\_implies\_mc} \\
\quad \textbf{DH $\Rightarrow$ MC} (cofinal hitting route) & Proved & \code{dynamical\_hitting\_implies\_mullin} \\
\quad \textbf{CCSB $\Rightarrow$ MC} (single hypothesis) & Proved & \code{complex\_csb\_mc'} \\
\quad PE $\Rightarrow$ MC & Proved & \code{pe\_implies\_mullin} \\
\quad HH $\Rightarrow$ MC & Proved & \code{hh\_implies\_mullin} \\
\quad SE + MH $\Rightarrow$ MC & Proved & \code{se\_mixing\_implies\_mullin} \\
\quad \textbf{WE $\Rightarrow$ MC} (single Prop) & Proved & \code{walk\_equidist\_mc} \\
\midrule
\multicolumn{3}{l}{\emph{Inductive bootstrap}} \\
\quad PrimeResidueEscape (elementary) & Proved & \code{prime\_residue\_escape} \\
\quad MC($<\!p$) + PRE $\Rightarrow$ SE($p$) & Proved & \code{mc\_below\_pre\_implies\_se} \\
\quad $q$-roughness from MC($<\!q$) & Proved & \code{mc\_below\_implies\_seq\_ge} \\
\quad One-prime gap & Proved & \code{mc\_below\_cofinal\_hit\_implies\_mc\_at} \\
\quad mc\_below 11 & Proved & \code{concrete\_mc\_below\_11} \\
\midrule
\multicolumn{3}{l}{\emph{Algebraic framework}} \\
\quad Confinement Theorem & Proved & \code{confinement\_forward/reverse} \\
\quad PRE $\Leftrightarrow$ SE & Proved & \code{pre\_iff\_se} \\
\quad SE $\Leftrightarrow$ character detection & Proved & \code{se\_iff\_char\_detection} \\
\quad Maximal subgroup reduction & Proved & \code{se\_of\_maximal\_escape} \\
\quad WHP $\Leftrightarrow$ HH & Proved & \code{whp\_iff\_hh} \\
\quad SE density argument (CRT + QR) & Proved & \code{se\_qr\_obstruction} \\
\midrule
\multicolumn{3}{l}{\emph{Character sum chain}} \\
\quad Fourier bridge: CCSB $\Rightarrow$ hit count lb & Proved & \code{complex\_csb\_implies\_hit\_count\_lb\_proved} \\
\quad Decorrelation $\Rightarrow$ PED & Proved & \code{decorrelation\_implies\_ped} \\
\quad NoLongRuns$(L)$ $\Rightarrow$ PED & Proved & \code{noLongRuns\_implies\_ped} \\
\quad BRE $\Rightarrow$ PEDImpliesCSB & Proved & \code{block\_rotation\_implies\_ped\_csb} \\
\quad CME $\Rightarrow$ CCSB (all $d$, bypasses BRE) & Proved & \code{cme\_implies\_ccsb} \\
\quad CME $\Rightarrow$ MC & Proved & \code{cme\_implies\_mc} \\
\quad Walk char recurrence ($\mathbb{C}$-valued) & Proved & \code{char\_walk\_recurrence} \\
\quad Telescoping identity & Proved & \code{walk\_telescope\_identity} \\
\quad Telescoping norm $\leq 2$ & Proved & \code{walk\_telescope\_norm\_bound} \\
\quad Shift-one autocorrelation & Proved & \code{walk\_shift\_one\_correlation} \\
\quad Order-2 sign-flip chain & Proved & \code{order2\_noLongRuns\_mc} \\
\midrule
\multicolumn{3}{l}{\emph{Walk dynamics}} \\
\quad Walk--divisibility bridge & Proved & \code{walkZ\_eq\_neg\_one\_iff} \\
\quad Products strictly monotone & Proved & \code{prod\_strictMono} \\
\quad Fundamental trichotomy & Proved & \code{avoidance\_contradicts\_se\_mixing} \\
\quad Self-avoidance dichotomy & Proved & \code{self\_avoidance\_dichotomy} \\
\quad Scheduled walk coverage & Proved & \code{scheduled\_walk\_covers\_all} \\
\midrule
\multicolumn{3}{l}{\emph{Selectability analysis}} \\
\quad Divisor freshness & Proved & \code{divisor\_not\_yet\_in\_seq} \\
\quad Passed-over persistence & Proved & \code{passed\_over\_persists} \\
\quad Selectability extinction & Proved & \code{selectability\_extinguished} \\
\quad MC $\Rightarrow$ $\neg$InfinitelySelectable & Proved & \code{mc\_implies\_not\_infinitely\_selectable} \\
\quad DH $\Rightarrow$ InfinitelySelectable & Proved & \code{dh\_implies\_infinitely\_selectable} \\
\midrule
\multicolumn{3}{l}{\emph{Sieve and orbit analysis}} \\
\quad EMFE $\Leftrightarrow$ TailSE & Proved & \code{emfe\_iff\_tail\_se\_at} \\
\quad TailSE $\Rightarrow$ CofinalEscape $\Rightarrow$ QuotientDH & Proved & \code{tail\_se\_gives\_sub\_dh} \\
\quad Dirichlet: $\infty$ primes per residue class & Proved & \code{dirichlet\_residues\_independent} \\
\quad Minimality sieve + coupling & Proved & \code{minimality\_sieve} \\
\quad StrongSieveEquidist $\Rightarrow$ NoLongRunsAt & Proved & \code{strongSieveEquidist\_noLongRunsAt} \\
\quad NoLongRunsAt $\Rightarrow$ PEDAt & Proved & \code{noLongRunsAt\_ped} \\
\quad DPED $\Rightarrow$ PED & Proved & \code{dped\_implies\_ped} \\
\quad PDE $+$ sieve chain $\Rightarrow$ MC & Proved & \code{primeDensity\_chain\_mc} \\
\quad GLPFE $+$ SieveTransfer $\Rightarrow$ MC & Proved & \code{genericLPF\_chain\_mc} \\
\quad SieveEquidist $\Rightarrow$ Dec & Proved & \code{sieve\_equidist\_implies\_decorrelation} \\
\quad SieveEquidist $\Rightarrow$ PED & Proved & \code{sieve\_equidist\_implies\_ped} \\
\midrule
\multicolumn{3}{l}{\emph{Large sieve route}} \\
\quad MultiModularCSB $\Rightarrow$ MC & Proved & \code{mmcsb\_implies\_mc} \\
\quad BV chain $\Rightarrow$ MC & Proved & \code{bv\_chain\_mc} \\
\quad ArithLS chain $\Rightarrow$ MC & Proved & \code{arith\_ls\_chain\_mc} \\
\quad ALS chain $\Rightarrow$ MC & Proved & \code{als\_prime\_arith\_ls\_chain\_mc} \\
\quad \textbf{WeakALS} (\S58) & \textbf{Proved} & \code{weak\_als\_from\_card\_bound} \\
\quad Gauss sum inversion (\S57) & Proved & \code{char\_sum\_to\_exp\_sum} \\
\quad \textbf{ALS $\Rightarrow$ PrimeArithLS} (\S65) & \textbf{Proved} & \code{als\_implies\_prime\_arith\_ls} \\
\quad Jordan's inequality (\S56) & Proved & \code{sin\_pi\_ge\_two\_mul} \\
\quad Geometric sum bound (\S56) & Proved & \code{norm\_eAN\_geom\_sum\_le\_inv} \\
\quad Parseval for ZMod.dft (\S53) & Proved & \code{zmod\_dft\_parseval} \\
\quad Gauss sum norm $\|\tau\|^2 = p$ (\S54) & Proved & \code{gaussSum\_norm\_sq\_eq\_prime} \\
\quad Walk autocorrelation identities (\S53) & Proved & \code{walkAutocorrelation\_*} \\
\quad Character Parseval (\S60) & Proved & \code{char\_parseval\_units} \\
\quad All 8 GCT internal lemmas (\S56--\S62) & Proved & \code{gct\_nontrivial\_char\_sum\_le} \\
\midrule
\multicolumn{3}{l}{\emph{Open hypotheses --- live targets}} \\
\quad \textbf{SingleHitHypothesis} (weakest sufficient) & \textbf{Open} & \code{SingleHitHypothesis} \\
\quad \textbf{DynamicalHitting} & \textbf{Open} & \code{DynamicalHitting} \\
\quad \textbf{ComplexCharSumBound} & \textbf{Open} & \code{ComplexCharSumBound} \\
\quad \textbf{MultiModularCSB} & \textbf{Open} & \code{MultiModularCSB} \\
\quad DecorrelationHypothesis & \textbf{Open} & \code{DecorrelationHypothesis} \\
\quad PositiveEscapeDensity & \textbf{Open} & \code{PositiveEscapeDensity} \\
\quad \textbf{PEDImpliesComplexCSB} (sole sieve-route gap) & \textbf{Open} & \code{PEDImpliesComplexCSB} \\
\quad NoLongRuns$(L)$ & \textbf{Open} & \code{NoLongRuns} \\
\quad SieveEquidistribution & \textbf{Open} & \code{SieveEquidistribution} \\
\quad MertensEscape & \textbf{Open} & \code{MertensEscape} \\
\quad SieveAmplification & \textbf{Open} & \code{SieveAmplification} \\
\quad \textbf{SieveTransfer} (genuine frontier) & \textbf{Open} & \code{SieveTransfer} \\
\quad StrongSieveEquidist & \textbf{Open} & \code{StrongSieveEquidist} \\
\quad DistributionalPED & \textbf{Open} & \code{DistributionalPED} \\
\quad \textbf{BVImpliesMMCSB} (genuine frontier) & \textbf{Open} & \code{BVImpliesMMCSB} \\
\quad GaussConductorTransfer (all lemmas proved) & \textbf{Open} & \code{GaussConductorTransfer} \\
\quad PrimeArithLSImpliesMMCSB & \textbf{Open} & \code{PrimeArithLSImpliesMMCSB} \\
\midrule
\multicolumn{3}{l}{\emph{Known theorems --- not yet in Mathlib}} \\
\quad PrimeDensityEquipartition (PNT in APs) & Known & \code{PrimeDensityEquipartition} \\
\quad GenericLPFEquidist (Alladi~\cite{Alladi1977}) & Known & \code{GenericLPFEquidist} \\
\quad BombieriVinogradov & Known & \code{BombieriVinogradov} \\
\quad AnalyticLargeSieve & Known & \code{AnalyticLargeSieve} \\
\quad ArithmeticLargeSieve & Known & \code{ArithmeticLargeSieve} \\
\midrule
\multicolumn{3}{l}{\emph{Dead ends --- false or blocked}} \\
\quad MultCSBImpliesMMCSB (false in general, \S\ref{sec:walk-bridge-false}) & \textbf{Dead} & \code{MultCSBImpliesMMCSB} \\
\quad BlockRotationEstimate (impossible for $d \geq 3$, Remark~\ref{rem:bre-impossible}) & \textbf{Dead} & \code{BlockRotationEstimate} \\
\end{longtable}
\renewcommand{\arraystretch}{1.0}
\normalsize


