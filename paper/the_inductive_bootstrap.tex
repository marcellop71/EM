%% =========================================================================
\section{The Inductive Bootstrap}
\label{sec:bootstrap}
%% =========================================================================

With the first missing prime's situation established (\S\ref{sec:walk}), we now show that the algebraic precondition---SubgroupEscape---comes for free, and that a single hit on~$-1$ past the sieve gap suffices for MC\@.

The walk avoids~$-1$ forever.  Can it at least \emph{reach}~$-1$?
That is: does SE hold?

\begin{theorem}[\defname{PrimeResidueEscape} (\abbr{PRE});
  \lean{EM/EquidistBootstrap.lean}{prime\_residue\_escape}]\label{thm:pre}
For every prime $p \geq 5$ and every proper subgroup
$H < (\ZZ/p\ZZ)^\times$, some odd prime $r < p$ has residue
$r \bmod p \notin H$.
\end{theorem}

\begin{proof}
Suppose every odd prime $r \in [3, p)$ satisfies $r \bmod p \in H$.
Since $H$ is a subgroup, every \emph{product} of such primes is
in~$H$.  Every odd number in~$[1, p)$ factors into odd primes $< p$,
so every odd number in~$[1, p)$ maps into~$H$.  In particular,
$p - 2 \equiv -2$ and $p - 4 \equiv -4$ are both in~$H$ (both odd
and $< p$ for $p \geq 5$).  Then $2 = (-4)(-2)^{-1} \in H$, so every
even number in $[1, p)$ is in~$H$ as well.  Hence
$H = (\ZZ/p\ZZ)^\times$, contradicting $H$ proper.
\end{proof}

\begin{theorem}[\lean{EM/EquidistBootstrap.lean}{mc\_below\_pre\_implies\_se}]
\label{thm:bootstrap}
$\MC({<}\,p) + \PRE \;\Longrightarrow\; \SE(p)$.
\end{theorem}

\begin{proof}[Proof sketch]
Let $H < (\ZZ/p\ZZ)^\times$ be proper.  By PRE, some odd prime
$r < p$ has $r \bmod p \notin H$.  By $\MC({<}\,p)$, the prime~$r$
appears as $\seq(k)$ for some~$k$.  Then $\multZ{p}{k{-}1} \equiv r
\pmod{p} \notin H$.
\end{proof}

\begin{corollary}\label{cor:fmp-se}
For the first missing prime~$q$: the multipliers generate all of
$(\ZZ/q\ZZ)^\times$.  SubgroupEscape holds, and $-1$ is reachable.
\end{corollary}

The one-prime gap (Theorem~\ref{thm:one-prime-gap}) shows that
$\MC({<}\,q)$ plus a single divisibility event $q \mid \Prod(n)+1$
past the sieve gap gives $\MC(q)$.  The bootstrap
(Theorem~\ref{thm:bootstrap}) shows that $\MC({<}\,q)$ gives $\SE(q)$
for free.  These two facts compose into the primary reduction of
Mullin's Conjecture.

\begin{definition}[\defname{SingleHitHypothesis} (\abbr{SHH})]\label{def:shh}
For every prime~$q$: if $\MC({<}\,q)$ and $\SE(q)$ hold and $q$ is
missing, then there exists~$n$ past the sieve gap with
$q \mid \Prod(n)+1$.

\

\noindent Equivalently: for every prime~$q$, if $\MC({<}\,q)$ and $\SE(q)$
hold, then either $q$ already appears in the sequence, or there
exists $n \geq N_0(q)$ with $\walkZ{q}{n} = -1$.
\end{definition}

\begin{theorem}[\defname{Single Hit Theorem} --- \lean{EM/EquidistBootstrap.lean}{single\_hit\_implies\_mc}]
\label{thm:single-hit}
$\mathrm{SingleHitHypothesis} \;\Longrightarrow\; \MC$.
\end{theorem}

\begin{proof}
By strong induction on~$p$.  Assume $\MC({<}\,p)$.

\

\begin{enumerate}[nosep]
\item \textbf{Bootstrap gives SE\@.}
    $\MC({<}\,p) + \PRE \Rightarrow \SE(p)$ (Theorem~\ref{thm:bootstrap}).
    Since PRE is proved unconditionally (Theorem~\ref{thm:pre}), we
    obtain $\SE(p)$.

\item \textbf{SHH gives a hit.}
    Since $\MC({<}\,p)$ and $\SE(p)$ hold, SHH provides
    $n \geq N_0$ (past the sieve gap) with $p \mid \Prod(n)+1$.

\item \textbf{The sieve gap closes.}
    Past $N_0$, all primes $< p$ divide $\Prod(n)$ and hence cannot
    divide $\Prod(n)+1$.  So $p = \minFac(\Prod(n)+1)$, giving
    $\seq(n{+}1) = p$.
\end{enumerate}
\end{proof}

\subsection{The death channel avoidance paradox}

We now have a precise and sharp picture of the first missing prime~$q$:

\

\begin{enumerate}[nosep]
\item The walk lives in $(\ZZ/q\ZZ)^\times$ forever.
\item The multipliers generate the full group---$-1$ is reachable.
\item The walk never reaches $-1$---the death channel is avoided at
  every step past the sieve gap.
\item At each step, the death channel is a single residue class out
  of $q - 1$---a ``target'' of density $1/(q-1)$.
\end{enumerate}

\

\noindent Mullin's Conjecture is therefore equivalent to: \textbf{this situation
is impossible.}  The open problem is now:

\begin{center}
\emph{Does the multiplicative walk on $(\ZZ/q\ZZ)^\times$ defined by
the EM sequence, whose multipliers generate the full group, hit~$-1$
at least once past the sieve gap?}
\end{center}

``At least once past the sieve gap'' is the precise requirement.  Not
cofinally, not equidistributed---once.
The intuition for impossibility rests on an information-theoretic
asymmetry.  At step~$n$, the death channel
$f(n) = -\walkZ{q}{n}^{-1}$ is determined by the walk position
$\walkZ{q}{n} = \Prod(n) \bmod q$, which carries $O(\log q)$ bits of
information.  The multiplier
$\multZ{q}{n} = \minFac(\Prod(n)+1) \bmod q$ is determined by the
full integer $\Prod(n)+1$, which has $\sim 2^n$ bits.  The factoring
operation $\minFac$ extracts global arithmetic information from all
$\sim 2^n$ bits; the death channel is determined by $O(\log q)$ bits.
For the multiplier to systematically avoid the death channel, the
$O(\log q)$-bit residue would have to predict the outcome of an
operation on a $2^n$-bit integer---a ``prediction'' whose information
content vanishes exponentially.

More precisely: for a generic $q$-rough integer $N \equiv a \pmod{q}$,
the conditional distribution of $\minFac(N) \bmod q$ is
asymptotically uniform over $(\ZZ/q\ZZ)^\times$, by CRT\@.  Knowing
$N \bmod q$ does not constrain $N \bmod p$ for any other prime~$p$,
so it does not constrain which primes $\leq N^{1/2}$ divide~$N$.  The
density of the death channel among all primes is $1/(q-1)$, and the
conditional probability that $\minFac(N)$ falls in the death channel
is $1/(q-1) + o(1)$ as $N \to \infty$, \emph{independently of the
residue class~$a$}.

For the EM sequence, $\Prod(n)+1$ grows super-exponentially
($\Prod(n)+1 \geq 2^{2^n}$), so the $o(1)$ error at each step
shrinks exponentially fast.  The ``probability'' of avoiding the death
channel for~$N$ consecutive steps is heuristically
$(1 - 1/(q-1))^N \to 0$.  Converting this heuristic into a proof is
the content of the conjecture.

\subsection{Sufficient conditions for the single hit}
\label{sec:sufficient-conditions}

Several formally verified strategies produce the required single hit
past the sieve gap.  Each implies MC via the Single Hit Theorem.

\begin{definition}[\defname{DynamicalHitting} (\abbr{DH})]\label{def:dh}\label{def:hh}
For every missing prime~$q$: $\SE(q) \Rightarrow \HH(q)$, where
$\HH(q)$ (\defname{HittingHypothesis}) asks for cofinal hitting:
$\forall\, N,\; \exists\, n \geq N,\; q \mid (\Prod(n) + 1)$.
\end{definition}

\begin{theorem}[\lean{EM/EquidistBootstrap.lean}{dynamical\_hitting\_implies\_mullin}]
\label{thm:dh-mc}
$\DH \;\Longrightarrow\; \MC$.
\end{theorem}

DH is stronger than SHH: it does not assume $\MC({<}\,q)$ and asks for
infinitely many hits rather than one.  But the extra strength is
convenient---DH interfaces cleanly with character sum methods.
In the death channel language, DH asserts: if the multipliers generate
$(\ZZ/q\ZZ)^\times$, then the multiplier cannot avoid the forbidden
residue $-\walkZ{q}{n}^{-1}$ forever.  Equivalently: there is no
infinite walk on a cyclic group, with a generating set of multipliers,
that dodges a single moving target of density $1/(q-1)$ at every step.

The key structural point: SHH is \textbf{strictly weaker} than
DynamicalHitting in two ways.
First, SHH assumes $\MC({<}\,q)$, which DH does not (DH only assumes SE).
Second, SHH asks for one hit past the sieve gap, while DH asks for
cofinal hitting.  Both extra assumptions are harmless in the inductive
proof---$\MC({<}\,q)$ is always available at the inductive step, and
one hit is all the Single Hit Theorem needs---but they make SHH
genuinely easier to satisfy as a mathematical statement.

\begin{remark}\label{rem:shh-vs-dh}
The logical relationships among the hitting hypotheses are:
\[
  \HH \;\Longrightarrow\; \DH \;\Longrightarrow\; \mathrm{SHH},
\]
where $\HH$ on its own denotes the \emph{unconditional} version of
$\HH(q)$ from Definition~\ref{def:dh}: cofinal hitting asserted for
every missing~$q$ without assuming $\SE(q)$.
DH conditions cofinal hitting on SE, and SHH asks for a
single hit past the sieve gap given both $\MC({<}\,q)$ and SE\@.
All three imply MC\@.  SHH is the weakest: it assumes the most
($\MC({<}\,q)$ and SE, both provided free by the inductive bootstrap)
and demands the least (one hit, not infinitely many).
The converses need not hold.
\end{remark}

\begin{definition}[\defname{ComplexCharSumBound} (\abbr{CCSB})]\label{def:ccsb-intro}
For every missing prime~$q$ and every non-trivial character~$\chi$:
$\|S_\chi(N)\| = o(N)$, where
$S_\chi(N) = \sum_{n<N}\chi(\walkZ{q}{n})$.
\end{definition}

\begin{theorem}[\lean{EM/EquidistSelfCorrecting.lean}{complex\_csb\_mc'}]
\label{thm:ccsb-mc-intro}
$\mathrm{CCSB} \;\Longrightarrow\; \MC$.
\end{theorem}

CCSB produces the hit via Fourier inversion: if all non-trivial
character sums are $o(N)$, the hit count at~$-1$ is
$N/(q{-}1) + o(N)$, which is eventually positive.  One hit past
the sieve gap is lethal (Section~\ref{sec:character}).

\begin{definition}[\defname{ConditionalMultiplierEquidist} (\abbr{CME})]\label{def:cme-intro}
For every missing prime~$q$, every non-trivial~$\chi$, and every walk
position~$c$: the multiplier characters
$\chi(\multZ{q}{n})$ conditioned on $\walkZ{q}{n} = c$ are equidistributed
(their partial sums are $o(N)$).
\end{definition}

\begin{theorem}[\lean{EM/LargeSieveSpectral.lean}{cme\_implies\_mc}]
\label{thm:cme-mc-intro}
$\mathrm{CME} \;\Longrightarrow\; \MC$ \emph{(sharpest sufficient condition)}.
\end{theorem}

CME is the sharpest hypothesis: it asserts that the factoring
operation destroys correlation between consecutive multiplier residues
conditioned on the walk position.  It implies CCSB for all character
orders, bypassing the $d \geq 3$ barrier that blocks the
BRE route.

\begin{theorem}[\lean{EM/LargeSieveSpectral.lean}{vcb\_ped\_implies\_mc}]
\label{thm:vcb-ped-mc-intro}
$\mathrm{VCB} + \mathrm{PED} \;\Longrightarrow\; \MC$.
\end{theorem}

VCB (Vanishing Conditional Bias) is a weakening of CME that permits
fiber sums to be proportional to visit counts rather than small;
combined with PED (Positive Escape Density), it reaches CCSB\@.

Each of these strategies produces at least one hit past the sieve gap,
which the Single Hit Theorem (Theorem~\ref{thm:single-hit}) converts
into MC\@.

\paragraph{Why it's hard: the walk controls nothing}

The difficulty is entirely in the coupling between walk position and
multiplier.  At step~$n$:

\begin{itemize}[nosep]
\item The \textbf{walk position} $\walkZ{q}{n} = \Prod(n) \bmod q$
  determines the death channel $f(n) = -\walkZ{q}{n}^{-1}$.
\item The \textbf{multiplier}
  $\multZ{q}{n} = \minFac(\Prod(n)+1) \bmod q$ is what must fall
  into the death channel.
\item Both depend on $\Prod(n)$: the walk position depends on
  $\Prod(n) \bmod q$, and the multiplier depends on $\Prod(n)+1$ as
  a full integer.
\end{itemize}

The walk position sees $O(\log q)$ bits; the multiplier sees all
$\sim 2^n$ bits.  But they share the same underlying object
$\Prod(n)$.  The question is whether this shared dependence creates a
correlation strong enough for the multiplier to systematically avoid
one residue class out of $q-1$.

For a \emph{random} sequence of multipliers drawn uniformly from
$(\ZZ/q\ZZ)^\times$, the probability of avoiding the death channel
for $N$ steps is $(1 - 1/(q-1))^N \to 0$.  The EM walk is not
random---it is deterministic---but the factoring operation that
determines each multiplier is, heuristically, a decorrelating ``hash.''
This is the \textbf{factorization independence heuristic}: the
function $\minFac$, applied to a $2^n$-bit integer, produces an
output that is effectively uncorrelated with the $O(\log q)$-bit
residue class of the input.

Every open hypothesis in this paper---DH, CCSB, CME---is a precise
formalization of this heuristic.

\subsection{Supporting Infrastructure}

The remainder of this section collects the supporting results that
flesh out the bootstrap: the one-prime gap, the sieve gap, and the power residue decomposition.

\paragraph{The Sieve Gap and the One-Prime Gap}

The $q$-roughness theorem (Theorem~\ref{thm:roughness}) resolves the
selectability problem described in Section~\ref{sec:intro}: past the
sieve gap, $q$ is the smallest available factor whenever it divides
the Euclid number.  This gives the one-prime gap:

\begin{theorem}[\defname{One-prime gap} --- \lean{EM/EquidistThreshold.lean}{mc\_below\_cofinal\_hit\_implies\_mc\_at}]
\label{thm:one-prime-gap}
$\MC({<}\,q)$ plus a single hitting event
$q \mid \Prod(n)+1$ for some $n$ past the sieve gap
implies $\MC(q)$.
\end{theorem}

\paragraph{The Power Residue Decomposition}

Since $(\ZZ/q\ZZ)^\times$ is cyclic of order $q\!-\!1$, its subgroup
lattice is determined by the prime factorization of $q\!-\!1$.  A
multiplier set generates the full group if and only if it escapes every
maximal subgroup---and the maximal subgroups correspond to the prime
divisors $\ell$ of~$q\!-\!1$.  This decomposition converts SE into
independent conditions, one per prime~$\ell \mid q\!-\!1$.

\begin{definition}[\defname{PowerResidueEscape} ($\abbr{PRE}_\ell$)]
$\exists\, n : \multZ{q}{n}^{(q-1)/\ell} \neq 1$ in $(\ZZ/q\ZZ)^\times$.
\end{definition}

\begin{theorem}[\abbr{PRE} $\Leftrightarrow$ \abbr{SE} --- \lean{EM/EquidistCharPRE.lean}{pre\_iff\_se}]
$\mathrm{PRE} \;\Longleftrightarrow\; \SE$.
The forward direction uses only Lagrange's theorem; the reverse uses
cyclicity of $(\ZZ/q\ZZ)^\times$.
\end{theorem}

\paragraph{Quadratic Reciprocity Obstruction}

The power residue decomposition raises a natural question: for how many
primes~$q$ could SE actually \emph{fail}?  Since each PRE$_\ell$
condition asks only that \emph{one} multiplier among infinitely many
escapes the $\ell$-th power subgroup, SE failure requires all
multipliers to be $\ell$-th power residues for some $\ell \mid q{-}1$.
By CRT and quadratic reciprocity, the density of primes~$q$ for which
the first~$k$ multiplier primes all fail to escape the index-2
subgroup is at most $O(2^{-k})$.  Extending this to a rigorous
density-zero statement for SE failure would require controlling all
prime divisors $\ell \mid q{-}1$ simultaneously, which remains open.

