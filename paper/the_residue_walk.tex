%% =========================================================================
\section{The Residue Walk}
\label{sec:walk}
%% =========================================================================

We define two sequences by mutual recursion:
\begin{alignat}{2}
  \seq(0) &\coloneqq 2, &\qquad
  \Prod(0) &\coloneqq 2, \label{eq:base}\\
  \seq(n+1) &\coloneqq \minFac\bigl(\Prod(n) + 1\bigr), &\qquad
  \Prod(n+1) &\coloneqq \Prod(n) \cdot \seq(n+1). \label{eq:step}
\end{alignat}

\begin{theorem}[\lean{EM/MullinDefs.lean}{seq\_isPrime, seq\_injective}]\label{thm:seq-props}
Every $\seq(n)$ is prime, and the sequence is injective: no prime
appears twice.
\end{theorem}

\begin{definition}[\defname{Walk} and \defname{Multiplier}]\label{def:walk-mult}
For any prime~$q$, define the \textbf{residue walk} and the \textbf{multiplier} by projecting the accumulator and the next prime onto $\ZZ/q\ZZ$:

$$\begin{aligned}\label{eq:walk}
  &\walkZ{q}{n} \coloneqq \Prod(n) \bmod q \\
  &\multZ{q}{n} \coloneqq \seq(n{+}1) \bmod q
\end{aligned}$$
\end{definition}

\begin{proposition}[\defname{Walk recurrence} --- \lean{EM/MullinGroupCore.lean}{walkZ\_succ}]\label{prop:recurrence}
$\walkZ{q}{n\!+\!1} = \walkZ{q}{n} \cdot \multZ{q}{n}$ in
$\ZZ/q\ZZ$.
\end{proposition}

\noindent The walk can be in one of two regimes:

\

\begin{enumerate}[nosep]
\item \textbf{Living regime.}  If $q$ has not yet appeared in the
  sequence up to step~$n$, then $q \nmid \Prod(n)$ (since $\Prod(n)$
  is a product of the primes $\seq(0), \ldots, \seq(n)$, none equal
  to~$q$).  So $\walkZ{q}{n} \in (\ZZ/q\ZZ)^\times$---the walk lives
  in the group of units.

\item \textbf{Dead regime.}  If $q = \seq(k)$ for some $k \leq n$,
  then $q \mid \Prod(n)$, so $\walkZ{q}{n} = 0$.  From this point on,
  $q \mid \Prod(m)$ for all $m \geq k$, so $\walkZ{q}{m} = 0$
  forever.  The walk has \emph{collapsed to zero}
  (\lean{EM/EquidistBootstrap.lean}{walkZ\_capture\_then\_collapse}).
\end{enumerate}

\

The transition from living to dead can happen only through $-1$:

\begin{theorem}[\defname{Walk--Divisibility Bridge} --- \lean{EM/MullinGroupCore.lean}{walkZ\_eq\_neg\_one\_iff}]\label{thm:bridge}
For any prime~$q$ and any~$n$ such that $\walkZ{q}{n} \in
(\ZZ/q\ZZ)^\times$:
\[
  \walkZ{q}{n} = -1
  \quad\Longleftrightarrow\quad
  q \mid \bigl(\Prod(n) + 1\bigr).
\]
\end{theorem}

Hitting $-1$ is a \emph{necessary} condition for the walk to die (for
$q$ to enter the sequence): if $q \mid \Prod(n)+1$, then
$\walkZ{q}{n} = -1$.  But it is not sufficient by itself.  When
$\walkZ{q}{n} = -1$, the walk dies only if $q = \minFac(\Prod(n)+1)$,
i.e.\ no smaller prime divides $\Prod(n)+1$.  If a smaller prime~$p$
also divides $\Prod(n)+1$, then $\minFac(\Prod(n)+1) \leq p < q$
and $q$ is \emph{not} selected---the walk \emph{bounces off}~$-1$ and continues living.
The walk may therefore hit~$-1$ multiple times without dying.  Death
occurs only at a hit where $q$ is the smallest available factor:
\[
  \cdots \to \underbrace{\walkZ{q}{n} = -1}_{\text{bounce}}
  \to (\ZZ/q\ZZ)^\times \to \cdots \to
  \underbrace{\walkZ{q}{n_0} = -1}_{\substack{\text{lethal hit:}\\
    q = \minFac}}
  \to \underbrace{\walkZ{q}{n_0{+}1} = 0}_{\text{dead}}
  \to 0 \to \cdots
\]

What determines whether a hit on~$-1$ is a bounce or a lethal hit?
The answer depends on which other primes divide $\Prod(n)+1$.  This
is controlled by the \emph{sieve gap}.

\begin{definition}[\defname{$\MC({<}\,q)$}]\label{def:mc-below}
For a prime~$q$, write $\MC({<}\,q)$ for the statement that every
prime smaller than~$q$ appears in the EM sequence:
$\forall\, p < q,\; p \text{ prime} \;\Rightarrow\; \exists\, k,\;
\seq(k) = p$.
\end{definition}

\noindent When $\MC({<}\,q)$ holds, every prime $p < q$ has entered the sequence
at some stage~$k_p$, so $p \mid \Prod(n)$ for all $n \geq k_p$.
Past a uniform stage $N_0 = \max_p k_p$, every prime $p < q$ divides
$\Prod(n)$, and therefore cannot divide $\Prod(n)+1$ (since
$\gcd(\Prod(n), \Prod(n)+1) = 1$).  This is the \textbf{sieve gap}:
past~$N_0$, the only primes that can divide $\Prod(n)+1$ are
$\geq q$.

\begin{theorem}[\defname{$q$-roughness} --- \lean{EM/EquidistThreshold.lean}{mc\_below\_implies\_seq\_ge}]\label{thm:roughness}
If $\MC({<}\,q)$ holds, then $\exists\, N_0$ such that
$\seq(n{+}1) \geq q$ for all $n \geq N_0$.
\end{theorem}

\noindent The sieve gap transforms hitting~$-1$ from a necessary condition into
a sufficient one:

\begin{theorem}[\defname{Lethal hit} --- \lean{EM/EquidistThreshold.lean}{mc\_below\_hit\_is\_lethal}]\label{thm:lethal-hit}
If $\MC({<}\,q)$ holds and $\walkZ{q}{n} = -1$ for some $n$ past the
sieve gap, then $\seq(n{+}1) = q$.
\end{theorem}

\begin{proof}
Since $n$ is past the sieve gap, no prime $< q$ divides
$\Prod(n)+1$ (Theorem~\ref{thm:roughness}).  But $\walkZ{q}{n} = -1$
gives $q \mid \Prod(n)+1$, so $q = \minFac(\Prod(n)+1)$, and
$\seq(n{+}1) = q$.
\end{proof}

\begin{definition}[Missing prime]\label{def:missing}
A prime~$q$ is \textbf{missing} if $\seq(n) \neq q$ for all~$n$.
\end{definition}

\noindent Mullin's Conjecture asserts that no prime is missing.
Under $\MC({<}\,q)$, the first hit on~$-1$ past the sieve gap is
lethal.  This immediately constrains missing primes:

\begin{theorem}[\lean{EM/EquidistThreshold.lean}{mc\_below\_missing\_walk\_ne\_neg\_one}]\label{thm:no-cofinal-hit}
If $\MC({<}\,q)$ holds and $q$ is missing, then the walk \emph{never}
hits~$-1$ past the sieve gap: $\exists\, N_0$ such that
$\walkZ{q}{n} \neq -1$ for all $n \geq N_0$.
\end{theorem}

\begin{proof}
Any hit past the sieve gap would give $\seq(n{+}1) = q$
(Theorem~\ref{thm:lethal-hit}), contradicting $q$ missing.
\end{proof}

\noindent For the first missing prime~$q$, the hypothesis $\MC({<}\,q)$ holds
by definition.  So its walk avoids~$-1$ permanently past the sieve
gap, the walk $\walkZ{q}{\cdot}$ is infinite: it stays
in~$(\ZZ/q\ZZ)^\times$ forever, never collapsing to zero.

\subsection{The forbidden multiplier and the death channel}

Suppose the walk is at position $c \in (\ZZ/q\ZZ)^\times$ at step~$n$.
The walk would hit $-1$ at step $n{+}1$ if and only if
\[
  \walkZ{q}{n} \cdot \multZ{q}{n} = -1,
  \quad\text{i.e.,}\quad
  \multZ{q}{n} = -c^{-1}.
\]

\noindent So at every step, there is exactly \textbf{one forbidden multiplier}
$f(n) = -\walkZ{q}{n}^{-1}$---the unique residue class that would
send the walk to~$-1$.  This is $1$~element out of $q-1$ possible
unit residues.

\begin{definition}[\defname{Death channel}]\label{def:death-channel}
The \textbf{death channel} at position $c$ is the residue class
$b(c) = -c^{-1} \bmod q$.
\end{definition}

\noindent When the walk is at~$c$, the death channel is the set of primes
$p \equiv -c^{-1} \pmod{q}$, which by Dirichlet's theorem has density $1/(q-1)$ among all primes.
The death channel \emph{moves} with the walk: as $c$ changes from step
to step, the forbidden class changes with it.  The map $c \mapsto
-c^{-1}$ is a bijection on $(\ZZ/q\ZZ)^\times$ (it is the composition
of inversion and negation).

\begin{theorem}[\lean{EM/EquidistSieve.lean}{walk\_hits\_neg\_one\_iff\_mult\_eq\_forbidden}]\label{thm:forbidden}
For any step~$n$ with $\walkZ{q}{n} \in (\ZZ/q\ZZ)^\times$:
\[
  \walkZ{q}{n{+}1} = -1
  \quad\Longleftrightarrow\quad
  \multZ{q}{n} = -\walkZ{q}{n}^{-1}.
\]
\end{theorem}

\noindent The statement ``$q$ is missing'' is therefore equivalent to:
\[
  \text{For all } n: \quad \multZ{q}{n} \neq -\walkZ{q}{n}^{-1}.
\]
In words: the multiplier avoids the death channel at every step, forever.
This brings us to discuss multipliers.

\subsection{The Confinement Theorem and SubgroupEscape}

Before asking whether the walk hits~$-1$, we must ask whether it
\emph{can}.  The walk is multiplicative:
$\walkZ{q}{n} = \walkZ{q}{0} \cdot \prod_{k < n} \multZ{q}{k}$.
The reachable set is the coset
$\walkZ{q}{0} \cdot \langle \text{multipliers} \rangle$.

\begin{theorem}[\defname{Confinement} --- \lean{EM/MullinGroupCore.lean}{confinement\_forward}]\label{thm:confinement}
If every multiplier lies in a subgroup $H \leq (\ZZ/q\ZZ)^\times$,
then the walk is confined to the coset $\walkZ{q}{0} \cdot H$.
In particular, if $-1 \notin \walkZ{q}{0} \cdot H$, the walk
\emph{never} hits~$-1$.
\end{theorem}

\noindent This motivates:

\begin{definition}[\defname{SubgroupEscape} (\abbr{SE})]\label{def:se}
For a prime~$q$: no proper subgroup $H < (\ZZ/q\ZZ)^\times$ contains
all multipliers.  Equivalently,
$\langle \multZ{q}{n} : n \in \NN \rangle = (\ZZ/q\ZZ)^\times$.
\end{definition}

\begin{theorem}[\lean{EM/MullinGroupEscape.lean}{se\_of\_maximal\_escape}]
SE holds iff multipliers escape every \emph{maximal} proper subgroup.
Since $(\ZZ/q\ZZ)^\times$ is cyclic, the maximal subgroups are the
index-$\ell$ subgroups for prime $\ell \mid q\!-\!1$.
\end{theorem}

\noindent When SE holds, the walk is not confined to any proper coset: every
element of $(\ZZ/q\ZZ)^\times$, including~$-1$, is \emph{reachable}.
But reachability does not imply hitting---that is a dynamical question.

\subsection{The first missing prime}

We now assemble the key argument.  Suppose, toward contradiction, that
Mullin's Conjecture is false.  Then there exists a \textbf{first
missing prime}: the smallest prime~$q$ that never appears in the EM
sequence. For this~$q$:

\begin{enumerate}[label=(\alph*),nosep]
\item $\MC({<}\,q)$ holds---by definition, every prime smaller than~$q$
  is in the sequence.
\item The walk $\walkZ{q}{\cdot}$ is infinite---since $q$ is missing,
  $\walkZ{q}{n} \in (\ZZ/q\ZZ)^\times$ for all~$n$.
\item Past the sieve gap, the walk never hits~$-1$.  Here is why.
  Eventually (say for $n \geq N_0$), all primes $< q$ have entered
  the sequence and divide $\Prod(n)$.  For $n \geq N_0$, no prime
  $< q$ can divide $\Prod(n)+1$ (since it divides $\Prod(n)$ and
  $\gcd(\Prod(n), \Prod(n)+1) = 1$).  So if $\walkZ{q}{n} = -1$ for
  some $n \geq N_0$, then $q \mid \Prod(n)+1$, and $q$ is the smallest
  prime factor of $\Prod(n)+1$ (all smaller primes are excluded).
  Then $\seq(n{+}1) = q$, contradicting $q$ being missing.
\end{enumerate}

\

\noindent Therefore: \textbf{for the first missing prime~$q$, the walk is an
infinite trajectory on $(\ZZ/q\ZZ)^\times$ that avoids the death
channel at every step past the sieve gap.} Equivalently: $\multZ{q}{n} \neq -\walkZ{q}{n}^{-1}$ for all $n \geq N_0$.

