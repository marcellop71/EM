%% =========================================================================
\appendix

\section{History and Computational Status}
\label{app:history}
%% =========================================================================

Mullin posed the conjecture in 1963~\cite{Mullin1963}.  In over sixty
years, no proof has been found, and no theoretical approach has come
close.  The problem sits in an unusual position: it is elementary to
state, each individual step is deterministic, yet the global behavior
of the sequence appears completely intractable.

\paragraph{The largest-factor variant.}
Cox and van der Poorten~\cite{CoxVdP1968} showed that the related sequence using
the \emph{largest} prime factor---where each term is
$\operatorname{gpf}(\Prod(n) + 1)$ instead of $\minFac$---provably
misses infinitely many primes (for instance,~$5$ never appears).  By
always jumping to the largest factor of $\Prod(n) + 1$, the sequence
leaps past small primes and can never return to them, since each Euclid
number is coprime to every earlier term.  This refuted the natural
strengthening that surjectivity holds regardless of the factor-selection
rule, and showed that the $\minFac$ rule is essential to the conjecture.

The two selection rules are also structurally different.  The largest
prime factor $P^+(n) = \operatorname{gpf}(n)$ has rich algebraic
structure: the sets $\{n : P^+(n) \leq y\}$ (smooth numbers) are
controlled by the Dickman function, and for polynomial sequences
$f(n) = \Prod(n)+1$ one can sometimes exploit algebraic curves and
Hasse--Weil bounds.  In contrast, the smallest prime factor
$P^-(n) = \minFac(n)$ is purely a \emph{sieve} object: controlling
$P^-(n)$ requires excluding small prime divisors one at a time (CRT
$+$ Mertens), and no algebraic-geometric handle exists.  This
$\operatorname{gpf}$/$\minFac$ asymmetry is one reason the largest-factor
variant admits a short proof while the smallest-factor conjecture
remains open.

\paragraph{Variants and surveys.}
Booker~\cite{BoSa2012} showed that a carefully chosen variant of the
Euclid--Mullin sequence \emph{does} contain every prime: by selecting a
specific (not necessarily smallest) prime factor at each step, one can
steer the sequence to hit every prime.  This demonstrates that the
conjecture is \emph{delicate}: the surjectivity depends on the precise
rule, not just the Euclidean structure.  Pollack and
Trevi\~{n}o~\cite{PollackTrevino2014} surveyed the problem's place in
the broader landscape of Euclid-inspired sequences, and studied
distributional properties of primes ``forgotten'' by Euclid-type
constructions.

\paragraph{Computational status.}
The sequence has been extended through a series of large-scale
factoring efforts:
\begin{itemize}[nosep]
\item Wagstaff~(1993) computed through the 43rd term.
\item In 2010, the 180-digit number $\Prod(43) + 1$ was factored via
  GNFS (General Number Field Sieve), yielding a 68-digit prime as
  $a(44)$.  Terms $a(45)$--$a(47)$ followed.
\item In 2012, Propper factored the 256-digit number $\Prod(47)+1$ by
  ECM (Elliptic Curve Method), discovering a 75-digit factor and
  extending the sequence to 51~terms
  (Booker--Irvine~\cite{BoIr2016}).
\item Finding $a(52)$ requires factoring a 335-digit number.  No
  factorization is known as of 2025.
\end{itemize}
After 51~terms, the smallest primes not yet observed are $41$ and $47$.
Note that $31$---a smaller prime---does not appear until position~$50$.

\paragraph{Why computation cannot resolve the conjecture.}
Even heroic computation is fundamentally unable to address the
conjecture.  Each new term requires factoring a number whose digit
count grows roughly linearly with the number of terms, quickly
exceeding the reach of any known factoring algorithm.  But even if we
could compute millions of terms, this would prove nothing: the
conjecture is a $\forall$-statement over all primes, and no finite
computation can rule out the possibility that some prime first appears
at an astronomically large index.

More fundamentally, the sequence exhibits a sensitive dependence on its
full history.  Each term $a(n+1) = \minFac(\Prod(n)+1)$ depends on the
\emph{complete factorization} of a number that encodes all previous
terms.  Changing a single early term alters every subsequent one.  This
global coupling is what makes the sequence appear random despite being
deterministic, and it means that local or statistical reasoning
about ``typical'' behavior is unreliable.  There are no known density
arguments, probabilistic heuristics, or sieve-theoretic bounds that
bear on the conjecture.  The problem requires a structural argument
about the sequence's long-term dynamics---which is precisely what the
formalization in the body of this paper provides.

%% =========================================================================
\section{The Bag-Theoretic Structure of the Euclid--Mullin Dynamics}
\label{app:bag}
%% =========================================================================

\paragraph{The EM sequence as a dynamical system on bags of primes.}
The state of the Euclid--Mullin construction at step~$n$ is fully captured
by the \emph{bag} (finite set) of primes collected so far:
$S_n = \{\seq(0), \seq(1), \ldots, \seq(n)\}$.
The dynamics is a deterministic map on finite subsets of primes:
\[
S_{n+1} = S_n \cup \{\min \mathcal{F}(S_n)\},
\]
where $\mathcal{F}(S) = \text{PrimeFactors}(\prod S + 1)$ produces the
\emph{factor bag} of the Euclid number $E_n = \Prod(n) + 1$.

The trajectory is entirely determined by $S_0 = \{2\}$.  The Markov
property holds: the future depends only on the current bag, not on the
order in which its elements were assembled.  The product
$\Prod(n) = \prod S_n$ is a \emph{lossless encoding} of the bag (by
unique factorization of the squarefree integer~$\Prod(n)$), and the
order of insertion is forgotten.

\paragraph{The pipeline: from bag to next prime.}
The selection of the next prime passes through a pipeline of operations,
each with distinct information-theoretic character:
\[
S_n \;\xrightarrow[\text{lossless}]{\prod(\cdot)}\;
\Prod(n) \;\xrightarrow[+1]{\text{shift}}\;
\Prod(n)+1 \;\xrightarrow[\text{factor}]{}\;
\mathcal{F}(S_n) \;\xrightarrow[\text{compress}]{\min}\;
\seq(n+1).
\]

\emph{Stage~1: Product (lossless).}  The map $S_n \mapsto \Prod(n)$ is
an injection (on sets of distinct primes), encoding ${\sim}\, n \log n$
bits of bag information into a single integer of ${\sim}\, 2^n$ digits.
This stage \emph{expands} the representation while preserving all
information.

\emph{Stage~2: The $+1$ shift (multiplicative $\to$ additive).}  The map
$\Prod(n) \mapsto \Prod(n)+1$ is the critical bridge between two worlds.
The integer~$\Prod(n)$ lives in the multiplicative world: its structure
(as a product of known primes) is completely understood.  The
integer~$\Prod(n)+1$ lives in the additive world: its factorization
bears no algebraic relationship to that of~$\Prod(n)$, beyond the
guaranteed coprimality $\gcd(\Prod(n),\, \Prod(n)+1) = 1$.  This shift
is the \emph{source of mixing} in the dynamics.

\emph{Stage~3: Factorization (revealing the new bag).}  The factorization
$\Prod(n)+1 = p_1^{a_1} \cdots p_k^{a_k}$ reveals the factor bag
$\mathcal{F}(S_n) = \{p_1, \ldots, p_k\}$.  This step is formally
lossless, but computationally opaque: predicting $\mathcal{F}(S_n)$ from
$S_n$ requires factoring a number of ${\sim}\, 2^n$ digits.

\emph{Stage~4: Minimum (catastrophic compression).}  The map
$\mathcal{F}(S_n) \mapsto \min \mathcal{F}(S_n)$ discards all but the
smallest element of the factor bag.  The compression ratio is
exponential: the input has ${\sim}\, 2^n$ bits; the output has $O(n)$
bits.  The composition of stages~2--4 is the \emph{minFac bottleneck}:
an information-destroying channel that maps ${\sim}\, 2^n$ bits to
$O(n)$ bits through arithmetically chaotic operations.

\paragraph{The orthogonal bag property.}
A fundamental structural fact distinguishes the factor bag
$\mathcal{F}(S_n)$ from the current bag~$S_n$:

\begin{proposition}[Orthogonal bags]
$\mathcal{F}(S_n) \cap S_n = \emptyset$.
That is, the primes dividing $\Prod(n)+1$ are entirely disjoint from
the primes in the current bag.
\end{proposition}

\begin{proof}
For any $p \in S_n$: $p \mid \Prod(n)$, hence
$\Prod(n)+1 \equiv 1 \pmod{p}$, so $p \nmid \Prod(n)+1$, i.e.\
$p \notin \mathcal{F}(S_n)$.
\end{proof}

This is the Euclidean coprimality at the heart of the construction, but
its bag-theoretic formulation reveals its dynamical significance: at each
step, the factorization produces a \emph{fresh sample} from the
complementary set $\mathbb{P} \setminus S_n$.  The new bag
$\mathcal{F}(S_n)$ contains no ``recycled'' primes---every element is
drawn from the universe of primes not yet seen.

Moreover, $\Prod(n)$ is \emph{squarefree} (a product of distinct primes),
so $S_n$ is genuinely an unordered set with no multiplicities.  The
accumulator~$\Prod(n)$ is the unique squarefree integer with prime
support exactly~$S_n$.

\paragraph{The CRT perspective.}
The Chinese Remainder Theorem provides the natural coordinate system for
the bag dynamics.  The integer $\Prod(n)$ is determined by its CRT
vector:
\[
\mathbf{c}(n) = (\Prod(n) \bmod 2,\;
\Prod(n) \bmod 3,\; \Prod(n) \bmod 5,\; \ldots)
\]
indexed by all primes.  For primes $p \in S_n$:
$\Prod(n) \bmod p = 0$ (since $p \mid \Prod(n)$).  For primes
$r \notin S_n$: $\Prod(n) \bmod r = \prod_{p \in S_n} (p \bmod r)
\in (\ZZ/r\ZZ)^\times$.

The $+1$ shift maps $\mathbf{c}(n) \mapsto \mathbf{c}(n) + \mathbf{1}$
componentwise.  The factorization of $\Prod(n)+1$ is determined by which
components satisfy $c_r(n) + 1 \equiv 0 \pmod{r}$, i.e., $c_r(n) = -1$.
The minFac is the \emph{winner of a race} across CRT coordinates: the
smallest prime $r \notin S_n$ for which $c_r(n) = -1$.  This race
depends on all coordinates simultaneously, but each coordinate is
determined independently (by CRT) from every other.

\paragraph{Fiber structure and non-reconstructibility.}
A natural question is whether the output of minFac carries information
about the bag that produced it.  The answer is: almost none.

\begin{proposition}[Massive many-to-one]
For any prime~$q$ and any set size~$k$, the number of bags $T$ of $k$
distinct primes satisfying $\minFac(\prod T + 1) = q$ grows
exponentially in~$k$.
\end{proposition}

The constraints on such a bag~$T$ are purely modular:
$\prod T \equiv -1 \pmod{q}$ (one congruence condition), plus
$\prod T \not\equiv -1 \pmod{p}$ for each prime $p < q$ with
$p \notin T$ (at most $\pi(q)$ non-congruence conditions).  The total
information content of the output is $O(q)$~bits, while the bag carries
${\sim}\, 2^k$~bits.  The fraction of information preserved is
$O(q / 2^k) \to 0$ super-exponentially.  Consequently, the map
$S \mapsto \minFac(\prod S + 1)$ is an \emph{exponentially lossy
summarization} of the bag.

\paragraph{MinFac as a greedy algorithm.}
The minimum operation in Stage~4 admits a natural optimization
interpretation.  Among all prime factors of $\Prod(n)+1$, the minFac
rule selects the one that increases $\log \Prod$ by the least amount:
\[
\seq(n+1) = \arg\min_{p \in \mathcal{F}(S_n)} \log p.
\]
This is a \emph{greedy algorithm for slow growth}: at each step, the
accumulator grows as slowly as possible (given the constraint of
selecting a prime factor of $\Prod(n)+1$).

Slow growth has a dynamical consequence.  Since $\Prod(n)$ grows slowly
(relative to the maxFac alternative), each target prime~$q$ receives
\emph{more trials}---more steps during which $\Prod(n)+1$ might be
divisible by~$q$---before the accumulator becomes so large that the
probability of $q$-divisibility becomes negligible.  The minFac rule
maximizes the number of opportunities each prime has to enter the
sequence.

By contrast, the maxFac rule (selecting the largest prime factor) causes
$\Prod(n)$ to grow so rapidly that small primes quickly lose any chance
of being selected, which is why the second Euclid--Mullin sequence
provably misses infinitely many primes
(Cox--van der Poorten~\cite{CoxVdP1968}, Booker~\cite{BoSa2012}).

\paragraph{The $0$-accessibility of minFac.}
We formalize the advantage of minFac over other selection rules.  Define
a selection rule $\sigma\colon 2^{\mathbb{P}} \to \mathbb{P}$ mapping a
nonempty set of primes to one of its elements.  Say $\sigma$ is
\emph{$0$-accessible} for a prime~$q$ past the sieve gap if: whenever
$q \in F$ and all primes $p < q$ have been screened, $\sigma(F) = q$.

MinFac is $0$-accessible for every~$q$: past the sieve gap, all primes
below~$q$ are excluded from $\mathcal{F}(S_n)$, so if
$q \in \mathcal{F}(S_n)$, then $q = \min \mathcal{F}(S_n)$.  The event
``$q$ divides $\Prod(n)+1$'' is \emph{sufficient} for~$q$ to enter the
sequence.

MaxFac is not $0$-accessible: even if $q \mid \Prod(n)+1$, a larger
factor is selected instead, and the probability that~$q$ is the largest
factor of a number of size ${\sim}\, 2^{2^n}$ decays faster than any
polynomial.

The $0$-accessibility of minFac ensures that the walk mod~$q$ hitting
$-1$ is both necessary and sufficient for~$q$ to enter the sequence
(past the sieve gap).  This reduces the question of whether~$q$ appears
to a question about the walk alone, with no additional selection barrier.

\paragraph{Deterministic pseudorandomness.}
The EM sequence is fully deterministic, yet exhibits pseudorandom
behavior.  The pseudorandomness arises from the composition of two
sources of arithmetic opacity:
\begin{enumerate}[nosep]
\item \emph{The $+1$ shift} moves from the multiplicative world (where
  $\Prod(n)$ is fully understood) to the additive world (where
  $\Prod(n)+1$ is arithmetically opaque).  The relationship between the
  multiplicative structures of consecutive integers is the central
  mystery of analytic number theory.
\item \emph{The factoring step} reveals the prime structure of
  $\Prod(n)+1$, but this structure is computationally hard to predict
  from~$\Prod(n)$.  The computational difficulty of factoring is, in a
  precise sense, the \emph{source of pseudorandomness} for the EM
  sequence.
\end{enumerate}
Small changes to the bag (adding one prime~$p$) transform $\Prod(n)$ to
$\Prod(n) \cdot p$ and hence $\Prod(n)+1$ to $\Prod(n) \cdot p + 1$, a
completely different integer with a completely different factorization.
The dynamics is \emph{sensitive to the bag contents}, analogous to
sensitive dependence on initial conditions in chaotic systems, but
operating through number-theoretic rather than geometric mechanisms.

The orthogonal bag property reinforces this pseudorandomness: at each
step, the factor bag is drawn from $\mathbb{P} \setminus S_n$, a set
that has never been ``used'' in the construction.  Each step provides a
genuinely fresh arithmetic input, preventing the buildup of systematic
correlations.

%% =========================================================================
\section{Analogies and Context}
\label{app:analogies}
%% =========================================================================

Mullin's Conjecture has no known applications: if proved tomorrow, no
other theorem in number theory would follow from it.  The value of the
problem lies instead in what it \emph{is an instance of}---and in the
methods its resolution would require.  The orbit-specificity barrier
identified by this formalization appears, in recognizable form, across
several active areas of mathematics.

\paragraph{Artin's conjecture and the orbit-specificity gap.}
The closest structural analogue to MC is Artin's conjecture on
primitive roots~\cite{Artin1927}: for any integer $a \neq -1$ that is not a
perfect square, the group $(\ZZ/p\ZZ)^\times$ is generated by~$a$ for
infinitely many primes~$p$.  The parallel is precise:
\begin{itemize}[nosep]
\item \textbf{Artin} asks whether the orbit of a fixed generator~$a$
  under repeated multiplication fills $(\ZZ/p\ZZ)^\times$.
\item \textbf{Mullin} asks whether the orbit of~$2$ under
  multiplication by successive EM primes in $(\ZZ/q\ZZ)^\times$ hits
  the single element~$-1$.
\end{itemize}
Both conjectures are blocked by the same fundamental obstacle:
transferring \emph{averaged} equidistribution results (which hold for
most moduli or most generators) to \emph{one specific deterministic
orbit}.  Hooley~\cite{Hooley1967} proved Artin's conjecture conditional on GRH
for Dedekind zeta functions of Kummer extensions---because GRH
provides the uniformity across individual characters needed to control
a single orbit.  In our setting, this uniformity is exactly what CCSB
demands.

The analogy is not merely structural.  The Kummer extensions
$\mathbb{Q}(\zeta_\ell, a^{1/\ell})$ that appear in Hooley's proof
are the same extensions that arise in the EffectiveKummerEscape
approach to SubgroupEscape (Section~\ref{sec:open}).  The elementary
PRE lemma (Theorem~\ref{thm:pre}) sidesteps this Chebotarev machinery
entirely for the algebraic component, but the dynamical
component---does the walk \emph{hit}~$-1$, not merely \emph{generate}
the full group?---remains exactly the orbit-specificity gap that GRH
closes for Artin and that no known tool closes for Mullin.

\paragraph{Multiplicative walks on finite groups.}
The walk reformulation (Section~\ref{sec:walk}) places MC in the
framework of random walks on finite groups, studied systematically by
Diaconis and others since the
1980s~\cite{ChungDiaconisGraham1987}.  The Diaconis--Shahshahani upper
bound lemma~\cite{DiaconisShahshahani1981} shows that a random walk on a finite group~$G$
driven by i.i.d.\ multipliers from a conjugation-invariant
distribution mixes in $O(\log |G|)$ steps, with mixing measured by
character sums.

The EM walk has the same algebraic structure---a multiplicative walk
on the cyclic group $(\ZZ/q\ZZ)^\times$---but violates every
assumption of the classical mixing theory.  The multipliers are
deterministic, not random; they are not identically distributed; and
most critically, the multiplier at step~$n$ depends on the walk
position at step~$n$, creating exactly the position-multiplier
correlation that the Diaconis--Shahshahani framework assumes away.
CCSB is the precise derandomization of the mixing-time bound: it asks
that the Fourier coefficients of the walk's occupation measure tend to
zero for every non-trivial character.

\paragraph{Smallest prime factor distribution.}
The SieveTransfer hypothesis (Section~\ref{sec:open}) connects MC to
the distribution of the smallest prime factor function
$P^-(n) = \min\{p : p \mid n\}$, studied by Alladi~\cite{Alladi1977},
Hildebrand~\cite{Hildebrand1986}, and others.  For generic integers in an arithmetic
progression $n \equiv a \pmod{q}$, the distribution of $P^-(n)$ is
controlled by the Dickman function and CRT-based equidistribution
results.  MC asks whether this equidistribution transfers from generic
integers to the specific subsequence $\{\Prod(n)+1\}_{n \geq 0}$.
The same orbit-specificity transfer problem arises for Mersenne
numbers $2^p - 1$, Fibonacci numbers, and polynomial iterates
$f^{\circ n}(a)$ in arithmetic dynamics.

\paragraph{The marginal/joint barrier and Sarnak's conjecture.}
The formalization identifies a precise meta-obstacle
(Section~\ref{sec:hard}): \emph{marginal} equidistribution of the
multiplier residues is provable (the EM primes are equidistributed in
residue classes, by Dirichlet's theorem); what MC requires is
\emph{joint} equidistribution of the pair (walk position, multiplier),
conditioned on the walk's history.  This barrier is an instance of a
broader phenomenon.  Sarnak's conjecture~\cite{Sarnak2010} asserts that the
M\"obius function $\mu(n)$ is orthogonal to every bounded
deterministic sequence:
$\frac{1}{N} \sum_{n \leq N} \mu(n) \, a_n \to 0$.
CCSB is a M\"obius-orthogonality-type statement for the EM sequence,
placing MC squarely within the Sarnak program's conceptual framework,
even though the EM sequence falls outside the technical scope of
existing results (which require zero topological entropy).

\paragraph{Greedy sieves and orbit-hitting.}
MC is the simplest nontrivial instance of a broader question: does a
greedy, deterministic prime-selection process eventually cover all
primes?  The Cox--van der Poorten result~\cite{CoxVdP1968} shows that this
determinism is fragile: choosing the \emph{largest} factor instead
provably misses primes.  More generally, MC belongs to the family of
\emph{orbit-hitting problems} in arithmetic dynamics: given a map~$T$
on a space~$X$ and a target set $S \subset X$, does the orbit
eventually enter~$S$?  Unlike Artin (where the map $x \mapsto ax$ is
the same at every step) or Collatz (where the map depends only on the
current state), the EM map varies at each step, determined by the
factorization of a number that depends on the entire orbit history.
This is the accumulator coupling of Section~\ref{sec:intro}: the
running product $\Prod(n)$ is a cumulative digest of the full orbit,
and the walk--multiplier framework makes it precise.  The
formalization shows it is the sole source of difficulty: once the
coupling is controlled (via CCSB or CME), MC follows by
machine-checked deduction.

%% =========================================================================
\section{Additional Sieve and Spectral Routes}
\label{app:routes}
%% =========================================================================

This appendix collects the sieve and spectral-energy routes to MC that
complement the three principal reductions (DH, CCSB, BV) presented in
the body.  All reduction arrows are machine-verified; the sole open
content in each route is the orbit-specificity transfer.

\subsection*{Arithmetic Large Sieve Route}

\begin{theorem}[\lean{EM/LargeSieve.lean\#L403}{arith\_ls\_chain\_mc}]
$\mathrm{ArithLS} + \mathrm{ArithLSImpliesMMCSB} \;\Longrightarrow\; \MC$.
\end{theorem}
The arithmetic large sieve gives character sum bounds for Dirichlet
characters (a known result, not in Mathlib).  The transfer
ArithLSImpliesMMCSB is open and is in fact a \textbf{dead
end}: universal coefficient bounds cannot distinguish equidistributed
walks from clumped walks.

\subsection*{Analytic Large Sieve Route}

The most developed route connects the analytic large sieve to MC via
Gauss sum inversion.

\begin{definition}[\defname{AnalyticLargeSieve} (\abbr{ALS})]
For well-separated points $\{\alpha_r\} \subset \mathbb{R}/\mathbb{Z}$
with $\min_{r \neq s} \|\alpha_r - \alpha_s\| \geq \delta$:
\[
\sum_r \left\|\sum_{n < N} a_n\, e(n\alpha_r)\right\|^2
\;\leq\; (N - 1 + \delta^{-1}) \sum_{n < N} \|a_n\|^2.
\]
\end{definition}

\begin{theorem}[\lean{EM/LargeSieveAnalytic.lean\#L490}{weak\_als\_from\_card\_bound}]
\label{thm:weak-als}
A \emph{weak} version with constant $N \cdot (\delta^{-1} + 1)$ is
proved (the optimal constant is $N - 1 + \delta^{-1}$, but the
difference is immaterial since MMCSB requires only~$o(N)$).
\end{theorem}

The key bridge is \defname{Gauss sum inversion}: a Gauss sum
$\tau(\chi) = \sum_{a} \chi(a)\, e(a/p)$ intertwines multiplication
and addition on~$\ZZ/p\ZZ$, converting character sums to exponential
sums.

\begin{theorem}[\lean{EM/LargeSieveAnalytic.lean\#L304}{char\_sum\_to\_exp\_sum}]
\label{thm:gauss-inversion}
For a non-trivial character~$\chi$ mod~$p$ prime:
$\sum_n f(n)\, \chi(n)
= \tau^{-1} \sum_{b=1}^{p-1} \chi^{-1}(b) \sum_n f(n)\, \psi(bn)$.
\end{theorem}

The GaussConductorTransfer composes eight internal lemmas (all proved,
\S56--\S62) into the bridge from ALS to the prime arithmetic large sieve:

\begin{theorem}[\lean{EM/LargeSieveAnalytic.lean\#L1358}{als\_implies\_prime\_arith\_ls}]
\label{thm:als-prime}
$\mathrm{AnalyticLargeSieve} \;\Longrightarrow\; \mathrm{PrimeArithLS}$.
\end{theorem}

\begin{theorem}[\lean{EM/LargeSieveAnalytic.lean\#L1535}{als\_prime\_arith\_ls\_chain\_mc}]
$\mathrm{ALS} + \mathrm{PrimeArithLSImpliesMMCSB} \;\Longrightarrow\; \MC$.
\end{theorem}

The remaining open content is PrimeArithLSImpliesMMCSB: transferring
prime-modulus arithmetic large sieve bounds to multi-modular character
sum bounds for the specific EM orbit.

\subsection*{The Spectral Energy Route}
\label{sec:sve}

Instead of individual character sums, this route examines the
\emph{total energy} of the walk occupation measure
$V_N(a) = |\{n < N : \walkZ{q}{n} = a\}|$.

\begin{theorem}[\lean{EM/LargeSieveSpectral.lean\#L89}{walk\_energy\_parseval}]
\label{thm:walk-energy}
$\sum_\chi \|S_\chi(N)\|^2 = (q{-}1) \sum_{a \in (\ZZ/q\ZZ)^\times} V_N(a)^2$
\quad (Parseval).
\end{theorem}

The \emph{excess energy} $E(N) = \sum_{\chi \neq 1} \|S_\chi(N)\|^2$.
If $E(N) = o(N^2)$, then every non-trivial character sum is
individually~$o(N)$, which is CCSB.

\begin{definition}[\defname{SubquadraticVisitEnergy} (\abbr{SVE})]
For every missing prime~$q$ and $\varepsilon > 0$, there exists
$N_0$ such that for $N \geq N_0$: $E(N) \leq \varepsilon N^2$.
\end{definition}

\begin{theorem}[\lean{EM/LargeSieveSpectral.lean\#L238}{sve\_implies\_mmcsb}]
\label{thm:sve-mc}
$\mathrm{SVE} \Longrightarrow \mathrm{MMCSB} \Longrightarrow \MC$.
\end{theorem}

\paragraph{Van der Corput and higher-order decorrelation.}
The van der Corput inequality bounds $|\sum z_n|$ via autocorrelations:

\begin{theorem}[\lean{EM/LargeSieveSpectral.lean\#L826}{van\_der\_corput\_bound}]
\label{thm:vdc}
$\left\|\sum_{n \leq N} z_n\right\|^2
\leq \frac{N + H}{H+1}\bigl(N + 2\sum_{h=1}^{H}
  |\mathrm{Re}\sum_{n \leq N-h} z_n \overline{z_{n+h}}|\bigr)$.
\end{theorem}

For the EM walk, lag-$h$ autocorrelations involve $h$-step multiplier
products.  At $h = 1$, the autocorrelation equals the multiplier
character sum (Theorem~\ref{thm:shift-one}), so VdC with a single
shift gives only $O(N)$.  Higher lags may decorrelate:

\begin{definition}[\defname{HigherOrderDecorrelation} (\abbr{HOD})]
For every missing prime~$q$, non-trivial~$\chi$, and $\varepsilon > 0$:
there exists $H_0$ such that for $H \geq H_0$, $N_0$ such that for
$N \geq N_0$ and all $1 \leq h \leq H$:
$\|R_h(N)\| \leq \varepsilon N$.
\end{definition}

\begin{theorem}[\lean{EM/LargeSieveSpectral.lean\#L1038}{hod\_implies\_ccsb}]
\label{thm:hod-mc}
$\mathrm{HOD} \Longrightarrow \mathrm{CCSB} \Longrightarrow \MC$.
\end{theorem}

\paragraph{Conditional multiplier equidistribution.}

\begin{definition}[\defname{ConditionalMultiplierEquidist} (\abbr{CME})]
For every missing prime~$q$, non-trivial~$\chi$, $\varepsilon > 0$,
$N_0$ such that for $N \geq N_0$ and every
$c \in (\ZZ/q\ZZ)^\times$:
$\|\sum_{\substack{n < N \\ w(n) = c}} \chi(m(n))\| \leq \varepsilon N$.
\end{definition}

\begin{theorem}[\lean{EM/LargeSieveSpectral.lean}{cme\_implies\_dec}]
$\mathrm{CME} \Longrightarrow \mathrm{DecorrelationHypothesis}$.
\end{theorem}

\begin{theorem}[\lean{EM/LargeSieveSpectral.lean}{cme\_implies\_ccsb}]
\label{thm:cme-ccsb}
$\mathrm{CME} \Longrightarrow \mathrm{CCSB}$.
\end{theorem}

\begin{proof}[Proof sketch]
The walk telescoping identity gives
$\sum \chi(w(n)) = \sum \chi(w(n))\chi(m(n)) - (\chi(w(N)) - \chi(w(0)))$.
The product sum decomposes by fiber:
$\sum \chi(w(n))\chi(m(n)) = \sum_a \chi(a) \cdot \sum_{w(n)=a} \chi(m(n))$.
CME bounds each fiber sum by $\varepsilon' N$; the triangle inequality
sums over at most $|(\ZZ/q\ZZ)^\times|$ fibers; the boundary term
$\chi(w(N)) - \chi(w(0))$ has norm~$\leq 2$ and is absorbed for large~$N$.
\end{proof}

This is the key reduction that bypasses PED, BRE, and the $d \geq 3$
barrier.  The proof works for all character orders because it uses
only the fiber decomposition and telescoping---no block rotation
estimate is needed.

\begin{theorem}[\lean{EM/LargeSieveSpectral.lean}{cme\_implies\_mc}]
$\mathrm{CME} \Longrightarrow \MC$.
\end{theorem}

\begin{proof}
Compose \texttt{cme\_implies\_ccsb} with \texttt{complex\_csb\_mc'}.
\end{proof}

\begin{theorem}[\lean{EM/LargeSieveSpectral.lean}{cme\_chain\_mc}]
$\mathrm{CME} + \mathrm{PEDImpliesCSB} \Longrightarrow \MC$.
\end{theorem}

This older route through the Dec $\to$ PED $\to$ CCSB chain is
superseded by the direct CME $\to$ CCSB reduction above, which
requires no additional hypotheses.

\subsection*{The Complete Hypothesis Hierarchy}

\[
\mathrm{PED} \;<\; \mathrm{Dec} \;<\; \mathrm{CME}
\;\xrightarrow{\text{proved}}\;
\mathrm{CCSB} \;\approx\; \mathrm{HOD}
\;\approx\; \mathrm{SVE},
\]
where ``$<$'' means strictly weaker (proved implication, known not
to reverse) and ``$\approx$'' means equivalent.
HOD~$\Leftrightarrow$~CCSB via van der Corput;
SVE~$\Leftrightarrow$~CCSB via Parseval;
CME~$\Rightarrow$~CCSB via telescoping + fiber decomposition
(Theorem~\ref{thm:cme-ccsb}).

Every hypothesis implies MC.  The PED route has an open BRE bridge
for $d \geq 3$ characters, but this is now bypassed: the direct
CME~$\to$~CCSB arrow is proved for all character orders.  CME is the
\emph{sharpest sufficient condition}---the weakest hypothesis known
to imply MC.

%% =========================================================================
\section{Methodology: Human--AI Collaboration}
\label{app:agents}
%% =========================================================================

This work was produced through a sustained collaboration between a
human author and an AI system (Claude, Anthropic) across 72+~sessions.
The human author directed the mathematical strategy---proof
architecture, dead-end identification, and editorial control---while
the AI system handled Lean~4 formalization, Mathlib API search,
literature scouting, and exploration of candidate proof strategies.

The interaction was organized at scale via an \emph{agent swarm}:
a multi-agent system built on the Claude Agent SDK\@.  The swarm comprises
seven specialized agents, each with its own system prompt, tool access,
and model:

\begin{itemize}[nosep]
\item A \emph{coordinator} that reads the current proof state,
  selects the most promising action, dispatches specialists, and updates
  shared state files.
\item A \emph{formalizer} that writes and compiles Lean code
  in rapid iteration cycles.
\item A \emph{literature scout} that searches papers and Mathlib
  for relevant results.
\item Four \emph{attack vector specialists} focused on analytic,
  algebraic, combinatorial, and information-theoretic approaches.
\item A \emph{paper writer} that maintains this document.
\end{itemize}

Agent prompts are \emph{self-evolving}: after each session the
coordinator updates them to record dead ends, new Mathlib discoveries,
and shifted priorities.  This prevents agents from rediscovering settled
territory.  All agent state (progress, strategy log, findings) is stored
as git-tracked markdown, making the exploration history fully
reproducible.

The division of labor between human and AI was sharp:

\begin{itemize}[nosep]
\item \textbf{Human:} mathematical direction, proof strategy,
  identification of dead ends, evaluation of intermediate results,
  architectural decisions on the reduction hierarchy, and editorial
  control over the final formalization and paper.
\item \textbf{AI (Claude):} Lean~4 formalization using Mathlib,
  Mathlib API search, literature scouting, exploration of candidate proof
  strategies, and drafting of this paper.
\end{itemize}

The human author guided the proof effort across 72+~sessions, suggesting
attack vectors (algebraic, analytic, combinatorial, sieve-theoretic),
identifying when an approach had reached a dead end, and pushing toward
the sharpest possible reductions.  The AI agents wrote all Lean code,
searched Mathlib for relevant lemmas, explored dozens of proof strategies
to completion or refutation, and maintained the evolving paper.

The swarm is optimized for formalization and reduction, not mathematical
discovery.  The next breakthrough, if it comes, will probably be a human
insight about the structure of $\minFac$ on EM products---not something
an agent finds by systematic search.

%% =========================================================================
\section{Glossary of Definitions and Hypotheses}
\label{sec:glossary}
%% =========================================================================

The table below collects every named definition, hypothesis, and key
theorem introduced in this paper, with abbreviations and the section
where each is defined.

\medskip
\renewcommand{\arraystretch}{1.25}
\footnotesize
\begin{longtable}{@{}l l p{7.2cm} l@{}}
\toprule
\textbf{Abbr.} & \textbf{Name} & \textbf{Meaning} & \textbf{Ref.} \\
\midrule
\endfirsthead
\toprule
\textbf{Abbr.} & \textbf{Name} & \textbf{Meaning} & \textbf{Ref.} \\
\midrule
\endhead
\midrule
\multicolumn{4}{r}{\emph{continued on next page}} \\
\endfoot
\bottomrule
\endlastfoot

\multicolumn{4}{l}{\emph{Core sequence and walk}} \\
--- & Walk / Multiplier
    & $\walkZ{q}{n} = \Prod(n) \bmod q$;\; $\multZ{q}{n} = \seq(n\!+\!1) \bmod q$
    & Def.~\ref{def:walk-mult} \\
\abbr{MC} & MullinConjecture
    & Every prime appears in the Euclid--Mullin sequence
    & Conj.~\ref{conj:mullin} \\
\midrule

\multicolumn{4}{l}{\emph{Algebraic hypotheses (\S\ref{sec:walk}--\S\ref{sec:bootstrap})}} \\
\abbr{SE} & SubgroupEscape
    & No proper subgroup of $(\ZZ/q\ZZ)^\times$ contains all multipliers
    & Def.~\ref{def:se} \\
\abbr{HH} & HittingHypothesis
    & The walk reaches $-1$ cofinally: $\forall N,\,\exists n \geq N,\; q \mid \Prod(n)+1$
    & Def.~\ref{def:hh} \\
\abbr{DH} & DynamicalHitting
    & $\SE(q) \Rightarrow \HH(q)$ for every missing prime~$q$
    & Def.~\ref{def:dh} \\
\abbr{SHH} & SingleHitHypothesis
    & $\MC({<}\,q) + \SE(q) + q$ missing $\Rightarrow$ $\exists\, n \geq N_0$
      with $q \mid \Prod(n)+1$
    & Def.~\ref{def:shh} \\
\abbr{PRE} & PrimeResidueEscape
    & Every proper subgroup of $(\ZZ/p\ZZ)^\times$ is escaped by some odd prime $< p$
    & Thm.~\ref{thm:pre} \\
$\abbr{PRE}_\ell$ & PowerResidueEscape
    & Multipliers escape the index-$\ell$ subgroup of $(\ZZ/q\ZZ)^\times$
    & \S\ref{sec:bootstrap} \\
--- & ThresholdHitting
    & DH restricted to primes $q \geq B$
    & \S\ref{sec:large-sieve} \\
\midrule

\multicolumn{4}{l}{\emph{Character-analytic hypotheses (\S\ref{sec:character})}} \\
\abbr{CCSB} & ComplexCharSumBound
    & Walk char sums $S_\chi(N) = o(N)$ for all non-trivial $\chi$
    & \S\ref{sec:character} \\
\abbr{MMCSB} & MultiModularCSB
    & CCSB simultaneously for all primes $q$ in a range
    & \S\ref{sec:character} \\
\abbr{ALS} & AnalyticLargeSieve
    & Large sieve inequality adapted to EM walk
    & \S\ref{sec:character} \\
\abbr{PED} & PositiveEscapeDensity
    & Positive density of $n$ with $\multZ{q}{n} \notin H$, for every proper $H$
    & \S\ref{sec:character} \\
--- & DecorrelationHypothesis
    & $\multZ{q}{n}$ and $\multZ{q}{n{+}1}$ are asymptotically independent
    & \S\ref{sec:character} \\
\abbr{BRE} & BlockRotationEstimate
    & Cancellation in block sums of characters applied to walk
    & \S\ref{sec:character} \\
\abbr{SVE} & SubquadraticVisitEnergy
    & $\sum_{a} |\{n \leq N : \walkZ{q}{n}=a\}|^2 = o(N^2/(q{-}1))$
    & \S\ref{sec:character} \\
\abbr{HOD} & HigherOrderDecorrelation
    & Higher-order correlation bounds for walk increments
    & \S\ref{sec:character} \\
\abbr{CME} & ConditionalMultiplierEquidist
    & Conditional equidist.\ of multipliers given walk state; implies CCSB (proved)
    & \S\ref{sec:character} \\
\midrule

\multicolumn{4}{l}{\emph{Named theorems}} \\
--- & Confinement
    & If SE fails, the walk is confined to a proper coset
    & Thm.~\ref{thm:confinement} \\
--- & Walk--Divisibility Bridge
    & $\walkZ{q}{n} = -1 \Leftrightarrow q \mid \Prod(n)+1$
    & Thm.~\ref{thm:bridge} \\
--- & One-prime gap
    & $\MC({<}\,q)$ + cofinal hit $\Rightarrow$ $\MC(q)$
    & Thm.~\ref{thm:one-prime-gap} \\
--- & QR Obstruction
    & SE failure has density $O(2^{-k})$ by CRT + quadratic reciprocity
    & \S\ref{sec:bootstrap} \\
--- & Gauss sum inversion
    & Character sums $\leftrightarrow$ exponential sums via Gauss sums
    & \S\ref{sec:character} \\
\end{longtable}
\renewcommand{\arraystretch}{1.0}
\normalsize

