\documentclass[11pt,a4paper]{article}

\usepackage[utf8]{inputenc}
\usepackage[T1]{fontenc}
\usepackage{amsmath,amssymb,amsthm}
\usepackage{mathtools}
\usepackage{hyperref}
\usepackage{enumitem}
\usepackage{array}
\usepackage{booktabs}
\usepackage{longtable}
\usepackage{tikz-cd}
\usepackage[margin=1in]{geometry}
\usepackage{scalerel}

% Inline code: sans-serif, small, dark teal, with breakable underscores.
\definecolor{codeink}{HTML}{1B6B6D}
\newcommand{\code}[1]{%
  {\color{codeink}\sffamily\small #1}%
}

% Lean 4 source links for formally verified results.
% Set \leanurl to the repository blob URL for clickable source references.
\newcommand{\leanurl}{https://github.com/marcellop71/EM/blob/main}
\definecolor{leanink}{HTML}{4338CA}
\newcommand{\lean}[2]{%
  \texorpdfstring{%
    \href{\leanurl/#1}{%
      {\color{leanink}\sffamily\footnotesize%
        \scalerel*{\checkmark}{X}\kern0.12em#2}}%
  }{#2}%
}

% Allow line breaks at underscores everywhere (tables, \code, etc.).
\renewcommand{\_}{\textunderscore\hspace{0pt}}

% Named definitions/theorems: small-caps for names, bold small-caps for abbreviations.
\definecolor{defink}{HTML}{7C3AED}
\newcommand{\defname}[1]{\texorpdfstring{{\color{defink}\textsc{#1}}}{#1}}
\newcommand{\abbr}[1]{\texorpdfstring{{\color{defink}\textsc{\textbf{#1}}}}{#1}}

\theoremstyle{plain}
\newtheorem{theorem}{Theorem}[section]
\newtheorem{lemma}[theorem]{Lemma}
\newtheorem{proposition}[theorem]{Proposition}
\newtheorem{corollary}[theorem]{Corollary}
\newtheorem{conjecture}[theorem]{Conjecture}
\theoremstyle{definition}
\newtheorem{definition}[theorem]{Definition}
\newtheorem{example}[theorem]{Example}
\theoremstyle{remark}
\newtheorem{remark}[theorem]{Remark}

\newcommand{\ZZ}{\mathbb{Z}}
\newcommand{\NN}{\mathbb{N}}
\newcommand{\FF}{\mathbb{F}}
\newcommand{\seq}{\mathrm{em}}
\newcommand{\Prod}{\mathrm{P}}
\newcommand{\minFac}{\mathrm{minFac}}
\newcommand{\walkZ}[2]{\mathrm{P}_{#1}(#2)}
\newcommand{\multZ}[2]{\mathrm{m}_{#1}(#2)}
\newcommand{\SE}{\mathrm{SE}}
\newcommand{\MH}{\mathrm{MH}}
\newcommand{\HH}{\mathrm{HH}}
\renewcommand{\DH}{\mathrm{DH}}
\newcommand{\MC}{\mathrm{MC}}
\newcommand{\PE}{\mathrm{PE}}
\newcommand{\ME}{\mathrm{ME}}
\newcommand{\EMPR}{\mathrm{EMPR}}
\newcommand{\WHP}{\mathrm{WHP}}
\newcommand{\EKE}{\mathrm{EKE}}
\newcommand{\PRE}{\mathrm{PRE}}

\title{A formalized reduction of the Mullin's Conjecture}

\author{Marcello Paris}
\date{February 2026}

\begin{document}

\maketitle

\begin{abstract}
The Euclid--Mullin sequence is defined by $a(0)=2$,
$a(n{+}1) = \text{lpf}(a(0)\cdots a(n)+1)$, where $\text{lpf}$ is
the least prime factor.
Mullin's Conjecture (MC, 1963) asserts that every prime eventually
appears.  We present a Lean~4 formalization
(${\sim}26{,}900$~lines, 35~files, \textbf{zero sorry}) that reduces MC to a
single open hypothesis.

An \emph{inductive bootstrap} yields the primary reduction: the
\textbf{Single Hit Theorem} shows that MC follows if, for each
prime~$q$, the multiplicative walk on $(\ZZ/q\ZZ)^\times$ hits~$-1$
at least once past a computable bound.  The algebraic precondition
(SubgroupEscape) is free: for any prime $p \geq 5$, some odd prime
$r < p$ escapes every proper subgroup of $(\ZZ/p\ZZ)^\times$, proved
using only modular arithmetic.  A separate \emph{Fourier bridge}
gives a parallel reduction: MC follows whenever certain walk character
sums are $o(N)$.

Multiple reduction routes---algebraic, character-analytic,
sieve-theoretic---all converge on the same
\emph{orbit-specificity gap}: transferring generic equidistribution
to one deterministic orbit.  The sharpest sufficient condition is
Conditional Multiplier Equidistribution~(CME)---the statement that
the factoring operation in the EM construction destroys correlation
between consecutive multiplier residues---proved to imply the
Complex Character Sum Bound~(CCSB) for all character orders,
bypassing the $d \geq 3$ barrier.  CME decomposes as
$\mathrm{CME} = \mathrm{VCB} + \mathrm{Dec}$, where
Vanishing Conditional Bias~(VCB) is a strictly weaker hypothesis
permitting fiber sums to be proportional to visit counts rather than
small.  We prove $\mathrm{VCB} + \mathrm{PED} \Rightarrow
\mathrm{CCSB}$, giving nine independent routes to MC.
Over a hundred dead ends are
documented, precisely delineating the boundary of current methods.
\end{abstract}

\newpage
\tableofcontents

\newpage

%% =========================================================================
\section{Introduction}
\label{sec:intro}
%% =========================================================================

Euclid's proposition IX.20 of the \emph{Elements} shows that for any
finite set of primes, each prime factor of their product plus one is outside the
set: to grow your set of primes, you can pick any of them. The \textbf{Euclid--Mullin
sequence} (OEIS A000945), introduced by Mullin~\cite{Mullin1963}, makes a
definite choice: always take the \emph{smallest} prime factor.
\begin{align}
  a(0) &= 2, &
  a(n+1) &= \text{smallest prime factor of } \bigl(a(0) \cdots a(n) + 1\bigr).
  \label{eq:em-def}
\end{align}
The first twenty terms (0-indexed) are
\begin{gather*}
  \underbrace{2}_{a(0)},\;
  3,\; 7,\; 43,\; 13,\; 53,\;
  \underbrace{5}_{a(6)},\;
  \underbrace{6221671}_{a(7)},\;
  38709183810571, \\
  139,\; 2801,\;
  \underbrace{11}_{a(11)},\;
  17,\; 5471,\; 52662739,\; 23003,\;
  30693651606209, \\
  \underbrace{37}_{a(17)},\;
  1741,\;
  \underbrace{1313797957}_{a(19)},\; \ldots
\end{gather*}
The sequence shows an erratic behavior: small primes appear out of their
natural order ($5$ not until position~$6$, $11$ at position~$11$,
$37$ at position~$17$), while enormous primes---a 7-digit number at
position~$7$, a 14-digit number at position~$8$---appear early.  As of
2025, only 51~terms are known and some primes like $41$ and $47$ are
not yet observed.  Computing further terms requires \emph{complete}
factorization of $\Prod(n)+1$ (where $\Prod(n) \coloneqq a(0)\cdots a(n)$
is the running product): finding any prime factor is not
enough---one must certify that no smaller factor exists.  Since
$\Prod(n)$ grows super-exponentially, this quickly exceeds the reach
of all known factoring algorithms.  By construction, no prime can
appear twice.

\begin{conjecture}[Mullin, 1963]\label{conj:mullin}
Every prime number eventually appears in the Euclid--Mullin sequence.
\end{conjecture}

The conjecture has resisted proof for over sixty years.  The difficulty
is showing that the deterministic $\minFac$ rule eventually \emph{selects} each prime.
That the rule matters is not idle speculation: Cox and van der
Poorten~\cite{CoxVdP1968} showed that replacing $\minFac$ with the
\emph{largest} prime factor provably misses infinitely many primes.

Each step couples the next prime to the full factorization history, creating a recursive dependency that
defeats both probabilistic heuristics and standard sieve methods.

At each step, $\Prod(n)+1$ may have many prime factors, and just the \emph{smallest} is chosen.
So, a target prime~$q$ may possibly divide $\Prod(n)+1$ for many (maybe infinite) $n$,
yet never be selected if a smaller prime always divides $\Prod(n)+1$ as well.
Our formalization tries to make this tension precise.

\paragraph{The accumulator structure.}
The running product $\Prod(n)$ is an \emph{accumulator}: a single
number that commits to the entire sequence history via irreversible
multiplication, analogous to the cumulative digest in a hash chain.
The accumulator poses a challenge~$\Prod(n)+1$; the response
$\minFac(\Prod(n)+1)$ extends the chain; and the updated accumulator
$\Prod(n{+}1) = \Prod(n) \cdot \minFac(\Prod(n){+}1)$ absorbs the
response irreversibly---once a prime enters the product, it divides
every future product and can never appear again
(Theorem~\ref{thm:extinct}).

This accumulator coupling is what makes standard tools fail.  Sieve
methods require approximate independence between the events
``$p \mid m$'' for different~$m$; here successive challenges
$\Prod(n)+1$ share a cumulative history.  Ergodic methods require a
fixed or state-dependent map; here the map at step~$n$ depends on the
full accumulator, not just the current residue.  The walk
reformulation (Section~\ref{sec:walk}) tames this coupling by
projecting the accumulator onto a finite group: $\walkZ{q}{n} =
\Prod(n) \bmod q$ preserves the divisibility information relevant
to~$q$ while discarding the accumulator's combinatorial complexity.
But the information lost in this projection is exactly the source of
difficulty: the walk position determines \emph{whether} $q$ can divide
$\Prod(n)+1$; the full accumulator determines \emph{which} prime is
actually selected.  Every open hypothesis in this paper---DH, CCSB,
CME---addresses this gap.

\paragraph{The factoring channel.}
The accumulator structure of the EM sequence is closely analogous to a
cryptographic hash chain: each term is deterministically derived from
its predecessor through an operation---integer factorization---that
destroys algebraic relationships.  In an iterated SHA-256 chain
$x_0, H(x_0), H(H(x_0)), \ldots$, each value is uniquely determined
by the seed, yet consecutive terms pass every reasonable statistical
test for independence.  In the EM sequence, the running product
$\Prod(n)$ determines $\Prod(n)+1$, whose smallest prime factor
becomes the next multiplier; but the $O(\log q)$ bits visible in the
residue $\Prod(n) \bmod q$ cannot control the outcome of factoring the
${\sim}2^n$-bit integer $\Prod(n)+1$.  This \emph{information
bottleneck} is why the multiplier residues behave as if
independent---and why the walk on $(\ZZ/q\ZZ)^\times$ should visit
every element, including the ``death state'' $-1$ that would put~$q$
in the sequence.  The analogy is tighter than it may appear: in both
cases the pseudorandomness claim concerns a fully deterministic process
whose forward map is easy but whose outputs resist structural
prediction---the difference being that $\minFac$ achieves this not by
cryptographic design but accidentally, through the information
bottleneck of projecting a ${\sim}2^n$-bit integer onto $O(\log q)$
bits.  Our formalization makes this intuition precise: we
identify the exact mathematical content---Conditional Multiplier
Equidistribution---needed to convert ``the factoring channel destroys
correlation'' into a proof of MC, and show it implies the conjecture
through a verified chain of reductions.

\

Our main result is a formally verified reduction of MC to a single
dynamical question: for each prime~$q$, does the walk on
$(\ZZ/q\ZZ)^\times$ hit~$-1$ at least once past a computable bound?
The strategy proceeds in three stages:

\

\begin{enumerate}[nosep]
\item \textbf{Reformulation.}  We recast ``does prime~$q$ appear?'' as
  ``does a multiplicative walk on the cyclic group $(\ZZ/q\ZZ)^\times$
  hit the element~$-1$?''  This translation is exact
  (Section~\ref{sec:walk}).
\item \textbf{Bootstrap.}  We show that the algebraic precondition for
  the walk to reach~$-1$---that the multipliers generate the full
  group---is \emph{free}, following from the inductive hypothesis
  $\MC({<}\,p)$ via an elementary lemma
  (Section~\ref{sec:bootstrap}).  A single hit on~$-1$ past the
  sieve gap suffices for MC\@.
\item \textbf{Diagnosis.}  We develop the harmonic-analytic and
  sieve-theoretic infrastructure to determine \emph{precisely} what
  kind of statement would produce the required hit, and why known methods
  fall short (Sections~\ref{sec:character}--\ref{sec:hard}).
\end{enumerate}

\

The formalization serves two purposes:
(i)~it guarantees that every reduction is logically sound, and (ii)~it
precisely delineates the boundary between what is proved and what
remains open, preventing the kind of subtle gap that plagues pencil-and-paper
reductions involving multiple interacting hypotheses.

\

The primary reduction is:

\begin{theorem}[\defname{Single Hit Theorem} --- \lean{EM/EquidistBootstrap.lean}{single\_hit\_implies\_mc}]
\label{thm:main}
$\mathrm{SingleHitHypothesis} \;\Longrightarrow\; \MC$.
\end{theorem}

SingleHitHypothesis asks: for every missing prime~$q$, if $\MC({<}\,q)$
and $\SE(q)$ hold, then $q \mid \Prod(n)+1$ for some~$n$ past the
sieve gap.  The proof is by strong induction on~$p$: the inductive
hypothesis gives $\MC({<}\,p)$, PrimeResidueEscape (proved elementarily)
bootstraps $\SE(p)$, and the single hit past the sieve gap gives
$\seq(n{+}1) = p$.

The algebraic precondition---that the multipliers generate
$(\ZZ/q\ZZ)^\times$---is free; the open problem is purely dynamical.
Multiple strategies for producing the required hit are formally verified:

\begin{theorem}[\lean{EM/EquidistBootstrap.lean}{dynamical\_hitting\_implies\_mullin}]
\label{thm:dh-main}
$\mathrm{DynamicalHitting} \;\Longrightarrow\; \MC$.
\end{theorem}

\begin{theorem}[\lean{EM/EquidistSelfCorrecting.lean}{complex\_csb\_mc'}]
$\mathrm{ComplexCharSumBound} \;\Longrightarrow\; \MC$.
\end{theorem}

\begin{theorem}[\lean{EM/LargeSieveSpectral.lean}{cme\_implies\_mc}]
$\mathrm{CME} \;\Longrightarrow\; \MC$ \emph{(sharpest sufficient condition)}.
\end{theorem}

All reductions are fully machine-verified with zero \code{sorry}.
Each strategy produces at least one hit past the sieve gap, which the
Single Hit Theorem converts into MC\@.  The formalization thus
provides a precise ``roadmap'': prove DynamicalHitting,
ComplexCharSumBound, CME, VCB~$+$~PED, or any of the equivalent
formulations in \S\ref{sec:character} and \S\ref{sec:open}, and the
rest follows by machine-checked deduction.

\paragraph{Notation.}
Theorems marked {\color{leanink}\sffamily\footnotesize$\checkmark$\,name}
are formally verified in Lean~4; clicking the identifier links to the
source code.

\paragraph{Organization.}
The paper follows the logical structure of the reduction.
Section~\ref{sec:walk} reformulates MC as a walk-hitting problem,
establishes the algebraic prerequisites, and derives the first missing prime's death channel avoidance.
Section~\ref{sec:bootstrap} presents the inductive bootstrap---the
core insight that SubgroupEscape is free---and proves the Single Hit
Theorem, reducing MC to producing one hit at each prime.
Section~\ref{sec:character} develops the character-analytic reduction
(CCSB~$\Rightarrow$~MC), including the large sieve infrastructure, the
spectral energy bridge, and the van der Corput--autocorrelation route.
Section~\ref{sec:hard} explains \emph{why} the remaining hypothesis is
difficult by analyzing dead ends and structural barriers.
Section~\ref{sec:lean} describes the Lean formalization.
Section~\ref{sec:open} discusses open problems and paths forward.
Appendix~\ref{app:history} collects the historical background;
Appendix~\ref{app:analogies} discusses analogies with
Artin's conjecture, multiplicative walks, Sarnak's program, and
sieve theory;
Appendix~\ref{app:routes} presents additional sieve and spectral routes to MC;
Appendix~\ref{app:agents} describes the human--AI methodology;
and Appendix~\ref{sec:glossary} provides a glossary of all definitions and hypotheses.

%% =========================================================================
\section{The Residue Walk}
\label{sec:walk}
%% =========================================================================

We define two sequences by mutual recursion:
\begin{alignat}{2}
  \seq(0) &\coloneqq 2, &\qquad
  \Prod(0) &\coloneqq 2, \label{eq:base}\\
  \seq(n+1) &\coloneqq \minFac\bigl(\Prod(n) + 1\bigr), &\qquad
  \Prod(n+1) &\coloneqq \Prod(n) \cdot \seq(n+1). \label{eq:step}
\end{alignat}

\begin{theorem}[\lean{EM/MullinDefs.lean}{seq\_isPrime, seq\_injective}]\label{thm:seq-props}
Every $\seq(n)$ is prime, and the sequence is injective: no prime
appears twice.
\end{theorem}

\begin{definition}[\defname{Walk} and \defname{Multiplier}]\label{def:walk-mult}
For any prime~$q$, define the \textbf{residue walk} and the \textbf{multiplier} by projecting the accumulator and the next prime onto $\ZZ/q\ZZ$:

$$\begin{aligned}\label{eq:walk}
  &\walkZ{q}{n} \coloneqq \Prod(n) \bmod q \\
  &\multZ{q}{n} \coloneqq \seq(n{+}1) \bmod q
\end{aligned}$$
\end{definition}

\begin{proposition}[\defname{Walk recurrence} --- \lean{EM/MullinGroupCore.lean}{walkZ\_succ}]\label{prop:recurrence}
$\walkZ{q}{n\!+\!1} = \walkZ{q}{n} \cdot \multZ{q}{n}$ in
$\ZZ/q\ZZ$.
\end{proposition}

\noindent The walk can be in one of two regimes:

\

\begin{enumerate}[nosep]
\item \textbf{Living regime.}  If $q$ has not yet appeared in the
  sequence up to step~$n$, then $q \nmid \Prod(n)$ (since $\Prod(n)$
  is a product of the primes $\seq(0), \ldots, \seq(n)$, none equal
  to~$q$).  So $\walkZ{q}{n} \in (\ZZ/q\ZZ)^\times$---the walk lives
  in the group of units.

\item \textbf{Dead regime.}  If $q = \seq(k)$ for some $k \leq n$,
  then $q \mid \Prod(n)$, so $\walkZ{q}{n} = 0$.  From this point on,
  $q \mid \Prod(m)$ for all $m \geq k$, so $\walkZ{q}{m} = 0$
  forever.  The walk has \emph{collapsed to zero}
  (\lean{EM/EquidistBootstrap.lean}{walkZ\_capture\_then\_collapse}).
\end{enumerate}

\

The transition from living to dead can happen only through $-1$:

\begin{theorem}[\defname{Walk--Divisibility Bridge} --- \lean{EM/MullinGroupCore.lean}{walkZ\_eq\_neg\_one\_iff}]\label{thm:bridge}
For any prime~$q$ and any~$n$ such that $\walkZ{q}{n} \in
(\ZZ/q\ZZ)^\times$:
\[
  \walkZ{q}{n} = -1
  \quad\Longleftrightarrow\quad
  q \mid \bigl(\Prod(n) + 1\bigr).
\]
\end{theorem}

Hitting $-1$ is a \emph{necessary} condition for the walk to die (for
$q$ to enter the sequence): if $q \mid \Prod(n)+1$, then
$\walkZ{q}{n} = -1$.  But it is not sufficient by itself.  When
$\walkZ{q}{n} = -1$, the walk dies only if $q = \minFac(\Prod(n)+1)$,
i.e.\ no smaller prime divides $\Prod(n)+1$.  If a smaller prime~$p$
also divides $\Prod(n)+1$, then $\minFac(\Prod(n)+1) \leq p < q$
and $q$ is \emph{not} selected---the walk \emph{bounces off}~$-1$ and continues living.
The walk may therefore hit~$-1$ multiple times without dying.  Death
occurs only at a hit where $q$ is the smallest available factor:
\[
  \cdots \to \underbrace{\walkZ{q}{n} = -1}_{\text{bounce}}
  \to (\ZZ/q\ZZ)^\times \to \cdots \to
  \underbrace{\walkZ{q}{n_0} = -1}_{\substack{\text{lethal hit:}\\
    q = \minFac}}
  \to \underbrace{\walkZ{q}{n_0{+}1} = 0}_{\text{dead}}
  \to 0 \to \cdots
\]

What determines whether a hit on~$-1$ is a bounce or a lethal hit?
The answer depends on which other primes divide $\Prod(n)+1$.  This
is controlled by the \emph{sieve gap}.

\begin{definition}[\defname{$\MC({<}\,q)$}]\label{def:mc-below}
For a prime~$q$, write $\MC({<}\,q)$ for the statement that every
prime smaller than~$q$ appears in the EM sequence:
$\forall\, p < q,\; p \text{ prime} \;\Rightarrow\; \exists\, k,\;
\seq(k) = p$.
\end{definition}

\noindent When $\MC({<}\,q)$ holds, every prime $p < q$ has entered the sequence
at some stage~$k_p$, so $p \mid \Prod(n)$ for all $n \geq k_p$.
Past a uniform stage $N_0 = \max_p k_p$, every prime $p < q$ divides
$\Prod(n)$, and therefore cannot divide $\Prod(n)+1$ (since
$\gcd(\Prod(n), \Prod(n)+1) = 1$).  This is the \textbf{sieve gap}:
past~$N_0$, the only primes that can divide $\Prod(n)+1$ are
$\geq q$.

\begin{theorem}[\defname{$q$-roughness} --- \lean{EM/EquidistThreshold.lean}{mc\_below\_implies\_seq\_ge}]\label{thm:roughness}
If $\MC({<}\,q)$ holds, then $\exists\, N_0$ such that
$\seq(n{+}1) \geq q$ for all $n \geq N_0$.
\end{theorem}

\noindent The sieve gap transforms hitting~$-1$ from a necessary condition into
a sufficient one:

\begin{theorem}[\defname{Lethal hit} --- \lean{EM/EquidistThreshold.lean}{mc\_below\_hit\_is\_lethal}]\label{thm:lethal-hit}
If $\MC({<}\,q)$ holds and $\walkZ{q}{n} = -1$ for some $n$ past the
sieve gap, then $\seq(n{+}1) = q$.
\end{theorem}

\begin{proof}
Since $n$ is past the sieve gap, no prime $< q$ divides
$\Prod(n)+1$ (Theorem~\ref{thm:roughness}).  But $\walkZ{q}{n} = -1$
gives $q \mid \Prod(n)+1$, so $q = \minFac(\Prod(n)+1)$, and
$\seq(n{+}1) = q$.
\end{proof}

\begin{definition}[Missing prime]\label{def:missing}
A prime~$q$ is \textbf{missing} if $\seq(n) \neq q$ for all~$n$.
\end{definition}

\noindent Mullin's Conjecture asserts that no prime is missing.
Under $\MC({<}\,q)$, the first hit on~$-1$ past the sieve gap is
lethal.  This immediately constrains missing primes:

\begin{theorem}[\lean{EM/EquidistThreshold.lean}{mc\_below\_missing\_walk\_ne\_neg\_one}]\label{thm:no-cofinal-hit}
If $\MC({<}\,q)$ holds and $q$ is missing, then the walk \emph{never}
hits~$-1$ past the sieve gap: $\exists\, N_0$ such that
$\walkZ{q}{n} \neq -1$ for all $n \geq N_0$.
\end{theorem}

\begin{proof}
Any hit past the sieve gap would give $\seq(n{+}1) = q$
(Theorem~\ref{thm:lethal-hit}), contradicting $q$ missing.
\end{proof}

\noindent For the first missing prime~$q$, the hypothesis $\MC({<}\,q)$ holds
by definition.  So its walk avoids~$-1$ permanently past the sieve
gap, the walk $\walkZ{q}{\cdot}$ is infinite: it stays
in~$(\ZZ/q\ZZ)^\times$ forever, never collapsing to zero.

\subsection{The forbidden multiplier and the death channel}

Suppose the walk is at position $c \in (\ZZ/q\ZZ)^\times$ at step~$n$.
The walk would hit $-1$ at step $n{+}1$ if and only if
\[
  \walkZ{q}{n} \cdot \multZ{q}{n} = -1,
  \quad\text{i.e.,}\quad
  \multZ{q}{n} = -c^{-1}.
\]

\noindent So at every step, there is exactly \textbf{one forbidden multiplier}
$f(n) = -\walkZ{q}{n}^{-1}$---the unique residue class that would
send the walk to~$-1$.  This is $1$~element out of $q-1$ possible
unit residues.

\begin{definition}[\defname{Death channel}]\label{def:death-channel}
The \textbf{death channel} at position $c$ is the residue class
$b(c) = -c^{-1} \bmod q$.
\end{definition}

\noindent When the walk is at~$c$, the death channel is the set of primes
$p \equiv -c^{-1} \pmod{q}$, which by Dirichlet's theorem has density $1/(q-1)$ among all primes.
The death channel \emph{moves} with the walk: as $c$ changes from step
to step, the forbidden class changes with it.  The map $c \mapsto
-c^{-1}$ is a bijection on $(\ZZ/q\ZZ)^\times$ (it is the composition
of inversion and negation).

\begin{theorem}[\lean{EM/EquidistSieve.lean}{walk\_hits\_neg\_one\_iff\_mult\_eq\_forbidden}]\label{thm:forbidden}
For any step~$n$ with $\walkZ{q}{n} \in (\ZZ/q\ZZ)^\times$:
\[
  \walkZ{q}{n{+}1} = -1
  \quad\Longleftrightarrow\quad
  \multZ{q}{n} = -\walkZ{q}{n}^{-1}.
\]
\end{theorem}

\noindent The statement ``$q$ is missing'' is therefore equivalent to:
\[
  \text{For all } n: \quad \multZ{q}{n} \neq -\walkZ{q}{n}^{-1}.
\]
In words: the multiplier avoids the death channel at every step, forever.
This brings us to discuss multipliers.

\subsection{The Confinement Theorem and SubgroupEscape}

Before asking whether the walk hits~$-1$, we must ask whether it
\emph{can}.  The walk is multiplicative:
$\walkZ{q}{n} = \walkZ{q}{0} \cdot \prod_{k < n} \multZ{q}{k}$.
The reachable set is the coset
$\walkZ{q}{0} \cdot \langle \text{multipliers} \rangle$.

\begin{theorem}[\defname{Confinement} --- \lean{EM/MullinGroupCore.lean}{confinement\_forward}]\label{thm:confinement}
If every multiplier lies in a subgroup $H \leq (\ZZ/q\ZZ)^\times$,
then the walk is confined to the coset $\walkZ{q}{0} \cdot H$.
In particular, if $-1 \notin \walkZ{q}{0} \cdot H$, the walk
\emph{never} hits~$-1$.
\end{theorem}

\noindent This motivates:

\begin{definition}[\defname{SubgroupEscape} (\abbr{SE})]\label{def:se}
For a prime~$q$: no proper subgroup $H < (\ZZ/q\ZZ)^\times$ contains
all multipliers.  Equivalently,
$\langle \multZ{q}{n} : n \in \NN \rangle = (\ZZ/q\ZZ)^\times$.
\end{definition}

\begin{theorem}[\lean{EM/MullinGroupEscape.lean}{se\_of\_maximal\_escape}]
SE holds iff multipliers escape every \emph{maximal} proper subgroup.
Since $(\ZZ/q\ZZ)^\times$ is cyclic, the maximal subgroups are the
index-$\ell$ subgroups for prime $\ell \mid q\!-\!1$.
\end{theorem}

\noindent When SE holds, the walk is not confined to any proper coset: every
element of $(\ZZ/q\ZZ)^\times$, including~$-1$, is \emph{reachable}.
But reachability does not imply hitting---that is a dynamical question.

\subsection{The first missing prime}

We now assemble the key argument.  Suppose, toward contradiction, that
Mullin's Conjecture is false.  Then there exists a \textbf{first
missing prime}: the smallest prime~$q$ that never appears in the EM
sequence. For this~$q$:

\begin{enumerate}[label=(\alph*),nosep]
\item $\MC({<}\,q)$ holds---by definition, every prime smaller than~$q$
  is in the sequence.
\item The walk $\walkZ{q}{\cdot}$ is infinite---since $q$ is missing,
  $\walkZ{q}{n} \in (\ZZ/q\ZZ)^\times$ for all~$n$.
\item Past the sieve gap, the walk never hits~$-1$.  Here is why.
  Eventually (say for $n \geq N_0$), all primes $< q$ have entered
  the sequence and divide $\Prod(n)$.  For $n \geq N_0$, no prime
  $< q$ can divide $\Prod(n)+1$ (since it divides $\Prod(n)$ and
  $\gcd(\Prod(n), \Prod(n)+1) = 1$).  So if $\walkZ{q}{n} = -1$ for
  some $n \geq N_0$, then $q \mid \Prod(n)+1$, and $q$ is the smallest
  prime factor of $\Prod(n)+1$ (all smaller primes are excluded).
  Then $\seq(n{+}1) = q$, contradicting $q$ being missing.
\end{enumerate}

\

\noindent Therefore: \textbf{for the first missing prime~$q$, the walk is an
infinite trajectory on $(\ZZ/q\ZZ)^\times$ that avoids the death
channel at every step past the sieve gap.} Equivalently: $\multZ{q}{n} \neq -\walkZ{q}{n}^{-1}$ for all $n \geq N_0$.

%% =========================================================================
\section{The Inductive Bootstrap}
\label{sec:bootstrap}
%% =========================================================================

With the first missing prime's situation established (\S\ref{sec:walk}), we now show that the algebraic precondition---SubgroupEscape---comes for free, and that a single hit on~$-1$ past the sieve gap suffices for MC\@.

The walk avoids~$-1$ forever.  Can it at least \emph{reach}~$-1$?
That is: does SE hold?

\begin{theorem}[\defname{PrimeResidueEscape} (\abbr{PRE});
  \lean{EM/EquidistBootstrap.lean}{prime\_residue\_escape}]\label{thm:pre}
For every prime $p \geq 5$ and every proper subgroup
$H < (\ZZ/p\ZZ)^\times$, some odd prime $r < p$ has residue
$r \bmod p \notin H$.
\end{theorem}

\begin{proof}
Suppose every odd prime $r \in [3, p)$ satisfies $r \bmod p \in H$.
Since $H$ is a subgroup, every \emph{product} of such primes is
in~$H$.  Every odd number in~$[1, p)$ factors into odd primes $< p$,
so every odd number in~$[1, p)$ maps into~$H$.  In particular,
$p - 2 \equiv -2$ and $p - 4 \equiv -4$ are both in~$H$ (both odd
and $< p$ for $p \geq 5$).  Then $2 = (-4)(-2)^{-1} \in H$, so every
even number in $[1, p)$ is in~$H$ as well.  Hence
$H = (\ZZ/p\ZZ)^\times$, contradicting $H$ proper.
\end{proof}

\begin{theorem}[\lean{EM/EquidistBootstrap.lean}{mc\_below\_pre\_implies\_se}]
\label{thm:bootstrap}
$\MC({<}\,p) + \PRE \;\Longrightarrow\; \SE(p)$.
\end{theorem}

\begin{proof}[Proof sketch]
Let $H < (\ZZ/p\ZZ)^\times$ be proper.  By PRE, some odd prime
$r < p$ has $r \bmod p \notin H$.  By $\MC({<}\,p)$, the prime~$r$
appears as $\seq(k)$ for some~$k$.  Then $\multZ{p}{k{-}1} \equiv r
\pmod{p} \notin H$.
\end{proof}

\begin{corollary}\label{cor:fmp-se}
For the first missing prime~$q$: the multipliers generate all of
$(\ZZ/q\ZZ)^\times$.  SubgroupEscape holds, and $-1$ is reachable.
\end{corollary}

The one-prime gap (Theorem~\ref{thm:one-prime-gap}) shows that
$\MC({<}\,q)$ plus a single divisibility event $q \mid \Prod(n)+1$
past the sieve gap gives $\MC(q)$.  The bootstrap
(Theorem~\ref{thm:bootstrap}) shows that $\MC({<}\,q)$ gives $\SE(q)$
for free.  These two facts compose into the primary reduction of
Mullin's Conjecture.

\begin{definition}[\defname{SingleHitHypothesis} (\abbr{SHH})]\label{def:shh}
For every prime~$q$: if $\MC({<}\,q)$ and $\SE(q)$ hold and $q$ is
missing, then there exists~$n$ past the sieve gap with
$q \mid \Prod(n)+1$.

\

\noindent Equivalently: for every prime~$q$, if $\MC({<}\,q)$ and $\SE(q)$
hold, then either $q$ already appears in the sequence, or there
exists $n \geq N_0(q)$ with $\walkZ{q}{n} = -1$.
\end{definition}

\begin{theorem}[\defname{Single Hit Theorem} --- \lean{EM/EquidistBootstrap.lean}{single\_hit\_implies\_mc}]
\label{thm:single-hit}
$\mathrm{SingleHitHypothesis} \;\Longrightarrow\; \MC$.
\end{theorem}

\begin{proof}
By strong induction on~$p$.  Assume $\MC({<}\,p)$.

\

\begin{enumerate}[nosep]
\item \textbf{Bootstrap gives SE\@.}
    $\MC({<}\,p) + \PRE \Rightarrow \SE(p)$ (Theorem~\ref{thm:bootstrap}).
    Since PRE is proved unconditionally (Theorem~\ref{thm:pre}), we
    obtain $\SE(p)$.

\item \textbf{SHH gives a hit.}
    Since $\MC({<}\,p)$ and $\SE(p)$ hold, SHH provides
    $n \geq N_0$ (past the sieve gap) with $p \mid \Prod(n)+1$.

\item \textbf{The sieve gap closes.}
    Past $N_0$, all primes $< p$ divide $\Prod(n)$ and hence cannot
    divide $\Prod(n)+1$.  So $p = \minFac(\Prod(n)+1)$, giving
    $\seq(n{+}1) = p$.
\end{enumerate}
\end{proof}

\subsection{The death channel avoidance paradox}

We now have a precise and sharp picture of the first missing prime~$q$:

\

\begin{enumerate}[nosep]
\item The walk lives in $(\ZZ/q\ZZ)^\times$ forever.
\item The multipliers generate the full group---$-1$ is reachable.
\item The walk never reaches $-1$---the death channel is avoided at
  every step past the sieve gap.
\item At each step, the death channel is a single residue class out
  of $q - 1$---a ``target'' of density $1/(q-1)$.
\end{enumerate}

\

\noindent Mullin's Conjecture is therefore equivalent to: \textbf{this situation
is impossible.}  The open problem is now:

\begin{center}
\emph{Does the multiplicative walk on $(\ZZ/q\ZZ)^\times$ defined by
the EM sequence, whose multipliers generate the full group, hit~$-1$
at least once past the sieve gap?}
\end{center}

``At least once past the sieve gap'' is the precise requirement.  Not
cofinally, not equidistributed---once.
The intuition for impossibility rests on an information-theoretic
asymmetry.  At step~$n$, the death channel
$f(n) = -\walkZ{q}{n}^{-1}$ is determined by the walk position
$\walkZ{q}{n} = \Prod(n) \bmod q$, which carries $O(\log q)$ bits of
information.  The multiplier
$\multZ{q}{n} = \minFac(\Prod(n)+1) \bmod q$ is determined by the
full integer $\Prod(n)+1$, which has $\sim 2^n$ bits.  The factoring
operation $\minFac$ extracts global arithmetic information from all
$\sim 2^n$ bits; the death channel is determined by $O(\log q)$ bits.
For the multiplier to systematically avoid the death channel, the
$O(\log q)$-bit residue would have to predict the outcome of an
operation on a $2^n$-bit integer---a ``prediction'' whose information
content vanishes exponentially.

More precisely: for a generic $q$-rough integer $N \equiv a \pmod{q}$,
the conditional distribution of $\minFac(N) \bmod q$ is
asymptotically uniform over $(\ZZ/q\ZZ)^\times$, by CRT\@.  Knowing
$N \bmod q$ does not constrain $N \bmod p$ for any other prime~$p$,
so it does not constrain which primes $\leq N^{1/2}$ divide~$N$.  The
density of the death channel among all primes is $1/(q-1)$, and the
conditional probability that $\minFac(N)$ falls in the death channel
is $1/(q-1) + o(1)$ as $N \to \infty$, \emph{independently of the
residue class~$a$}.

For the EM sequence, $\Prod(n)+1$ grows super-exponentially
($\Prod(n)+1 \geq 2^{2^n}$), so the $o(1)$ error at each step
shrinks exponentially fast.  The ``probability'' of avoiding the death
channel for~$N$ consecutive steps is heuristically
$(1 - 1/(q-1))^N \to 0$.  Converting this heuristic into a proof is
the content of the conjecture.

\subsection{Sufficient conditions for the single hit}
\label{sec:sufficient-conditions}

Several formally verified strategies produce the required single hit
past the sieve gap.  Each implies MC via the Single Hit Theorem.

\begin{definition}[\defname{DynamicalHitting} (\abbr{DH})]\label{def:dh}\label{def:hh}
For every missing prime~$q$: $\SE(q) \Rightarrow \HH(q)$, where
$\HH(q)$ (\defname{HittingHypothesis}) asks for cofinal hitting:
$\forall\, N,\; \exists\, n \geq N,\; q \mid (\Prod(n) + 1)$.
\end{definition}

\begin{theorem}[\lean{EM/EquidistBootstrap.lean}{dynamical\_hitting\_implies\_mullin}]
\label{thm:dh-mc}
$\DH \;\Longrightarrow\; \MC$.
\end{theorem}

DH is stronger than SHH: it does not assume $\MC({<}\,q)$ and asks for
infinitely many hits rather than one.  But the extra strength is
convenient---DH interfaces cleanly with character sum methods.
In the death channel language, DH asserts: if the multipliers generate
$(\ZZ/q\ZZ)^\times$, then the multiplier cannot avoid the forbidden
residue $-\walkZ{q}{n}^{-1}$ forever.  Equivalently: there is no
infinite walk on a cyclic group, with a generating set of multipliers,
that dodges a single moving target of density $1/(q-1)$ at every step.

The key structural point: SHH is \textbf{strictly weaker} than
DynamicalHitting in two ways.
First, SHH assumes $\MC({<}\,q)$, which DH does not (DH only assumes SE).
Second, SHH asks for one hit past the sieve gap, while DH asks for
cofinal hitting.  Both extra assumptions are harmless in the inductive
proof---$\MC({<}\,q)$ is always available at the inductive step, and
one hit is all the Single Hit Theorem needs---but they make SHH
genuinely easier to satisfy as a mathematical statement.

\begin{remark}\label{rem:shh-vs-dh}
The logical relationships among the hitting hypotheses are:
\[
  \HH \;\Longrightarrow\; \DH \;\Longrightarrow\; \mathrm{SHH},
\]
where $\HH$ on its own denotes the \emph{unconditional} version of
$\HH(q)$ from Definition~\ref{def:dh}: cofinal hitting asserted for
every missing~$q$ without assuming $\SE(q)$.
DH conditions cofinal hitting on SE, and SHH asks for a
single hit past the sieve gap given both $\MC({<}\,q)$ and SE\@.
All three imply MC\@.  SHH is the weakest: it assumes the most
($\MC({<}\,q)$ and SE, both provided free by the inductive bootstrap)
and demands the least (one hit, not infinitely many).
The converses need not hold.
\end{remark}

\begin{definition}[\defname{ComplexCharSumBound} (\abbr{CCSB})]\label{def:ccsb-intro}
For every missing prime~$q$ and every non-trivial character~$\chi$:
$\|S_\chi(N)\| = o(N)$, where
$S_\chi(N) = \sum_{n<N}\chi(\walkZ{q}{n})$.
\end{definition}

\begin{theorem}[\lean{EM/EquidistSelfCorrecting.lean}{complex\_csb\_mc'}]
\label{thm:ccsb-mc-intro}
$\mathrm{CCSB} \;\Longrightarrow\; \MC$.
\end{theorem}

CCSB produces the hit via Fourier inversion: if all non-trivial
character sums are $o(N)$, the hit count at~$-1$ is
$N/(q{-}1) + o(N)$, which is eventually positive.  One hit past
the sieve gap is lethal (Section~\ref{sec:character}).

\begin{definition}[\defname{ConditionalMultiplierEquidist} (\abbr{CME})]\label{def:cme-intro}
For every missing prime~$q$, every non-trivial~$\chi$, and every walk
position~$c$: the multiplier characters
$\chi(\multZ{q}{n})$ conditioned on $\walkZ{q}{n} = c$ are equidistributed
(their partial sums are $o(N)$).
\end{definition}

\begin{theorem}[\lean{EM/LargeSieveSpectral.lean}{cme\_implies\_mc}]
\label{thm:cme-mc-intro}
$\mathrm{CME} \;\Longrightarrow\; \MC$ \emph{(sharpest sufficient condition)}.
\end{theorem}

CME is the sharpest hypothesis: it asserts that the factoring
operation destroys correlation between consecutive multiplier residues
conditioned on the walk position.  It implies CCSB for all character
orders, bypassing the $d \geq 3$ barrier that blocks the
BRE route.

\begin{theorem}[\lean{EM/LargeSieveSpectral.lean}{vcb\_ped\_implies\_mc}]
\label{thm:vcb-ped-mc-intro}
$\mathrm{VCB} + \mathrm{PED} \;\Longrightarrow\; \MC$.
\end{theorem}

VCB (Vanishing Conditional Bias) is a weakening of CME that permits
fiber sums to be proportional to visit counts rather than small;
combined with PED (Positive Escape Density), it reaches CCSB\@.

Each of these strategies produces at least one hit past the sieve gap,
which the Single Hit Theorem (Theorem~\ref{thm:single-hit}) converts
into MC\@.

\paragraph{Why it's hard: the walk controls nothing}

The difficulty is entirely in the coupling between walk position and
multiplier.  At step~$n$:

\begin{itemize}[nosep]
\item The \textbf{walk position} $\walkZ{q}{n} = \Prod(n) \bmod q$
  determines the death channel $f(n) = -\walkZ{q}{n}^{-1}$.
\item The \textbf{multiplier}
  $\multZ{q}{n} = \minFac(\Prod(n)+1) \bmod q$ is what must fall
  into the death channel.
\item Both depend on $\Prod(n)$: the walk position depends on
  $\Prod(n) \bmod q$, and the multiplier depends on $\Prod(n)+1$ as
  a full integer.
\end{itemize}

The walk position sees $O(\log q)$ bits; the multiplier sees all
$\sim 2^n$ bits.  But they share the same underlying object
$\Prod(n)$.  The question is whether this shared dependence creates a
correlation strong enough for the multiplier to systematically avoid
one residue class out of $q-1$.

For a \emph{random} sequence of multipliers drawn uniformly from
$(\ZZ/q\ZZ)^\times$, the probability of avoiding the death channel
for $N$ steps is $(1 - 1/(q-1))^N \to 0$.  The EM walk is not
random---it is deterministic---but the factoring operation that
determines each multiplier is, heuristically, a decorrelating ``hash.''
This is the \textbf{factorization independence heuristic}: the
function $\minFac$, applied to a $2^n$-bit integer, produces an
output that is effectively uncorrelated with the $O(\log q)$-bit
residue class of the input.

Every open hypothesis in this paper---DH, CCSB, CME---is a precise
formalization of this heuristic.

\subsection{Supporting Infrastructure}

The remainder of this section collects the supporting results that
flesh out the bootstrap: the one-prime gap, the sieve gap, and the power residue decomposition.

\paragraph{The Sieve Gap and the One-Prime Gap}

The $q$-roughness theorem (Theorem~\ref{thm:roughness}) resolves the
selectability problem described in Section~\ref{sec:intro}: past the
sieve gap, $q$ is the smallest available factor whenever it divides
the Euclid number.  This gives the one-prime gap:

\begin{theorem}[\defname{One-prime gap} --- \lean{EM/EquidistThreshold.lean}{mc\_below\_cofinal\_hit\_implies\_mc\_at}]
\label{thm:one-prime-gap}
$\MC({<}\,q)$ plus a single hitting event
$q \mid \Prod(n)+1$ for some $n$ past the sieve gap
implies $\MC(q)$.
\end{theorem}

\paragraph{The Power Residue Decomposition}

Since $(\ZZ/q\ZZ)^\times$ is cyclic of order $q\!-\!1$, its subgroup
lattice is determined by the prime factorization of $q\!-\!1$.  A
multiplier set generates the full group if and only if it escapes every
maximal subgroup---and the maximal subgroups correspond to the prime
divisors $\ell$ of~$q\!-\!1$.  This decomposition converts SE into
independent conditions, one per prime~$\ell \mid q\!-\!1$.

\begin{definition}[\defname{PowerResidueEscape} ($\abbr{PRE}_\ell$)]
$\exists\, n : \multZ{q}{n}^{(q-1)/\ell} \neq 1$ in $(\ZZ/q\ZZ)^\times$.
\end{definition}

\begin{theorem}[\abbr{PRE} $\Leftrightarrow$ \abbr{SE} --- \lean{EM/EquidistCharPRE.lean}{pre\_iff\_se}]
$\mathrm{PRE} \;\Longleftrightarrow\; \SE$.
The forward direction uses only Lagrange's theorem; the reverse uses
cyclicity of $(\ZZ/q\ZZ)^\times$.
\end{theorem}

\paragraph{Quadratic Reciprocity Obstruction}

The power residue decomposition raises a natural question: for how many
primes~$q$ could SE actually \emph{fail}?  Since each PRE$_\ell$
condition asks only that \emph{one} multiplier among infinitely many
escapes the $\ell$-th power subgroup, SE failure requires all
multipliers to be $\ell$-th power residues for some $\ell \mid q{-}1$.
By CRT and quadratic reciprocity, the density of primes~$q$ for which
the first~$k$ multiplier primes all fail to escape the index-2
subgroup is at most $O(2^{-k})$.  Extending this to a rigorous
density-zero statement for SE failure would require controlling all
prime divisors $\ell \mid q{-}1$ simultaneously, which remains open.

%% =========================================================================
\section{The Character Sum Reduction}
\label{sec:character}
%% =========================================================================

Sections~\ref{sec:walk}--\ref{sec:bootstrap} established the picture: for the first
missing prime~$q$, the walk lives in $(\ZZ/q\ZZ)^\times$ forever, the
multipliers generate the full group, yet the walk avoids~$-1$
permanently past the sieve gap.  The death channel
$f(n) = -\walkZ{q}{n}^{-1}$ is dodged at every step.  The Single Hit
Theorem (\S\ref{sec:single-hit}) reduces MC to producing one hit
on~$-1$ past the sieve gap at each prime.

The question is not ``does the walk equidistribute?''---that is far
stronger than needed.  The question is: \emph{can the walk really
avoid one class out of~$q{-}1$ forever?}  This section develops the
harmonic-analytic tools to show it cannot: character sums detect the
anomaly that permanent avoidance would create, and bounding them rules
it out.

\paragraph{Why character sums?}
Permanent avoidance of~$-1$ means the hit count
$|\{n < N : \walkZ{q}{n} = -1\}|$ stays at zero past the sieve gap.
Character orthogonality decomposes this count into a uniform share
$N/(q{-}1)$ plus correction terms built from non-trivial character
sums $S_\chi(N) = \sum_{n<N}\chi(\walkZ{q}{n})$.  The uniform share
grows linearly; for the hit count to remain zero, the correction terms
must cancel this growth.  That requires at least one non-trivial
character sum to be $\Omega(N/(q{-}1))$---a sustained asymmetry in
the character spectrum.  The hypothesis CCSB (all $|S_\chi(N)| =
o(N)$) rules this out: it forces the hit count to be eventually
positive, contradicting permanent avoidance.

\medskip\noindent\textbf{What is new vs.\ what is infrastructure.}\enspace
The CCSB~$\Rightarrow$~MC reduction (Definition~\ref{thm:ccsb-mc}),
the Fourier bridge (Theorem~\ref{thm:fourier-step}), the
Decorrelation--PED chain (\S\ref{sec:dec-chain}), and the telescoping
no-go results (\S\ref{sec:telescope}) are original contributions of
this formalization.  The large sieve infrastructure
(\S\ref{sec:large-sieve}, Appendix~\ref{sec:sve})---including the weak ALS,
Gauss sum inversion, van der Corput, and Parseval---formalizes known
results; its purpose is to identify the precise \emph{transfer gap}
between classical tools and the EM orbit, which is itself a
contribution (see \S\ref{sec:large-sieve}).

The formalization develops this Fourier-analytic reduction because
character sums provide the cleanest interface between the
death-channel avoidance problem and the toolkit of analytic number
theory: the Bombieri--Vinogradov theorem, the large sieve inequality,
and Gauss sum inversion all produce character sum bounds, and the
formalization shows exactly how each connects to ruling out permanent
avoidance.

\subsection{Character Sums and Permanent Avoidance}

For a Dirichlet character $\chi : (\ZZ/q\ZZ)^\times \to \mathbb{C}^\times$,
the \emph{walk character sum} is
\[
  S_\chi(N) \;=\; \sum_{n < N} \chi\bigl(\walkZ{q}{n}\bigr).
\]

\begin{definition}[\defname{ComplexCharSumBound} (\abbr{CCSB})]
For every missing prime~$q$, every non-trivial character~$\chi$, and
every $\varepsilon > 0$, there exists $N_0$ such that for all
$N \geq N_0$:
\[
  \|S_\chi(N)\| \;\leq\; \varepsilon \cdot N.
\]
In other words, the walk character sums are $o(N)$---they grow
strictly slower than linearly.  The contrapositive: if the walk
permanently avoids~$-1$, some non-trivial character sum must be
$\Omega(N)$ (to cancel the uniform share in the Fourier bridge).
CCSB rules this out.
\end{definition}

\begin{theorem}[\lean{EM/EquidistSelfCorrecting.lean}{complex\_csb\_mc'}]
\label{thm:ccsb-mc}
$\mathrm{CCSB} \;\Longrightarrow\; \MC$.
\end{theorem}

This is a single-hypothesis reduction with no additional parameters.
The proof composes three bridges:

\begin{enumerate}[nosep]
\item \textbf{Fourier inversion} (\lean{EM/EquidistFourier.lean}{complex\_csb\_implies\_hit\_count\_lb\_proved}):
  CCSB implies that the hit count at~$-1$ satisfies
  $|\{n < N : \walkZ{q}{n} = -1\}| = N/(q{-}1) + o(N)$,
  which is eventually positive.  This directly contradicts permanent
  avoidance past the sieve gap (the character orthogonality formula,
  Theorem~\ref{thm:fourier-step}, gives this for any target~$t$;
  specializing to $t = -1$ is what matters).

\item \textbf{Cofinal hitting gives HH}
  (\lean{EM/EquidistFourier.lean}{walk\_equidist\_mc}):
  the eventually-positive hit count at~$-1$ gives cofinal hitting,
  hence HH\@.  (In fact, the Fourier bridge yields equidistribution
  across all classes, but only the $t = -1$ case is needed.  SE is a
  side effect: a walk visiting~$-1$ cannot be confined to a proper
  coset.)

\item \textbf{DH implies MC} (Theorem~\ref{thm:dh-mc}): via the
  inductive bootstrap.
\end{enumerate}

\subsection{The Fourier Bridge}

The Fourier bridge is the single most important proved result after
DH~$\Rightarrow$~MC itself: it converts character sum bounds into hit
count lower bounds, and hence into MC.

\begin{theorem}[\lean{EM/EquidistFourier.lean}{walk\_hit\_count\_fourier\_step}]
\label{thm:fourier-step}
For any target $t \in (\ZZ/q\ZZ)^\times$:
\[
  \bigl|\{n < N : \walkZ{q}{n} = t\}\bigr|
  \;=\; \frac{1}{q-1}\sum_{\chi} \overline{\chi(t)}\, S_\chi(N),
\]
where the sum is over all Dirichlet characters mod~$q$.
\end{theorem}

This is a standard Fourier inversion formula on the finite group
$(\ZZ/q\ZZ)^\times$, stated for all~$t$.  Specializing to $t = -1$
and splitting the trivial character $\chi_0$ (with $\chi_0(a) = 1$
for all~$a$) from the non-trivial ones:
\[
  \bigl|\{n < N : \walkZ{q}{n} = -1\}\bigr|
  \;=\; \underbrace{\frac{N}{q-1}}_{\text{uniform share}}
  \;+\; \underbrace{\frac{1}{q-1}\sum_{\chi \neq \chi_0}
    \overline{\chi(-1)}\, S_\chi(N)}_{\text{correction}}.
\]
The first term grows linearly.  The correction term is bounded by
$\tfrac{1}{q-1}\sum_{\chi \neq \chi_0} |S_\chi(N)|$, since
$|\overline{\chi(-1)}| = 1$.  If CCSB holds---all
$|S_\chi(N)| = o(N)$---the correction is $o(N)$, and the hit count
is $N/(q{-}1) + o(N)$, which is eventually positive.

\paragraph{The contrapositive.}
Section~\ref{sec:bootstrap} showed that for the first missing
prime~$q$, the hit count at~$-1$ is exactly~0 past the sieve gap.
By the Fourier bridge, this forces the correction terms to
cancel the entire main term $N/(q{-}1)$.  Since there are only
$q{-}2$ non-trivial characters, at least one must satisfy
$|S_\chi(N)| = \Omega(N/(q{-}1)^2)$---a sustained linear-scale
bias in the character spectrum.  CCSB ($= o(N)$) rules this out.
The Fourier bridge thus converts the geometric statement ``the walk
avoids~$-1$ permanently'' into the spectral statement ``some character
sum has sustained bias,'' and CCSB negates the latter.

Returning to our running example ($q = 41$): the walk must accumulate
a deficit of $\sim N/40$ hits at~$-1$ compared to uniform, which
requires sustained character sum bias across the 39~non-trivial
characters.  Proving CCSB for $q = 41$ would rule out this bias and
establish that 41~eventually appears.

\subsection{The Decorrelation--PED--CCSB Chain}
\label{sec:dec-chain}

CCSB rules out permanent avoidance of~$-1$.  But the walk is built
from \emph{multipliers}: $\walkZ{q}{n{+}1} = \walkZ{q}{n} \cdot
\multZ{q}{n}$.  Since $\chi$ is a group homomorphism,
$\chi(\walkZ{q}{n}) = \chi(\walkZ{q}{0}) \cdot \prod_{k<n}
\chi(\multZ{q}{k})$: the walk character sum is a sum of
\emph{partial products} of the multiplier characters.  The question
becomes: what properties of the multiplier sequence would prevent
the walk from permanently dodging the death channel?

This subsection formalizes a chain of progressively weaker hypotheses
about the multipliers, each implying the next via proved bridges.
The goal is to decompose CCSB into sharper conditions on the multiplier
sequence, and to identify exactly where the irreducible difficulty
lies.

\begin{definition}[\defname{PositiveEscapeDensity} (\abbr{PED})]
For every missing prime~$q$ and non-trivial~$\chi$, there exist
$\delta > 0$ and $N_0$ such that for $N \geq N_0$:
$|\{k < N : \chi(\multZ{q}{k}) \neq 1\}| \geq \delta N$.
\end{definition}

The name ``escape'' comes from the SubgroupEscape perspective: the
kernel $\ker(\chi)$ is a proper subgroup of $(\ZZ/q\ZZ)^\times$, and
$\chi(\multZ{q}{k}) \neq 1$ means the $k$-th multiplier ``escapes''
from $\ker(\chi)$.  PED asks that a positive fraction of multipliers
escape \emph{every} proper subgroup, not just occasionally but with
positive density.  The connection to the death channel: since the
death channel is a single class and multipliers that escape
$\ker(\chi)$ ``rotate'' the walk character value $\chi(\walkZ{q}{n})$
by a non-trivial amount, enough escapes should prevent the systematic
avoidance needed for permanent death-channel dodging.  PED is a weak
condition---it says nothing about cancellation, only that the
multipliers are not asymptotically trapped in any subgroup.

\begin{definition}[\defname{DecorrelationHypothesis}]
For every missing prime~$q$ and non-trivial~$\chi$, the multiplier
character sums are $o(N)$:
$\|\sum_{n<N} \chi(\multZ{q}{n})\| \leq \varepsilon N$ for large~$N$.
\end{definition}

Decorrelation is stronger than PED: it asks not merely that many
multipliers escape $\ker(\chi)$, but that they do so with enough
balance that the character values cancel.  If the multipliers were
independent random elements of $(\ZZ/q\ZZ)^\times$, the sum would be
$O(\sqrt{N})$ by the law of large numbers---far smaller than
$\varepsilon N$.  Decorrelation asks for the much weaker $o(N)$.

\begin{definition}[\defname{NoLongRuns}$(L)$]
For every missing prime~$q$ and non-trivial~$\chi$, past some
threshold, no $L$ consecutive multipliers all lie in $\ker(\chi)$.
\end{definition}

NoLongRuns is a qualitative cousin of PED: if multipliers never stay
inside $\ker(\chi)$ for $L$~steps in a row, then at least $1/(2L)$ of
them escape.  This condition is easier to verify in practice because it
only requires checking short blocks.

\begin{definition}[\defname{BlockRotationEstimate} (\abbr{BRE})]
If the escape count is $\geq \delta N$, then the walk character sums
are $o(N)$.  This encapsulates the Cauchy--Schwarz / van der Corput
step in harmonic analysis.
\end{definition}

BRE is the bridge between the multiplier-level conditions
(PED/Decorrelation) and the walk-level condition (CCSB).  It says:
given that multipliers escape with positive density, the walk
character sums must cancel.  The intuition is that each escape event
``rotates'' the walk character value $\chi(\walkZ{q}{n})$ by a
non-trivial amount, and sufficiently many such rotations produce
cancellation in the sum.  BRE is the sole unproved bridge in the PED
route.

The hypotheses above---PED, Decorrelation, NoLongRuns---all treat
the multiplier sequence as a single stream, ignoring what the walk is
doing at the moment.  But permanent death-channel avoidance is a
statement about the \emph{coupling} between walk position and
multiplier: the death channel $-\walkZ{q}{n}^{-1}$ is a function of
walk position, so avoiding it means the multiplier distribution at
each walk position is biased away from one class.  A more refined
hypothesis should address this coupling head-on.

\begin{definition}[\defname{ConditionalMultiplierEquidist} (\abbr{CME})]
\label{def:cme}%
For every missing prime~$q$, non-trivial~$\chi$, $\varepsilon > 0$,
there exists~$N_0$ such that for $N \geq N_0$ and every walk
position~$c \in (\ZZ/q\ZZ)^\times$:
$\|\sum_{\substack{n < N \\ \walkZ{q}{n} = c}} \chi(\multZ{q}{n})\|
\leq \varepsilon N$.
\end{definition}

CME says the multiplier distribution is the same at every walk
position.  Since the death channel is a function of walk position,
CME implies the death channel has no special status---the multiplier
is no more likely to avoid the forbidden class than any other.
Formally, CME is strictly stronger than Decorrelation: it bounds the
fiber sums
$\sum_{\substack{n<N \\ \walkZ{q}{n} = c}} \chi(\multZ{q}{n}) = o(N)$
separately for each position~$c$, not just the global sum.
Since the global sum is the sum of the fiber sums, CME implies
Decorrelation by the triangle inequality
(\lean{EM/LargeSieveSpectral.lean}{cme\_implies\_dec}).
The significance of CME is that it also implies CCSB
\emph{directly}, bypassing PED and BRE entirely: the fiber
decomposition and the telescoping identity together convert
conditional multiplier cancellation into walk character sum
cancellation, for \emph{all} character orders, without needing the
intermediate PED $\to$ BRE $\to$ CCSB chain.

\begin{theorem}[\lean{EM/EquidistSelfCorrecting.lean}{decorrelation\_implies\_ped}]
Decorrelation $\Rightarrow$ PED.
\end{theorem}

\begin{proof}[Proof sketch]
Contrapositive.  If few multipliers escape $\ker(\chi)$---say fewer
than $\delta N$---then most contribute $\chi(m(n)) = 1$ to the sum.
The at most $\delta N$ exceptions contribute values of norm~$\leq 1$.
By the reverse triangle inequality, $|\sum \chi(m(n))| \geq N - 2\delta N$,
which is $\geq \varepsilon N$ for $\delta$ small enough.  This
contradicts Decorrelation.
\end{proof}

\begin{theorem}[\lean{EM/EquidistSelfCorrecting.lean}{noLongRuns\_implies\_ped}]
NoLongRuns$(L) \Rightarrow$ PED with $\delta = 1/(2L)$.
\end{theorem}

\begin{proof}[Proof sketch]
Partition $\{0, \ldots, N{-}1\}$ into blocks of length~$L$.  Each block
contains at least one escape (by assumption), so the total escape count
is $\geq N/(2L)$.
\end{proof}

\begin{theorem}[\lean{EM/EquidistSelfCorrecting.lean}{block\_rotation\_implies\_ped\_csb}]
BRE $\Rightarrow$ PEDImpliesComplexCSB.
\end{theorem}

The PED route, with all proved arrows:
\[
\mathrm{Dec} \;\xrightarrow{\text{proved}}\; \mathrm{PED}
\;\xleftarrow{\text{proved}}\; \mathrm{NoLongRuns}(L)
\;\xrightarrow{\text{BRE, open}}\;
\mathrm{CCSB}
\;\xrightarrow{\text{proved}}\;
\mathrm{MC}.
\]
The sole open bridge in this route is BRE: converting positive escape
density into walk character sum cancellation.

However, the PED route is not the only path.  CME implies CCSB
\emph{directly}, bypassing PED and BRE entirely
(\lean{EM/LargeSieveSpectral.lean}{cme\_implies\_ccsb}):
\[
\mathrm{CME}
\;\xrightarrow{\text{proved}}\;
\mathrm{CCSB}
\;\xrightarrow{\text{proved}}\;
\mathrm{MC}.
\]
This bypass is significant: the $d \geq 3$ barrier
(Remark~\ref{rem:bre-impossible}) blocks the PED $\to$ CCSB
factorization for characters of order $\geq 3$, but CME $\to$ CCSB
works for \emph{all} character orders, using only the telescoping
identity and fiber decomposition.

\subsection{Vanishing Conditional Bias}
\label{sec:vcb}

CME is a strong hypothesis: it asks that \emph{every} fiber character
sum $F(c, \chi) = \sum_{\substack{n < N \\ w(n) = c}} \chi(m(n))$ be
$o(N)$---in other words, that $\mu = 0$ in the proportionality
$F(c, \chi) \approx \mu \cdot V_N(c)$.  But the telescoping route to
CCSB does not need~$\mu = 0$; it needs only that $\mu \neq 1$.  This
observation motivates a strictly weaker hypothesis.

\begin{definition}[\defname{VanishingConditionalBias} (\abbr{VCB})]
\label{def:vcb}%
For every missing prime~$q$, non-trivial~$\chi$, and $\varepsilon > 0$,
there exists~$N_0$ such that for $N \geq N_0$ there is a complex
number~$\mu = \mu(N, \chi)$ with $|\mu| \leq 1$ satisfying, for all
$c \in (\ZZ/q\ZZ)^\times$:
\[
  \left\|\sum_{\substack{n < N \\ w(n) = c}} \chi(m(n))
    \;-\; \mu \cdot V_N(c)\right\|
  \;\leq\; \varepsilon \cdot N,
\]
where $V_N(c) = |\{n < N : w(n) = c\}|$ is the visit count.
\end{definition}

In words: the factoring channel treats all walk positions
proportionally (the formal content of the factoring channel analogy
from \S\ref{sec:intro})---the character statistics of multipliers are
the same at every position, up to a common constant~$\mu$ (which may vary
with~$N$ and~$\chi$ but must be common across all positions~$c$).
Combined with PED (enough multipliers escape), the proportionality
constant cannot equal~1; the telescope then forces the walk character
sum to be $o(N)$, ruling out the sustained bias needed for permanent
avoidance.

\begin{proposition}[\lean{EM/LargeSieveSpectral.lean\#L1997}{cme\_implies\_vcb}]
$\mathrm{CME} \Rightarrow \mathrm{VCB}$ with $\mu = 0$.
Conversely, $\mathrm{VCB} + \mathrm{Dec} \Rightarrow \mathrm{CME}$.
Thus CME decomposes as $\mathrm{CME} = \mathrm{VCB} + \mathrm{Dec}$.
VCB alone is strictly weaker than CME: it permits the fiber sums to
be $\Theta(N)$, provided they are proportional to the visit counts.
\end{proposition}

\begin{theorem}[\lean{EM/LargeSieveSpectral.lean\#L2049}{vcbPedImpliesCcsb}]
\label{thm:vcb-ped-ccsb}
$\mathrm{VCB} + \mathrm{PED} \;\Longrightarrow\; \mathrm{CCSB}$.
\end{theorem}

\begin{proof}[Proof sketch]
From VCB, the telescope identity gives
$(\mu - 1) \cdot S_N = O(1) + O(\varepsilon N (q{-}1))$.
If $|1 - \mu|$ is bounded away from zero, then
$S_N = O(\varepsilon N / |1{-}\mu|) = o(N)$.

To show $|1 - \mu| \geq c_0 > 0$, PED provides $\delta > 0$ such
that at least $\delta N$ multipliers escape $\ker(\chi)$.  For each
escaping multiplier, $\chi(m(n))$ is a non-trivial $d$-th root of
unity, satisfying $\mathrm{Re}(\chi(m(n))) \leq 1 - \eta_0^2/2$
where $\eta_0 = \min_{\zeta \in \mu_d \setminus \{1\}} |\zeta - 1| > 0$
(\lean{EM/LargeSieveSpectral.lean\#L2014}{norm\_sub\_one\_sq\_eq},
\lean{EM/LargeSieveSpectral.lean\#L2027}{unit\_norm\_re\_le\_of\_dist}).
This real-part defect accumulates over $\delta N$ escaping steps,
forcing $|\sum \chi(m(n)) - N| \geq c_0 N$ for
$c_0 = \delta \eta_0^2 / 2$.  Combined with VCB's control
$|\sum \chi(m(n)) - \mu N| \leq C \varepsilon N$, the reverse
triangle inequality gives $|1 - \mu| \geq c_0/2$ for small
$\varepsilon$.
\end{proof}

\begin{theorem}[\lean{EM/LargeSieveSpectral.lean\#L2343}{vcb\_ped\_implies\_mc}]
$\mathrm{VCB} + \mathrm{PED} \;\Longrightarrow\; \MC$.
\end{theorem}

This decomposition splits the monolithic CME hypothesis into two
independently attackable pieces:
\begin{itemize}[nosep]
\item \textbf{VCB} (``the factoring channel preserves proportionality''):
  the character distribution of multipliers is the same at every walk
  position, up to a common rate.  This captures the ``factorization
  independence'' intuition---that the factoring operation destroys the
  correlation between walk position and multiplier residue.
\item \textbf{PED} (``enough multipliers escape every character kernel''):
  a positive fraction of multiplier residues fall outside any proper
  subgroup.
\end{itemize}
VCB is a recent contribution of the formalization.

\subsection{Walk Telescoping Identities}
\label{sec:telescope}

The hypotheses above attack CCSB by decomposing it into conditions on
the multiplier sequence.  Before proceeding to the large sieve route,
we pause to examine what the walk recurrence
$\chi(w(n{+}1)) = \chi(w(n)) \cdot \chi(m(n))$ itself forces.  This
multiplicative structure is the \emph{only} algebraic relation
connecting walk and multiplier character values, and it constrains
any possible proof of CCSB\@.  The following identities make these
constraints explicit.  They are formalized not because they solve
CCSB, but because they reveal the structural landscape: they show
which proof strategies are compatible with the walk's algebra and
which are ruled out.

\begin{theorem}[\lean{EM/EquidistSelfCorrecting.lean}{walk\_telescope\_identity}]
\label{thm:telescope}
For any $\chi$ and $N$:
\[
  \sum_{n < N} \chi(w(n))\cdot\bigl(\chi(m(n)) - 1\bigr)
  \;=\; \chi(w(N)) - \chi(w(0)).
\]
\end{theorem}

This identity follows immediately from the walk recurrence
$\chi(w(n{+}1)) = \chi(w(n)) \cdot \chi(m(n))$: writing
$\chi(w(n)) \cdot (\chi(m(n)) - 1) = \chi(w(n{+}1)) - \chi(w(n))$,
the sum telescopes to $\chi(w(N)) - \chi(w(0))$.

\begin{theorem}[\lean{EM/EquidistSelfCorrecting.lean}{walk\_telescope\_norm\_bound}]
The telescoping sum has norm $\leq 2$ (triangle inequality on
unit-norm terms).
\end{theorem}

The $\leq 2$ bound looks innocent, but it constrains how the walk
can respond to the death channel.  Each summand
$\chi(w(n))\cdot(\chi(m(n)) - 1)$ is non-zero exactly when
$\chi(m(n)) \neq 1$---i.e., when the multiplier escapes
$\ker(\chi)$, ``rotating'' the character value.  The $\leq 2$ bound
means the net rotation over all $N$ steps is negligible, tightly
coupling the walk character sum $S_N = \sum_{n<N} \chi(w(n))$ to the
multiplier character sum $M_N = \sum_{n<N} \chi(m(n))$.  Splitting
the product in Theorem~\ref{thm:telescope} gives
$S_N \cdot \overline{M_N/N} - S_N = O(1)$ (after normalization).

\begin{theorem}[\lean{EM/EquidistSelfCorrecting.lean}{walk\_shift\_one\_correlation}]
\label{thm:shift-one}
$\sum_{n < N} \chi(w(n)) \cdot \overline{\chi(w(n+1))}
  = \overline{\sum_{n < N} \chi(m(n))}$.
\end{theorem}

This identity says that the lag-1 autocorrelation of the walk
character equals the conjugate of the multiplier character sum.  It is
a \emph{no-go result} for the van der Corput method with $H = 1$:
VdC bounds $|S_N|^2$ in terms of autocorrelations, but at lag $h = 1$,
the autocorrelation is exactly $|M_N|$---the multiplier character
sum---which need not be small.  So VdC with a single shift gives only
$|S_N| \leq O(\sqrt{N \cdot |M_N|})$, which is $O(N)$ in the worst
case, not the $o(N)$ that CCSB requires.  This means any proof of CCSB
must either (i)~use higher-order correlations (HOD,
Appendix~\ref{sec:sve}) or (ii)~establish multiplier decorrelation
first.

\subsection{The Large Sieve Route}
\label{sec:large-sieve}

The preceding subsections attacked the question ``can the walk avoid
$-1$ permanently?'' for a single modulus~$q$.  A different strategy
works with \emph{many moduli simultaneously}: classical analytic
number theory produces character sum estimates averaged over all
moduli $q \leq Q$, and such averaged results are often easier to
prove than pointwise ones.  The large sieve inequality and the
Bombieri--Vinogradov theorem are the two central tools for this.

The formalization develops this multi-modular route across three files
(\code{LargeSieve.lean}, \code{Large\-Sieve\-Harmonic.lean},
\code{Large\-Sieve\-Analytic.lean}) totaling ${\sim}5{,}870$~lines,
connecting classical analytic number theory to MC via a multi-modular
character sum bound.

\paragraph{Why formalize the large sieve?}
The analytic large sieve inequality and the Bombieri--Vinogradov theorem
are among the most powerful tools in analytic number theory for
controlling the distribution of primes in arithmetic progressions.
If these tools could be applied to the EM walk, MC would follow.
We formalize the connection---not the deep theorems themselves (which
are known results, stated as open Props)---for two reasons:
\begin{enumerate}[nosep]
\item To identify \emph{precisely} what transfer hypothesis is needed to
  apply each classical result to the specific EM orbit, and
\item To verify that six apparently independent routes (BV, ArithLS, ALS,
  PrimeArithLS, LoD, sieve transfer) all reduce to the \emph{same}
  orbit-specificity gap.
\end{enumerate}
This diagnosis is itself a mathematical contribution: it shows that the
difficulty of MC is not a failure of existing analytic tools but a
fundamental obstacle in applying ensemble-averaged results to a single
deterministic orbit.

\begin{definition}[\defname{MultiModularCSB} (\abbr{MMCSB})]
There exists $Q_0$ such that for all $q \geq Q_0$ prime, every
non-trivial character~$\chi$ mod~$q$, and every $\varepsilon > 0$,
there exists $N_0$ such that for $N \geq N_0$:
$\|S_\chi(N)\| \leq \varepsilon N$.
\end{definition}

MultiModularCSB is weaker than CCSB in that it allows finitely many
exceptional primes below~$Q_0$.  This weakening is crucial because
averaged results like BV naturally produce bounds that fail for
finitely many moduli.

\begin{theorem}[\lean{EM/LargeSieve.lean\#L320}{mmcsb\_implies\_mc}]
\label{thm:mmcsb-mc}
$\mathrm{MultiModularCSB} \;\Longrightarrow\; \MC$.
\end{theorem}

The proof composes the per-prime Fourier bridge
(Theorem~\ref{thm:fourier-step}) with the inductive bootstrap: for
$q \geq Q_0$, MMCSB gives hit count $\sim N/(q{-}1) > 0$ at~$-1$,
contradicting permanent avoidance; for the finitely many primes
$q < Q_0$, the bootstrap (Section~\ref{sec:bootstrap}) handles
them---once MC holds for all primes below~$q$, the sieve gap and
one-prime gap (Theorem~\ref{thm:one-prime-gap}) reduce MC($q$) to a
single hitting event.

Three parallel routes to MultiModularCSB are formalized:

\paragraph{Bombieri--Vinogradov route.}
BV says primes are equidistributed among arithmetic progressions ``on
average over moduli'': for most $q \leq Q = \sqrt{x}/(\log x)^A$, the
count of primes $\leq x$ in any progression $a \bmod q$ is close to
the expected $\pi(x)/\phi(q)$.  If this applies to the EM multipliers,
the death channel---a single progression---gets its fair share of
multipliers, breaking the avoidance.

\begin{theorem}[\lean{EM/LargeSieve.lean\#L379}{bv\_chain\_mc}]
$\mathrm{BV} + \mathrm{BVImpliesMMCSB} \;\Longrightarrow\; \MC$.
\end{theorem}
The transfer hypothesis BVImpliesMMCSB is a \textbf{genuine frontier}:
it requires transferring the averaged equidistribution statement of BV
(valid for primes in generic progressions) to the specific EM walk orbit.
The EM sequence is not a generic sample of primes---it is a deterministic
sequence defined by iterated factorization---so its multipliers could
exhibit special correlations that BV's averaged estimate cannot detect.

The route was decomposed into two stages:
\[
\mathrm{BV}
\xrightarrow{\text{sieve transfer}}
\mathrm{EMMultCSB}
\xrightarrow{\text{walk bridge}}
\mathrm{MMCSB}
\xrightarrow{\text{proved}}
\mathrm{MC},
\]
separating the number-theoretic content (BV~$\Rightarrow$~EMMultCSB,
where EMMultCSB bounds the \emph{multiplier} character sums)
from the dynamical content (EMMultCSB~$\Rightarrow$~MMCSB, converting
multiplier bounds to \emph{walk} bounds).  However, the walk bridge
\textbf{MultCSBImpliesMMCSB is false in general}
(\lean{EM/LargeSieve.lean\#L1243}{MultCSBImpliesMMCSB}):
the walk character sum $\sum \chi(w(n))$ is a \emph{partial product}
$\prod_{k<n} \chi(m(k))$ of the multiplier characters, and partial
products of equidistributed unit complex numbers need not cancel---they
perform a random walk on the unit circle whose norm grows as~$\sqrt{N}$,
not as~$o(N)$.  The telescope identity (Theorem~\ref{thm:shift-one})
makes this obstruction precise: the $h\!=\!1$ autocorrelation equals
the multiplier character sum, so van der Corput with a single shift
gives only~$O(N)$, not~$o(N)$.
This is why the CME bypass (fiber decomposition $+$ telescoping,
Definition~\ref{def:cme}) is essential: it goes directly from
conditional multiplier equidistribution to CCSB without ever
requiring the walk bridge.

\paragraph{Additional sieve routes.}
Two further routes are formalized---the arithmetic large sieve
(ArithLS~$\Rightarrow$~MC, a dead end) and the analytic
large sieve (ALS~$\Rightarrow$~PrimeArithLS~$\Rightarrow$~MC), where
the ALS-to-PrimeArithLS bridge via Gauss sum inversion is fully proved
across eight internal lemmas.  In both cases, the genuine open content
is the same \emph{orbit-specificity transfer}: applying averaged
results to one deterministic orbit.  The full details, including the
ALS definition, weak ALS proof, Gauss sum inversion theorem, and a
spectral energy route (SVE, van der Corput, HOD, CME) with its
complete hypothesis hierarchy, appear in Appendix~\ref{app:routes}.

%% =========================================================================
\section{Why It's Hard}
\label{sec:hard}
%% =========================================================================

\noindent\textbf{Original contribution.}\enspace
The selectability analysis, oracle barrier, and CCSB-as-frontier
argument below are new.  They explain \emph{why} the remaining
hypothesis resists both computation and existing analytic tools.

\subsection{The Selectability Perspective}
\label{sec:selectability}

The Euclid construction guarantees fresh primes at every step: every
prime factor of~$\Prod(n)+1$ is new (coprime to the running product).
The difficulty of MC is not the \emph{existence} of new primes but
whether the $\minFac$ rule eventually \emph{selects} each one.  The
formalization makes this contrast precise.

\begin{theorem}[\lean{EM/EquidistOrbitAnalysis.lean}{divisor\_not\_yet\_in\_seq}]
\label{thm:divisor-fresh}
If $p \mid \Prod(n)+1$, then $\seq(m) \neq p$ for all $m \leq n$.
\end{theorem}

\begin{proof}
Any $\seq(m)$ with $m \leq n$ divides $\Prod(n)$.  A number $\geq 2$
cannot divide both~$a$ and~$a+1$.
\end{proof}

\begin{theorem}[\lean{EM/EquidistOrbitAnalysis.lean}{passed\_over\_persists}]
If $p \mid \Prod(n)+1$ but $\seq(n\!+\!1) \neq p$ (the $\minFac$ rule
chose a smaller prime), then $\seq(m) \neq p$ for all $m \leq n+1$.
The prime survives to potentially divide future Euclid numbers.
\end{theorem}

\begin{theorem}[\lean{EM/EquidistOrbitAnalysis.lean}{selectability\_extinguished}]
\label{thm:extinct}
Once $\seq(m) = p$, we have $p \mid \Prod(n)$ for all $n \geq m$, so
$p \nmid \Prod(n)+1$ ever again.  Selectability is a one-shot resource.
\end{theorem}

\begin{definition}[\defname{InfinitelySelectable}]\label{def:inf-sel}
A prime~$p$ is \emph{infinitely selectable} if $p \mid \Prod(n)+1$ for
cofinally many~$n$: $\forall\, N,\, \exists\, n \geq N,\,
p \mid \Prod(n)+1$.
\end{definition}

By Theorem~\ref{thm:extinct}, MC$(p)$ and
Infinitely\-Selectable$(p)$ are mutually exclusive
(\lean{EM/EquidistOrbitAnalysis.lean}%
{mc\_implies\_not\_infinitely\_selectable}):
a prime that enters the sequence can never be selectable again.

\begin{theorem}[\lean{EM/EquidistOrbitAnalysis.lean}{dh\_implies\_infinitely\_selectable}]
\label{thm:dh-inf-sel}
Under DH, every prime that never appears in the sequence (with SE
satisfied) is infinitely selectable.
\end{theorem}

\paragraph{The random-factor variant is easy.}
Consider a variant of the Euclid--Mullin construction where, instead of
the smallest prime factor, one picks a \emph{random} prime factor
of~$\Prod(n)+1$ at each step.  In this variant, MC follows from DH
alone: whenever $p \mid \Prod(n)+1$, simply choose~$p$.  Under DH with
SE, this happens infinitely often (Theorem~\ref{thm:dh-inf-sel}), so
$p$ eventually gets picked.

The argument is even simpler probabilistically.  For any target
prime~$p$, the residue $r_n = \Prod(n) \bmod p$ performs a
multiplicative walk on $(\ZZ/p\ZZ)^\times$.  In the random-factor
variant, each multiplier is a random element of the group; once the
multipliers generate the full group (which PRE guarantees), the walk
is a genuine random walk with full support.  By classical
equidistribution on finite groups, $r_n$ converges to uniform,
so $r_n = -1$ (i.e., $p \mid \Prod(n)+1$) occurs with
probability $\to 1/(p\!-\!1)$---infinitely often with
probability~1.

\paragraph{The lpf variant is hard.}
The actual Euclid--Mullin sequence uses
$\seq(n{+}1) = \minFac(\Prod(n){+}1)$, a \emph{deterministic}
function of the walk position.  This creates the exact correlation
identified by the oracle analysis (\S\ref{sec:oracle}): the
multiplier at step~$n$ depends on the full value of~$\Prod(n)+1$,
coupling walk position to multiplier.  The random-factor variant
breaks this coupling by choosing multipliers independently of
position; the $\minFac$ rule preserves it.

The difficulty of Mullin's Conjecture is entirely in the minimality of
the prime selection, not in the Euclidean construction itself.  The
inductive bootstrap (Section~\ref{sec:bootstrap}) bridges this gap:
$\MC({<}\,p)$ ensures all primes below~$p$ are already in the sequence,
hence divide~$\Prod(n)$, hence cannot divide~$\Prod(n)+1$.  Past a
computable stage, $p$ is the \emph{smallest} available factor whenever
it divides the Euclid number---reducing the $\minFac$ variant to the
``any-factor'' variant for the tail of the sequence.

\subsection{The Marginal/Joint Barrier}
\label{sec:oracle}

The verified reductions (TailSE, CofinalEscape, QuotientDH) exhaust
what can be proved about the \emph{marginal} distribution of multiplier
residues.

\begin{theorem}[\lean{EM/EquidistOrbitAnalysis.lean}{emfe\_iff\_tail\_se\_at}]
$\mathrm{EuclidMinFacEscape}(q) \Leftrightarrow \mathrm{TailSE}(q)$.
\end{theorem}

Even perfect per-position equidistribution of multipliers is consistent
with HH failure.  DH is a \emph{joint} statement---the
(position, multiplier) pair must hit the \emph{death curve}
$\multZ{q}{n} = -\walkZ{q}{n}^{-1}$---and no marginal statement can
force this.

\paragraph{The orbit chain gap.}
The cofinal orbit analysis picks one cofinal multiplier~$s_x$ per
walk position, producing a cycle $x_0 \to x_1 \to \cdots \to x_0$
in $(\ZZ/q\ZZ)^\times$.  Even when the cofinal multipliers generate
the full group, the cycle size~$k$ can be less than~$|(\ZZ/q\ZZ)^\times|$.
Example: in $\ZZ/6\ZZ$, the cycle $0 \to 1 \to 0$ has
$\langle 1, 5 \rangle = \ZZ/6\ZZ$ but misses~3.

Closing this gap requires showing that at each cofinally visited
position, \emph{multiple} multiplier classes appear---expanding the
cycle until it covers $-1$.  This is the ``specific-orbit problem'':
transferring generic equidistribution of $\minFac$ residues to the
particular EM orbit.

\subsection{The BRE Impossibility for $d \geq 3$}

\begin{remark}\label{rem:bre-impossible}
Positive escape density (PED) alone does \emph{not} imply
CCSB for characters of order $d \geq 3$.

\emph{Counterexample}: a walk on $\ZZ/3\ZZ$ that alternates between
only two of the three $d$-th roots of unity (phase-aligned escapes)
achieves positive escape density yet has walk sum
$\approx N/2 \cdot (1 + \omega) \neq o(N)$.

For $d = 2$ this degeneracy vanishes: the only non-trivial rotation
is~$-1$, so escape frequency \emph{is} the rotation distribution.
The order-2 BRE from NoLongRuns$(L)$ is proved in the formalization
(\lean{EM/EquidistSelfCorrecting.lean}{order2\_noLongRuns\_mc}).
But for $d \geq 3$, PED constrains how often the walk rotates without
constraining the \emph{distribution} among $d-1$ non-identity
rotations.  The $\mathrm{PED} \Rightarrow \mathrm{BRE} \Rightarrow
\mathrm{CCSB}$ factorization is invalid for $d \geq 3$.

This barrier is specific to the PED route.  The CME $\to$ CCSB
reduction (\lean{EM/LargeSieveSpectral.lean}{cme\_implies\_ccsb})
bypasses PED and BRE entirely, working for all character orders $d$
via the telescoping identity.  The $d \geq 3$ problem is therefore
not a barrier for the \emph{reduction}---only for the particular
factorization through PED.
\end{remark}

\paragraph{The Van der Corput Barrier}

The van der Corput inequality (Theorem~\ref{thm:vdc}, now fully proved
in the formalization) converts character sum bounds into autocorrelation
bounds.  Theorem~\ref{thm:shift-one} gives $R_1 = o(N)$ under the
Decorrelation Hypothesis.  VdC with $H = 1$
yields $|S_N|^2 \leq \frac{N+1}{2}(N + 2|R_1|) = N^2/2 + o(N^2)$,
hence $|S_N| \leq N/\sqrt{2}$.  This is non-trivial but \emph{not}
$o(N)$.  To get $o(N)$, one needs higher-order correlations $R_h = o(N)$
for $h \geq 2$, which requires HigherOrderDecorrelation
(Theorem~\ref{thm:hod-mc}).
The telescoping identity $\sum_n \chi(w(n))(\chi(m(n))-1) = O(1)$
is a precise structural constraint.

\paragraph{The Walk Bridge Falsity}
\label{sec:walk-bridge-false}

The BV route decomposes into two stages: sieve transfer
(BV~$\Rightarrow$~EMMultCSB, bounding \emph{multiplier} character sums)
and the walk bridge (EMMultCSB~$\Rightarrow$~MMCSB, converting multiplier
bounds to \emph{walk} bounds).  The walk bridge
\textbf{MultCSBImpliesMMCSB}
(\lean{EM/LargeSieve.lean\#L1243}{MultCSBImpliesMMCSB})
is stated as an open \code{Prop} and is \textbf{false in general}.

The obstruction is structural: the walk character sum is a
\emph{partial product} $\chi(w(n)) = \prod_{k<n} \chi(m(k))$
of the multiplier characters.  Even when the individual factors
$\chi(m(k))$ are equidistributed on the unit circle (so their
\emph{sum} cancels), their \emph{partial products} perform a
multiplicative walk whose norm grows as~$\sqrt{N}$,
not~$o(N)$.  Cancellation of sums does not imply cancellation
of cumulative products.

This negative result explains why the CME bypass
(Definition~\ref{def:cme}) is essential.  CME uses fiber
decomposition and telescoping to go directly from conditional
multiplier equidistribution to CCSB, circumventing the walk bridge
entirely.

\subsection{The Factorization Independence Heuristic}
\label{sec:hash-heuristic}

The preceding barriers explain why MC is hard to \emph{prove}.  This
subsection explains why it should be \emph{true}---and why the
formalization's sole remaining hypothesis (CME) is the precise
mathematical content of a natural intuition about factorization.

\paragraph{The information bottleneck.}
The information bottleneck of \S\ref{sec:sufficient-conditions}---$O(\log q)$
bits visible to the walk versus ${\sim}2^n$ bits determining the
multiplier---means the mutual information vanishes exponentially.
This is directly analogous to the pseudorandomness of iterated hash
functions: if~$H$ is a cryptographic hash, the sequence
$x, H(x), H(H(x)), \ldots$ is deterministic but statistically
indistinguishable from random, because the hash destroys recoverable
correlations.  For the EM sequence, integer factorization plays the
role of the hash function.  Both processes are fully deterministic
(given the seed, every term is uniquely determined), one-way
(computing forward is trivial; extracting structure from the output
is computationally hard), and \emph{de facto} uncorrelated
(consecutive terms pass every reasonable test for independence).

The analogy is heuristic, not rigorous.  SHA-256 is \emph{designed}
for pseudorandomness; $\minFac$ is not designed for anything.  We do
not claim that computational hardness of factoring implies MC---what
MC requires is a number-theoretic statement (CME), not a
complexity-theoretic one.  But the analogy explains the structure of
the problem: proving pseudorandomness of a deterministic process
requires showing that \emph{no exploitable structure exists}, which is
harder than finding structure.  This is why the conjecture resists
proof despite overwhelming heuristic evidence.

\paragraph{Negative analogies: why the selection rule matters.}
The factorization independence heuristic explains why the EM sequence
($\minFac$ variant) should contain all primes, while related sequences
do not.

\begin{itemize}[nosep]
\item The \textbf{Sylvester sequence} $s(n{+}1) = s(0) \cdots s(n) + 1$
  has density-zero prime divisors~\cite{Odoni1985}.  Its terms grow
  doubly exponentially, giving each prime only $O(1)$ chances to appear
  as a factor.  The ``hash chain'' runs too fast.

\item \textbf{Fermat numbers} $F_n = 2^{2^n} + 1$ have the even stronger
  property that $\sum 1/p$ converges over their prime divisors.  Again,
  doubly exponential growth limits opportunities.

\item The \textbf{second EM sequence} ($\mathrm{maxFac}$ variant)
  provably omits infinitely many primes~\cite{CoxVdP1968,
  PollackTrevino2014}.  Here the ``hash function'' ($\mathrm{maxFac}$
  instead of $\minFac$) produces large multipliers, causing the product
  to grow rapidly---analogous to using a hash function that amplifies
  rather than compresses.

\item The \textbf{first EM sequence} ($\minFac$ variant) keeps the
  product growing slowly---heuristically as $\exp(n^2/2)$
  (see~\S\ref{sec:lean}).  This is analogous to a hash function that
  compresses, giving exponentially many iterations within any fixed
  modulus.  The conjecture is that this compression ensures coverage.
\end{itemize}

\paragraph{What the formalization adds.}
The formalization identifies three equivalent formulations of the
factoring channel's decorrelation:
\begin{enumerate}[nosep]
\item \textbf{Conditional Multiplier Equidistribution (CME)}:
  the distribution of $\minFac(\Prod(n)+1) \bmod q$, conditioned on
  $\Prod(n) \equiv c \pmod{q}$, is asymptotically independent of~$c$
  (Definition~\ref{def:cme}).
\item \textbf{Decorrelation Hypothesis}: the character sum
  $\sum \chi(\multZ{q}{n}) = o(N)$ for nontrivial~$\chi$---the
  multiplier residues have cancelling character sums, as independent
  random variables would.
\item \textbf{ComplexCharSumBound (CCSB)}: the walk character sums
  $\sum \chi(\walkZ{q}{n}) = o(N)$ for nontrivial~$\chi$---ruling out
  permanent avoidance of any residue class.
\end{enumerate}
These are related by proved implications
($\mathrm{CME} \Rightarrow \mathrm{Dec} \Rightarrow \mathrm{PED}$)
and direct bypass ($\mathrm{CME} \Rightarrow \mathrm{CCSB}$).
Each implies MC through the verified reduction chain.  The irreducible
mathematical content is: does the factoring operation destroy enough
correlation for character sums to cancel?

\subsection{Dead Ends as a Roadmap}

Over a hundred potential approaches
were explored and found to be dead ends (catalogued in full in the
project's \code{dead\_ends.md}).  Each elimination is informative:
it narrows the space of viable strategies.  A representative selection,
grouped by the type of obstruction:

\smallskip
\begin{center}
\renewcommand{\arraystretch}{1.15}
\small
\begin{tabular}{@{}p{3.6cm}p{8.8cm}@{}}
\toprule
\textbf{Dead end} & \textbf{Why it fails} \\
\midrule
\multicolumn{2}{@{}l}{\textsc{Ensemble-to-orbit transfer}:
  \emph{tool applies to generic sequences, not the specific EM orbit}} \\[2pt]
BV for EM subsequence
  & BV applies to all primes in APs, not to a greedy subsequence. \\
Furstenberg / ergodic theory
  & Standard ergodic methods assume classical multiplicativity;
    the EM sequence is recursive and non-multiplicative. \\
Diaconis--Shahshahani lemma
  & Requires i.i.d.\ random steps; inapplicable to the
    deterministic EM walk. \\
\midrule
\multicolumn{2}{@{}l}{\textsc{Independence / linearity violated}:
  \emph{tool requires additive or independent structure the walk lacks}} \\[2pt]
Large sieve for partial products
  & The large sieve handles linear sums, not multiplicative walks. \\
Abel summation
  & Converts multiplier decorrelation to walk-sum bounds, but the
    summation weights \emph{amplify} rather than cancel. \\
Self-avoidance $\Rightarrow$ CCSB
  & Self-avoidance (no repeated $\hat{\ZZ}$ positions) is invisible
    to characters, which see only residues. \\
\midrule
\multicolumn{2}{@{}l}{\textsc{Wrong algebraic structure}:
  \emph{the group or decomposition has no room for the desired bound}} \\[2pt]
Non-abelian / representation
  & $(\ZZ/q\ZZ)^\times$ is cyclic; all irreps are 1-d characters.
    No higher-dimensional structure to exploit. \\
CRT product group
  & Reformulating on $\prod_{q \leq Q}(\ZZ/q\ZZ)^\times$ makes
    the problem harder: the product group is exponentially large. \\
NoLongRuns + PED $\Rightarrow$ BRE ($d \!\geq\! 3$)
  & Variable block lengths align adversarially with character phases. \\
DPED $\Rightarrow$ CCSB ($d \!\geq\! 3$)
  & Alternating $\omega,\omega^2$ rotations satisfy DPED yet produce
    $\Theta(N)$ walk sums.  All PED-to-CCSB intermediates ruled out. \\
\midrule
\multicolumn{2}{@{}l}{\textsc{Reduces to single-modulus CCSB}:
  \emph{no genuine simplification}} \\[2pt]
Multi-modular approaches
  & All variants (BV + threshold, CRT, death coupling) collapse
    to single-modulus CCSB. \\
Death set coupling across moduli
  & Death sets $\{m : \minFac(m) \equiv -c^{-1}\}$ vary per step;
    no uniform coupling bound exists. \\
Spectral gap (deterministic walk)
  & Spectral gap theory applies to probability measures on groups
    (convergence of random sampling); the EM walk is a single
    deterministic path. \\
Information-theoretic bounds
  & Category error: entropy/mutual-information tools assume a
    random variable; the EM sequence is deterministic with zero
    entropy. \\
\bottomrule
\end{tabular}
\renewcommand{\arraystretch}{1.0}
\end{center}

\paragraph{The Four-Way Blocker.}
The majority of the 105~dead ends reduce to a single meta-obstacle:
every known technique for proving equidistribution of sequences on
finite groups requires at least \emph{one} of (1)~independence of
steps, (2)~multiplicativity of the generating function,
(3)~algebraic-geometric structure (parameter families, monodromy),
or (4)~ergodic stationarity.  The EM walk satisfies \emph{none} of
these: the steps are deterministic and mutually dependent, the walk
is not a multiplicative function, the multipliers have no known
algebraic-geometric parametrization, and the non-autonomous dynamics
(a different multiplier at each step) rule out stationarity.  This
explains why classical tools---random-walk mixing, Hal\'asz's theorem,
Katz monodromy, Birkhoff averages---all fail.

\paragraph{The telescope exhausts the algebra.}
The telescoping identity $\chi(w(n{+}1)) = \chi(w(n)) \cdot
\chi(m(n))$ is the \emph{complete} algebraic content of the walk.
Only two decomposition strategies for the character sum $S_N =
\sum_{n<N} \chi(w(n))$ exist: grouping by \emph{value} (fiber
decomposition $\to$ CME) and grouping by \emph{lag} (autocorrelation
$\to$ HOD).  Every other rearrangement---Abel summation, M\"obius
inversion, Dirichlet series, block decomposition---either reduces to
one of these two or fails outright (Abel gives $O(N^2)$ remainder in
the wrong direction; M\"obius/Dirichlet require multiplicativity).
There is no third algebraic route to CCSB\@.

\smallskip\noindent
The pattern: every approach that avoids the specific EM orbit's
joint distribution either reduces to CCSB or fails.  Since CME
implies CCSB (proved), and CME decomposes as VCB~$+$~Dec
(\S\ref{sec:vcb}), the sharpest targets are now VCB~$+$~PED or CME:
conditional equidistribution of multipliers given walk position, or
the weaker proportionality condition combined with escape density.
CME is strictly weaker than CCSB and is the \emph{irreducible
analytic content}.

\paragraph{The Mathematical Landscape}

We need to prove one of these equivalent statements for every missing
prime~$q$:
\begin{itemize}[nosep]
\item \textbf{DH}: If the multipliers generate $(\ZZ/q\ZZ)^\times$,
  the walk $\walkZ{q}{n} = \Prod(n) \bmod q$ hits $-1$ cofinally.
\item \textbf{CCSB}: For every non-trivial character
  $\chi \colon (\ZZ/q\ZZ)^\times \to \mathbb{C}^\times$, the sum
  $\sum_{n < N} \chi(\walkZ{q}{n}) = o(N)$.
\item \textbf{$d{=}2$ special case} (as a stepping stone): For every
  quadratic character~$\chi$, the $\pm 1$-valued walk character sum is
  $o(N)$.
\end{itemize}
The formalization has conclusively shown that every tool requiring
independence, classical multiplicativity, or ensemble averaging fails.
So we need ideas that exploit what the EM walk specifically has.

\paragraph{Structural Features of the EM Walk}

The dead ends above show what does not work.  Complementarily, the EM
walk has four structural features---all proved or formalized---that no
dead-end approach has successfully exploited.  Any proof of MC will
almost certainly use at least one.

\medskip\noindent\textbf{Feature~1: Super-exponential growth.}\enspace
$\Prod(n) \geq 2^n$
(\lean{EM/LargeSieve.lean}{prod\_lower\_bound\_for\_sieve}).  The
Euclid numbers $\Prod(n)+1$ grow absurdly fast.  This means the
\emph{sieve level}---the threshold below which all prime factors have
been excluded---grows super-exponentially.  By step~$n$, the Euclid
number $\Prod(n)+1$ is coprime to each of $\seq(0), \ldots, \seq(n)$,
a growing set of distinct primes.  The pool of ``available'' small
primes as factors of $\Prod(n)+1$ shrinks, but the size of
$\Prod(n)+1$ grows so fast that it must have enormous prime factors
most of the time.

\medskip\noindent\textbf{Feature~2: Mutual coprimality of Euclid
numbers.}\enspace For $m > n$, $\Prod(m)$ is divisible by
$\seq(n+1)$, which divides $\Prod(n)+1$.  So
$\Prod(m)+1 \equiv 1 \pmod{\seq(n+1)}$: successive Euclid numbers
live in different residue classes modulo earlier sequence terms.  This
coprimality structure means the Euclid numbers cannot all ``avoid'' a
residue class in a coordinated way---their residues are forced apart
by the construction.

\medskip\noindent\textbf{Feature~3: The multiplier is the smallest
prime factor.}\enspace  This is the key constraint that everyone
mentions but nobody has quantified.  If $\walkZ{q}{n} \neq -1$ (so
that $q$ does not divide $\Prod(n)+1$ as the smallest factor), then
the multiplier $\multZ{q}{n} = \minFac(\Prod(n)+1)$ satisfies
$\multZ{q}{n} \leq (\Prod(n)+1)^{1/2}$.  For a number of size
${\sim}2^n$, this smallest factor could be as small as~$3$ or as
large as~${\sim}2^{n/2}$.  The $\minFac$ rule creates a deterministic
coupling between walk position and multiplier: the multiplier at
step~$n$ depends on the full value of~$\Prod(n)+1$, not just its
residue.

\medskip\noindent\textbf{Feature~4: Self-correcting feedback.}\enspace
If the walk concentrates on certain residues mod~$q$---say
$\walkZ{q}{n} \equiv a \pmod{q}$ for many~$n$---then
$\Prod(n)+1 \equiv a+1 \pmod{q}$ for many~$n$.  The smallest prime
factor of numbers $\equiv a+1 \pmod{q}$ depends on~$a+1$, creating a
feedback loop: concentration in one residue class biases the
multiplier distribution, which in turn pushes the walk away from that
class.  This self-correcting mechanism has been formalized
(\code{EquidistSelfCorrecting.lean}), but all paths from it lead to
\textsc{SieveTransfer}---the open hypothesis that generic
$\minFac$~equidistribution transfers to the specific EM orbit.
We return to this feature in our assessment of whether DH is true
(\S\ref{sec:open}).

\medskip
These four features---growth, coprimality, the $\minFac$ selection
rule, and self-correcting feedback---are the raw material that any
successful approach must engage with.  The dead ends above fail
precisely because they treat the walk generically (as a random walk,
or as an arbitrary multiplicative walk) rather than exploiting the
specific arithmetic of the EM construction.

%% =========================================================================
\section{The Lean Formalization}
\label{sec:lean}
%% =========================================================================

Having developed the mathematical reduction and identified the
structural obstacles, we describe the Lean~4 formalization that
certifies these results.

\paragraph{Codebase Structure}

The formalization uses Lean~4 with Mathlib~v4.27.0 across 35~files
totaling ${\sim}26{,}900$~lines.  The dependency chain is linear, with
three leaf modules:

\begin{center}
\small
\begin{tabular}{llr}
\toprule
\textbf{File} & \textbf{Content} & \textbf{Lines} \\
\midrule
\code{Euclid.lean} & Constructive Euclid's theorem & 422 \\
\code{MullinDefs.lean} & \code{seq}, \code{prod}, \code{aux}, identities & 527 \\
\code{MullinConjectures.lean} & MC, Conjecture A (FALSE), HH & 490 \\
\code{MullinDWH.lean} & DivisorWalkHypothesis (leaf) & 547 \\
\code{MullinResidueWalk.lean} & WalkCoverage, residue walk, concrete MC & 603 \\
\code{MullinGroupCore.lean} & walkZ, multZ, confinement, SE & 422 \\
\code{MullinGroupEscape.lean} & Escape lemmas, 8-element argument & 673 \\
\code{MullinGroupSEInstances.lean} & 29 concrete SE instances ($q \leq 157$) & 364 \\
\code{MullinGroupPumping.lean} & Gordon sequenceability (leaf) & 343 \\
\code{MullinGroupQR.lean} & QR obstruction ($\leq 1.6\%$) (leaf) & 683 \\
\code{MullinCRT.lean} & CRT multiplier invariance, walk recurrence & 160 \\
\code{MullinDepartureGraph.lean} & Departure graph, infinite recurrence, safe prime lattice & 393 \\
\code{RotorRouter.lean} & Scheduled walk coverage (standalone) & 421 \\
\code{MullinRotorBridge.lean} & EMPR + SE $\Rightarrow$ MC bridge & 87 \\
\code{EquidistPreamble.lean} & PE $\Rightarrow$ MC, bootstrapping & 234 \\
\code{EquidistSieve.lean} & Sieve, WHP $\Leftrightarrow$ HH, forbidden multiplier, M\"obius death & 755 \\
\code{EquidistSelfAvoidance.lean} & Self-avoidance, periodicity & 450 \\
\code{EquidistCharPRE.lean} & Character non-vanishing, PRE $\Leftrightarrow$ SE & 811 \\
\code{EquidistBootstrap.lean} & Inductive bootstrap, DH $\Rightarrow$ MC, first passage & 617 \\
\code{EquidistThreshold.lean} & Sieve gap, one-prime gap, cofinal pair avoidance & 325 \\
\code{EquidistOrbitAnalysis.lean} & Cofinal orbits, quotient walk, sieve, selectability & 1441 \\
\code{EquidistFourier.lean} & Character sums, Fourier bridge & 1298 \\
\code{EquidistSelfCorrecting.lean} & Decorrelation, BRE, telescoping, kernel (\S31--\S37, \S72) & 1163 \\
\code{EquidistSieveTransfer.lean} & Sieve transfer, coprimality, neg-inv involution (\S38--\S78) & 1457 \\
\code{CMEVariants.lean} & CME weaker variants (CME\_d, CME\_avg, CME\_subseq, CME\_target) & 113 \\
\code{LargeSieve.lean} & BV, ALS, ArithLS, MMCSB, sieve bridge (\S41--\S52, \S79) & 1812 \\
\code{LargeSieveHarmonic.lean} & Parseval, Gauss sums, DFT, kernel (\S53--\S55) & 892 \\
\code{LargeSieveAnalytic.lean} & Gauss inversion, WeakALS, GCT, dead ends (\S56--\S65, \S81--\S82) & 1683 \\
\code{LargeSieveSpectral.lean} & Walk energy, HOD, VdC, CME, VCB, SVE, transition matrix (\S66--\S86) & 2670 \\
\midrule
\code{IKCh1.lean} & Iwaniec--Kowalski~\cite{IwaniecKowalski2004} Ch.\,1: arithmetic functions & 437 \\
\code{IKCh2.lean} & Iwaniec--Kowalski Ch.\,2: summation formulas & 270 \\
\code{IKCh3.lean} & Iwaniec--Kowalski Ch.\,3: combinatorial sieve & 557 \\
\code{IKCh4.lean} & Iwaniec--Kowalski Ch.\,4: summation formulas & 593 \\
\code{IKCh5.lean} & Iwaniec--Kowalski Ch.\,5: Kloosterman sums & 877 \\
\code{IKCh7.lean} & Iwaniec--Kowalski Ch.\,7: bilinear forms, duality, Gram, MLS, sieve & 2269 \\
\bottomrule
\end{tabular}
\end{center}

\subsection{Axiom Usage: What's Constructive}

The core definitions (\code{seq}, \code{prod}, \code{aux}) and
their basic properties (\code{seq\_isPrime}, \code{seq\_injective})
are \textbf{fully constructive}: they use only \code{propext} and
\code{Quot.sound} (no \code{Classical.choice}, no
\code{Decidable} instances beyond~$\NN$).  Euclid's theorem
itself (\code{Euclid.lean}) is constructive.

Classical reasoning enters at the reduction level:
\begin{itemize}[nosep]
\item The $\HH \Rightarrow \MC$ proof uses well-founded induction
  (strong induction on $\NN$), which in Lean~4 is constructive but
  relies on \code{Classical.choice} for the
  cofinal-implies-hit argument.
\item Character theory (orthogonality, Fourier inversion) is
  inherently classical via \code{open Classical}.
\item All open hypotheses are stated as \code{def \ldots : Prop},
  never as \code{sorry}'d theorems.  The type-checker guarantees
  that no proof obligation is silently assumed.
\end{itemize}

\subsection{Mathlib Dependencies}

The formalization draws on several Mathlib libraries:
\begin{itemize}[nosep]
\item \textbf{Group theory}: \code{Subgroup}, \code{QuotientGroup},
  \code{orderOf}, cyclic group structure, maximal subgroups
  (\code{Subgroup.IsCoatom}).
\item \textbf{Number theory}: \code{Nat.minFac}, Legendre symbols,
  quadratic reciprocity, \code{ZMod}, Dirichlet characters, Gauss sums.
\item \textbf{Character theory}: \code{DirichletCharacter.Orthogonality},
  roots of unity in algebraically closed fields, character bounds,
  \code{MulChar.sum\_eq\_zero\_of\_ne\_one}.
\item \textbf{Analysis}: \code{norm\_sum\_le}, complex norms,
  \code{IsOfFinOrder.norm\_eq\_one}, Fourier analysis on $\ZZ/n\ZZ$
  (\code{ZMod.dft}, discrete Fourier transform).
\item \textbf{Dirichlet's theorem}: \code{Nat.infinite\_setOf\_prime\_and\_eq\_mod}
  (primes in arithmetic progressions, via $L$-series).
\item \textbf{Harmonic analysis}: Parseval's theorem for finite abelian
  groups, trigonometric exponentials, geometric series identities.
\end{itemize}

\begin{center}
\begin{tabular}{lr}
\toprule
Lines of Lean code & ${\sim}26{,}900$ \\
Files & 35 \\
Theorems/lemmas & ${\sim}910$ \\
Definitions & ${\sim}460$ \\
\code{sorry} occurrences & \textbf{0} \\
Open hypotheses (stated as \code{def}) & ${\sim}26$ \\
Mathlib version & v4.27.0 \\
\bottomrule
\end{tabular}
\end{center}

%% =========================================================================
\section{Open Problems}
\label{sec:open}
%% =========================================================================

The formalization certifies the reductions above and pinpoints the
frontier.  This section identifies the open problems whose resolution
would close Mullin's Conjecture.

\subsection{CCSB as the Precise Frontier}

The formalization identifies \textbf{ComplexCharSumBound} as the
irreducible analytic content.  The question:

\begin{center}
\emph{Are the walk character sums $\sum_{n<N} \chi(\walkZ{q}{n})$ bounded
$o(N)$ for every non-trivial $\chi$?}
\end{center}

CCSB is a single hypothesis that implies MC with no additional
conditions.  As shown in Section~\ref{sec:character}, CCSB rules out
permanent avoidance of~$-1$: the Fourier bridge forces the hit count
at~$-1$ to be $N/(q{-}1) + o(N)$, contradicting the zero hits that
the first missing prime's walk must achieve past the sieve gap.

The walk telescoping identities (Section~\ref{sec:character}) provide
precise structural constraints.  The identity
$\sum_n \chi(w(n))(\chi(m(n))-1) = O(1)$ means that the walk sum
$S_N$ and the multiplier sum $M_N = \sum_n \chi(m(n))$ satisfy
$S_N \approx S_N + (M_N - S_N) = M_N + O(1)$ only in the crude sense;
the telescoping does \emph{not} separate them.

\subsection{Connection to Bombieri--Vinogradov}

A Bombieri--Vinogradov type result for EM walk residues would give:
\[
\sum_{\substack{q \leq Q \\ q\text{ prime}}}
\max_{a}
\Bigl| |\{n \leq N : w(n) \equiv a \pmod{q}\}|
  - \frac{N}{q-1} \Bigr|
\;\ll\; \frac{NQ}{(\log N)^A}.
\]
For non-exceptional primes, the hit count at~$-1$ would be
$\sim N/(q{-}1) > 0$, contradicting permanent avoidance; the finitely
many exceptional primes below~$Q_0$ are handled by the inductive
bootstrap (\S\ref{sec:bootstrap}), closing the conjecture.

The difficulty is that BV applies to the set of \emph{all} primes,
not to a specific subsequence.  The EM walk is deterministic and
self-referential: the walk at step~$n$ depends on the factorization
of~$\Prod(n)+1$, which depends on all previous walk values.  Standard
BV does not apply.

\subsection{Connection to Chebotarev}

The \textbf{EffectiveKummerEscape} hypothesis asserts: for each
prime~$\ell$, there exists~$B$ such that for $q \geq B$ with
$\ell \mid q\!-\!1$, some multiplier among the first~$B$ escapes the
$\ell$-th power kernel.  This is a Chebotarev-type statement for the
Kummer extension $\mathbb{Q}(\zeta_\ell, 3^{1/\ell}, \ldots,
53^{1/\ell})$: the Frobenius at~$q$ determines which multiplier
primes are $\ell$-th power residues.

An effective Chebotarev density theorem for this fixed number field
would give EKE for all but finitely many~$q$ (effectively bounded).
Combined with finite verification for the remaining~$q$, this would
prove PRE and hence SE unconditionally---but SE is
\emph{already} proved unconditionally via the elementary PRE.
The Chebotarev approach would give a stronger \emph{effective}
bound on how quickly SE kicks in, refining the density argument of
\S\ref{sec:bootstrap}.

\subsection{The Sieve-Theoretic Approach}

\textbf{Mertens\-Escape}: for any prime~$q$ and proper
subgroup~$H$, infinitely many primes outside~$H$ exist
(Dirichlet content).
\textbf{Sieve\-Amplification}: Mertens escape should force
eventual $\minFac(\Prod(n){+}1)$ escape from~$H$, via
super-exponential growth and mutual coprimality of
successive Euclid numbers.

The formally verified chain:
$\mathrm{MertensEscape} + \mathrm{SieveAmplification}
\xrightarrow{\text{proved}}
\mathrm{TailSE} \xrightarrow{\text{proved}}
\mathrm{CofinalEscape} \xrightarrow{\text{proved}}
\mathrm{QuotientDH}$.

The formalization articulates a richer sieve infrastructure in two parallel
routes:

\paragraph{Cumulative route.}
\begin{gather*}
\mathrm{PDE}
\xrightarrow{\text{Alladi}}
\mathrm{GLPFE}
\xrightarrow{\mathrm{SieveTransfer}}
\mathrm{SieveEquidist}
\xrightarrow{\text{open}}
\mathrm{NoLongRuns} \\
\xrightarrow{\text{proved}}
\mathrm{PED}
\xrightarrow{\text{open}}
\mathrm{CCSB}
\xrightarrow{\text{proved}}
\mathrm{MC}.
\end{gather*}
Here PDE is PrimeDensityEquipartition (PNT in arithmetic progressions,
a known theorem not yet in Mathlib), and GLPFE is GenericLPFEquidist
(Alladi's theorem~\cite{Alladi1977} on $\minFac$ distribution of generic integers,
also known but not formalized).  Both ends of the chain---from PDE to
GLPFE via Alladi, and from CCSB to MC via Fourier inversion---are
formally proved.

\paragraph{Window route.}
\[
\mathrm{StrongSieveEquidist}
\xrightarrow{\text{proved}}
\mathrm{NoLongRunsAt}
\xrightarrow{\text{proved}}
\mathrm{PEDAt}
\xrightarrow{\text{open}}
\mathrm{CCSB}
\xrightarrow{\text{proved}}
\mathrm{MC}.
\]
StrongSieveEquidist asserts that EM multipliers are equidistributed
within sliding windows; NoLongRunsAt and PEDAt are per-prime variants
proved by pigeonhole and block-counting respectively
(\lean{EM/EquidistSieveTransfer.lean\#L185}{strongSieveEquidist\_noLongRunsAt},
\lean{EM/EquidistSieveTransfer.lean\#L254}{noLongRunsAt\_ped}).

\paragraph{The genuine frontier.}
\textbf{SieveTransfer} is the critical open hypothesis: does the
equidistribution of $\minFac$ residues for generic integers transfer
to the specific EM orbit?  Everything above SieveTransfer is known
mathematics; everything below it is proved.  SieveTransfer is where
``known but not formalized'' meets ``genuinely open.''

The difficulty: for \emph{generic} $q$-rough integers,
$\minFac$ residues are equidistributed (by CRT + Mertens)---the
death channel gets its fair share of multipliers.  But the EM
Euclid numbers are not generic: each is determined by the entire
walk history.  Transferring the generic equidistribution to this
specific orbit is the open step.

\paragraph{Sieve-to-harmonic convergence.}
The sieve hierarchy (\S36--\S39 of \code{EquidistSelfCorrecting.lean})
and the harmonic hierarchy (\S30--\S35) converge: both produce
DecorrelationHypothesis as output.  The full chain
\[
\mathrm{SieveEquidist}
\xrightarrow{\text{proved}}
\mathrm{Dec}
\xrightarrow{\text{proved}}
\mathrm{PED}
\xrightarrow[\text{sole gap}]{\text{open}}
\mathrm{CCSB}
\xrightarrow{\text{proved}}
\mathrm{MC}
\]
is formalized, with the first two arrows machine-verified
(\lean{EM/LargeSieve.lean\#L1634}{sieve\_equidist\_implies\_decorrelation},
\lean{EM/EquidistSelfCorrecting.lean\#L139}{decorrelation\_implies\_ped}).
The sieve route achieves SieveEquidist~$\Rightarrow$~Dec via a
counting-to-character-sum bridge: SieveEquidistribution produces
\code{EMMultCharSumBound} with $Q_0 = 0$, meaning multiplier character
sums cancel for \emph{all} primes~$q$, which is exactly
DecorrelationHypothesis.  The sole remaining gap on this route is
\textbf{PEDImpliesComplexCSB}
(\lean{EM/EquidistSelfCorrecting.lean\#L109}{PEDImpliesComplexCSB}):
does positive escape density for all primes imply walk character sum
cancellation?  Any proof of SieveEquidistribution (e.g., from PNT
in APs $+$ Alladi's theorem) would immediately yield Dec and PED
for free, isolating this single bridge as the only open step.

\subsection{What Would Close the Conjecture}

The cleanest paths to MC:
\begin{enumerate}
\item \textbf{Prove CME} (sharpest target): show that the multiplier
  character sum $\sum_{\substack{n < N \\ w(n) = c}} \chi(m(n))$ is
  $o(N)$ for each walk position~$c$.  CME is strictly weaker than
  CCSB, and $\mathrm{CME} \Rightarrow \mathrm{CCSB}$ is proved
  (\lean{EM/LargeSieveSpectral.lean}{cme\_implies\_ccsb}).
  CME asks only about the \emph{conditional} distribution of
  multipliers given walk state---it does not require controlling the
  walk character sum itself.  This bypasses the $d \geq 3$ barrier
  entirely.

\item \textbf{Prove CCSB directly}: show that no non-trivial
  character sum can sustain the $\Omega(N)$ bias that permanent
  avoidance of~$-1$ would require.  The self-correcting sieve
  (concentration of EM primes in a residue class is exponentially
  self-limiting) is the strongest heuristic argument.

\item \textbf{Prove a BV-type estimate}: even an averaged bound
  over~$q$ would suffice---for $q \geq Q_0$ it rules out permanent
  avoidance, and the finitely many $q < Q_0$ are handled by the
  inductive bootstrap (\S\ref{sec:bootstrap}).

\item \textbf{Close the orbit chain gap}: show that at each cofinally
  visited walk position, at least two distinct multiplier classes
  appear.  This would force the orbit chain to expand to the full group.

\item \textbf{Prove DH directly}: show that a multiplicative walk
  on a cyclic group with a generating set of multipliers must hit
  every element cofinally.  This is a combinatorial question about
  deterministic walks.

\item \textbf{Prove SieveTransfer}: show that the EM orbit's $\minFac$
  distribution matches that of generic integers, at least on average.
  The cumulative sieve route (\S38) reduces this to known number theory
  (PNT in APs + Alladi's theorem); closing SieveTransfer gives CME
  (and hence CCSB) via the conditional multiplier equidistribution
  framework.

\item \textbf{Prove BVImpliesMMCSB or PrimeArithLSImpliesMMCSB}: the
  large sieve route (\S41--\S65) reduces MC to a transfer hypothesis.
  Given Bombieri--Vinogradov or the analytic large sieve, the remaining
  open step is showing that the EM orbit's multipliers receive their
  fair share of each residue class---in particular, that the death
  channel is not systematically avoided.  This is the same
  orbit-specificity gap as SieveTransfer, approached from a different
  mathematical toolkit.
\end{enumerate}

\subsection{Does the Walk Hit $-1$?}

The irreducible open question is not whether DH holds (cofinal
hitting is a convenient sufficient condition) but whether the walk
hits~$-1$ \emph{at least once} past the sieve gap.  The Single Hit
Theorem (Theorem~\ref{thm:single-hit}) shows that a single hit at
each prime suffices for MC; DH, CCSB, and CME are strategies for
producing that hit.

\paragraph{Evidence for.}
Two independent lines of evidence suggest the walk does hit~$-1$.
(1)~\emph{Self-correcting feedback}: the formalized sieve analysis
(\code{EquidistSelfCorrecting.lean}) shows that concentration of EM
primes in a residue class is exponentially self-limiting---a walk
biased toward missing $-1$ automatically biases the multiplier
distribution toward correcting that miss.
(2)~\emph{Analogy with Artin's conjecture}: Artin's conjecture
(that every non-square integer is a primitive root for infinitely
many primes) has the same orbit-specificity structure and is believed
true; Hooley~\cite{Hooley1967} proved it conditional on GRH.  The
walk-hitting question is the analogous statement for the EM walk
and would follow from an analogous uniformity hypothesis.

\paragraph{Evidence against.}
Two features of the EM sequence give pause.
(1)~\emph{Cox--van der Poorten}~\cite{CoxVdP1968}: the ``largest factor'' variant of
the Euclid--Mullin sequence provably misses primes.  The EM
sequence's completeness is not a soft consequence of the Euclid
construction but depends sensitively on the $\minFac$ selection
rule.  This fragility means heuristic arguments
(``it should work because Euclid numbers have many factors'') are
not reliable.
(2)~\emph{The $d \geq 3$ barrier}: the formalization proves that
the most natural route from multiplier escape density to walk
character sum cancellation (PED~$\Rightarrow$~BRE~$\Rightarrow$~CCSB) is
\emph{impossible} for character orders $d \geq 3$.  This is not
evidence against the hit, but it shows that producing one
requires mechanisms beyond the simplest equidistribution
framework.  The CME bypass sidesteps this barrier, but the barrier's
existence means any proof must be genuinely subtle.

\paragraph{Assessment.}
We believe the walk does hit~$-1$ at every prime, primarily because
the self-correcting sieve mechanism provides a concrete dynamical
reason (not merely a probabilistic heuristic) for the walk to
eventually hit~$-1$.  The strongest form of this belief: CME should
hold because the EM multipliers, conditioned on walk position, have
no arithmetic reason to avoid the death channel systematically.
But we acknowledge that no existing technique can prove this, and
the $d \geq 3$ barrier shows that the proof, when found, will need
to exploit the specific structure of the EM walk in ways that
current analytic number theory does not.

%% =========================================================================
\section{Summary of Verified Results}
\label{sec:summary}
%% =========================================================================

\renewcommand{\arraystretch}{1.25}
\footnotesize
\begin{longtable}{@{}p{5.5cm}l>{\raggedright\arraybackslash}p{4.5cm}@{}}
\toprule
\textbf{Result} & \textbf{Status} & \textbf{Lean identifier} \\
\midrule
\endfirsthead
\toprule
\textbf{Result} & \textbf{Status} & \textbf{Lean identifier} \\
\midrule
\endhead
\midrule
\multicolumn{3}{r}{\emph{continued on next page}} \\
\endfoot
\bottomrule
\endlastfoot
\multicolumn{3}{l}{\emph{Sequence foundations}} \\
\quad Every $\seq(n)$ is prime & Proved & \code{seq\_isPrime} \\
\quad No prime repeats & Proved & \code{seq\_injective} \\
\midrule
\multicolumn{3}{l}{\emph{Main reductions to MC}} \\
\quad \textbf{SHH $\Rightarrow$ MC} (Single Hit Theorem) & Proved & \code{single\_hit\_implies\_mc} \\
\quad \textbf{DH $\Rightarrow$ MC} (cofinal hitting route) & Proved & \code{dynamical\_hitting\_implies\_mullin} \\
\quad \textbf{CCSB $\Rightarrow$ MC} (single hypothesis) & Proved & \code{complex\_csb\_mc'} \\
\quad PE $\Rightarrow$ MC & Proved & \code{pe\_implies\_mullin} \\
\quad HH $\Rightarrow$ MC & Proved & \code{hh\_implies\_mullin} \\
\quad SE + MH $\Rightarrow$ MC & Proved & \code{se\_mixing\_implies\_mullin} \\
\quad \textbf{WE $\Rightarrow$ MC} (single Prop) & Proved & \code{walk\_equidist\_mc} \\
\midrule
\multicolumn{3}{l}{\emph{Inductive bootstrap}} \\
\quad PrimeResidueEscape (elementary) & Proved & \code{prime\_residue\_escape} \\
\quad MC($<\!p$) + PRE $\Rightarrow$ SE($p$) & Proved & \code{mc\_below\_pre\_implies\_se} \\
\quad $q$-roughness from MC($<\!q$) & Proved & \code{mc\_below\_implies\_seq\_ge} \\
\quad One-prime gap & Proved & \code{mc\_below\_cofinal\_hit\_implies\_mc\_at} \\
\quad mc\_below 11 & Proved & \code{concrete\_mc\_below\_11} \\
\midrule
\multicolumn{3}{l}{\emph{Algebraic framework}} \\
\quad Confinement Theorem & Proved & \code{confinement\_forward/reverse} \\
\quad PRE $\Leftrightarrow$ SE & Proved & \code{pre\_iff\_se} \\
\quad SE $\Leftrightarrow$ character detection & Proved & \code{se\_iff\_char\_detection} \\
\quad Maximal subgroup reduction & Proved & \code{se\_of\_maximal\_escape} \\
\quad WHP $\Leftrightarrow$ HH & Proved & \code{whp\_iff\_hh} \\
\quad SE density argument (CRT + QR) & Proved & \code{se\_qr\_obstruction} \\
\midrule
\multicolumn{3}{l}{\emph{Character sum chain}} \\
\quad Fourier bridge: CCSB $\Rightarrow$ hit count lb & Proved & \code{complex\_csb\_implies\_hit\_count\_lb\_proved} \\
\quad Decorrelation $\Rightarrow$ PED & Proved & \code{decorrelation\_implies\_ped} \\
\quad NoLongRuns$(L)$ $\Rightarrow$ PED & Proved & \code{noLongRuns\_implies\_ped} \\
\quad BRE $\Rightarrow$ PEDImpliesCSB & Proved & \code{block\_rotation\_implies\_ped\_csb} \\
\quad CME $\Rightarrow$ CCSB (all $d$, bypasses BRE) & Proved & \code{cme\_implies\_ccsb} \\
\quad CME $\Rightarrow$ MC & Proved & \code{cme\_implies\_mc} \\
\quad Walk char recurrence ($\mathbb{C}$-valued) & Proved & \code{char\_walk\_recurrence} \\
\quad Telescoping identity & Proved & \code{walk\_telescope\_identity} \\
\quad Telescoping norm $\leq 2$ & Proved & \code{walk\_telescope\_norm\_bound} \\
\quad Shift-one autocorrelation & Proved & \code{walk\_shift\_one\_correlation} \\
\quad Order-2 sign-flip chain & Proved & \code{order2\_noLongRuns\_mc} \\
\midrule
\multicolumn{3}{l}{\emph{Walk dynamics}} \\
\quad Walk--divisibility bridge & Proved & \code{walkZ\_eq\_neg\_one\_iff} \\
\quad Products strictly monotone & Proved & \code{prod\_strictMono} \\
\quad Fundamental trichotomy & Proved & \code{avoidance\_contradicts\_se\_mixing} \\
\quad Self-avoidance dichotomy & Proved & \code{self\_avoidance\_dichotomy} \\
\quad Scheduled walk coverage & Proved & \code{scheduled\_walk\_covers\_all} \\
\midrule
\multicolumn{3}{l}{\emph{Selectability analysis}} \\
\quad Divisor freshness & Proved & \code{divisor\_not\_yet\_in\_seq} \\
\quad Passed-over persistence & Proved & \code{passed\_over\_persists} \\
\quad Selectability extinction & Proved & \code{selectability\_extinguished} \\
\quad MC $\Rightarrow$ $\neg$InfinitelySelectable & Proved & \code{mc\_implies\_not\_infinitely\_selectable} \\
\quad DH $\Rightarrow$ InfinitelySelectable & Proved & \code{dh\_implies\_infinitely\_selectable} \\
\midrule
\multicolumn{3}{l}{\emph{Sieve and orbit analysis}} \\
\quad EMFE $\Leftrightarrow$ TailSE & Proved & \code{emfe\_iff\_tail\_se\_at} \\
\quad TailSE $\Rightarrow$ CofinalEscape $\Rightarrow$ QuotientDH & Proved & \code{tail\_se\_gives\_sub\_dh} \\
\quad Dirichlet: $\infty$ primes per residue class & Proved & \code{dirichlet\_residues\_independent} \\
\quad Minimality sieve + coupling & Proved & \code{minimality\_sieve} \\
\quad StrongSieveEquidist $\Rightarrow$ NoLongRunsAt & Proved & \code{strongSieveEquidist\_noLongRunsAt} \\
\quad NoLongRunsAt $\Rightarrow$ PEDAt & Proved & \code{noLongRunsAt\_ped} \\
\quad DPED $\Rightarrow$ PED & Proved & \code{dped\_implies\_ped} \\
\quad PDE $+$ sieve chain $\Rightarrow$ MC & Proved & \code{primeDensity\_chain\_mc} \\
\quad GLPFE $+$ SieveTransfer $\Rightarrow$ MC & Proved & \code{genericLPF\_chain\_mc} \\
\quad SieveEquidist $\Rightarrow$ Dec & Proved & \code{sieve\_equidist\_implies\_decorrelation} \\
\quad SieveEquidist $\Rightarrow$ PED & Proved & \code{sieve\_equidist\_implies\_ped} \\
\midrule
\multicolumn{3}{l}{\emph{Large sieve route}} \\
\quad MultiModularCSB $\Rightarrow$ MC & Proved & \code{mmcsb\_implies\_mc} \\
\quad BV chain $\Rightarrow$ MC & Proved & \code{bv\_chain\_mc} \\
\quad ArithLS chain $\Rightarrow$ MC & Proved & \code{arith\_ls\_chain\_mc} \\
\quad ALS chain $\Rightarrow$ MC & Proved & \code{als\_prime\_arith\_ls\_chain\_mc} \\
\quad \textbf{WeakALS} (\S58) & \textbf{Proved} & \code{weak\_als\_from\_card\_bound} \\
\quad Gauss sum inversion (\S57) & Proved & \code{char\_sum\_to\_exp\_sum} \\
\quad \textbf{ALS $\Rightarrow$ PrimeArithLS} (\S65) & \textbf{Proved} & \code{als\_implies\_prime\_arith\_ls} \\
\quad Jordan's inequality (\S56) & Proved & \code{sin\_pi\_ge\_two\_mul} \\
\quad Geometric sum bound (\S56) & Proved & \code{norm\_eAN\_geom\_sum\_le\_inv} \\
\quad Parseval for ZMod.dft (\S53) & Proved & \code{zmod\_dft\_parseval} \\
\quad Gauss sum norm $\|\tau\|^2 = p$ (\S54) & Proved & \code{gaussSum\_norm\_sq\_eq\_prime} \\
\quad Walk autocorrelation identities (\S53) & Proved & \code{walkAutocorrelation\_*} \\
\quad Character Parseval (\S60) & Proved & \code{char\_parseval\_units} \\
\quad All 8 GCT internal lemmas (\S56--\S62) & Proved & \code{gct\_nontrivial\_char\_sum\_le} \\
\midrule
\multicolumn{3}{l}{\emph{Open hypotheses --- live targets}} \\
\quad \textbf{SingleHitHypothesis} (weakest sufficient) & \textbf{Open} & \code{SingleHitHypothesis} \\
\quad \textbf{DynamicalHitting} & \textbf{Open} & \code{DynamicalHitting} \\
\quad \textbf{ComplexCharSumBound} & \textbf{Open} & \code{ComplexCharSumBound} \\
\quad \textbf{MultiModularCSB} & \textbf{Open} & \code{MultiModularCSB} \\
\quad DecorrelationHypothesis & \textbf{Open} & \code{DecorrelationHypothesis} \\
\quad PositiveEscapeDensity & \textbf{Open} & \code{PositiveEscapeDensity} \\
\quad \textbf{PEDImpliesComplexCSB} (sole sieve-route gap) & \textbf{Open} & \code{PEDImpliesComplexCSB} \\
\quad NoLongRuns$(L)$ & \textbf{Open} & \code{NoLongRuns} \\
\quad SieveEquidistribution & \textbf{Open} & \code{SieveEquidistribution} \\
\quad MertensEscape & \textbf{Open} & \code{MertensEscape} \\
\quad SieveAmplification & \textbf{Open} & \code{SieveAmplification} \\
\quad \textbf{SieveTransfer} (genuine frontier) & \textbf{Open} & \code{SieveTransfer} \\
\quad StrongSieveEquidist & \textbf{Open} & \code{StrongSieveEquidist} \\
\quad DistributionalPED & \textbf{Open} & \code{DistributionalPED} \\
\quad \textbf{BVImpliesMMCSB} (genuine frontier) & \textbf{Open} & \code{BVImpliesMMCSB} \\
\quad GaussConductorTransfer (all lemmas proved) & \textbf{Open} & \code{GaussConductorTransfer} \\
\quad PrimeArithLSImpliesMMCSB & \textbf{Open} & \code{PrimeArithLSImpliesMMCSB} \\
\midrule
\multicolumn{3}{l}{\emph{Known theorems --- not yet in Mathlib}} \\
\quad PrimeDensityEquipartition (PNT in APs) & Known & \code{PrimeDensityEquipartition} \\
\quad GenericLPFEquidist (Alladi~\cite{Alladi1977}) & Known & \code{GenericLPFEquidist} \\
\quad BombieriVinogradov & Known & \code{BombieriVinogradov} \\
\quad AnalyticLargeSieve & Known & \code{AnalyticLargeSieve} \\
\quad ArithmeticLargeSieve & Known & \code{ArithmeticLargeSieve} \\
\midrule
\multicolumn{3}{l}{\emph{Dead ends --- false or blocked}} \\
\quad MultCSBImpliesMMCSB (false in general, \S\ref{sec:walk-bridge-false}) & \textbf{Dead} & \code{MultCSBImpliesMMCSB} \\
\quad BlockRotationEstimate (impossible for $d \geq 3$, Remark~\ref{rem:bre-impossible}) & \textbf{Dead} & \code{BlockRotationEstimate} \\
\end{longtable}
\renewcommand{\arraystretch}{1.0}
\normalsize

%% =========================================================================
\appendix

\section{History and Computational Status}
\label{app:history}
%% =========================================================================

Mullin posed the conjecture in 1963~\cite{Mullin1963}.  In over sixty
years, no proof has been found, and no theoretical approach has come
close.  The problem sits in an unusual position: it is elementary to
state, each individual step is deterministic, yet the global behavior
of the sequence appears completely intractable.

\paragraph{The largest-factor variant.}
Cox and van der Poorten~\cite{CoxVdP1968} showed that the related sequence using
the \emph{largest} prime factor---where each term is
$\operatorname{gpf}(\Prod(n) + 1)$ instead of $\minFac$---provably
misses infinitely many primes (for instance,~$5$ never appears).  By
always jumping to the largest factor of $\Prod(n) + 1$, the sequence
leaps past small primes and can never return to them, since each Euclid
number is coprime to every earlier term.  This refuted the natural
strengthening that surjectivity holds regardless of the factor-selection
rule, and showed that the $\minFac$ rule is essential to the conjecture.

The two selection rules are also structurally different.  The largest
prime factor $P^+(n) = \operatorname{gpf}(n)$ has rich algebraic
structure: the sets $\{n : P^+(n) \leq y\}$ (smooth numbers) are
controlled by the Dickman function, and for polynomial sequences
$f(n) = \Prod(n)+1$ one can sometimes exploit algebraic curves and
Hasse--Weil bounds.  In contrast, the smallest prime factor
$P^-(n) = \minFac(n)$ is purely a \emph{sieve} object: controlling
$P^-(n)$ requires excluding small prime divisors one at a time (CRT
$+$ Mertens), and no algebraic-geometric handle exists.  This
$\operatorname{gpf}$/$\minFac$ asymmetry is one reason the largest-factor
variant admits a short proof while the smallest-factor conjecture
remains open.

\paragraph{Variants and surveys.}
Booker~\cite{BoSa2012} showed that a carefully chosen variant of the
Euclid--Mullin sequence \emph{does} contain every prime: by selecting a
specific (not necessarily smallest) prime factor at each step, one can
steer the sequence to hit every prime.  This demonstrates that the
conjecture is \emph{delicate}: the surjectivity depends on the precise
rule, not just the Euclidean structure.  Pollack and
Trevi\~{n}o~\cite{PollackTrevino2014} surveyed the problem's place in
the broader landscape of Euclid-inspired sequences, and studied
distributional properties of primes ``forgotten'' by Euclid-type
constructions.

\paragraph{Computational status.}
The sequence has been extended through a series of large-scale
factoring efforts:
\begin{itemize}[nosep]
\item Wagstaff~(1993) computed through the 43rd term.
\item In 2010, the 180-digit number $\Prod(43) + 1$ was factored via
  GNFS (General Number Field Sieve), yielding a 68-digit prime as
  $a(44)$.  Terms $a(45)$--$a(47)$ followed.
\item In 2012, Propper factored the 256-digit number $\Prod(47)+1$ by
  ECM (Elliptic Curve Method), discovering a 75-digit factor and
  extending the sequence to 51~terms
  (Booker--Irvine~\cite{BoIr2016}).
\item Finding $a(52)$ requires factoring a 335-digit number.  No
  factorization is known as of 2025.
\end{itemize}
After 51~terms, the smallest primes not yet observed are $41$ and $47$.
Note that $31$---a smaller prime---does not appear until position~$50$.

\paragraph{Why computation cannot resolve the conjecture.}
Even heroic computation is fundamentally unable to address the
conjecture.  Each new term requires factoring a number whose digit
count grows roughly linearly with the number of terms, quickly
exceeding the reach of any known factoring algorithm.  But even if we
could compute millions of terms, this would prove nothing: the
conjecture is a $\forall$-statement over all primes, and no finite
computation can rule out the possibility that some prime first appears
at an astronomically large index.

More fundamentally, the sequence exhibits a sensitive dependence on its
full history.  Each term $a(n+1) = \minFac(\Prod(n)+1)$ depends on the
\emph{complete factorization} of a number that encodes all previous
terms.  Changing a single early term alters every subsequent one.  This
global coupling is what makes the sequence appear random despite being
deterministic, and it means that local or statistical reasoning
about ``typical'' behavior is unreliable.  There are no known density
arguments, probabilistic heuristics, or sieve-theoretic bounds that
bear on the conjecture.  The problem requires a structural argument
about the sequence's long-term dynamics---which is precisely what the
formalization in the body of this paper provides.

%% =========================================================================
\section{Analogies and Context}
\label{app:analogies}
%% =========================================================================

Mullin's Conjecture has no known applications: if proved tomorrow, no
other theorem in number theory would follow from it.  The value of the
problem lies instead in what it \emph{is an instance of}---and in the
methods its resolution would require.  The orbit-specificity barrier
identified by this formalization appears, in recognizable form, across
several active areas of mathematics.

\paragraph{Artin's conjecture and the orbit-specificity gap.}
The closest structural analogue to MC is Artin's conjecture on
primitive roots~\cite{Artin1927}: for any integer $a \neq -1$ that is not a
perfect square, the group $(\ZZ/p\ZZ)^\times$ is generated by~$a$ for
infinitely many primes~$p$.  The parallel is precise:
\begin{itemize}[nosep]
\item \textbf{Artin} asks whether the orbit of a fixed generator~$a$
  under repeated multiplication fills $(\ZZ/p\ZZ)^\times$.
\item \textbf{Mullin} asks whether the orbit of~$2$ under
  multiplication by successive EM primes in $(\ZZ/q\ZZ)^\times$ hits
  the single element~$-1$.
\end{itemize}
Both conjectures are blocked by the same fundamental obstacle:
transferring \emph{averaged} equidistribution results (which hold for
most moduli or most generators) to \emph{one specific deterministic
orbit}.  Hooley~\cite{Hooley1967} proved Artin's conjecture conditional on GRH
for Dedekind zeta functions of Kummer extensions---because GRH
provides the uniformity across individual characters needed to control
a single orbit.  In our setting, this uniformity is exactly what CCSB
demands.

The analogy is not merely structural.  The Kummer extensions
$\mathbb{Q}(\zeta_\ell, a^{1/\ell})$ that appear in Hooley's proof
are the same extensions that arise in the EffectiveKummerEscape
approach to SubgroupEscape (Section~\ref{sec:open}).  The elementary
PRE lemma (Theorem~\ref{thm:pre}) sidesteps this Chebotarev machinery
entirely for the algebraic component, but the dynamical
component---does the walk \emph{hit}~$-1$, not merely \emph{generate}
the full group?---remains exactly the orbit-specificity gap that GRH
closes for Artin and that no known tool closes for Mullin.

\paragraph{Multiplicative walks on finite groups.}
The walk reformulation (Section~\ref{sec:walk}) places MC in the
framework of random walks on finite groups, studied systematically by
Diaconis and others since the
1980s~\cite{ChungDiaconisGraham1987}.  The Diaconis--Shahshahani upper
bound lemma~\cite{DiaconisShahshahani1981} shows that a random walk on a finite group~$G$
driven by i.i.d.\ multipliers from a conjugation-invariant
distribution mixes in $O(\log |G|)$ steps, with mixing measured by
character sums.

The EM walk has the same algebraic structure---a multiplicative walk
on the cyclic group $(\ZZ/q\ZZ)^\times$---but violates every
assumption of the classical mixing theory.  The multipliers are
deterministic, not random; they are not identically distributed; and
most critically, the multiplier at step~$n$ depends on the walk
position at step~$n$, creating exactly the position-multiplier
correlation that the Diaconis--Shahshahani framework assumes away.
CCSB is the precise derandomization of the mixing-time bound: it asks
that the Fourier coefficients of the walk's occupation measure tend to
zero for every non-trivial character.

\paragraph{Smallest prime factor distribution.}
The SieveTransfer hypothesis (Section~\ref{sec:open}) connects MC to
the distribution of the smallest prime factor function
$P^-(n) = \min\{p : p \mid n\}$, studied by Alladi~\cite{Alladi1977},
Hildebrand~\cite{Hildebrand1986}, and others.  For generic integers in an arithmetic
progression $n \equiv a \pmod{q}$, the distribution of $P^-(n)$ is
controlled by the Dickman function and CRT-based equidistribution
results.  MC asks whether this equidistribution transfers from generic
integers to the specific subsequence $\{\Prod(n)+1\}_{n \geq 0}$.
The same orbit-specificity transfer problem arises for Mersenne
numbers $2^p - 1$, Fibonacci numbers, and polynomial iterates
$f^{\circ n}(a)$ in arithmetic dynamics.

\paragraph{The marginal/joint barrier and Sarnak's conjecture.}
The formalization identifies a precise meta-obstacle
(Section~\ref{sec:hard}): \emph{marginal} equidistribution of the
multiplier residues is provable (the EM primes are equidistributed in
residue classes, by Dirichlet's theorem); what MC requires is
\emph{joint} equidistribution of the pair (walk position, multiplier),
conditioned on the walk's history.  This barrier is an instance of a
broader phenomenon.  Sarnak's conjecture~\cite{Sarnak2010} asserts that the
M\"obius function $\mu(n)$ is orthogonal to every bounded
deterministic sequence:
$\frac{1}{N} \sum_{n \leq N} \mu(n) \, a_n \to 0$.
CCSB is a M\"obius-orthogonality-type statement for the EM sequence,
placing MC squarely within the Sarnak program's conceptual framework,
even though the EM sequence falls outside the technical scope of
existing results (which require zero topological entropy).

\paragraph{Greedy sieves and orbit-hitting.}
MC is the simplest nontrivial instance of a broader question: does a
greedy, deterministic prime-selection process eventually cover all
primes?  The Cox--van der Poorten result~\cite{CoxVdP1968} shows that this
determinism is fragile: choosing the \emph{largest} factor instead
provably misses primes.  More generally, MC belongs to the family of
\emph{orbit-hitting problems} in arithmetic dynamics: given a map~$T$
on a space~$X$ and a target set $S \subset X$, does the orbit
eventually enter~$S$?  Unlike Artin (where the map $x \mapsto ax$ is
the same at every step) or Collatz (where the map depends only on the
current state), the EM map varies at each step, determined by the
factorization of a number that depends on the entire orbit history.
This is the accumulator coupling of Section~\ref{sec:intro}: the
running product $\Prod(n)$ is a cumulative digest of the full orbit,
and the walk--multiplier framework makes it precise.  The
formalization shows it is the sole source of difficulty: once the
coupling is controlled (via CCSB or CME), MC follows by
machine-checked deduction.

%% =========================================================================
\section{Additional Sieve and Spectral Routes}
\label{app:routes}
%% =========================================================================

This appendix collects the sieve and spectral-energy routes to MC that
complement the three principal reductions (DH, CCSB, BV) presented in
the body.  All reduction arrows are machine-verified; the sole open
content in each route is the orbit-specificity transfer.

\subsection*{Arithmetic Large Sieve Route}

\begin{theorem}[\lean{EM/LargeSieve.lean\#L420}{arith\_ls\_chain\_mc}]
$\mathrm{ArithLS} + \mathrm{ArithLSImpliesMMCSB} \;\Longrightarrow\; \MC$.
\end{theorem}
The arithmetic large sieve gives character sum bounds for Dirichlet
characters (a known result, not in Mathlib).  The transfer
ArithLSImpliesMMCSB is open and is in fact a \textbf{dead
end}: universal coefficient bounds cannot distinguish equidistributed
walks from clumped walks.

\subsection*{Analytic Large Sieve Route}

The most developed route connects the analytic large sieve to MC via
Gauss sum inversion.

\begin{definition}[\defname{AnalyticLargeSieve} (\abbr{ALS})]
For well-separated points $\{\alpha_r\} \subset \mathbb{R}/\mathbb{Z}$
with $\min_{r \neq s} \|\alpha_r - \alpha_s\| \geq \delta$:
\[
\sum_r \left\|\sum_{n < N} a_n\, e(n\alpha_r)\right\|^2
\;\leq\; (N - 1 + \delta^{-1}) \sum_{n < N} \|a_n\|^2.
\]
\end{definition}

\begin{theorem}[\lean{EM/LargeSieveAnalytic.lean\#L490}{weak\_als\_from\_card\_bound}]
\label{thm:weak-als}
A \emph{weak} version with constant $N \cdot (\delta^{-1} + 1)$ is
proved (the optimal constant is $N - 1 + \delta^{-1}$, but the
difference is immaterial since MMCSB requires only~$o(N)$).
\end{theorem}

The key bridge is \defname{Gauss sum inversion}: a Gauss sum
$\tau(\chi) = \sum_{a} \chi(a)\, e(a/p)$ intertwines multiplication
and addition on~$\ZZ/p\ZZ$, converting character sums to exponential
sums.

\begin{theorem}[\lean{EM/LargeSieveAnalytic.lean\#L304}{char\_sum\_to\_exp\_sum}]
\label{thm:gauss-inversion}
For a non-trivial character~$\chi$ mod~$p$ prime:
$\sum_n f(n)\, \chi(n)
= \tau^{-1} \sum_{b=1}^{p-1} \chi^{-1}(b) \sum_n f(n)\, \psi(bn)$.
\end{theorem}

The GaussConductorTransfer composes eight internal lemmas (all proved,
\S56--\S62) into the bridge from ALS to the prime arithmetic large sieve:

\begin{theorem}[\lean{EM/LargeSieveAnalytic.lean\#L1358}{als\_implies\_prime\_arith\_ls}]
\label{thm:als-prime}
$\mathrm{AnalyticLargeSieve} \;\Longrightarrow\; \mathrm{PrimeArithLS}$.
\end{theorem}

\begin{theorem}[\lean{EM/LargeSieveAnalytic.lean\#L1535}{als\_prime\_arith\_ls\_chain\_mc}]
$\mathrm{ALS} + \mathrm{PrimeArithLSImpliesMMCSB} \;\Longrightarrow\; \MC$.
\end{theorem}

The remaining open content is PrimeArithLSImpliesMMCSB: transferring
prime-modulus arithmetic large sieve bounds to multi-modular character
sum bounds for the specific EM orbit.

\subsection*{The Spectral Energy Route}
\label{sec:sve}

Instead of individual character sums, this route examines the
\emph{total energy} of the walk occupation measure
$V_N(a) = |\{n < N : \walkZ{q}{n} = a\}|$.

\begin{theorem}[\lean{EM/LargeSieveSpectral.lean\#L89}{walk\_energy\_parseval}]
\label{thm:walk-energy}
$\sum_\chi \|S_\chi(N)\|^2 = (q{-}1) \sum_{a \in (\ZZ/q\ZZ)^\times} V_N(a)^2$
\quad (Parseval).
\end{theorem}

The \emph{excess energy} $E(N) = \sum_{\chi \neq 1} \|S_\chi(N)\|^2$.
If $E(N) = o(N^2)$, then every non-trivial character sum is
individually~$o(N)$, which is CCSB.

\begin{definition}[\defname{SubquadraticVisitEnergy} (\abbr{SVE})]
For every missing prime~$q$ and $\varepsilon > 0$, there exists
$N_0$ such that for $N \geq N_0$: $E(N) \leq \varepsilon N^2$.
\end{definition}

\begin{theorem}[\lean{EM/LargeSieveSpectral.lean\#L240}{sve\_implies\_mmcsb}]
\label{thm:sve-mc}
$\mathrm{SVE} \Longrightarrow \mathrm{MMCSB} \Longrightarrow \MC$.
\end{theorem}

\paragraph{Van der Corput and higher-order decorrelation.}
The van der Corput inequality bounds $|\sum z_n|$ via autocorrelations:

\begin{theorem}[\lean{EM/LargeSieveSpectral.lean\#L591}{vanDerCorputBound}]
\label{thm:vdc}
$\left\|\sum_{n \leq N} z_n\right\|^2
\leq \frac{N + H}{H+1}\bigl(N + 2\sum_{h=1}^{H}
  |\mathrm{Re}\sum_{n \leq N-h} z_n \overline{z_{n+h}}|\bigr)$.
\end{theorem}

For the EM walk, lag-$h$ autocorrelations involve $h$-step multiplier
products.  At $h = 1$, the autocorrelation equals the multiplier
character sum (Theorem~\ref{thm:shift-one}), so VdC with a single
shift gives only $O(N)$.  Higher lags may decorrelate:

\begin{definition}[\defname{HigherOrderDecorrelation} (\abbr{HOD})]
For every missing prime~$q$, non-trivial~$\chi$, and $\varepsilon > 0$:
there exists $H_0$ such that for $H \geq H_0$, $N_0$ such that for
$N \geq N_0$ and all $1 \leq h \leq H$:
$\|R_h(N)\| \leq \varepsilon N$.
\end{definition}

\begin{theorem}[\lean{EM/LargeSieveSpectral.lean\#L1014}{hod\_implies\_ccsb}]
\label{thm:hod-mc}
$\mathrm{HOD} \Longrightarrow \mathrm{CCSB} \Longrightarrow \MC$.
\end{theorem}

\paragraph{Conditional multiplier equidistribution.}

\begin{definition}[\defname{ConditionalMultiplierEquidist} (\abbr{CME})]
For every missing prime~$q$, non-trivial~$\chi$, $\varepsilon > 0$,
$N_0$ such that for $N \geq N_0$ and every
$c \in (\ZZ/q\ZZ)^\times$:
$\|\sum_{\substack{n < N \\ w(n) = c}} \chi(m(n))\| \leq \varepsilon N$.
\end{definition}

\begin{theorem}[\lean{EM/LargeSieveSpectral.lean}{cme\_implies\_dec}]
$\mathrm{CME} \Longrightarrow \mathrm{DecorrelationHypothesis}$.
\end{theorem}

\begin{theorem}[\lean{EM/LargeSieveSpectral.lean}{cme\_implies\_ccsb}]
\label{thm:cme-ccsb}
$\mathrm{CME} \Longrightarrow \mathrm{CCSB}$.
\end{theorem}

\begin{proof}[Proof sketch]
The walk telescoping identity gives
$\sum \chi(w(n)) = \sum \chi(w(n))\chi(m(n)) - (\chi(w(N)) - \chi(w(0)))$.
The product sum decomposes by fiber:
$\sum \chi(w(n))\chi(m(n)) = \sum_a \chi(a) \cdot \sum_{w(n)=a} \chi(m(n))$.
CME bounds each fiber sum by $\varepsilon' N$; the triangle inequality
sums over at most $|(\ZZ/q\ZZ)^\times|$ fibers; the boundary term
$\chi(w(N)) - \chi(w(0))$ has norm~$\leq 2$ and is absorbed for large~$N$.
\end{proof}

This is the key reduction that bypasses PED, BRE, and the $d \geq 3$
barrier.  The proof works for all character orders because it uses
only the fiber decomposition and telescoping---no block rotation
estimate is needed.

\begin{theorem}[\lean{EM/LargeSieveSpectral.lean}{cme\_implies\_mc}]
$\mathrm{CME} \Longrightarrow \MC$.
\end{theorem}

\begin{proof}
Compose \texttt{cme\_implies\_ccsb} with \texttt{complex\_csb\_mc'}.
\end{proof}

\begin{theorem}[\lean{EM/LargeSieveSpectral.lean}{cme\_chain\_mc}]
$\mathrm{CME} + \mathrm{PEDImpliesCSB} \Longrightarrow \MC$.
\end{theorem}

This older route through the Dec $\to$ PED $\to$ CCSB chain is
superseded by the direct CME $\to$ CCSB reduction above, which
requires no additional hypotheses.

\subsection*{The Complete Hypothesis Hierarchy}

\[
\mathrm{PED} \;<\; \mathrm{Dec} \;<\; \mathrm{CME}
\;\xrightarrow{\text{proved}}\;
\mathrm{CCSB} \;\approx\; \mathrm{HOD}
\;\approx\; \mathrm{SVE},
\]
where ``$<$'' means strictly weaker (proved implication, known not
to reverse) and ``$\approx$'' means equivalent.
HOD~$\Leftrightarrow$~CCSB via van der Corput;
SVE~$\Leftrightarrow$~CCSB via Parseval;
CME~$\Rightarrow$~CCSB via telescoping + fiber decomposition
(Theorem~\ref{thm:cme-ccsb}).

Every hypothesis implies MC.  The PED route has an open BRE bridge
for $d \geq 3$ characters, but this is now bypassed: the direct
CME~$\to$~CCSB arrow is proved for all character orders.  CME is the
\emph{sharpest sufficient condition}---the weakest hypothesis known
to imply MC.

%% =========================================================================
\section{Methodology: Human--AI Collaboration}
\label{app:agents}
%% =========================================================================

This work was produced through a sustained collaboration between a
human author and an AI system (Claude, Anthropic) across 72+~sessions.
The human author directed the mathematical strategy---proof
architecture, dead-end identification, and editorial control---while
the AI system handled Lean~4 formalization, Mathlib API search,
literature scouting, and exploration of candidate proof strategies.

The interaction was organized at scale via an \emph{agent swarm}:
a multi-agent system built on the Claude Agent SDK\@.  The swarm comprises
seven specialized agents, each with its own system prompt, tool access,
and model:

\begin{itemize}[nosep]
\item A \emph{coordinator} that reads the current proof state,
  selects the most promising action, dispatches specialists, and updates
  shared state files.
\item A \emph{formalizer} that writes and compiles Lean code
  in rapid iteration cycles.
\item A \emph{literature scout} that searches papers and Mathlib
  for relevant results.
\item Four \emph{attack vector specialists} focused on analytic,
  algebraic, combinatorial, and information-theoretic approaches.
\item A \emph{paper writer} that maintains this document.
\end{itemize}

Agent prompts are \emph{self-evolving}: after each session the
coordinator updates them to record dead ends, new Mathlib discoveries,
and shifted priorities.  This prevents agents from rediscovering settled
territory.  All agent state (progress, strategy log, findings) is stored
as git-tracked markdown, making the exploration history fully
reproducible.

The division of labor between human and AI was sharp:

\begin{itemize}[nosep]
\item \textbf{Human:} mathematical direction, proof strategy,
  identification of dead ends, evaluation of intermediate results,
  architectural decisions on the reduction hierarchy, and editorial
  control over the final formalization and paper.
\item \textbf{AI (Claude):} Lean~4 formalization using Mathlib,
  Mathlib API search, literature scouting, exploration of candidate proof
  strategies, and drafting of this paper.
\end{itemize}

The human author guided the proof effort across 72+~sessions, suggesting
attack vectors (algebraic, analytic, combinatorial, sieve-theoretic),
identifying when an approach had reached a dead end, and pushing toward
the sharpest possible reductions.  The AI agents wrote all Lean code,
searched Mathlib for relevant lemmas, explored dozens of proof strategies
to completion or refutation, and maintained the evolving paper.

The swarm is optimized for formalization and reduction, not mathematical
discovery.  The next breakthrough, if it comes, will probably be a human
insight about the structure of $\minFac$ on EM products---not something
an agent finds by systematic search.

%% =========================================================================
\section{Glossary of Definitions and Hypotheses}
\label{sec:glossary}
%% =========================================================================

The table below collects every named definition, hypothesis, and key
theorem introduced in this paper, with abbreviations and the section
where each is defined.

\medskip
\renewcommand{\arraystretch}{1.25}
\footnotesize
\begin{longtable}{@{}l l p{7.2cm} l@{}}
\toprule
\textbf{Abbr.} & \textbf{Name} & \textbf{Meaning} & \textbf{Ref.} \\
\midrule
\endfirsthead
\toprule
\textbf{Abbr.} & \textbf{Name} & \textbf{Meaning} & \textbf{Ref.} \\
\midrule
\endhead
\midrule
\multicolumn{4}{r}{\emph{continued on next page}} \\
\endfoot
\bottomrule
\endlastfoot

\multicolumn{4}{l}{\emph{Core sequence and walk}} \\
--- & Walk / Multiplier
    & $\walkZ{q}{n} = \Prod(n) \bmod q$;\; $\multZ{q}{n} = \seq(n\!+\!1) \bmod q$
    & Def.~\ref{def:walk-mult} \\
\abbr{MC} & MullinConjecture
    & Every prime appears in the Euclid--Mullin sequence
    & Conj.~\ref{conj:mullin} \\
\midrule

\multicolumn{4}{l}{\emph{Algebraic hypotheses (\S\ref{sec:walk}--\S\ref{sec:bootstrap})}} \\
\abbr{SE} & SubgroupEscape
    & No proper subgroup of $(\ZZ/q\ZZ)^\times$ contains all multipliers
    & Def.~\ref{def:se} \\
\abbr{HH} & HittingHypothesis
    & The walk reaches $-1$ cofinally: $\forall N,\,\exists n \geq N,\; q \mid \Prod(n)+1$
    & Def.~\ref{def:hh} \\
\abbr{DH} & DynamicalHitting
    & $\SE(q) \Rightarrow \HH(q)$ for every missing prime~$q$
    & Def.~\ref{def:dh} \\
\abbr{SHH} & SingleHitHypothesis
    & $\MC({<}\,q) + \SE(q) + q$ missing $\Rightarrow$ $\exists\, n \geq N_0$
      with $q \mid \Prod(n)+1$
    & Def.~\ref{def:shh} \\
\abbr{PRE} & PrimeResidueEscape
    & Every proper subgroup of $(\ZZ/p\ZZ)^\times$ is escaped by some odd prime $< p$
    & Thm.~\ref{thm:pre} \\
$\abbr{PRE}_\ell$ & PowerResidueEscape
    & Multipliers escape the index-$\ell$ subgroup of $(\ZZ/q\ZZ)^\times$
    & \S\ref{sec:bootstrap} \\
--- & ThresholdHitting
    & DH restricted to primes $q \geq B$
    & \S\ref{sec:large-sieve} \\
\midrule

\multicolumn{4}{l}{\emph{Character-analytic hypotheses (\S\ref{sec:character})}} \\
\abbr{CCSB} & ComplexCharSumBound
    & Walk char sums $S_\chi(N) = o(N)$ for all non-trivial $\chi$
    & \S\ref{sec:character} \\
\abbr{MMCSB} & MultiModularCSB
    & CCSB simultaneously for all primes $q$ in a range
    & \S\ref{sec:character} \\
\abbr{ALS} & AnalyticLargeSieve
    & Large sieve inequality adapted to EM walk
    & \S\ref{sec:character} \\
\abbr{PED} & PositiveEscapeDensity
    & Positive density of $n$ with $\multZ{q}{n} \notin H$, for every proper $H$
    & \S\ref{sec:character} \\
--- & DecorrelationHypothesis
    & $\multZ{q}{n}$ and $\multZ{q}{n{+}1}$ are asymptotically independent
    & \S\ref{sec:character} \\
\abbr{BRE} & BlockRotationEstimate
    & Cancellation in block sums of characters applied to walk
    & \S\ref{sec:character} \\
\abbr{SVE} & SubquadraticVisitEnergy
    & $\sum_{a} |\{n \leq N : \walkZ{q}{n}=a\}|^2 = o(N^2/(q{-}1))$
    & \S\ref{sec:character} \\
\abbr{HOD} & HigherOrderDecorrelation
    & Higher-order correlation bounds for walk increments
    & \S\ref{sec:character} \\
\abbr{CME} & ConditionalMultiplierEquidist
    & Conditional equidist.\ of multipliers given walk state; implies CCSB (proved)
    & \S\ref{sec:character} \\
\midrule

\multicolumn{4}{l}{\emph{Named theorems}} \\
--- & Confinement
    & If SE fails, the walk is confined to a proper coset
    & Thm.~\ref{thm:confinement} \\
--- & Walk--Divisibility Bridge
    & $\walkZ{q}{n} = -1 \Leftrightarrow q \mid \Prod(n)+1$
    & Thm.~\ref{thm:bridge} \\
--- & One-prime gap
    & $\MC({<}\,q)$ + cofinal hit $\Rightarrow$ $\MC(q)$
    & Thm.~\ref{thm:one-prime-gap} \\
--- & QR Obstruction
    & SE failure has density $O(2^{-k})$ by CRT + quadratic reciprocity
    & \S\ref{sec:bootstrap} \\
--- & Gauss sum inversion
    & Character sums $\leftrightarrow$ exponential sums via Gauss sums
    & \S\ref{sec:character} \\
\end{longtable}
\renewcommand{\arraystretch}{1.0}
\normalsize

%% =========================================================================
\newpage
\begin{thebibliography}{99}

\bibitem{Mullin1963}
A.\,A.~Mullin.
\newblock Recursive function theory (a modern look at a Euclidean idea).
\newblock \emph{Bull.\ Amer.\ Math.\ Soc.}, 69:737, 1963.

\bibitem{BoIr2016}
A.\,R.~Booker and S.\,A.~Irvine.
\newblock The {E}uclid--{M}ullin graph.
\newblock \emph{J.\ Number Theory}, 165:30--57, 2016.

\bibitem{BoSa2012}
A.\,R.~Booker.
\newblock A variant of the {E}uclid--{M}ullin sequence containing every prime.
\newblock \emph{J.\ Integer Sequences}, 19:Article 16.6.4, 2016.

\bibitem{Gordon1961}
B.~Gordon.
\newblock Sequences in groups with distinct partial products.
\newblock \emph{Pacific J.\ Math.}, 11(4):1309--1313, 1961.

\bibitem{PollackTrevino2014}
P.~Pollack and E.~Trevi{\~n}o.
\newblock The primes that {E}uclid forgot.
\newblock \emph{Amer.\ Math.\ Monthly}, 121(5):433--437, 2014.

\bibitem{Hardy2009}
M.~Hardy and C.~Woodgold.
\newblock Prime simplicity.
\newblock \emph{Math.\ Intelligencer}, 31:44--52, 2009.

\bibitem{Hooley1967}
C.~Hooley.
\newblock On {A}rtin's conjecture.
\newblock \emph{J.\ Reine Angew.\ Math.}, 225:209--220, 1967.

\bibitem{LagariasOdlyzko1977}
J.\,C.~Lagarias and A.\,M.~Odlyzko.
\newblock Effective versions of the {C}hebotarev density theorem.
\newblock In \emph{Algebraic Number Fields ({D}urham Symposium)},
  pages 409--464. Academic Press, 1977.

\bibitem{Alladi1977}
K.~Alladi.
\newblock On the distribution of the largest prime factor.
\newblock \emph{Stud.\ Sci.\ Math.\ Hungar.}, 12:1--9, 1977.

\bibitem{Hildebrand1986}
A.~Hildebrand.
\newblock On the number of positive integers $\leq x$ and free of
  prime factors $> y$.
\newblock \emph{J.\ Number Theory}, 22(3):289--307, 1986.

\bibitem{CoxVdP1968}
C.\,D.~Cox and A.\,J.~van~der~Poorten.
\newblock On a sequence of prime numbers.
\newblock \emph{J.\ Austral.\ Math.\ Soc.}, 8:571--574, 1968.

\bibitem{IwaniecKowalski2004}
H.~Iwaniec and E.~Kowalski.
\newblock \emph{Analytic Number Theory}.
\newblock Amer.\ Math.\ Soc.\ Colloq.\ Publ., vol.~53, 2004.

\bibitem{DiaconisShahshahani1981}
P.~Diaconis and M.~Shahshahani.
\newblock Generating a random permutation with random transpositions.
\newblock \emph{Z.\ Wahrsch.\ Verw.\ Gebiete}, 57:159--179, 1981.

\bibitem{ChungDiaconisGraham1987}
F.\,R.\,K.~Chung, P.~Diaconis, and R.\,L.~Graham.
\newblock Random walks arising in random number generation.
\newblock \emph{Ann.\ Probab.}, 15(3):1148--1165, 1987.

\bibitem{Sarnak2010}
P.~Sarnak.
\newblock Three lectures on the {M}\"obius function, randomness,
  and dynamics.
\newblock Lecture notes, IAS, 2010.
\newblock Available at \url{https://publications.ias.edu/sarnak/paper/512}.

\bibitem{Artin1927}
E.~Artin.
\newblock Beweis des allgemeinen {R}eziprozit\"atsgesetzes.
\newblock \emph{Abh.\ Math.\ Semin.\ Univ.\ Hambg.}, 5:353--363, 1927.

\bibitem{Odoni1985}
R.\,W.\,K.~Odoni.
\newblock On the prime divisors of the sequence $w_{n+1} = 1 + w_1 \cdots w_n$.
\newblock \emph{J.\ London Math.\ Soc.~(2)}, 32(1):1--11, 1985.

\end{thebibliography}

\end{document}
