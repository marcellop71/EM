\documentclass[11pt,a4paper]{article}

\usepackage[utf8]{inputenc}
\usepackage[T1]{fontenc}
\usepackage{amsmath,amssymb,amsthm}
\usepackage{mathtools}
\usepackage{hyperref}
\usepackage{enumitem}
\usepackage{array}
\usepackage{booktabs}
\usepackage{longtable}
\usepackage{tikz-cd}
\usepackage[margin=1in]{geometry}
\usepackage{scalerel}

% Inline code: sans-serif, small, dark teal, with breakable underscores.
\definecolor{codeink}{HTML}{1B6B6D}
\newcommand{\code}[1]{%
  {\color{codeink}\sffamily\small #1}%
}

% Lean 4 source links for formally verified results.
% Set \leanurl to the repository blob URL for clickable source references.
\newcommand{\leanurl}{https://github.com/mparis/EM/blob/main}
\definecolor{leanink}{HTML}{4338CA}
\newcommand{\lean}[2]{%
  \texorpdfstring{%
    \href{\leanurl/#1}{%
      {\color{leanink}\sffamily\footnotesize%
        \scalerel*{\checkmark}{X}\kern0.12em#2}}%
  }{#2}%
}

% Allow line breaks at underscores everywhere (tables, \code, etc.).
\renewcommand{\_}{\textunderscore\hspace{0pt}}

% Named definitions/theorems: small-caps for names, bold small-caps for abbreviations.
\definecolor{defink}{HTML}{7C3AED}
\newcommand{\defname}[1]{\texorpdfstring{{\color{defink}\textsc{#1}}}{#1}}
\newcommand{\abbr}[1]{\texorpdfstring{{\color{defink}\textsc{\textbf{#1}}}}{#1}}

\theoremstyle{plain}
\newtheorem{theorem}{Theorem}[section]
\newtheorem{lemma}[theorem]{Lemma}
\newtheorem{proposition}[theorem]{Proposition}
\newtheorem{corollary}[theorem]{Corollary}
\newtheorem{conjecture}[theorem]{Conjecture}
\theoremstyle{definition}
\newtheorem{definition}[theorem]{Definition}
\newtheorem{example}[theorem]{Example}
\theoremstyle{remark}
\newtheorem{remark}[theorem]{Remark}

\newcommand{\ZZ}{\mathbb{Z}}
\newcommand{\NN}{\mathbb{N}}
\newcommand{\FF}{\mathbb{F}}
\newcommand{\seq}{\mathrm{seq}}
\newcommand{\Prod}{\mathrm{prod}}
\newcommand{\minFac}{\mathrm{minFac}}
\newcommand{\walkZ}{\mathrm{walkZ}}
\newcommand{\multZ}{\mathrm{multZ}}
\newcommand{\SE}{\mathrm{SE}}
\newcommand{\MH}{\mathrm{MH}}
\newcommand{\HH}{\mathrm{HH}}
\renewcommand{\DH}{\mathrm{DH}}
\newcommand{\MC}{\mathrm{MC}}
\newcommand{\PE}{\mathrm{PE}}
\newcommand{\ME}{\mathrm{ME}}
\newcommand{\EMPR}{\mathrm{EMPR}}
\newcommand{\WHP}{\mathrm{WHP}}
\newcommand{\EKE}{\mathrm{EKE}}
\newcommand{\PRE}{\mathrm{PRE}}

\title{The Irreducible Core of Mullin's Conjecture:\\
A Machine-Verified Reduction}

\author{Marcello Paris}
\date{February 2026}

\begin{document}

\maketitle

\begin{abstract}
The Euclid--Mullin sequence is defined by $a(0)=2$,
$a(n{+}1) = \text{lpf}(a(0)\cdots a(n)+1)$, where $\text{lpf}$ is
the least prime factor.
Mullin's Conjecture (MC, 1963) asserts that every prime eventually
appears.  We present a Lean~4 formalization
(${\sim}22{,}400$~lines, \textbf{zero sorry}) that reduces MC to a
single open hypothesis.

An \emph{inductive bootstrap} eliminates the algebraic half: we prove
that for any prime $p \geq 5$, some odd prime $r < p$ escapes every
proper subgroup of $(\ZZ/p\ZZ)^\times$, using only modular
arithmetic.  A \emph{Fourier bridge} handles the analytic half: MC
follows whenever certain walk character sums are $o(N)$.  Together
these yield \textbf{DynamicalHitting}~(DH)~$\Rightarrow$~MC\@: if a
multiplicative walk whose multipliers generate
$(\ZZ/q\ZZ)^\times$ hits every element cofinally, MC holds.

Multiple reduction routes---algebraic, character-analytic,
sieve-theoretic---all converge on the same
\emph{orbit-specificity gap}: transferring generic equidistribution
to one deterministic orbit.  The sharpest sufficient condition is
Conditional Multiplier Equidistribution~(CME), proved to imply the
Complex Character Sum Bound~(CCSB) for all character orders,
bypassing the $d \geq 3$ barrier.  Twelve dead ends are documented.

The formalization comprises ${\sim}770$~theorems and
${\sim}460$~definitions across 32~files.
\end{abstract}

\newpage
\tableofcontents

\newpage

%% =========================================================================
\section{Introduction}
\label{sec:intro}
%% =========================================================================

\subsection{Mullin's Conjecture}

Euclid's proposition IX.20 of the \emph{Elements} shows that for any
finite set of primes, each prime factor of their product plus one is outside the
set: to grow your set of primes, you can pick any of them. The \textbf{Euclid--Mullin
sequence} (OEIS A000945), introduced by Mullin~\cite{Mullin1963}, makes a
definite choice: always take the \emph{smallest} prime factor.
\begin{align}
  a(0) &= 2, &
  a(n+1) &= \text{smallest prime factor of } \bigl(a(0) \cdots a(n) + 1\bigr).
  \label{eq:em-def}
\end{align}
The first twenty terms (0-indexed) are
\begin{gather*}
  \underbrace{2}_{a(0)},\;
  3,\; 7,\; 43,\; 13,\; 53,\;
  \underbrace{5}_{a(6)},\;
  \underbrace{6221671}_{a(7)},\;
  38709183810571, \\
  139,\; 2801,\;
  \underbrace{11}_{a(11)},\;
  17,\; 5471,\; 52662739,\; 23003,\;
  30693651606209, \\
  \underbrace{37}_{a(17)},\;
  1741,\;
  \underbrace{1313797957}_{a(19)},\; \ldots
\end{gather*}
The sequence behaves almost randomly: small primes appear out of their
natural order ($5$ not until position~$6$, $11$ at position~$11$,
$37$ at position~$17$), while enormous primes---a 7-digit number at
position~$7$, a 14-digit number at position~$8$---appear early.  As of
2025, 51~terms are known.  Remarkably, $41$ and $47$---the two smallest
primes not yet observed---have not appeared even after 51~terms, while
$31$ does not show up until position~50.
By construction, no prime can appear twice.

\begin{conjecture}[Mullin, 1963]\label{conj:mullin}
Every prime number eventually appears in the Euclid--Mullin sequence.
\end{conjecture}

The conjecture has resisted proof for over sixty years.  The difficulty
is showing that the deterministic $\minFac$ rule eventually \emph{selects} each prime.
Each step couples the next prime to the full factorization history, creating a recursive dependency that
defeats both probabilistic heuristics and standard sieve methods.

The crux of the difficulty is a \emph{selectability} problem.  At each
step, $\Prod(n)+1$ has many prime factors, and any of them could serve
as the next term---but only the \emph{smallest} is chosen.  A target
prime~$q$ may divide $\Prod(n)+1$ infinitely often (it is
``selectable'') yet never be selected if a smaller prime always divides
$\Prod(n)+1$ as well.  Our formalization makes this tension precise and
shows that the inductive structure of MC eliminates this obstruction for
the tail of the sequence.

\subsection{History and Computational Status}

Mullin posed the conjecture in 1963~\cite{Mullin1963}.  In over sixty
years, no proof has been found, and no theoretical approach has come
close.  The problem sits in an unusual position: it is elementary to
state, each individual step is deterministic, yet the global behavior
of the sequence appears completely intractable.

\paragraph{The largest-factor variant.}
Cox and van der Poorten~\cite{CoxVdP1968} showed that the related sequence using
the \emph{largest} prime factor---where each term is
$\operatorname{gpf}(\Prod(n) + 1)$ instead of $\minFac$---provably
misses infinitely many primes (for instance,~$5$ never appears).  By
always jumping to the largest factor of $\Prod(n) + 1$, the sequence
leaps past small primes and can never return to them, since each Euclid
number is coprime to every earlier term.  This refuted the natural
strengthening that surjectivity holds regardless of the factor-selection
rule, and showed that the $\minFac$ rule is essential to the conjecture.

\paragraph{Variants and surveys.}
Booker~\cite{BoSa2012} showed that a carefully chosen variant of the
Euclid--Mullin sequence \emph{does} contain every prime: by selecting a
specific (not necessarily smallest) prime factor at each step, one can
steer the sequence to hit every prime.  This demonstrates that the
conjecture is \emph{delicate}: the surjectivity depends on the precise
rule, not just the Euclidean structure.  Pollack and
Trevi\~{n}o~\cite{PollackTrevino2014} surveyed the problem's place in
the broader landscape of Euclid-inspired sequences, and studied
distributional properties of primes ``forgotten'' by Euclid-type
constructions.

\paragraph{Computational status.}
The sequence has been extended through a series of large-scale
factoring efforts:
\begin{itemize}[nosep]
\item Wagstaff~(1993) computed through the 43rd term.
\item In 2010, the 180-digit number $\Prod(43) + 1$ was factored via
  GNFS (General Number Field Sieve), yielding a 68-digit prime as
  $a(44)$.  Terms $a(45)$--$a(47)$ followed.
\item In 2012, Propper factored the 256-digit number $\Prod(47)+1$ by
  ECM (Elliptic Curve Method), discovering a 75-digit factor and
  extending the sequence to 51~terms
  (Booker--Irvine~\cite{BoIr2016}).
\item Finding $a(52)$ requires factoring a 335-digit number.  No
  factorization is known as of 2025.
\end{itemize}
After 51~terms, the smallest primes not yet observed are $41$ and $47$.
Note that $31$---a smaller prime---does not appear until position~$50$.

\paragraph{Why computation cannot resolve the conjecture.}
Even heroic computation is fundamentally unable to address the
conjecture.  Each new term requires factoring a number whose digit
count grows roughly linearly with the number of terms, quickly
exceeding the reach of any known factoring algorithm.  But even if we
could compute millions of terms, this would prove nothing: the
conjecture is a $\forall$-statement over all primes, and no finite
computation can rule out the possibility that some prime first appears
at an astronomically large index.

More fundamentally, the sequence exhibits a sensitive dependence on its
full history.  Each term $a(n+1) = \minFac(\Prod(n)+1)$ depends on the
\emph{complete factorization} of a number that encodes all previous
terms.  Changing a single early term alters every subsequent one.  This
global coupling is what makes the sequence appear random despite being
deterministic, and it means that local or statistical reasoning
about ``typical'' behavior is unreliable.  There are no known density
arguments, probabilistic heuristics, or sieve-theoretic bounds that
bear on the conjecture.  The problem requires a structural argument
about the sequence's long-term dynamics---which is precisely what our
formalization provides.

\subsection{Main Result and Approach}

Our main result is a formally verified reduction of MC to a single
hypothesis about the dynamics of a residue walk.  The strategy proceeds
in three stages:
\begin{enumerate}[nosep]
\item \textbf{Reformulation.}  We recast ``does prime~$q$ appear?'' as
  ``does a multiplicative walk on the cyclic group $(\ZZ/q\ZZ)^\times$
  hit the element~$-1$?''  This translation is exact
  (Section~\ref{sec:walk}).
\item \textbf{Bootstrap.}  We show that the algebraic precondition for
  walk equidistribution---that the multipliers generate the full
  group---is \emph{free}, following from the inductive hypothesis
  $\MC({<}\,p)$ via an elementary lemma
  (Section~\ref{sec:bootstrap}).  This reduces MC to a single
  dynamical question.
\item \textbf{Diagnosis.}  We develop the harmonic-analytic and
  sieve-theoretic infrastructure to determine \emph{precisely} what
  kind of statement would close the conjecture, and why known methods
  fall short (Sections~\ref{sec:character}--\ref{sec:hard}).
\end{enumerate}
The formalization serves two purposes:
(i)~it guarantees that every reduction is logically sound, and (ii)~it
precisely delineates the boundary between what is proved and what
remains open, preventing the kind of subtle gap that plagues pencil-and-paper
reductions involving multiple interacting hypotheses.

\begin{theorem}[\lean{EM/EquidistBootstrap.lean}{dynamical\_hitting\_implies\_mullin}]
\label{thm:main}
$\mathrm{DynamicalHitting} \;\Longrightarrow\; \MC$.
\end{theorem}

DynamicalHitting asserts: if the multiplier residues generate
$(\ZZ/q\ZZ)^\times$ (SubgroupEscape), then the walk hits~$-1$
cofinally (HittingHypothesis).  The proof is by strong induction
on~$p$, with PrimeResidueEscape (proved elementarily) bootstrapping SE
at each step.

A parallel reduction gives a clean character-analytic statement:

\begin{theorem}[\lean{EM/EquidistSelfCorrecting.lean}{complex\_csb\_mc'}]
$\mathrm{ComplexCharSumBound} \;\Longrightarrow\; \MC$.
\end{theorem}

Both reductions are fully machine-verified with zero \code{sorry}.
The formalization thus provides a complete ``roadmap'' for proving MC:
any future proof need only establish one of these open hypotheses
(DynamicalHitting, ComplexCharSumBound, or any of the equivalent
formulations in \S\ref{sec:character} and \S\ref{sec:open}), and the
rest follows by machine-checked deduction.

\subsection{Analogies and Context}

Mullin's Conjecture has no known applications: if proved tomorrow, no
other theorem in number theory would follow from it.  The value of the
problem lies instead in what it \emph{is an instance of}---and in the
methods its resolution would require.  The orbit-specificity barrier
identified by this formalization appears, in recognizable form, across
several active areas of mathematics.

\paragraph{Artin's conjecture and the orbit-specificity gap.}
The closest structural analogue to MC is Artin's conjecture on
primitive roots~\cite{Artin1927}: for any integer $a \neq -1$ that is not a
perfect square, the group $(\ZZ/p\ZZ)^\times$ is generated by~$a$ for
infinitely many primes~$p$.  The parallel is precise:
\begin{itemize}[nosep]
\item \textbf{Artin} asks whether the orbit of a fixed generator~$a$
  under repeated multiplication fills $(\ZZ/p\ZZ)^\times$.
\item \textbf{Mullin} asks whether the orbit of~$2$ under
  multiplication by successive EM primes in $(\ZZ/q\ZZ)^\times$ hits
  the single element~$-1$.
\end{itemize}
Both conjectures are blocked by the same fundamental obstacle:
transferring \emph{averaged} equidistribution results (which hold for
most moduli or most generators) to \emph{one specific deterministic
orbit}.  Hooley~\cite{Hooley1967} proved Artin's conjecture conditional on GRH
for Dedekind zeta functions of Kummer extensions---because GRH
provides the uniformity across individual characters needed to control
a single orbit.  In our setting, this uniformity is exactly what CCSB
demands.

The analogy is not merely structural.  The Kummer extensions
$\mathbb{Q}(\zeta_\ell, a^{1/\ell})$ that appear in Hooley's proof
are the same extensions that arise in the EffectiveKummerEscape
approach to SubgroupEscape (Section~\ref{sec:open}).  The elementary
PRE lemma (Theorem~\ref{thm:pre}) sidesteps this Chebotarev machinery
entirely for the algebraic component, but the dynamical
component---does the walk \emph{hit}~$-1$, not merely \emph{generate}
the full group?---remains exactly the orbit-specificity gap that GRH
closes for Artin and that no known tool closes for Mullin.

\paragraph{Multiplicative walks on finite groups.}
The walk reformulation (Section~\ref{sec:walk}) places MC in the
framework of random walks on finite groups, studied systematically by
Diaconis and others since the
1980s~\cite{ChungDiaconisGraham1987}.  The Diaconis--Shahshahani upper
bound lemma~\cite{DiaconisShahshahani1981} shows that a random walk on a finite group~$G$
driven by i.i.d.\ multipliers from a conjugation-invariant
distribution mixes in $O(\log |G|)$ steps, with mixing measured by
character sums.

The EM walk has the same algebraic structure---a multiplicative walk
on the cyclic group $(\ZZ/q\ZZ)^\times$---but violates every
assumption of the classical mixing theory.  The multipliers are
deterministic, not random; they are not identically distributed; and
most critically, the multiplier at step~$n$ depends on the walk
position at step~$n$, creating exactly the position-multiplier
correlation that the Diaconis--Shahshahani framework assumes away.
CCSB is the precise derandomization of the mixing-time bound: it asks
that the Fourier coefficients of the walk's occupation measure tend to
zero for every non-trivial character.

\paragraph{Smallest prime factor distribution.}
The SieveTransfer hypothesis (Section~\ref{sec:open}) connects MC to
the distribution of the smallest prime factor function
$P^-(n) = \min\{p : p \mid n\}$, studied by Alladi~\cite{Alladi1977},
Hildebrand~\cite{Hildebrand1986}, and others.  For generic integers in an arithmetic
progression $n \equiv a \pmod{q}$, the distribution of $P^-(n)$ is
controlled by the Dickman function and CRT-based equidistribution
results.  MC asks whether this equidistribution transfers from generic
integers to the specific subsequence $\{\Prod(n)+1\}_{n \geq 0}$.
The same orbit-specificity transfer problem arises for Mersenne
numbers $2^p - 1$, Fibonacci numbers, and polynomial iterates
$f^{\circ n}(a)$ in arithmetic dynamics.

\paragraph{The marginal/joint barrier and Sarnak's conjecture.}
The formalization identifies a precise meta-obstacle
(Section~\ref{sec:hard}): \emph{marginal} equidistribution of the
multiplier residues is provable (the EM primes are equidistributed in
residue classes, by Dirichlet's theorem); what MC requires is
\emph{joint} equidistribution of the pair (walk position, multiplier),
conditioned on the walk's history.  This barrier is an instance of a
broader phenomenon.  Sarnak's conjecture~\cite{Sarnak2010} asserts that the
M\"obius function $\mu(n)$ is orthogonal to every bounded
deterministic sequence:
$\frac{1}{N} \sum_{n \leq N} \mu(n) \, a_n \to 0$.
CCSB is a M\"obius-orthogonality-type statement for the EM sequence,
placing MC squarely within the Sarnak program's conceptual framework,
even though the EM sequence falls outside the technical scope of
existing results (which require zero topological entropy).

\paragraph{Greedy sieves and orbit-hitting.}
MC is the simplest nontrivial instance of a broader question: does a
greedy, deterministic prime-selection process eventually cover all
primes?  The Cox--van der Poorten result~\cite{CoxVdP1968} shows that this
determinism is fragile: choosing the \emph{largest} factor instead
provably misses primes.  More generally, MC belongs to the family of
\emph{orbit-hitting problems} in arithmetic dynamics: given a map~$T$
on a space~$X$ and a target set $S \subset X$, does the orbit
eventually enter~$S$?  Unlike Artin (where the map $x \mapsto ax$ is
the same at every step) or Collatz (where the map depends only on the
current state), the EM map varies at each step, determined by the
factorization of a number that depends on the entire orbit history.
This self-referential coupling is what the walk--multiplier framework
makes precise, and the formalization shows it is the sole source of
difficulty: once the coupling is controlled (via CCSB or CME), MC
follows by machine-checked deduction.

\paragraph{Notation.}
Theorems marked {\color{leanink}\sffamily\footnotesize$\checkmark$\,name}
are formally verified in Lean~4; clicking the identifier links to the
source code.

\paragraph{Organization.}
The paper follows the logical structure of the reduction.
Section~\ref{sec:walk} reformulates MC as a walk-hitting problem and
establishes the algebraic prerequisites.
Section~\ref{sec:bootstrap} presents the inductive bootstrap---the
core mathematical insight that makes SubgroupEscape free.
Section~\ref{sec:character} develops the character-analytic reduction
(CCSB~$\Rightarrow$~MC), including the large sieve infrastructure, the
spectral energy bridge, and the van der Corput--autocorrelation route.
Section~\ref{sec:hard} explains \emph{why} the remaining hypothesis is
difficult by analyzing dead ends and structural barriers.
Section~\ref{sec:lean} describes the Lean formalization.
Section~\ref{sec:open} discusses open problems and paths forward.

%% =========================================================================
\section{The Residue Walk Reformulation}
\label{sec:walk}
%% =========================================================================

The central idea is to replace a global number-theoretic question
(``does prime~$q$ eventually appear?'') with a local algebraic one
(``does a certain walk on a finite group hit a specific element?'').
We begin with the definitions, then explain why this reformulation
is powerful.

\subsection{Definitions}

Let $\minFac(m)$ denote the smallest prime factor of a natural number
$m \geq 2$. We define two sequences $\seq, \Prod : \NN \to \NN$ by
mutual recursion:
\begin{alignat}{2}
  \seq(0) &\coloneqq 2, &\qquad
  \Prod(0) &\coloneqq 2, \label{eq:base}\\
  \seq(n+1) &\coloneqq \minFac\bigl(\Prod(n) + 1\bigr), &\qquad
  \Prod(n+1) &\coloneqq \Prod(n) \cdot \seq(n+1). \label{eq:step}
\end{alignat}

\begin{theorem}[\lean{EM/MullinDefs.lean}{seq\_isPrime, seq\_injective}]\label{thm:seq-props}
\leavevmode
\begin{enumerate}[nosep]
\item \textup{(Primality)} For every $n \in \NN$, $\seq(n)$ is prime.
\item \textup{(Injectivity)} $\seq(i) = \seq(j)$
  implies $i = j$.  No prime appears twice.
\end{enumerate}
\end{theorem}

\begin{definition}[\defname{Walk} and \defname{Multiplier}]\label{def:walk-mult}
Let $q$ be a prime with $\seq(n) \neq q$ for all~$n$
(a \emph{missing prime}).
\begin{align}
  \walkZ(q, n) &\coloneqq \Prod(n) \bmod q \;\;\in (\ZZ/q\ZZ)^\times, \label{eq:walk}\\
  \multZ(q, n) &\coloneqq \seq(n\!+\!1) \bmod q \;\;\in (\ZZ/q\ZZ)^\times. \label{eq:mult}
\end{align}
\end{definition}

The restriction to missing primes is essential, not merely
conventional.  If $q = \seq(k)$ for some~$k$, then $q \mid \Prod(n)$
for every $n > k$, so $\Prod(n) \bmod q = 0$, which is \emph{not} a
unit in~$\ZZ/q\ZZ$; the walk and multiplier would not be
well-defined.  When~$q$ is missing, it never divides any sequence term
(by injectivity) and never divides the running product (which is a
product of primes~$\neq q$), so both residues land in~$(\ZZ/q\ZZ)^\times$.

Mullin's Conjecture asserts that no prime is missing.  The entire
framework therefore works by reduction: \emph{assuming}~$q$ is missing,
we build a walk that is well-defined, and then show it must hit~$-1$,
which forces~$q$ to appear---a contradiction.  Every quantifier ``for
every missing prime~$q$'' in the sequel should be read in this light.

\begin{proposition}[\defname{Walk recurrence} --- \lean{EM/MullinGroupCore.lean}{walkZ\_succ}]\label{prop:recurrence}
$\walkZ(q, n\!+\!1) = \walkZ(q, n) \cdot \multZ(q, n)$ in
$(\ZZ/q\ZZ)^\times$.
The walk is a multiplicative walk on the cyclic group
$(\ZZ/q\ZZ)^\times$: at each step, the position is multiplied by the
next EM prime reduced mod~$q$.
\end{proposition}

\begin{example}[The walk for $q = 41$]\label{ex:walk-41}
The prime $41$ has not appeared in the first 51~known terms of the
sequence, so we may form its walk.  The table below shows the first
ten steps.  At each step, $\walkZ(41, n)$ is updated by multiplying
by $\multZ(41, n) = \seq(n\!+\!1) \bmod 41$:

\medskip
\begin{center}
\begin{tabular}{r r r r l}
\toprule
$n$ & $\seq(n)$ & $\seq(n) \bmod 41$ & $\walkZ(41,n)$ & \\
\midrule
 0 &       $2$ &  $2$ &  $2$ & \\
 1 &       $3$ &  $3$ &  $6$ & \\
 2 &       $7$ &  $7$ &  $1$ & \\
 3 &      $43$ &  $2$ &  $2$ & \\
 4 &      $13$ & $13$ & $26$ & \\
 5 &      $53$ & $12$ & $25$ & \\
 6 &       $5$ &  $5$ &  $2$ & \\
 7 & $6221671$ &  $3$ &  $6$ & \raisebox{0pt}[0pt][0pt]{$\leftarrow$ pattern repeats} \\
 8 & $38709183810571$ &  $7$ &  $1$ & \\
 9 &     $139$ & $16$ & $16$ & \\
\bottomrule
\end{tabular}
\end{center}
\medskip

For $41$ to appear in the sequence at step~$n\!+\!1$, we need
$41 \mid \Prod(n)+1$, i.e.\ $\walkZ(41, n) = 40 \equiv -1 \pmod{41}$.
Over all 51~known terms, the walk takes values
\[
  2,\; 6,\; 1,\; 2,\; 26,\; 25,\; 2,\; 6,\; 1,\; 16,\;
  3,\; 33,\; 28,\; 12,\; 24,\; 7,\; 20,\; 2,\; 38,\; 7,\; \ldots
\]
and \emph{never reaches~$40$}.  The group $(\ZZ/41\ZZ)^\times$ has
$40$~elements; the walk wanders through them but has not yet found its
target.  Can we prove it eventually will?  That is the content of
Mullin's Conjecture for~$q = 41$.
\end{example}

\paragraph{Why the reformulation is powerful.}
The walk--multiplier framework is not merely cosmetic: it makes the
problem amenable to tools from group theory, harmonic analysis, and
ergodic theory, and it cleanly separates two independent components:
\begin{itemize}[nosep]
\item \textbf{Algebraic structure}: which subgroups of
  $(\ZZ/q\ZZ)^\times$ do the multipliers $\multZ(q, n)$ generate?  If
  a proper subgroup traps all multipliers, the walk is permanently
  confined and may never reach~$-1$.
\item \textbf{Dynamical content}: given that the multipliers generate
  the full group, does the walk actually visit every element?  This is
  a question about the \emph{ordering} of multipliers, not just their
  \emph{set}.
\end{itemize}
The bootstrap (Section~\ref{sec:bootstrap}) will show that the
algebraic component is free; the dynamical component is the sole
remaining content of MC.

\subsection{The Walk--Divisibility Bridge}

The walk and multiplier become useful only when connected back to the
original number-theoretic problem.  The following theorem is the
linchpin of the entire reduction: it converts divisibility (a
number-theoretic condition) into a walk event (a group-theoretic
condition), enabling all subsequent algebraic and harmonic-analytic
arguments.  Every route to MC in this paper---DH, CCSB, MMCSB, the
large sieve, the spectral energy approach---passes through this bridge.

\begin{theorem}[\lean{EM/MullinGroupCore.lean}{walkZ\_eq\_neg\_one\_iff}]\label{thm:bridge}
For every missing prime~$q$ and every $n \in \NN$:
\[
  \walkZ(q, n) = -1 \;\;\text{in } (\ZZ/q\ZZ)^\times
  \quad\Longleftrightarrow\quad
  q \mid \bigl(\Prod(n) + 1\bigr).
\]
\end{theorem}

This is the key translation: the group-theoretic event ``the walk
reaches $-1$'' is the number-theoretic event ``$q$ divides
$\Prod(n)+1$.''  If $q$ divides $\Prod(n)+1$, then $q$ could be a
factor---and the $\minFac$ selection has a chance of picking~$q$.

The bridge is formally verified because it is used in \emph{every}
reduction chain: DH~$\Rightarrow$~MC, CCSB~$\Rightarrow$~MC,
MMCSB~$\Rightarrow$~MC, and all sieve routes.  An error here would
invalidate the entire project.

\subsection{SubgroupEscape and the Confinement Theorem}

Before asking whether the walk reaches~$-1$, we must ask whether it
\emph{can}.  The key observation is that the walk is
\emph{multiplicative}: $\walkZ(q, n{+}1) = \walkZ(q, n) \cdot
\multZ(q, n)$.  Each step multiplies the current position by a new
multiplier.  This multiplicative structure is what makes
\emph{subgroups}---rather than arbitrary subsets---the natural
obstruction.

To see why, suppose all the multipliers happen to land in some subset
$S \subset (\ZZ/q\ZZ)^\times$.  If $S$ is merely a subset with no
algebraic structure, this tells us very little: multiplying elements of
$S$ together can produce anything, so the walk might still wander
freely across the whole group.  But if $S$ happens to be a
\emph{subgroup}~$H$, the situation changes dramatically.  Products of
elements of~$H$ stay in~$H$ (by closure), so $\walkZ(q, n)$ is trapped
in the coset $\walkZ(q, 0) \cdot H$ forever.  The walk visits at most
$|H|$ of the $q - 1$ residues and may never reach~$-1$.

This is not a theoretical curiosity.  For instance, if every EM prime
happened to be a quadratic residue mod~$q$, then every multiplier would
lie in the index-2 subgroup of squares, and the walk would be
permanently confined to a coset of that subgroup---potentially one that
does not contain~$-1$.

Because $(\ZZ/q\ZZ)^\times$ is cyclic of order~$q - 1$, its subgroup
structure is particularly simple: there is exactly one subgroup of
order~$d$ for each $d \mid q - 1$, and the maximal proper subgroups are
the index-$\ell$ subgroups for each prime~$\ell \mid q - 1$.  There are
relatively few of these (at most $\omega(q-1)$, the number of distinct
prime factors of~$q - 1$), which is what makes SubgroupEscape a
checkable condition.

\begin{definition}[\defname{SubgroupEscape} (\abbr{SE})]\label{def:se}
For every missing prime~$q$ and every proper subgroup $H \lneq
(\ZZ/q\ZZ)^\times$:
$\exists\, n,\; \multZ(q, n) \notin H$.
\end{definition}

In other words, SE says that no proper subgroup contains all
multipliers---equivalently, the multipliers \emph{generate} the full
group~$(\ZZ/q\ZZ)^\times$.  The following theorem makes precise why SE
is necessary.

\begin{theorem}[\defname{Confinement} --- \lean{EM/MullinGroupCore.lean}{confinement\_forward}]\label{thm:confinement}
If every multiplier lies in a subgroup $H \leq (\ZZ/q\ZZ)^\times$,
then the walk is confined to the coset $\walkZ(q,0) \cdot H$.
In particular, if $-1 \notin \walkZ(q,0) \cdot H$, the walk
\emph{never} hits~$-1$.
\end{theorem}

The proof is immediate from the multiplicative recurrence: if
$\multZ(q,n) \in H$ for all~$n$, then $\walkZ(q, n) = \walkZ(q,0)
\cdot \multZ(q,0) \cdot \multZ(q,1) \cdots \multZ(q,n{-}1) \in
\walkZ(q,0) \cdot H$.  The walk cannot leave this coset, and if $-1$
is not in it, the walk is geometrically prevented from ever reaching
$-1$---no amount of time will help.

SE eliminates this obstruction: once the multipliers generate the full
group, the walk is not confined to any proper coset, and the
possibility of hitting~$-1$ is restored.  SE does not by itself
\emph{guarantee} the walk reaches~$-1$---that is a separate dynamical
question---but without SE, the walk may be permanently locked out.
Every route to MC must therefore either assume SE or prove it.
(The bootstrap of Section~\ref{sec:bootstrap} will prove it for free.)

\begin{theorem}[\lean{EM/MullinGroupEscape.lean}{se\_of\_maximal\_escape}]
SE holds iff multipliers escape every \emph{maximal} proper subgroup.
Since $(\ZZ/q\ZZ)^\times$ is cyclic, the maximal subgroups are the
index-$\ell$ subgroups for prime $\ell \mid q\!-\!1$.
\end{theorem}

%% =========================================================================
\section{The Inductive Bootstrap}
\label{sec:bootstrap}
%% =========================================================================

The walk--divisibility bridge (Theorem~\ref{thm:bridge}) tells us
\emph{when} a missing prime~$q$ appears: it appears at step~$n\!+\!1$
if and only if $\walkZ(q,n) = -1$ and $q$ is the smallest prime factor
of $\Prod(n)+1$.  SubgroupEscape (Section~\ref{sec:walk}) tells us when
this is \emph{possible}: the walk can reach~$-1$ only if the multipliers
generate the full group.  What remains is the dynamical question: does
the walk actually reach~$-1$?

The Hitting Hypothesis asserts that it does---not just once, but
cofinally (infinitely often).  Cofinality is needed because a single
event $q \mid \Prod(n)+1$ does not guarantee $\seq(n\!+\!1) = q$: a
smaller prime might divide $\Prod(n)+1$ as well.  By asking for
infinitely many hits, we ensure that eventually all smaller primes
have already been selected, and $q$ is the smallest factor available.

\begin{definition}[\defname{HittingHypothesis} (\abbr{HH})]\label{def:hh}
For every missing prime~$q$:
$\forall\, N,\; \exists\, n \geq N,\; q \mid (\Prod(n) + 1)$.
Equivalently: the walk reaches $-1$ cofinally.
\end{definition}

\begin{theorem}[\lean{EM/MullinConjectures.lean}{hh\_implies\_mullin}]
$\HH \Rightarrow \MC$.
\end{theorem}

\begin{proof}[Proof sketch]
By strong induction on~$p$.  Suppose every prime $< p$ is in the
sequence (the IH), and assume $p$ is missing.  $\HH$ provides a cofinal
sequence of indices where $p \mid \Prod(n)+1$.  At each such~$n$,
$\seq(n\!+\!1) = \minFac(\Prod(n)+1) \leq p$.  If
$\seq(n\!+\!1) < p$, that prime is also missing---contradicting the IH.
So $\seq(n\!+\!1) = p$.
\end{proof}

Proving MC therefore requires two things for each missing prime~$q$:
(i)~SE, so the walk \emph{can} reach~$-1$, and (ii)~HH, so the walk
\emph{does} reach~$-1$ cofinally.  Without further machinery, these
are two independent open problems for every prime.

\medskip\noindent\textbf{Original contribution.}\enspace
This section presents the paper's main new result: an elementary
proof that SE is ``free'' at every step of the induction, eliminating
one of the two problems entirely.  Given $\MC({<}\,p)$ (the inductive
hypothesis), SE at~$p$ follows from a single, purely algebraic lemma
(PrimeResidueEscape).  This reduces the entire conjecture to the
dynamical question: does a walk with a generating set of multipliers
hit every element cofinally?

The proof of PRE uses no analytic number theory, no Chebotarev density
theorem, no character sums---only modular arithmetic.  This makes the
bootstrap both logically clean and Lean-friendly: the proof is short
(${\sim}100$~lines) and depends on minimal Mathlib infrastructure.

\subsection{PrimeResidueEscape}

\begin{definition}[\defname{PrimeResidueEscape} (\abbr{PRE})]
For every prime $p \geq 5$ and every proper subgroup
$H < (\ZZ/p\ZZ)^\times$, some odd prime $r < p$ has residue
$r \bmod p \notin H$.
\end{definition}

\begin{theorem}[\lean{EM/EquidistBootstrap.lean}{prime\_residue\_escape}]\label{thm:pre}
PrimeResidueEscape holds.
\end{theorem}

\begin{proof}
Suppose every odd prime $r \in [3, p)$ satisfies $r \bmod p \in H$.
Since $H$ is a subgroup, every \emph{product} of such primes is
in~$H$.  Every odd number in~$[1, p)$ factors into odd primes $< p$,
so every odd number in~$[1, p)$ maps into~$H$.  In particular:
\begin{itemize}[nosep]
\item $p - 2 \equiv -2 \pmod{p}$ is in $H$ (since $p - 2$ is odd and $< p$).
\item $p - 4 \equiv -4 \pmod{p}$ is in $H$ (since $p - 4$ is odd and $< p$, for $p \geq 5$).
\end{itemize}
Then $2 = (-4)(-2)^{-1} \in H$.  Now every integer in $[1, p)$ is
in~$H$: even numbers are $2^k \cdot (\text{odd})$, both factors
in~$H$.  So $H = (\ZZ/p\ZZ)^\times$---contradicting $H$ proper.
\end{proof}

The identity $2 = (-4)(-2)^{-1}$ is the only non-trivial step.
No analytic number theory, no Chebotarev density theorem, no
character sums---just modular arithmetic.

\begin{remark}[Why PRE is not trivial]\label{rem:pre-discussion}
The problem PRE solves---showing that the natural primes below~$p$
cannot all land in a proper subgroup---looks like it should be easy
but is surprisingly resistant to direct attack.

The obvious approach would be: use Dirichlet's theorem (infinitely
many primes in each residue class), or Chebotarev (primes are
equidistributed across cosets of any subgroup), or Linnik's theorem
(the least prime in each progression is bounded).  All of these
work---but they are nuclear weapons for a problem that \emph{feels}
elementary.  Worse, they are all absent from Mathlib, so they cannot
be formalized.  And they give much more than needed: one does not care
about the \emph{distribution} of primes across cosets, just that at
least one prime misses~$H$.

The other natural approach: argue directly about small primes.  The
first few EM primes are $2, 3, 5, 7, 13, 43, 53$---can one not just
check that these escape any subgroup?  One can for specific~$q$
(that is what the 30~SE instances for $q \leq 157$ do), but not
universally.  A proper subgroup of $(\ZZ/p\ZZ)^\times$ can have
index~2 and contain nearly half the elements---there is no reason a
fixed finite set of primes could not all be quadratic residues
mod some large~$p$.  The QR obstruction analysis shows this happens
for at most $1.6\%$ of primes, but ``at most $1.6\%$'' is not
``never.''

So one is stuck: the heavy theorems are not available, and direct
checking does not generalize.  PRE needs something else.

\medskip\noindent\textbf{Why the identity is delightful.}\enspace
The key move is: \emph{do not try to get any specific prime out
of~$H$.  Get~$2$ out of~$H$.}

If one can show $2 \in H$, one is done---because $H$ already contains
all odd numbers less than~$p$ (by the closure argument), so adding~$2$
gives all even numbers too, and $H = \top$.

But $2$ is not odd, so the closure argument for odd numbers does not
apply to~$2$ directly.  One cannot factor~$2$ as a product of odd
primes.  Seems like a dead end.

The trick: one does not need~$2$ to \emph{be} an odd number.  One
needs~$2$ to be \emph{expressible using elements already known to be
in~$H$}.  And $H$ is a subgroup---closed under multiplication
\emph{and inversion}.
\begin{align*}
p - 2 &\equiv -2 \pmod{p} \;\in H
  &&\text{($p{-}2$ is odd and $< p$)}, \\
p - 4 &\equiv -4 \pmod{p} \;\in H
  &&\text{($p{-}4$ is odd and $< p$, since $p \geq 5$)}.
\end{align*}
Therefore $(-4)\cdot(-2)^{-1} = 2 \in H$.  One division in the group.
The argument uses the specific arithmetic fact that $p-2$ and $p-4$
are both odd (which requires $p \geq 5$, hence the case split for
small primes), the subgroup property (closure under products and
inverses), and nothing else.  No analytic number theory, no character
theory, no sieve methods.

The delight is in the economy.  One is trying to show that primes
mod~$p$ escape every proper subgroup, and the proof does not mention
primes at all after the first two sentences.  It works entirely with
odd numbers, uses only that~$H$ is a subgroup, and extracts~$2$ from
the pair $(-2, -4)$ by a single division.  The entire argument is five
lines.

\medskip\noindent\textbf{Why PRE matters for the project.}\enspace
PRE is the engine of the inductive bootstrap.  Without it, the
$\DH \Rightarrow \MC$ reduction would need SE as a separate
hypothesis---a two-hypothesis reduction instead of a single-hypothesis
one.  The conjecture would be: ``if the multipliers generate the full
group \emph{and} the walk hits $-1$ cofinally, then MC holds.''  That
is weaker and less clean.

With PRE, SE becomes free.  The inductive hypothesis $\MC({<}\,p)$
says all primes below~$p$ are in the sequence.  PRE says some odd
prime $r < p$ escapes any proper subgroup $H < (\ZZ/p\ZZ)^\times$.
Since~$r$ is in the sequence (by the IH), its residue mod~$p$ appears
as a multiplier, and that multiplier escapes~$H$.  So the multipliers
generate the full group---automatically, at every step, with no
additional assumption.

This is what makes $\DH \Rightarrow \MC$ a \emph{genuine reduction}
rather than a \emph{reformulation}.  Without PRE, saying ``DH implies
MC'' would be like saying ``if the hard parts are true, the conjecture
follows''---technically correct but not illuminating.  With PRE, the
algebraic part (generation) is proved for free, and the entire content
of MC is concentrated in one dynamical question: does the walk
hit~$-1$ when the multipliers generate everything?  The simplicity of
the PRE proof---five lines, no analytic number theory---makes the
reduction sharp: the hard part of MC is \emph{purely dynamical}, not
algebraic.

There is also a methodological point.  The project explored
Chebotarev, Linnik, Kummer theory, effective density theorems---all of
which could prove SE but none of which are in Mathlib.  PRE bypasses
all of them with an argument that depends on nothing beyond the
definition of a subgroup and the fact that $p-2$ and $p-4$ are odd.
This is the kind of argument that makes formalization worthwhile: it is
short enough to verify by hand, but subtle enough that one would not
find it without specifically looking for a Mathlib-minimal proof of
subgroup escape.  The constraint of working within Lean forced a better
proof.
\end{remark}

\subsection{The Bootstrap Mechanism}

\begin{theorem}[\lean{EM/EquidistBootstrap.lean}{mc\_below\_pre\_implies\_se}]
\label{thm:bootstrap}
$\MC({<}\,p) + \PRE \;\Longrightarrow\; \SE(p)$.
\end{theorem}

\begin{proof}[Proof sketch]
Let $H < (\ZZ/p\ZZ)^\times$ be proper.  By PRE, some odd prime
$r < p$ has $r \bmod p \notin H$.  By $\MC({<}\,p)$, the prime~$r$
appears as $\seq(k)$ for some~$k$.  Then $\multZ(p, k{-}1) \equiv r
\pmod{p} \notin H$.  So the multipliers escape~$H$.
\end{proof}

The payoff of this mechanism---the full reduction from
DynamicalHitting to MC---is assembled in
Section~\ref{sec:dh-reduction}, after we establish the supporting
infrastructure.

\subsection{The Threshold Approach}

The threshold mechanism combines finite computation with the abstract
bootstrap.  If one can handle all primes below some bound~$B$ by
direct verification, then MC reduces to DH for primes~$\geq B$ only.
This is formalized because it interfaces with the large sieve route
(Section~\ref{sec:character}): MultiModularCSB gives DH for all
primes above its threshold parameter~$Q_0$, and the gap below~$Q_0$
is closed by threshold verification.

\begin{definition}[\defname{ThresholdHitting}]
$\mathrm{ThresholdHitting}(B)$: DH restricted to primes $q \geq B$.
\end{definition}

\begin{theorem}[\lean{EM/EquidistThreshold.lean}{threshold\_11\_implies\_mullin'}]
\label{thm:threshold-11}
$\mathrm{ThresholdHitting}(11) \;\Longrightarrow\; \MC$.
\end{theorem}

The computed values $\seq(0) = 2$, $\seq(1) = 3$, $\seq(2) = 7$,
$\seq(6) = 5$ prove $\MC$ for every prime $< 11$.  For $q \geq 11$,
the IH + PRE give SE, and ThresholdHitting gives HH.

\subsection{The Sieve Gap}

The inductive structure of the proof does more than provide SE: it also
resolves the selectability problem described in Section~\ref{sec:intro}.
Once all primes below~$q$ are in the sequence, they divide the running
product and hence \emph{cannot} divide any future Euclid number.  This
``sieve gap'' means that past a computable stage, every prime factor
of~$\Prod(n)+1$ is~$\geq q$, so $q$ is the \emph{smallest} available
factor whenever it divides the Euclid number.  The $\minFac$ rule,
which seemed like the source of difficulty, becomes an \emph{ally}:
it \emph{must} select~$q$ the next time $q$ divides an Euclid number.

This converts MC from ``$q$ divides $\Prod(n)+1$ \emph{and} is the
smallest such factor'' (hard) to simply ``$q$ divides $\Prod(n)+1$
cofinally'' (the walk hits $-1$ cofinally)---which is exactly what
DynamicalHitting asserts.

\begin{theorem}[\defname{$q$-roughness} --- \lean{EM/EquidistThreshold.lean}{mc\_below\_implies\_seq\_ge}]
If $\MC({<}\,q)$ holds, then $\exists\, N$ such that $\forall\, n \geq N$,
$\seq(n\!+\!1) \geq q$.
\end{theorem}

\begin{theorem}[\defname{One-prime gap} --- \lean{EM/EquidistThreshold.lean}{mc\_below\_cofinal\_hit\_implies\_mc\_at}]
\label{thm:one-prime-gap}
$\MC({<}\,q)$ plus a single cofinal hitting event
$(\forall\, N,\, \exists\, n \geq N,\, q \mid \Prod(n)+1)$
implies $\MC(q)$.
\end{theorem}

This is the \textbf{formal sieve gap}: the sieve at level $q\!-\!1$
is free from the IH, and extending it by one prime requires exactly one
cofinal divisibility event---which is what DynamicalHitting asserts.

\subsection{The Power Residue Decomposition}

Since $(\ZZ/q\ZZ)^\times$ is cyclic of order $q\!-\!1$, its subgroup
lattice is determined by the prime factorization of $q\!-\!1$.  A
multiplier set generates the full group if and only if it escapes every
maximal subgroup---and the maximal subgroups correspond to the prime
divisors $\ell$ of~$q\!-\!1$.  This decomposition converts SE into
independent conditions, one per prime~$\ell \mid q\!-\!1$.

\begin{definition}[\defname{PowerResidueEscape} ($\abbr{PRE}_\ell$)]
$\exists\, n : \multZ(q,n)^{(q-1)/\ell} \neq 1$ in $(\ZZ/q\ZZ)^\times$.
\end{definition}

\begin{theorem}[\abbr{PRE} $\Leftrightarrow$ \abbr{SE} --- \lean{EM/EquidistCharPRE.lean}{pre\_iff\_se}]
$\mathrm{PRE} \;\Longleftrightarrow\; \SE$.
The forward direction uses only Lagrange's theorem; the reverse uses
cyclicity of $(\ZZ/q\ZZ)^\times$.
\end{theorem}

\begin{theorem}[\lean{EM/EquidistCharPRE.lean}{eight\_elts\_escape\_order\_le\_seven}]
For $q \geq 59$ and $(q{-}1)/\ell \leq 7$, $\PRE_\ell(q)$ holds
automatically: the 8~known elements
$\{1, 3, 5, 7, 13, 43, 53, 21\}$ cannot fit in any subgroup of
order~$\leq 7$.
\end{theorem}

\subsection{Quadratic Reciprocity Obstruction}

The power residue decomposition raises a natural question: for how many
primes~$q$ could SE actually \emph{fail}?  The hardest case is the
index-2 subgroup (quadratic residues), since it is the largest maximal
subgroup.  Quadratic reciprocity gives a definitive answer:

\begin{theorem}[\defname{QR Obstruction} ---
  \lean{EM/MullinGroupQR.lean}{se\_qr\_obstruction}]
If all six multiplier primes $\{3,5,7,13,43,53\}$ are quadratic
residues mod~$q > 53$, then $q$ satisfies simultaneous Legendre
symbol conditions.  By CRT on
$\mathrm{lcm}(12,5,28,13,43,53) = 12{,}443{,}340$,
at most $1.6\%$ of primes satisfy all conditions.  For over
$98\%$ of primes, SE holds for the index-2 subgroup automatically.
\end{theorem}

This means that even without the bootstrap, SE fails for at most a
very sparse set of primes.  With the bootstrap, SE holds for
\emph{all} primes---but the QR obstruction analysis illustrates
why: the EM multiplier primes are arithmetically diverse enough to
escape any single subgroup.

\subsection{Concrete Verification}

\begin{theorem}[\lean{EM/MullinGroupSEInstances.lean}{se\_at\_11}]
SubgroupEscape holds for all 30~primes $q \leq 157$ not in the
sequence, verified via power checks.  For 29 of them, one of the
first six multipliers is a primitive root.  The exception is
$q = 131$, where the seventh multiplier
$\seq(7) = 6221671 \equiv 88 \pmod{131}$ has full order~130.
\end{theorem}

\subsection{The Reduction to DynamicalHitting}
\label{sec:dh-reduction}

We now have all the ingredients to assemble the main reduction.  The
goal is to identify the \emph{minimal} hypothesis that, combined with
the proved infrastructure, yields Mullin's Conjecture.

The Hitting Hypothesis (Definition~\ref{def:hh}) and its implication
$\HH \Rightarrow \MC$ show that cofinal hitting suffices---but HH is
a strong hypothesis, asserting cofinal hitting for \emph{every} missing
prime without any algebraic precondition.  Can we weaken it?

\subsubsection*{Conditioning on SubgroupEscape}

The confinement theorem (Theorem~\ref{thm:confinement}) tells us that
if SE fails at~$q$---meaning the multipliers are trapped in a proper
subgroup~$H$---then the walk is permanently confined to a coset of~$H$.
In this case, asking whether the walk hits~$-1$ may be the wrong
question: the walk might be structurally unable to reach~$-1$, and no
amount of dynamical analysis will help.

Conversely, when SE holds, the multipliers generate the full group
$(\ZZ/q\ZZ)^\times$, the walk is not confined to any proper coset, and
the question of whether it hits~$-1$ is a genuine dynamical question
about the walk's long-term behavior.

DynamicalHitting captures exactly this: it asks for HH only when the
algebraic precondition (SE) is satisfied.

\begin{definition}[\defname{DynamicalHitting} (\abbr{DH})]\label{def:dh}
For every missing prime~$q$: $\SE(q) \Rightarrow \HH(q)$.
\end{definition}

DH is \emph{strictly weaker} than HH: it makes no claim about primes
where SE fails.  And DH is strictly weaker than MC: MC implies
\emph{every} prime appears, while DH only promises appearance
\emph{conditional} on SE.  The remarkable fact is that DH is
nevertheless \emph{sufficient} for MC, because the bootstrap provides
SE for free.

\subsubsection*{The Full Reduction}

\begin{theorem}[\lean{EM/EquidistBootstrap.lean}{dynamical\_hitting\_implies\_mullin}]
\label{thm:dh-mc}
$\DH \;\Longrightarrow\; \MC$.
\end{theorem}

\begin{proof}
By strong induction on~$p$.  Assume $\MC({<}\,p)$ (every prime below~$p$
is in the sequence).  We must show $p$ appears.

\begin{enumerate}[nosep]
\item \textbf{SE is free.}
    Theorem~\ref{thm:bootstrap} gives
    $\MC({<}\,p) + \PRE \Rightarrow \SE(p)$.
    Since PRE is proved unconditionally (Theorem~\ref{thm:pre}), we
    obtain $\SE(p)$.

\item \textbf{DH gives HH.}
    Apply DH at the prime~$p$: since $\SE(p)$ holds, DH yields $\HH(p)$
    ---the walk $\walkZ(p, \cdot)$ reaches~$-1$ cofinally.

\item \textbf{The sieve gap closes.}
    By Theorem~\ref{thm:one-prime-gap}, $\MC({<}\,p)$ combined with
    cofinal hitting implies $\MC(p)$: once all primes below~$p$ are in
    the sequence, the $\minFac$ rule \emph{must} select~$p$ the next
    time it divides an Euclid number.
\end{enumerate}
\end{proof}

\subsubsection*{The Reduction Chain}

Assembling all the reductions proved in this paper, we obtain the
following chain.  Each arrow is a formally verified implication in Lean:

\[
\DH
\;\xrightarrow{\text{bootstrap}}\;
\SE \text{ (free)}
\;\xrightarrow{\DH}\;
\HH
\;\xrightarrow{\text{sieve gap}}\;
\MC
\]

The entire conjecture has been reduced to a single, cleanly stated
dynamical hypothesis: \emph{if the multipliers generate the full group,
then the walk hits every element cofinally}.  No algebraic hypothesis
remains open; the PRE lemma and the bootstrap mechanism have eliminated
SE completely.

The threshold variant (Theorem~\ref{thm:threshold-11}) offers a further
refinement: $\mathrm{ThresholdHitting}(11) \Rightarrow \MC$.  This
combines finite computation (MC for primes $< 11$, from the known
sequence values) with the abstract bootstrap, restricting DH to primes
$q \geq 11$ only.

%% =========================================================================
\section{The Character Sum Reduction}
\label{sec:character}
%% =========================================================================

The algebraic reduction (DH~$\Rightarrow$~MC) leaves the dynamical
content of DH untouched: \emph{does} the walk hit~$-1$ when the
multipliers generate the full group?  This section develops a parallel,
character-analytic approach that precisely characterizes what is needed.

\paragraph{Why character sums?}
The algebraic route (DH~$\Rightarrow$~MC) asks: does the walk hit a
\emph{single} element~($-1$)?  Character sums answer a stronger
question: does the walk visit \emph{every} element with equal
frequency?  Dirichlet characters $\chi : (\ZZ/q\ZZ)^\times \to
\mathbb{C}^\times$ are the Fourier basis of the cyclic group; they detect
imbalances in distribution.  By character orthogonality, the number of
times the walk visits a particular element~$t$ equals
$N/(q{-}1)$ (the uniform share) plus correction terms involving
character sums.  If all non-trivial character sums are
$o(N)$---negligible compared to the main term---the correction
terms vanish asymptotically, and the walk is equidistributed; in
particular, it visits~$-1$ cofinally, giving HH.

\medskip\noindent\textbf{What is new vs.\ what is infrastructure.}\enspace
The CCSB~$\Rightarrow$~MC reduction (Definition~\ref{thm:ccsb-mc}),
the Fourier bridge (Theorem~\ref{thm:fourier-step}), the
Decorrelation--PED chain (\S\ref{sec:dec-chain}), and the telescoping
no-go results (\S\ref{sec:telescope}) are original contributions of
this formalization.  The large sieve infrastructure
(\S\ref{sec:large-sieve}, Appendix~\ref{sec:sve})---including the weak ALS,
Gauss sum inversion, van der Corput, and Parseval---formalizes known
results; its purpose is to identify the precise \emph{transfer gap}
between classical tools and the EM orbit, which is itself a
contribution (see \S\ref{sec:large-sieve}).

The formalization develops this Fourier-analytic reduction because
it provides the cleanest interface between MC and the toolkit of
analytic number theory: the Bombieri--Vinogradov theorem, the large
sieve inequality, and Gauss sum inversion all produce character sum
bounds, and the formalization shows exactly how each connects to~MC.

\subsection{Character Sums and Walk Equidistribution}

For a Dirichlet character $\chi : (\ZZ/q\ZZ)^\times \to \mathbb{C}^\times$,
the \emph{walk character sum} is
\[
  S_\chi(N) \;=\; \sum_{n < N} \chi\bigl(\walkZ(q, n)\bigr).
\]

\begin{definition}[\defname{ComplexCharSumBound} (\abbr{CCSB})]
For every missing prime~$q$, every non-trivial character~$\chi$, and
every $\varepsilon > 0$, there exists $N_0$ such that for all
$N \geq N_0$:
\[
  \|S_\chi(N)\| \;\leq\; \varepsilon \cdot N.
\]
In other words, the walk character sums are $o(N)$---they grow
strictly slower than linearly.  The intuition: if $\chi$ is a
non-trivial character and the walk visits every group element equally
often, the sum $S_\chi(N)$ cancels out (positive and negative
contributions balance).  CCSB asks for exactly this cancellation.
\end{definition}

\begin{theorem}[\lean{EM/EquidistSelfCorrecting.lean}{complex\_csb\_mc'}]
\label{thm:ccsb-mc}
$\mathrm{CCSB} \;\Longrightarrow\; \MC$.
\end{theorem}

This is a single-hypothesis reduction with no additional parameters.
The proof composes three bridges:

\begin{enumerate}[nosep]
\item \textbf{Fourier inversion} (\lean{EM/EquidistFourier.lean}{complex\_csb\_implies\_hit\_count\_lb\_proved}):
  CCSB implies that the walk visits every unit class with positive
  lower density.  This follows from the character orthogonality
  formula (Theorem~\ref{thm:fourier-step} below): the number of hits
  to any target~$t$ equals $N/(q{-}1)$ plus correction terms of
  size~$o(N)$, so the count is eventually positive.

\item \textbf{Walk equidistribution implies DH}
  (\lean{EM/EquidistOrbitAnalysis.lean}{walk\_equidist\_mc}):
  if the walk visits every unit class cofinally, it visits~$-1$
  cofinally, giving HH.  SE is not needed as a separate hypothesis:
  equidistribution already implies the multipliers generate the full
  group (a walk confined to a proper coset cannot be equidistributed).

\item \textbf{DH implies MC} (Theorem~\ref{thm:dh-mc}): via the
  inductive bootstrap.
\end{enumerate}

\subsection{The Fourier Bridge}

The Fourier bridge is the single most important proved result after
DH~$\Rightarrow$~MC itself: it converts character sum bounds into hit
count lower bounds, and hence into MC.

\begin{theorem}[\lean{EM/EquidistFourier.lean}{walk\_hit\_count\_fourier\_step}]
\label{thm:fourier-step}
For any target $t \in (\ZZ/q\ZZ)^\times$:
\[
  \bigl|\{n < N : \walkZ(q,n) = t\}\bigr|
  \;=\; \frac{1}{q-1}\sum_{\chi} \overline{\chi(t)}\, S_\chi(N),
\]
where the sum is over all Dirichlet characters mod~$q$.
\end{theorem}

This is a standard Fourier inversion formula on the finite group
$(\ZZ/q\ZZ)^\times$.  To understand it, split the sum into the
contribution from the trivial character $\chi_0$ (which satisfies
$\chi_0(a) = 1$ for all~$a$) and the remaining non-trivial characters:
\[
  \bigl|\{n < N : \walkZ(q,n) = t\}\bigr|
  \;=\; \underbrace{\frac{N}{q-1}}_{\text{uniform share}}
  \;+\; \underbrace{\frac{1}{q-1}\sum_{\chi \neq \chi_0}
    \overline{\chi(t)}\, S_\chi(N)}_{\text{correction}}.
\]
The first term is the ``fair share'' count: if the walk were perfectly
equidistributed among the $q{-}1$ units, each would be visited
$N/(q{-}1)$ times.  The correction term measures the deviation from
uniformity.  Since $|\overline{\chi(t)}| = 1$, each summand in the
correction is bounded by $|S_\chi(N)|$, and there are $q{-}2$
non-trivial characters.  If CCSB holds---all $|S_\chi(N)| = o(N)$---the
correction is $o(N)$, and the hit count is $N/(q{-}1) + o(N)$, which is
eventually positive for every target~$t$, including~$t = -1$.

Returning to our running example ($q = 41$): the group
$(\ZZ/41\ZZ)^\times$ has 40~elements and 40~characters.  The Fourier
bridge says the number of times the walk reaches $\walkZ(41,n) = 40$
(i.e.\ $-1 \bmod 41$) in the first $N$ steps equals $N/40$ plus a
correction bounded by the 39~non-trivial character sums.  Proving CCSB
for $q = 41$ would establish that 41~eventually appears.

\subsection{The Decorrelation--PED--CCSB Chain}
\label{sec:dec-chain}

CCSB is a statement about the \emph{walk} character sums $S_\chi(N) =
\sum_{n<N} \chi(\walkZ(q,n))$.  But the walk is built from the
\emph{multipliers}: $\walkZ(q,n{+}1) = \walkZ(q,n) \cdot
\multZ(q,n)$.  Can we reduce walk equidistribution to a simpler
property of the multiplier sequence?  This subsection formalizes a
chain of progressively weaker hypotheses about the multipliers, each
implying the next via proved bridges, to identify exactly where the
irreducible difficulty lies.

\begin{definition}[\defname{PositiveEscapeDensity} (\abbr{PED})]
For every missing prime~$q$ and non-trivial~$\chi$, there exist
$\delta > 0$ and $N_0$ such that for $N \geq N_0$:
$|\{k < N : \chi(\multZ(q,k)) \neq 1\}| \geq \delta N$.
\end{definition}

The name ``escape'' comes from the SubgroupEscape perspective: the
kernel $\ker(\chi)$ is a proper subgroup of $(\ZZ/q\ZZ)^\times$, and
$\chi(\multZ(q,k)) \neq 1$ means the $k$-th multiplier ``escapes''
from $\ker(\chi)$.  PED asks that a positive fraction of multipliers
escape \emph{every} proper subgroup, not just occasionally but with
positive density.  This is a weak condition---it says nothing about
cancellation or equidistribution, only that the multipliers are not
asymptotically trapped in any subgroup.

\begin{definition}[\defname{DecorrelationHypothesis}]
For every missing prime~$q$ and non-trivial~$\chi$, the multiplier
character sums are $o(N)$:
$\|\sum_{n<N} \chi(\multZ(q,n))\| \leq \varepsilon N$ for large~$N$.
\end{definition}

Decorrelation is stronger than PED: it asks not merely that many
multipliers escape $\ker(\chi)$, but that they do so with enough
balance that the character values cancel.  If the multipliers were
independent random elements of $(\ZZ/q\ZZ)^\times$, the sum would be
$O(\sqrt{N})$ by the law of large numbers---far smaller than
$\varepsilon N$.  Decorrelation asks for the much weaker $o(N)$.

\begin{definition}[\defname{NoLongRuns}$(L)$]
For every missing prime~$q$ and non-trivial~$\chi$, past some
threshold, no $L$ consecutive multipliers all lie in $\ker(\chi)$.
\end{definition}

NoLongRuns is a qualitative cousin of PED: if multipliers never stay
inside $\ker(\chi)$ for $L$~steps in a row, then at least $1/(2L)$ of
them escape.  This condition is easier to verify in practice because it
only requires checking short blocks.

\begin{definition}[\defname{BlockRotationEstimate} (\abbr{BRE})]
If the escape count is $\geq \delta N$, then the walk character sums
are $o(N)$.  This encapsulates the Cauchy--Schwarz / van der Corput
step in harmonic analysis.
\end{definition}

BRE is the bridge between the multiplier-level conditions
(PED/Decorrelation) and the walk-level condition (CCSB).  It says:
given that multipliers escape with positive density, the walk
character sums must cancel.  The intuition is that each escape event
``rotates'' the walk character value $\chi(\walkZ(q,n))$ by a
non-trivial amount, and sufficiently many such rotations produce
cancellation in the sum.  BRE is the sole unproved bridge in the PED
route.

\begin{definition}[\defname{ConditionalMultiplierEquidist} (\abbr{CME})]
\label{sec:cme}%
For every missing prime~$q$, non-trivial~$\chi$, $\varepsilon > 0$,
there exists~$N_0$ such that for $N \geq N_0$ and every walk
position~$c \in (\ZZ/q\ZZ)^\times$:
$\|\sum_{\substack{n < N \\ \walkZ(q,n) = c}} \chi(\multZ(q,n))\|
\leq \varepsilon N$.
\end{definition}

CME is strictly stronger than Decorrelation: it asks for multiplier
character sum cancellation \emph{conditioned on walk position}.
Decorrelation bounds the global sum
$\sum_{n<N} \chi(\multZ(q,n)) = o(N)$; CME bounds the fiber sums
$\sum_{\substack{n<N \\ \walkZ(q,n) = c}} \chi(\multZ(q,n)) = o(N)$
separately for each~$c$.  Since the global sum is the sum of the
fiber sums, CME implies Decorrelation by the triangle inequality
(\lean{EM/LargeSieveAnalytic.lean\#L2481}{cme\_implies\_dec}).
The significance of CME is that it also implies CCSB
\emph{directly}, bypassing PED and BRE entirely.

\begin{theorem}[\lean{EM/EquidistSelfCorrecting.lean}{decorrelation\_implies\_ped}]
Decorrelation $\Rightarrow$ PED.
\end{theorem}

\begin{proof}[Proof sketch]
Contrapositive.  If few multipliers escape $\ker(\chi)$---say fewer
than $\delta N$---then most contribute $\chi(m(n)) = 1$ to the sum.
The at most $\delta N$ exceptions contribute values of norm~$\leq 1$.
By the reverse triangle inequality, $|\sum \chi(m(n))| \geq N - 2\delta N$,
which is $\geq \varepsilon N$ for $\delta$ small enough.  This
contradicts Decorrelation.
\end{proof}

\begin{theorem}[\lean{EM/EquidistSelfCorrecting.lean}{noLongRuns\_implies\_ped}]
NoLongRuns$(L) \Rightarrow$ PED with $\delta = 1/(2L)$.
\end{theorem}

\begin{proof}[Proof sketch]
Partition $\{0, \ldots, N{-}1\}$ into blocks of length~$L$.  Each block
contains at least one escape (by assumption), so the total escape count
is $\geq N/(2L)$.
\end{proof}

\begin{theorem}[\lean{EM/EquidistSelfCorrecting.lean}{block\_rotation\_implies\_ped\_csb}]
BRE $\Rightarrow$ PEDImpliesComplexCSB.
\end{theorem}

The PED route, with all proved arrows:
\[
\mathrm{Dec} \;\xrightarrow{\text{proved}}\; \mathrm{PED}
\;\xleftarrow{\text{proved}}\; \mathrm{NoLongRuns}(L)
\;\xrightarrow{\text{BRE, open}}\;
\mathrm{CCSB}
\;\xrightarrow{\text{proved}}\;
\mathrm{MC}.
\]
The sole open bridge in this route is BRE: converting positive escape
density into walk character sum cancellation.

However, the PED route is not the only path.  CME implies CCSB
\emph{directly}, bypassing PED and BRE entirely
(\lean{EM/LargeSieveAnalytic.lean\#L2567}{cme\_implies\_ccsb}):
\[
\mathrm{CME}
\;\xrightarrow{\text{proved}}\;
\mathrm{CCSB}
\;\xrightarrow{\text{proved}}\;
\mathrm{MC}.
\]
This bypass is significant: the $d \geq 3$ barrier
(Remark~\ref{rem:bre-impossible}) blocks the PED $\to$ CCSB
factorization for characters of order $\geq 3$, but CME $\to$ CCSB
works for \emph{all} character orders, using only the telescoping
identity and fiber decomposition.

\subsection{Walk Telescoping Identities}
\label{sec:telescope}

The following identities are formalized not because they solve CCSB,
but because they reveal \emph{structural constraints} that any proof
of CCSB must navigate.  They are ``no-go'' results that rule out
certain proof strategies.

\begin{theorem}[\lean{EM/EquidistSelfCorrecting.lean}{walk\_telescope\_identity}]
\label{thm:telescope}
For any $\chi$ and $N$:
\[
  \sum_{n < N} \chi(w(n))\cdot\bigl(\chi(m(n)) - 1\bigr)
  \;=\; \chi(w(N)) - \chi(w(0)).
\]
\end{theorem}

This identity follows immediately from the walk recurrence
$\chi(w(n{+}1)) = \chi(w(n)) \cdot \chi(m(n))$: writing
$\chi(w(n)) \cdot (\chi(m(n)) - 1) = \chi(w(n{+}1)) - \chi(w(n))$,
the sum telescopes to $\chi(w(N)) - \chi(w(0))$.

\begin{theorem}[\lean{EM/EquidistSelfCorrecting.lean}{walk\_telescope\_norm\_bound}]
The telescoping sum has norm $\leq 2$ (triangle inequality on
unit-norm terms).
\end{theorem}

The $\leq 2$ bound looks innocent, but it has a sharp consequence.
If we write $S_N = \sum_{n<N} \chi(w(n))$ for the walk character sum,
the telescope identity links $S_N$ to the multiplier character sum
$M_N = \sum_{n<N} \chi(m(n))$.  Specifically, splitting the product
in Theorem~\ref{thm:telescope} gives $S_N \cdot \overline{M_N/N} -
S_N = O(1)$ (after normalization), tightly coupling the walk and
multiplier sums.

\begin{theorem}[\lean{EM/EquidistSelfCorrecting.lean}{walk\_shift\_one\_correlation}]
\label{thm:shift-one}
$\sum_{n < N} \chi(w(n)) \cdot \overline{\chi(w(n+1))}
  = \overline{\sum_{n < N} \chi(m(n))}$.
\end{theorem}

This identity says that the lag-1 autocorrelation of the walk
character equals the conjugate of the multiplier character sum.  It is
a \emph{no-go result} for the van der Corput method with $H = 1$:
VdC bounds $|S_N|^2$ in terms of autocorrelations, but at lag $h = 1$,
the autocorrelation is exactly $|M_N|$---the multiplier character
sum---which need not be small.  So VdC with a single shift gives only
$|S_N| \leq O(\sqrt{N \cdot |M_N|})$, which is $O(N)$ in the worst
case, not the $o(N)$ that CCSB requires.  This means any proof of CCSB
must either (i)~use higher-order correlations (HOD,
Appendix~\ref{sec:sve}) or (ii)~establish multiplier decorrelation
first.

\subsection{The Large Sieve Route}
\label{sec:large-sieve}

Sessions~24--36 developed an extensive large sieve infrastructure
across three files (\code{LargeSieve.lean},
\code{Large\-Sieve\-Harmonic.lean},
\code{Large\-Sieve\-Analytic.lean}) totaling ${\sim}5{,}870$~lines.
This route connects classical analytic number theory to MC via a
multi-modular character sum bound.

\paragraph{Why formalize the large sieve?}
The analytic large sieve inequality and the Bombieri--Vinogradov theorem
are among the most powerful tools in analytic number theory for
controlling the distribution of primes in arithmetic progressions.
If these tools could be applied to the EM walk, MC would follow.
We formalize the connection---not the deep theorems themselves (which
are known results, stated as open Props)---for two reasons:
\begin{enumerate}[nosep]
\item To identify \emph{precisely} what transfer hypothesis is needed to
  apply each classical result to the specific EM orbit, and
\item To verify that six apparently independent routes (BV, ArithLS, ALS,
  PrimeArithLS, LoD, sieve transfer) all reduce to the \emph{same}
  orbit-specificity gap.
\end{enumerate}
This diagnosis is itself a mathematical contribution: it shows that the
difficulty of MC is not a failure of existing analytic tools but a
fundamental obstacle in applying ensemble-averaged results to a single
deterministic orbit.

\begin{definition}[\defname{MultiModularCSB} (\abbr{MMCSB})]
There exists $Q_0$ such that for all $q \geq Q_0$ prime, every
non-trivial character~$\chi$ mod~$q$, and every $\varepsilon > 0$,
there exists $N_0$ such that for $N \geq N_0$:
$\|S_\chi(N)\| \leq \varepsilon N$.
\end{definition}

MultiModularCSB is weaker than CCSB in that it allows finitely many
exceptional primes below~$Q_0$.  This weakening is crucial because
averaged results like BV naturally produce bounds that fail for
finitely many moduli; the threshold mechanism handles those exceptions.

\begin{theorem}[\lean{EM/LargeSieve.lean\#L1054}{mmcsb\_implies\_mc}]
\label{thm:mmcsb-mc}
$\mathrm{MultiModularCSB} \;\Longrightarrow\; \MC$.
\end{theorem}

The proof composes the per-prime Fourier bridge (Theorem~\ref{thm:fourier-step})
with the threshold mechanism (Theorem~\ref{thm:threshold-11}): for
$q \geq Q_0$, MultiModularCSB gives walk equidistribution and hence
HH; for $q < Q_0$, we verify MC directly.  This result was previously
an open hypothesis in \S36; proving it (Session~24) was a key milestone
that unified the large sieve and character sum approaches.

Three parallel routes to MultiModularCSB are formalized:

\paragraph{Bombieri--Vinogradov route.}
The Bombieri--Vinogradov theorem is one of the most powerful results in
analytic number theory.  Roughly, it says that primes are equidistributed
among arithmetic progressions ``on average over moduli'': for most
moduli $q \leq Q = \sqrt{x}/(\log x)^A$, the count of primes $\leq x$
in any progression $a \bmod q$ is close to the expected $\pi(x)/\phi(q)$.
If we could apply this to the EM walk, where the multipliers are primes,
we would get MMCSB and hence MC.

\begin{theorem}[\lean{EM/LargeSieve.lean\#L1145}{bv\_chain\_mc}]
$\mathrm{BV} + \mathrm{BVImpliesMMCSB} \;\Longrightarrow\; \MC$.
\end{theorem}
The transfer hypothesis BVImpliesMMCSB is a \textbf{genuine frontier}:
it requires transferring the averaged equidistribution statement of BV
(valid for primes in generic progressions) to the specific EM walk orbit.
The EM sequence is not a generic sample of primes---it is a deterministic
sequence defined by iterated factorization---so its multipliers could
exhibit special correlations that BV's averaged estimate cannot detect.

The route was decomposed into two stages:
\[
\mathrm{BV}
\xrightarrow{\text{sieve transfer}}
\mathrm{EMMultCSB}
\xrightarrow{\text{walk bridge}}
\mathrm{MMCSB}
\xrightarrow{\text{proved}}
\mathrm{MC},
\]
separating the number-theoretic content (BV~$\Rightarrow$~EMMultCSB,
where EMMultCSB bounds the \emph{multiplier} character sums)
from the dynamical content (EMMultCSB~$\Rightarrow$~MMCSB, converting
multiplier bounds to \emph{walk} bounds).  However, the walk bridge
\textbf{MultCSBImpliesMMCSB is false in general}
(\lean{EM/LargeSieve.lean\#L1243}{MultCSBImpliesMMCSB}):
the walk character sum $\sum \chi(w(n))$ is a \emph{partial product}
$\prod_{k<n} \chi(m(k))$ of the multiplier characters, and partial
products of equidistributed unit complex numbers need not cancel---they
perform a random walk on the unit circle whose norm grows as~$\sqrt{N}$,
not as~$o(N)$.  The telescope identity (Theorem~\ref{thm:shift-one})
makes this obstruction precise: the $h\!=\!1$ autocorrelation equals
the multiplier character sum, so van der Corput with a single shift
gives only~$O(N)$, not~$o(N)$.
This is why the CME bypass (fiber decomposition $+$ telescoping,
Section~\ref{sec:cme}) is essential: it goes directly from
conditional multiplier equidistribution to CCSB without ever
requiring the walk bridge.

\paragraph{Additional sieve routes.}
Two further routes are formalized---the arithmetic large sieve
(ArithLS~$\Rightarrow$~MC, a dead end per Session~35) and the analytic
large sieve (ALS~$\Rightarrow$~PrimeArithLS~$\Rightarrow$~MC), where
the ALS-to-PrimeArithLS bridge via Gauss sum inversion is fully proved
across eight internal lemmas.  In both cases, the genuine open content
is the same \emph{orbit-specificity transfer}: applying averaged
results to one deterministic orbit.  The full details, including the
ALS definition, weak ALS proof, Gauss sum inversion theorem, and a
spectral energy route (SVE, van der Corput, HOD, CME) with its
complete hypothesis hierarchy, appear in Appendix~\ref{app:routes}.

%% =========================================================================
\section{Why It's Hard}
\label{sec:hard}
%% =========================================================================

\noindent\textbf{Original contribution.}\enspace
The selectability analysis, oracle barrier, and CCSB-as-frontier
argument below are new.  They explain \emph{why} the remaining
hypothesis resists both computation and existing analytic tools.

\subsection{The Selectability Perspective}
\label{sec:selectability}

The Euclid construction guarantees fresh primes at every step: every
prime factor of~$\Prod(n)+1$ is new (coprime to the running product).
The difficulty of MC is not the \emph{existence} of new primes but
whether the $\minFac$ rule eventually \emph{selects} each one.  The
formalization makes this contrast precise.

\begin{theorem}[\lean{EM/EquidistOrbitAnalysis.lean}{divisor\_not\_yet\_in\_seq}]
\label{thm:divisor-fresh}
If $p \mid \Prod(n)+1$, then $\seq(m) \neq p$ for all $m \leq n$.
\end{theorem}

\begin{proof}
Any $\seq(m)$ with $m \leq n$ divides $\Prod(n)$.  A number $\geq 2$
cannot divide both~$a$ and~$a+1$.
\end{proof}

\begin{theorem}[\lean{EM/EquidistOrbitAnalysis.lean}{passed\_over\_persists}]
If $p \mid \Prod(n)+1$ but $\seq(n\!+\!1) \neq p$ (the $\minFac$ rule
chose a smaller prime), then $\seq(m) \neq p$ for all $m \leq n+1$.
The prime survives to potentially divide future Euclid numbers.
\end{theorem}

\begin{theorem}[\lean{EM/EquidistOrbitAnalysis.lean}{selectability\_extinguished}]
\label{thm:extinct}
Once $\seq(m) = p$, we have $p \mid \Prod(n)$ for all $n \geq m$, so
$p \nmid \Prod(n)+1$ ever again.  Selectability is a one-shot resource.
\end{theorem}

\begin{definition}[\defname{InfinitelySelectable}]\label{def:inf-sel}
A prime~$p$ is \emph{infinitely selectable} if $p \mid \Prod(n)+1$ for
cofinally many~$n$: $\forall\, N,\, \exists\, n \geq N,\,
p \mid \Prod(n)+1$.
\end{definition}

By Theorem~\ref{thm:extinct}, MC$(p)$ and
Infinitely\-Selectable$(p)$ are mutually exclusive
(\lean{EM/EquidistOrbitAnalysis.lean}%
{mc\_implies\_not\_infinitely\_selectable}):
a prime that enters the sequence can never be selectable again.

\begin{theorem}[\lean{EM/EquidistOrbitAnalysis.lean}{dh\_implies\_infinitely\_selectable}]
\label{thm:dh-inf-sel}
Under DH, every prime that never appears in the sequence (with SE
satisfied) is infinitely selectable.
\end{theorem}

\paragraph{The random-factor variant is easy.}
Consider a variant of the Euclid--Mullin construction where, instead of
the smallest prime factor, one picks a \emph{random} prime factor
of~$\Prod(n)+1$ at each step.  In this variant, MC follows from DH
alone: whenever $p \mid \Prod(n)+1$, simply choose~$p$.  Under DH with
SE, this happens infinitely often (Theorem~\ref{thm:dh-inf-sel}), so
$p$ eventually gets picked.

The argument is even simpler probabilistically.  For any target
prime~$p$, the residue $r_n = \Prod(n) \bmod p$ performs a
multiplicative walk on $(\ZZ/p\ZZ)^\times$.  In the random-factor
variant, each multiplier is a random element of the group; once the
multipliers generate the full group (which PRE guarantees), the walk
is a genuine random walk with full support.  By classical
equidistribution on finite groups, $r_n$ converges to uniform,
so $r_n = -1$ (i.e., $p \mid \Prod(n)+1$) occurs with
probability $\to 1/(p\!-\!1)$---infinitely often with
probability~1.

\paragraph{The lpf variant is hard.}
The actual Euclid--Mullin sequence uses
$\seq(n{+}1) = \minFac(\Prod(n){+}1)$, a \emph{deterministic}
function of the walk position.  This creates the exact correlation
identified by the oracle analysis (\S\ref{sec:oracle}): the
multiplier at step~$n$ depends on the full value of~$\Prod(n)+1$,
coupling walk position to multiplier.  The random-factor variant
breaks this coupling by choosing multipliers independently of
position; the $\minFac$ rule preserves it.

The difficulty of Mullin's Conjecture is entirely in the minimality of
the prime selection, not in the Euclidean construction itself.  The
inductive bootstrap (Section~\ref{sec:bootstrap}) bridges this gap:
$\MC({<}\,p)$ ensures all primes below~$p$ are already in the sequence,
hence divide~$\Prod(n)$, hence cannot divide~$\Prod(n)+1$.  Past a
computable stage, $p$ is the \emph{smallest} available factor whenever
it divides the Euclid number---reducing the $\minFac$ variant to the
``any-factor'' variant for the tail of the sequence.

\subsection{The Marginal/Joint Barrier}
\label{sec:oracle}

The verified reductions (TailSE, CofinalEscape, QuotientDH) exhaust
what can be proved about the \emph{marginal} distribution of multiplier
residues.

\begin{theorem}[\lean{EM/EquidistOrbitAnalysis.lean}{emfe\_iff\_tail\_se\_at}]
$\mathrm{EuclidMinFacEscape}(q) \Leftrightarrow \mathrm{TailSE}(q)$.
\end{theorem}

Even perfect per-position equidistribution of multipliers is consistent
with HH failure.  DH is a \emph{joint} statement---the
(position, multiplier) pair must hit the \emph{death curve}
$\multZ(q,n) = -\walkZ(q,n)^{-1}$---and no marginal statement can
force this.

\paragraph{The orbit chain gap.}
The cofinal orbit analysis picks one cofinal multiplier~$s_x$ per
walk position, producing a cycle $x_0 \to x_1 \to \cdots \to x_0$
in $(\ZZ/q\ZZ)^\times$.  Even when the cofinal multipliers generate
the full group, the cycle size~$k$ can be less than~$|(\ZZ/q\ZZ)^\times|$.
Example: in $\ZZ/6\ZZ$, the cycle $0 \to 1 \to 0$ has
$\langle 1, 5 \rangle = \ZZ/6\ZZ$ but misses~3.

Closing this gap requires showing that at each cofinally visited
position, \emph{multiple} multiplier classes appear---expanding the
cycle until it covers $-1$.  This is the ``specific-orbit problem'':
transferring generic equidistribution of $\minFac$ residues to the
particular EM orbit.

\subsection{The BRE Impossibility for $d \geq 3$}

\begin{remark}\label{rem:bre-impossible}
Positive escape density (PED) alone does \emph{not} imply
CCSB for characters of order $d \geq 3$.

\emph{Counterexample}: a walk on $\ZZ/3\ZZ$ that alternates between
only two of the three $d$-th roots of unity (phase-aligned escapes)
achieves positive escape density yet has walk sum
$\approx N/2 \cdot (1 + \omega) \neq o(N)$.

For $d = 2$ this degeneracy vanishes: the only non-trivial rotation
is~$-1$, so escape frequency \emph{is} the rotation distribution.
The order-2 BRE from NoLongRuns$(L)$ is proved in the formalization
(\lean{EM/EquidistSelfCorrecting.lean}{order2\_noLongRuns\_mc}).
But for $d \geq 3$, PED constrains how often the walk rotates without
constraining the \emph{distribution} among $d-1$ non-identity
rotations.  The $\mathrm{PED} \Rightarrow \mathrm{BRE} \Rightarrow
\mathrm{CCSB}$ factorization is invalid for $d \geq 3$.

This barrier is specific to the PED route.  The CME $\to$ CCSB
reduction (\lean{EM/LargeSieveAnalytic.lean\#L2567}{cme\_implies\_ccsb})
bypasses PED and BRE entirely, working for all character orders $d$
via the telescoping identity.  The $d \geq 3$ problem is therefore
not a barrier for the \emph{reduction}---only for the particular
factorization through PED.
\end{remark}

\subsection{The Van der Corput Barrier}

The van der Corput inequality (Theorem~\ref{thm:vdc}, now fully proved
in the formalization) converts character sum bounds into autocorrelation
bounds.  Theorem~\ref{thm:shift-one} gives $R_1 = o(N)$ under the
Decorrelation Hypothesis.  VdC with $H = 1$
yields $|S_N|^2 \leq \frac{N+1}{2}(N + 2|R_1|) = N^2/2 + o(N^2)$,
hence $|S_N| \leq N/\sqrt{2}$.  This is non-trivial but \emph{not}
$o(N)$.  To get $o(N)$, one needs higher-order correlations $R_h = o(N)$
for $h \geq 2$, which requires HigherOrderDecorrelation
(Theorem~\ref{thm:hod-mc}).
The telescoping identity $\sum_n \chi(w(n))(\chi(m(n))-1) = O(1)$
is a precise structural constraint.

\subsection{The Walk Bridge Falsity}
\label{sec:walk-bridge-false}

The BV route decomposes into two stages: sieve transfer
(BV~$\Rightarrow$~EMMultCSB, bounding \emph{multiplier} character sums)
and the walk bridge (EMMultCSB~$\Rightarrow$~MMCSB, converting multiplier
bounds to \emph{walk} bounds).  The walk bridge
\textbf{MultCSBImpliesMMCSB}
(\lean{EM/LargeSieve.lean\#L1243}{MultCSBImpliesMMCSB})
is stated as an open \code{Prop} and is \textbf{false in general}.

The obstruction is structural: the walk character sum is a
\emph{partial product} $\chi(w(n)) = \prod_{k<n} \chi(m(k))$
of the multiplier characters.  Even when the individual factors
$\chi(m(k))$ are equidistributed on the unit circle (so their
\emph{sum} cancels), their \emph{partial products} perform a
multiplicative walk whose norm grows as~$\sqrt{N}$,
not~$o(N)$.  Cancellation of sums does not imply cancellation
of cumulative products.

This negative result explains why the CME bypass
(Section~\ref{sec:cme}) is essential.  CME uses fiber
decomposition and telescoping to go directly from conditional
multiplier equidistribution to CCSB, circumventing the walk bridge
entirely.

\subsection{Dead Ends as a Roadmap}

Across 22~formalization sessions, dozens of potential approaches were
explored and found to be dead ends.  Each elimination is informative:
it narrows the space of viable strategies.  A selection of twelve,
grouped by the type of obstruction:

\smallskip
\begin{center}
\renewcommand{\arraystretch}{1.15}
\small
\begin{tabular}{@{}p{3.6cm}p{8.8cm}@{}}
\toprule
\textbf{Dead end} & \textbf{Why it fails} \\
\midrule
\multicolumn{2}{@{}l}{\textsc{Ensemble-to-orbit transfer}:
  \emph{tool applies to generic sequences, not the specific EM orbit}} \\[2pt]
BV for EM subsequence
  & BV applies to all primes in APs, not to a greedy subsequence. \\
Furstenberg / ergodic theory
  & Standard ergodic methods assume classical multiplicativity;
    the EM sequence is recursive and non-multiplicative. \\
Diaconis--Shahshahani lemma
  & Requires i.i.d.\ random steps; inapplicable to the
    deterministic EM walk. \\
\midrule
\multicolumn{2}{@{}l}{\textsc{Independence / linearity violated}:
  \emph{tool requires additive or independent structure the walk lacks}} \\[2pt]
Large sieve for partial products
  & The large sieve handles linear sums, not multiplicative walks. \\
Abel summation
  & Converts multiplier decorrelation to walk-sum bounds, but the
    summation weights \emph{amplify} rather than cancel. \\
Self-avoidance $\Rightarrow$ CCSB
  & Self-avoidance (no repeated $\hat{\ZZ}$ positions) is invisible
    to characters, which see only residues. \\
\midrule
\multicolumn{2}{@{}l}{\textsc{Wrong algebraic structure}:
  \emph{the group or decomposition has no room for the desired bound}} \\[2pt]
Non-abelian / representation
  & $(\ZZ/q\ZZ)^\times$ is cyclic; all irreps are 1-d characters.
    No higher-dimensional structure to exploit. \\
CRT product group
  & Reformulating on $\prod_{q \leq Q}(\ZZ/q\ZZ)^\times$ makes
    the problem harder: the product group is exponentially large. \\
NoLongRuns + PED $\Rightarrow$ BRE ($d \!\geq\! 3$)
  & Variable block lengths align adversarially with character phases. \\
DPED $\Rightarrow$ CCSB ($d \!\geq\! 3$)
  & Alternating $\omega,\omega^2$ rotations satisfy DPED yet produce
    $\Theta(N)$ walk sums.  All PED-to-CCSB intermediates ruled out. \\
\midrule
\multicolumn{2}{@{}l}{\textsc{Reduces to single-modulus CCSB}:
  \emph{no genuine simplification}} \\[2pt]
Multi-modular approaches
  & All variants (BV + threshold, CRT, death coupling) collapse
    to single-modulus CCSB. \\
Death set coupling across moduli
  & Death sets $\{m : \minFac(m) \equiv -c^{-1}\}$ vary per step;
    no uniform coupling bound exists. \\
\bottomrule
\end{tabular}
\renewcommand{\arraystretch}{1.0}
\end{center}

\smallskip\noindent
The pattern: every approach that avoids the specific EM orbit's
joint distribution either reduces to CCSB or fails.  Since CME
implies CCSB (proved), the sharpest target is now CME:
conditional equidistribution of multipliers given walk position.
CME is strictly weaker than CCSB and is the \emph{irreducible
analytic content}.

\subsection{The Mathematical Landscape}

We need to prove one of these equivalent statements for every missing
prime~$q$:
\begin{itemize}[nosep]
\item \textbf{DH}: If the multipliers generate $(\ZZ/q\ZZ)^\times$,
  the walk $\walkZ(q,n) = \Prod(n) \bmod q$ hits $-1$ cofinally.
\item \textbf{CCSB}: For every non-trivial character
  $\chi \colon (\ZZ/q\ZZ)^\times \to \mathbb{C}^\times$, the sum
  $\sum_{n < N} \chi(\walkZ(q,n)) = o(N)$.
\item \textbf{$d{=}2$ special case} (as a stepping stone): For every
  quadratic character~$\chi$, the $\pm 1$-valued walk character sum is
  $o(N)$.
\end{itemize}
The formalization has conclusively shown that every tool requiring
independence, classical multiplicativity, or ensemble averaging fails.
So we need ideas that exploit what the EM walk specifically has.

\subsection{Structural Features of the EM Walk}

The dead ends above show what does not work.  Complementarily, the EM
walk has four structural features---all proved or formalized---that no
dead-end approach has successfully exploited.  Any proof of MC will
almost certainly use at least one.

\medskip\noindent\textbf{Feature~1: Super-exponential growth.}\enspace
$\Prod(n) \geq 2^n$
(\lean{EM/LargeSieve.lean}{prod\_lower\_bound\_for\_sieve}).  The
Euclid numbers $\Prod(n)+1$ grow absurdly fast.  This means the
\emph{sieve level}---the threshold below which all prime factors have
been excluded---grows super-exponentially.  By step~$n$, the Euclid
number $\Prod(n)+1$ is coprime to each of $\seq(0), \ldots, \seq(n)$,
a growing set of distinct primes.  The pool of ``available'' small
primes as factors of $\Prod(n)+1$ shrinks, but the size of
$\Prod(n)+1$ grows so fast that it must have enormous prime factors
most of the time.

\medskip\noindent\textbf{Feature~2: Mutual coprimality of Euclid
numbers.}\enspace For $m > n$, $\Prod(m)$ is divisible by
$\seq(n+1)$, which divides $\Prod(n)+1$.  So
$\Prod(m)+1 \equiv 1 \pmod{\seq(n+1)}$: successive Euclid numbers
live in different residue classes modulo earlier sequence terms.  This
coprimality structure means the Euclid numbers cannot all ``avoid'' a
residue class in a coordinated way---their residues are forced apart
by the construction.

\medskip\noindent\textbf{Feature~3: The multiplier is the smallest
prime factor.}\enspace  This is the key constraint that everyone
mentions but nobody has quantified.  If $\walkZ(q,n) \neq -1$ (so
that $q$ does not divide $\Prod(n)+1$ as the smallest factor), then
the multiplier $\multZ(q,n) = \minFac(\Prod(n)+1)$ satisfies
$\multZ(q,n) \leq (\Prod(n)+1)^{1/2}$.  For a number of size
${\sim}2^n$, this smallest factor could be as small as~$3$ or as
large as~${\sim}2^{n/2}$.  The $\minFac$ rule creates a deterministic
coupling between walk position and multiplier: the multiplier at
step~$n$ depends on the full value of~$\Prod(n)+1$, not just its
residue.

\medskip\noindent\textbf{Feature~4: Self-correcting feedback.}\enspace
If the walk concentrates on certain residues mod~$q$---say
$\walkZ(q,n) \equiv a \pmod{q}$ for many~$n$---then
$\Prod(n)+1 \equiv a+1 \pmod{q}$ for many~$n$.  The smallest prime
factor of numbers $\equiv a+1 \pmod{q}$ depends on~$a+1$, creating a
feedback loop: concentration in one residue class biases the
multiplier distribution, which in turn pushes the walk away from that
class.  This self-correcting mechanism has been formalized
(\code{EquidistSelfCorrecting.lean}), but all paths from it lead to
\textsc{SieveTransfer}---the open hypothesis that generic
$\minFac$~equidistribution transfers to the specific EM orbit.

\medskip
These four features---growth, coprimality, the $\minFac$ selection
rule, and self-correcting feedback---are the raw material that any
successful approach must engage with.  The dead ends above fail
precisely because they treat the walk generically (as a random walk,
or as an arbitrary multiplicative walk) rather than exploiting the
specific arithmetic of the EM construction.

%% =========================================================================
\section{The Lean Formalization}
\label{sec:lean}
%% =========================================================================

\subsection{Codebase Structure}

The formalization uses Lean~4 with Mathlib~v4.27.0 across 32~files
totaling ${\sim}22{,}400$~lines.  The dependency chain is linear, with
three leaf modules:

\begin{center}
\small
\begin{tabular}{llr}
\toprule
\textbf{File} & \textbf{Content} & \textbf{Lines} \\
\midrule
\code{Euclid.lean} & Constructive Euclid's theorem & 425 \\
\code{MullinDefs.lean} & \code{seq}, \code{prod}, \code{aux}, identities & 527 \\
\code{MullinConjectures.lean} & MC, Conjecture A (FALSE), HH & 494 \\
\code{MullinDWH.lean} & DivisorWalkHypothesis (leaf) & 551 \\
\code{MullinResidueWalk.lean} & WalkCoverage, residue walk, concrete MC & 605 \\
\code{MullinGroupCore.lean} & walkZ, multZ, confinement, SE & 422 \\
\code{MullinGroupEscape.lean} & Escape lemmas, 8-element argument & 673 \\
\code{MullinGroupSEInstances.lean} & 30 concrete SE instances ($q \leq 157$) & 364 \\
\code{MullinGroupPumping.lean} & Gordon sequenceability (leaf) & 343 \\
\code{MullinGroupQR.lean} & QR obstruction ($\leq 1.6\%$) (leaf) & 683 \\
\code{RotorRouter.lean} & Scheduled walk coverage (standalone) & 421 \\
\code{MullinRotorBridge.lean} & EMPR + SE $\Rightarrow$ MC bridge & 87 \\
\code{EquidistPreamble.lean} & PE $\Rightarrow$ MC, bootstrapping & 234 \\
\code{EquidistSieve.lean} & Sieve, WHP $\Leftrightarrow$ HH & 297 \\
\code{EquidistSelfAvoidance.lean} & Self-avoidance, periodicity & 450 \\
\code{EquidistCharPRE.lean} & Character non-vanishing, PRE $\Leftrightarrow$ SE & 811 \\
\code{EquidistBootstrap.lean} & Inductive bootstrap, DH $\Rightarrow$ MC & 522 \\
\code{EquidistThreshold.lean} & ThresholdHitting(11) $\Rightarrow$ MC & 299 \\
\code{EquidistOrbitAnalysis.lean} & Cofinal orbits, quotient walk, sieve, selectability & 1441 \\
\code{EquidistFourier.lean} & Character sums, Fourier bridge & 1298 \\
\code{EquidistSelfCorrecting.lean} & Decorrelation, BRE, telescoping, kernel (\S31--\S37, \S72) & 1114 \\
\code{EquidistSieveTransfer.lean} & Prime density, sieve transfer, walk decomp (\S38--\S78) & 1319 \\
\code{LargeSieve.lean} & BV, ALS, ArithLS, MMCSB, sieve bridge (\S41--\S52, \S79) & 1812 \\
\code{LargeSieveHarmonic.lean} & Parseval, Gauss sums, DFT, kernel (\S53--\S55) & 892 \\
\code{LargeSieveAnalytic.lean} & Gauss inversion, WeakALS, GCT (\S56--\S65) & 1438 \\
\code{LargeSieveSpectral.lean} & Walk energy, HOD, VdC, CME, SVE (\S66--\S78) & 1685 \\
\midrule
\code{IKCh1.lean} & Iwaniec--Kowalski~\cite{IwaniecKowalski2004} Ch.\,1: arithmetic functions & 437 \\
\code{IKCh2.lean} & Iwaniec--Kowalski Ch.\,2: summation formulas & 270 \\
\code{IKCh3.lean} & Iwaniec--Kowalski Ch.\,3: combinatorial sieve & 557 \\
\code{IKCh4.lean} & Iwaniec--Kowalski Ch.\,4: summation formulas & 593 \\
\code{IKCh5.lean} & Iwaniec--Kowalski Ch.\,5: Kloosterman sums & 877 \\
\code{IKCh7.lean} & Iwaniec--Kowalski Ch.\,7: bilinear forms, large sieve & 455 \\
\bottomrule
\end{tabular}
\end{center}

\subsection{Axiom Usage: What's Constructive}

The core definitions (\code{seq}, \code{prod}, \code{aux}) and
their basic properties (\code{seq\_isPrime}, \code{seq\_injective})
are \textbf{fully constructive}: they use only \code{propext} and
\code{Quot.sound} (no \code{Classical.choice}, no
\code{Decidable} instances beyond~$\NN$).  Euclid's theorem
itself (\code{Euclid.lean}) is constructive.

Classical reasoning enters at the reduction level:
\begin{itemize}[nosep]
\item The $\HH \Rightarrow \MC$ proof uses well-founded induction
  (strong induction on $\NN$), which in Lean~4 is constructive but
  relies on \code{Classical.choice} for the
  cofinal-implies-hit argument.
\item Character theory (orthogonality, Fourier inversion) is
  inherently classical via \code{open Classical}.
\item All open hypotheses are stated as \code{def \ldots : Prop},
  never as \code{sorry}'d theorems.  The type-checker guarantees
  that no proof obligation is silently assumed.
\end{itemize}

\subsection{Mathlib Dependencies}

The formalization draws on several Mathlib libraries:
\begin{itemize}[nosep]
\item \textbf{Group theory}: \code{Subgroup}, \code{QuotientGroup},
  \code{orderOf}, cyclic group structure, maximal subgroups
  (\code{Subgroup.IsCoatom}).
\item \textbf{Number theory}: \code{Nat.minFac}, Legendre symbols,
  quadratic reciprocity, \code{ZMod}, Dirichlet characters, Gauss sums.
\item \textbf{Character theory}: \code{DirichletCharacter.Orthogonality},
  roots of unity in algebraically closed fields, character bounds,
  \code{MulChar.sum\_eq\_zero\_of\_ne\_one}.
\item \textbf{Analysis}: \code{norm\_sum\_le}, complex norms,
  \code{IsOfFinOrder.norm\_eq\_one}, Fourier analysis on $\ZZ/n\ZZ$
  (\code{ZMod.dft}, discrete Fourier transform).
\item \textbf{Dirichlet's theorem}: \code{Nat.infinite\_setOf\_prime\_and\_eq\_mod}
  (primes in arithmetic progressions, via $L$-series).
\item \textbf{Harmonic analysis}: Parseval's theorem for finite abelian
  groups, trigonometric exponentials, geometric series identities.
\end{itemize}

\subsection{Verification Statistics}

\begin{center}
\begin{tabular}{lr}
\toprule
Lines of Lean code & ${\sim}22{,}400$ \\
Files & 32 \\
Theorems/lemmas & ${\sim}770$ \\
Definitions & ${\sim}460$ \\
\code{sorry} occurrences & \textbf{0} \\
Open hypotheses (stated as \code{def}) & ${\sim}26$ \\
Concrete SE instances & 30 \\
Computed sequence terms & 8 \\
Mathlib version & v4.27.0 \\
\bottomrule
\end{tabular}
\end{center}

%% =========================================================================
\section{Open Problems}
\label{sec:open}
%% =========================================================================

\subsection{CCSB as the Precise Frontier}

The formalization identifies \textbf{ComplexCharSumBound} as the
irreducible analytic content.  The question:

\begin{center}
\emph{Are the walk character sums $\sum_{n<N} \chi(\walkZ(q,n))$ bounded
$o(N)$ for every non-trivial $\chi$?}
\end{center}

CCSB is a single hypothesis that implies MC with no additional
conditions.  It is equivalent to walk equidistribution mod~$q$, which
is the strongest ``uniform'' version of DH.

The walk telescoping identities (Section~\ref{sec:character}) provide
precise structural constraints.  The identity
$\sum_n \chi(w(n))(\chi(m(n))-1) = O(1)$ means that the walk sum
$S_N$ and the multiplier sum $M_N = \sum_n \chi(m(n))$ satisfy
$S_N \approx S_N + (M_N - S_N) = M_N + O(1)$ only in the crude sense;
the telescoping does \emph{not} separate them.

\subsection{Connection to Bombieri--Vinogradov}

A Bombieri--Vinogradov type result for EM walk residues would give:
\[
\sum_{\substack{q \leq Q \\ q\text{ prime}}}
\max_{a}
\Bigl| |\{n \leq N : w(n) \equiv a \pmod{q}\}|
  - \frac{N}{q-1} \Bigr|
\;\ll\; \frac{NQ}{(\log N)^A}.
\]
For non-exceptional primes, the walk equidistributes; exceptional
primes (finitely many) are handled by FiniteMCBelow.  Combining with
$\mathrm{ThresholdHitting}(11) \Rightarrow \MC$ would close the
conjecture.

The difficulty is that BV applies to the set of \emph{all} primes,
not to a specific subsequence.  The EM walk is deterministic and
self-referential: the walk at step~$n$ depends on the factorization
of~$\Prod(n)+1$, which depends on all previous walk values.  Standard
BV does not apply.

\subsection{Connection to Chebotarev}

The \textbf{EffectiveKummerEscape} hypothesis asserts: for each
prime~$\ell$, there exists~$B$ such that for $q \geq B$ with
$\ell \mid q\!-\!1$, some multiplier among the first~$B$ escapes the
$\ell$-th power kernel.  This is a Chebotarev-type statement for the
Kummer extension $\mathbb{Q}(\zeta_\ell, 3^{1/\ell}, \ldots,
53^{1/\ell})$: the Frobenius at~$q$ determines which multiplier
primes are $\ell$-th power residues.

An effective Chebotarev density theorem for this fixed number field
would give EKE for all but finitely many~$q$ (effectively bounded).
Combined with finite verification for the remaining~$q$, this would
prove PRE and hence SE unconditionally---but SE is
\emph{already} proved unconditionally via the elementary PRE.
The Chebotarev approach would give a stronger \emph{effective}
bound on how quickly SE kicks in.

\subsection{The Sieve-Theoretic Approach}

\textbf{Mertens\-Escape}: for any prime~$q$ and proper
subgroup~$H$, infinitely many primes outside~$H$ exist
(Dirichlet content).
\textbf{Sieve\-Amplification}: Mertens escape should force
eventual $\minFac(\Prod(n){+}1)$ escape from~$H$, via
super-exponential growth and mutual coprimality of
successive Euclid numbers.

The formally verified chain:
$\mathrm{MertensEscape} + \mathrm{SieveAmplification}
\xrightarrow{\text{proved}}
\mathrm{TailSE} \xrightarrow{\text{proved}}
\mathrm{CofinalEscape} \xrightarrow{\text{proved}}
\mathrm{QuotientDH}$.

Sessions~19--22 articulated a richer sieve infrastructure in two parallel
routes (\S38--\S39 of \code{EquidistSelfCorrecting.lean}):

\paragraph{Cumulative route.}
\begin{gather*}
\mathrm{PDE}
\xrightarrow{\text{Alladi}}
\mathrm{GLPFE}
\xrightarrow{\mathrm{SieveTransfer}}
\mathrm{SieveEquidist}
\xrightarrow{\text{open}}
\mathrm{NoLongRuns} \\
\xrightarrow{\text{proved}}
\mathrm{PED}
\xrightarrow{\text{open}}
\mathrm{CCSB}
\xrightarrow{\text{proved}}
\mathrm{MC}.
\end{gather*}
Here PDE is PrimeDensityEquipartition (PNT in arithmetic progressions,
a known theorem not yet in Mathlib), and GLPFE is GenericLPFEquidist
(Alladi's theorem~\cite{Alladi1977} on $\minFac$ distribution of generic integers,
also known but not formalized).  Both ends of the chain---from PDE to
GLPFE via Alladi, and from CCSB to MC via Fourier inversion---are
formally proved.

\paragraph{Window route.}
\[
\mathrm{StrongSieveEquidist}
\xrightarrow{\text{proved}}
\mathrm{NoLongRunsAt}
\xrightarrow{\text{proved}}
\mathrm{PEDAt}
\xrightarrow{\text{open}}
\mathrm{CCSB}
\xrightarrow{\text{proved}}
\mathrm{MC}.
\]
StrongSieveEquidist asserts window equidistribution of EM multipliers
within sliding windows; NoLongRunsAt and PEDAt are per-prime variants
proved by pigeonhole and block-counting respectively
(\lean{EM/EquidistSelfCorrecting.lean\#L1123}{strongSieveEquidist\_noLongRunsAt},
\lean{EM/EquidistSelfCorrecting.lean\#L1192}{noLongRunsAt\_ped}).

\paragraph{The genuine frontier.}
\textbf{SieveTransfer} is the critical open hypothesis: does the
equidistribution of $\minFac$ residues for generic integers transfer
to the specific EM orbit?  Everything above SieveTransfer is known
mathematics; everything below it is proved.  SieveTransfer is where
``known but not formalized'' meets ``genuinely open.''

The difficulty: this applies to \emph{generic} integers whose
$\minFac$ residues are equidistributed (by CRT + Mertens), not to
the specific EM orbit.  Transferring from ensemble to specific orbit
is the open step.

\paragraph{Sieve-to-harmonic convergence.}
The sieve hierarchy (\S36--\S39 of \code{EquidistSelfCorrecting.lean})
and the harmonic hierarchy (\S30--\S35) converge: both produce
DecorrelationHypothesis as output.  The full chain
\[
\mathrm{SieveEquidist}
\xrightarrow{\text{proved}}
\mathrm{Dec}
\xrightarrow{\text{proved}}
\mathrm{PED}
\xrightarrow[\text{sole gap}]{\text{open}}
\mathrm{CCSB}
\xrightarrow{\text{proved}}
\mathrm{MC}
\]
is formalized, with the first two arrows machine-verified
(\lean{EM/LargeSieve.lean\#L1634}{sieve\_equidist\_implies\_decorrelation},
\lean{EM/EquidistSelfCorrecting.lean\#L143}{decorrelation\_implies\_ped}).
The sieve route achieves SieveEquidist~$\Rightarrow$~Dec via a
counting-to-character-sum bridge: SieveEquidistribution produces
\code{EMMultCharSumBound} with $Q_0 = 0$, meaning multiplier character
sums cancel for \emph{all} primes~$q$, which is exactly
DecorrelationHypothesis.  The sole remaining gap on this route is
\textbf{PEDImpliesComplexCSB}
(\lean{EM/EquidistSelfCorrecting.lean\#L113}{PEDImpliesComplexCSB}):
does positive escape density for all primes imply walk character sum
cancellation?  Any proof of SieveEquidistribution (e.g., from PNT
in APs $+$ Alladi's theorem) would immediately yield Dec and PED
for free, isolating this single bridge as the only open step.

\subsection{What Would Close the Conjecture}

The cleanest paths to MC:
\begin{enumerate}
\item \textbf{Prove CME} (sharpest target): show that the multiplier
  character sum $\sum_{\substack{n < N \\ w(n) = c}} \chi(m(n))$ is
  $o(N)$ for each walk position~$c$.  CME is strictly weaker than
  CCSB, and $\mathrm{CME} \Rightarrow \mathrm{CCSB}$ is proved
  (\lean{EM/LargeSieveAnalytic.lean\#L2567}{cme\_implies\_ccsb}).
  CME asks only about the \emph{conditional} distribution of
  multipliers given walk state---it does not require controlling the
  walk character sum itself.  This bypasses the $d \geq 3$ barrier
  entirely.

\item \textbf{Prove CCSB directly}: show that the deterministic
  product walk cannot maintain character-sum bias $\Theta(N)$.  The
  self-correcting sieve (concentration of EM primes in a residue class
  is exponentially self-limiting) is the strongest heuristic argument.

\item \textbf{Prove a BV-type estimate}: even an averaged version
  over~$q$ would suffice, combined with ThresholdHitting(11).

\item \textbf{Close the orbit chain gap}: show that at each cofinally
  visited walk position, at least two distinct multiplier classes
  appear.  This would force the orbit chain to expand to the full group.

\item \textbf{Prove DH directly}: show that a multiplicative walk
  on a cyclic group with a generating set of multipliers must hit
  every element cofinally.  This is a combinatorial question about
  deterministic walks.

\item \textbf{Prove SieveTransfer}: show that the EM orbit's $\minFac$
  distribution matches that of generic integers, at least on average.
  The cumulative sieve route (\S38) reduces this to known number theory
  (PNT in APs + Alladi's theorem); closing SieveTransfer gives CME
  (and hence CCSB) via the conditional multiplier equidistribution
  framework.

\item \textbf{Prove BVImpliesMMCSB or PrimeArithLSImpliesMMCSB}: the
  large sieve route (\S41--\S65) reduces MC to a transfer hypothesis.
  Given Bombieri--Vinogradov or the analytic large sieve, the remaining
  open step is transferring the averaged or generic equidistribution to
  the specific EM walk.  This is the same orbit-specificity gap as
  SieveTransfer, approached from a different mathematical toolkit.
\end{enumerate}

\subsection{Is DynamicalHitting True?}

The formalization proves $\mathrm{DH} \Rightarrow \mathrm{MC}$ but
says nothing about whether DH itself holds.  Intellectual honesty
requires a frank assessment.

\paragraph{Evidence for.}
Three independent lines of evidence suggest DH is true.
(1)~\emph{Computation}: every prime below 41 has been verified to
appear in the EM sequence (51~terms computed), and the walk hits
$-1 \pmod{q}$ for all tested primes $q$ with no counterexample.
(2)~\emph{Self-correcting feedback}: the formalized sieve analysis
(\code{EquidistSelfCorrecting.lean}) shows that concentration of EM
primes in a residue class is exponentially self-limiting---a walk
biased toward missing $-1$ automatically biases the multiplier
distribution toward correcting that miss.
(3)~\emph{Analogy with Artin's conjecture}: Artin's conjecture
(that every non-square integer is a primitive root for infinitely
many primes) has the same orbit-specificity structure and is believed
true; Hooley~\cite{Hooley1967} proved it conditional on GRH.  DH is the analogous
statement for the EM walk and would follow from an analogous
uniformity hypothesis.

\paragraph{Evidence against.}
Two features of the EM sequence give pause.
(1)~\emph{Cox--van der Poorten}~\cite{CoxVdP1968}: the ``largest factor'' variant of
the Euclid--Mullin sequence provably misses primes.  The EM
sequence's completeness is not a soft consequence of the Euclid
construction but depends sensitively on the $\minFac$ selection
rule.  This fragility means heuristic arguments
(``it should work because Euclid numbers have many factors'') are
not reliable.
(2)~\emph{The $d \geq 3$ barrier}: the formalization proves that
the most natural route from multiplier equidistribution to walk
equidistribution (PED~$\Rightarrow$~BRE~$\Rightarrow$~CCSB) is
\emph{impossible} for character orders $d \geq 3$.  This is not
evidence that DH is false, but it shows that the truth of DH, if
it holds, requires mechanisms beyond the simplest equidistribution
framework.  The CME bypass sidesteps this barrier, but the barrier's
existence means any proof must be genuinely subtle.

\paragraph{Assessment.}
We believe DH is very likely true, primarily because the
self-correcting sieve mechanism provides a concrete dynamical
reason (not merely a probabilistic heuristic) for the walk to
equidistribute.  The strongest form of this belief: CME should hold
because the EM multipliers, conditioned on walk position, have no
arithmetic reason to correlate with characters of $(\ZZ/q\ZZ)^\times$.
But we acknowledge that no existing technique can prove this, and
the $d \geq 3$ barrier shows that the proof, when found, will need
to exploit the specific structure of the EM walk in ways that
current analytic number theory does not.

%% =========================================================================
\section{Summary of Verified Results}
\label{sec:summary}
%% =========================================================================

\renewcommand{\arraystretch}{1.25}
\footnotesize
\begin{longtable}{@{}p{5.5cm}l>{\raggedright\arraybackslash}p{4.5cm}@{}}
\toprule
\textbf{Result} & \textbf{Status} & \textbf{Lean identifier} \\
\midrule
\endfirsthead
\toprule
\textbf{Result} & \textbf{Status} & \textbf{Lean identifier} \\
\midrule
\endhead
\midrule
\multicolumn{3}{r}{\emph{continued on next page}} \\
\endfoot
\bottomrule
\endlastfoot
\multicolumn{3}{l}{\emph{Sequence foundations}} \\
\quad Every $\seq(n)$ is prime & Proved & \code{seq\_isPrime} \\
\quad No prime repeats & Proved & \code{seq\_injective} \\
\quad $\seq(0\text{--}7)$ computed & Proved & \code{seq\_zero..seq\_seven} \\
\midrule
\multicolumn{3}{l}{\emph{Main reductions to MC}} \\
\quad \textbf{DH $\Rightarrow$ MC} (irreducible core) & Proved & \code{dynamical\_hitting\_implies\_mullin} \\
\quad \textbf{ThHit(11) $\Rightarrow$ MC} & Proved & \code{threshold\_11\_implies\_mullin'} \\
\quad \textbf{CCSB $\Rightarrow$ MC} (single hypothesis) & Proved & \code{complex\_csb\_mc'} \\
\quad PE $\Rightarrow$ MC & Proved & \code{pe\_implies\_mullin} \\
\quad HH $\Rightarrow$ MC & Proved & \code{hh\_implies\_mullin} \\
\quad SE + MH $\Rightarrow$ MC & Proved & \code{se\_mixing\_implies\_mullin} \\
\quad \textbf{WE $\Rightarrow$ MC} (single Prop) & Proved & \code{walk\_equidist\_mc} \\
\midrule
\multicolumn{3}{l}{\emph{Inductive bootstrap}} \\
\quad PrimeResidueEscape (elementary) & Proved & \code{prime\_residue\_escape} \\
\quad MC($<\!p$) + PRE $\Rightarrow$ SE($p$) & Proved & \code{mc\_below\_pre\_implies\_se} \\
\quad $q$-roughness from MC($<\!q$) & Proved & \code{mc\_below\_implies\_seq\_ge} \\
\quad One-prime gap & Proved & \code{mc\_below\_cofinal\_hit\_implies\_mc\_at} \\
\quad mc\_below 11 & Proved & \code{concrete\_mc\_below\_11} \\
\midrule
\multicolumn{3}{l}{\emph{Algebraic framework}} \\
\quad Confinement Theorem & Proved & \code{confinement\_forward/reverse} \\
\quad PRE $\Leftrightarrow$ SE & Proved & \code{pre\_iff\_se} \\
\quad SE $\Leftrightarrow$ character detection & Proved & \code{se\_iff\_char\_detection} \\
\quad Maximal subgroup reduction & Proved & \code{se\_of\_maximal\_escape} \\
\quad WHP $\Leftrightarrow$ HH & Proved & \code{whp\_iff\_hh} \\
\quad QR obstruction ($\leq 1.6\%$) & Proved & \code{se\_qr\_obstruction} \\
\quad SE for 30 primes ($q \leq 157$) & Proved & \code{se\_at\_11..se\_at\_157} \\
\midrule
\multicolumn{3}{l}{\emph{Character sum chain}} \\
\quad Fourier bridge: CCSB $\Rightarrow$ hit count lb & Proved & \code{complex\_csb\_implies\_hit\_count\_lb\_proved} \\
\quad Decorrelation $\Rightarrow$ PED & Proved & \code{decorrelation\_implies\_ped} \\
\quad NoLongRuns$(L)$ $\Rightarrow$ PED & Proved & \code{noLongRuns\_implies\_ped} \\
\quad BRE $\Rightarrow$ PEDImpliesCSB & Proved & \code{block\_rotation\_implies\_ped\_csb} \\
\quad CME $\Rightarrow$ CCSB (all $d$, bypasses BRE) & Proved & \code{cme\_implies\_ccsb} \\
\quad CME $\Rightarrow$ MC & Proved & \code{cme\_implies\_mc} \\
\quad Walk char recurrence ($\mathbb{C}$-valued) & Proved & \code{char\_walk\_recurrence} \\
\quad Telescoping identity & Proved & \code{walk\_telescope\_identity} \\
\quad Telescoping norm $\leq 2$ & Proved & \code{walk\_telescope\_norm\_bound} \\
\quad Shift-one autocorrelation & Proved & \code{walk\_shift\_one\_correlation} \\
\quad Order-2 sign-flip chain & Proved & \code{order2\_noLongRuns\_mc} \\
\midrule
\multicolumn{3}{l}{\emph{Walk dynamics}} \\
\quad Walk--divisibility bridge & Proved & \code{walkZ\_eq\_neg\_one\_iff} \\
\quad Products strictly monotone & Proved & \code{prod\_strictMono} \\
\quad Fundamental trichotomy & Proved & \code{avoidance\_contradicts\_se\_mixing} \\
\quad Self-avoidance dichotomy & Proved & \code{self\_avoidance\_dichotomy} \\
\quad Scheduled walk coverage & Proved & \code{scheduled\_walk\_covers\_all} \\
\midrule
\multicolumn{3}{l}{\emph{Selectability analysis}} \\
\quad Divisor freshness & Proved & \code{divisor\_not\_yet\_in\_seq} \\
\quad Passed-over persistence & Proved & \code{passed\_over\_persists} \\
\quad Selectability extinction & Proved & \code{selectability\_extinguished} \\
\quad MC $\Rightarrow$ $\neg$InfinitelySelectable & Proved & \code{mc\_implies\_not\_infinitely\_selectable} \\
\quad DH $\Rightarrow$ InfinitelySelectable & Proved & \code{dh\_implies\_infinitely\_selectable} \\
\midrule
\multicolumn{3}{l}{\emph{Sieve and orbit analysis}} \\
\quad EMFE $\Leftrightarrow$ TailSE & Proved & \code{emfe\_iff\_tail\_se\_at} \\
\quad TailSE $\Rightarrow$ CofinalEscape $\Rightarrow$ QuotientDH & Proved & \code{tail\_se\_gives\_sub\_dh} \\
\quad Dirichlet: $\infty$ primes per residue class & Proved & \code{dirichlet\_residues\_independent} \\
\quad Minimality sieve + coupling & Proved & \code{minimality\_sieve} \\
\quad StrongSieveEquidist $\Rightarrow$ NoLongRunsAt & Proved & \code{strongSieveEquidist\_noLongRunsAt} \\
\quad NoLongRunsAt $\Rightarrow$ PEDAt & Proved & \code{noLongRunsAt\_ped} \\
\quad DPED $\Rightarrow$ PED & Proved & \code{dped\_implies\_ped} \\
\quad PDE $+$ sieve chain $\Rightarrow$ MC & Proved & \code{primeDensity\_chain\_mc} \\
\quad GLPFE $+$ SieveTransfer $\Rightarrow$ MC & Proved & \code{genericLPF\_chain\_mc} \\
\quad SieveEquidist $\Rightarrow$ Dec & Proved & \code{sieve\_equidist\_implies\_decorrelation} \\
\quad SieveEquidist $\Rightarrow$ PED & Proved & \code{sieve\_equidist\_implies\_ped} \\
\midrule
\multicolumn{3}{l}{\emph{Large sieve route}} \\
\quad MultiModularCSB $\Rightarrow$ MC & Proved & \code{mmcsb\_implies\_mc} \\
\quad BV chain $\Rightarrow$ MC & Proved & \code{bv\_chain\_mc} \\
\quad ArithLS chain $\Rightarrow$ MC & Proved & \code{arith\_ls\_chain\_mc} \\
\quad ALS chain $\Rightarrow$ MC & Proved & \code{als\_prime\_arith\_ls\_chain\_mc} \\
\quad \textbf{WeakALS} (\S58) & \textbf{Proved} & \code{weak\_als\_from\_card\_bound} \\
\quad Gauss sum inversion (\S57) & Proved & \code{char\_sum\_to\_exp\_sum} \\
\quad \textbf{ALS $\Rightarrow$ PrimeArithLS} (\S65) & \textbf{Proved} & \code{als\_implies\_prime\_arith\_ls} \\
\quad Jordan's inequality (\S56) & Proved & \code{sin\_pi\_ge\_two\_mul} \\
\quad Geometric sum bound (\S56) & Proved & \code{norm\_eAN\_geom\_sum\_le\_inv} \\
\quad Parseval for ZMod.dft (\S53) & Proved & \code{zmod\_dft\_parseval} \\
\quad Gauss sum norm $\|\tau\|^2 = p$ (\S54) & Proved & \code{gaussSum\_norm\_sq\_eq\_prime} \\
\quad Walk autocorrelation identities (\S53) & Proved & \code{walkAutocorrelation\_*} \\
\quad Character Parseval (\S60) & Proved & \code{char\_parseval\_units} \\
\quad All 8 GCT internal lemmas (\S56--\S62) & Proved & \code{gct\_nontrivial\_char\_sum\_le} \\
\midrule
\multicolumn{3}{l}{\emph{Open hypotheses --- live targets}} \\
\quad \textbf{DynamicalHitting} & \textbf{Open} & \code{DynamicalHitting} \\
\quad \textbf{ComplexCharSumBound} & \textbf{Open} & \code{ComplexCharSumBound} \\
\quad \textbf{MultiModularCSB} & \textbf{Open} & \code{MultiModularCSB} \\
\quad DecorrelationHypothesis & \textbf{Open} & \code{DecorrelationHypothesis} \\
\quad PositiveEscapeDensity & \textbf{Open} & \code{PositiveEscapeDensity} \\
\quad \textbf{PEDImpliesComplexCSB} (sole sieve-route gap) & \textbf{Open} & \code{PEDImpliesComplexCSB} \\
\quad NoLongRuns$(L)$ & \textbf{Open} & \code{NoLongRuns} \\
\quad SieveEquidistribution & \textbf{Open} & \code{SieveEquidistribution} \\
\quad MertensEscape & \textbf{Open} & \code{MertensEscape} \\
\quad SieveAmplification & \textbf{Open} & \code{SieveAmplification} \\
\quad \textbf{SieveTransfer} (genuine frontier) & \textbf{Open} & \code{SieveTransfer} \\
\quad StrongSieveEquidist & \textbf{Open} & \code{StrongSieveEquidist} \\
\quad DistributionalPED & \textbf{Open} & \code{DistributionalPED} \\
\quad \textbf{BVImpliesMMCSB} (genuine frontier) & \textbf{Open} & \code{BVImpliesMMCSB} \\
\quad GaussConductorTransfer (all lemmas proved) & \textbf{Open} & \code{GaussConductorTransfer} \\
\quad PrimeArithLSImpliesMMCSB & \textbf{Open} & \code{PrimeArithLSImpliesMMCSB} \\
\midrule
\multicolumn{3}{l}{\emph{Known theorems --- not yet in Mathlib}} \\
\quad PrimeDensityEquipartition (PNT in APs) & Known & \code{PrimeDensityEquipartition} \\
\quad GenericLPFEquidist (Alladi~\cite{Alladi1977}) & Known & \code{GenericLPFEquidist} \\
\quad BombieriVinogradov & Known & \code{BombieriVinogradov} \\
\quad AnalyticLargeSieve & Known & \code{AnalyticLargeSieve} \\
\quad ArithmeticLargeSieve & Known & \code{ArithmeticLargeSieve} \\
\midrule
\multicolumn{3}{l}{\emph{Dead ends --- false or blocked}} \\
\quad MultCSBImpliesMMCSB (false in general, \S\ref{sec:walk-bridge-false}) & \textbf{Dead} & \code{MultCSBImpliesMMCSB} \\
\quad BlockRotationEstimate (impossible for $d \geq 3$, \S\ref{rem:bre-impossible}) & \textbf{Dead} & \code{BlockRotationEstimate} \\
\end{longtable}
\renewcommand{\arraystretch}{1.0}
\normalsize

%% =========================================================================
\appendix

\section{Additional Sieve and Spectral Routes}
\label{app:routes}
%% =========================================================================

This appendix collects the sieve and spectral-energy routes to MC that
complement the three principal reductions (DH, CCSB, BV) presented in
the body.  All reduction arrows are machine-verified; the sole open
content in each route is the orbit-specificity transfer.

\subsection{Arithmetic Large Sieve Route}

\begin{theorem}[\lean{EM/LargeSieve.lean\#L1183}{arith\_ls\_chain\_mc}]
$\mathrm{ArithLS} + \mathrm{ArithLSImpliesMMCSB} \;\Longrightarrow\; \MC$.
\end{theorem}
The arithmetic large sieve gives character sum bounds for Dirichlet
characters (a known result, not in Mathlib).  The transfer
ArithLSImpliesMMCSB is open; Session~35 showed it is a \textbf{dead
end}---universal coefficient bounds cannot distinguish equidistributed
walks from clumped walks.

\subsection{Analytic Large Sieve Route}

The most developed route connects the analytic large sieve to MC via
Gauss sum inversion.

\begin{definition}[\defname{AnalyticLargeSieve} (\abbr{ALS})]
For well-separated points $\{\alpha_r\} \subset \mathbb{R}/\mathbb{Z}$
with $\min_{r \neq s} \|\alpha_r - \alpha_s\| \geq \delta$:
\[
\sum_r \left\|\sum_{n < N} a_n\, e(n\alpha_r)\right\|^2
\;\leq\; (N - 1 + \delta^{-1}) \sum_{n < N} \|a_n\|^2.
\]
\end{definition}

\begin{theorem}[\lean{EM/LargeSieveAnalytic.lean\#L876}{weak\_als\_from\_card\_bound}]
\label{thm:weak-als}
A \emph{weak} version with constant $N \cdot (\delta^{-1} + 1)$ is
proved (the optimal constant is $N - 1 + \delta^{-1}$, but the
difference is immaterial since MMCSB requires only~$o(N)$).
\end{theorem}

The key bridge is \defname{Gauss sum inversion}: a Gauss sum
$\tau(\chi) = \sum_{a} \chi(a)\, e(a/p)$ intertwines multiplication
and addition on~$\ZZ/p\ZZ$, converting character sums to exponential
sums.

\begin{theorem}[\lean{EM/LargeSieveAnalytic.lean\#L688}{char\_sum\_to\_exp\_sum}]
\label{thm:gauss-inversion}
For a non-trivial character~$\chi$ mod~$p$ prime:
$\sum_n f(n)\, \chi(n)
= \tau^{-1} \sum_{b=1}^{p-1} \chi^{-1}(b) \sum_n f(n)\, \psi(bn)$.
\end{theorem}

The GaussConductorTransfer composes eight internal lemmas (all proved,
\S56--\S62) into the bridge from ALS to the prime arithmetic large sieve:

\begin{theorem}[\lean{EM/LargeSieveAnalytic.lean\#L1259}{als\_implies\_prime\_arith\_ls}]
\label{thm:als-prime}
$\mathrm{AnalyticLargeSieve} \;\Longrightarrow\; \mathrm{PrimeArithLS}$.
\end{theorem}

\begin{theorem}[\lean{EM/LargeSieveAnalytic.lean\#L1422}{als\_prime\_arith\_ls\_chain\_mc}]
$\mathrm{ALS} + \mathrm{PrimeArithLSImpliesMMCSB} \;\Longrightarrow\; \MC$.
\end{theorem}

The remaining open content is PrimeArithLSImpliesMMCSB: transferring
prime-modulus arithmetic large sieve bounds to multi-modular character
sum bounds for the specific EM orbit.

\subsection{The Spectral Energy Route}
\label{sec:sve}

Instead of individual character sums, this route examines the
\emph{total energy} of the walk occupation measure
$V_N(a) = |\{n < N : \walkZ(q,n) = a\}|$.

\begin{theorem}[\lean{EM/LargeSieveAnalytic.lean\#L1503}{walk\_energy\_parseval}]
\label{thm:walk-energy}
$\sum_\chi \|S_\chi(N)\|^2 = (q{-}1) \sum_{a \in (\ZZ/q\ZZ)^\times} V_N(a)^2$
\quad (Parseval).
\end{theorem}

The \emph{excess energy} $E(N) = \sum_{\chi \neq 1} \|S_\chi(N)\|^2$.
If $E(N) = o(N^2)$, then every non-trivial character sum is
individually~$o(N)$, which is CCSB.

\begin{definition}[\defname{SubquadraticVisitEnergy} (\abbr{SVE})]
For every missing prime~$q$ and $\varepsilon > 0$, there exists
$N_0$ such that for $N \geq N_0$: $E(N) \leq \varepsilon N^2$.
\end{definition}

\begin{theorem}[\lean{EM/LargeSieveAnalytic.lean\#L1654}{sve\_implies\_mmcsb}]
\label{thm:sve-mc}
$\mathrm{SVE} \Longrightarrow \mathrm{MMCSB} \Longrightarrow \MC$.
\end{theorem}

\paragraph{Van der Corput and higher-order decorrelation.}
The van der Corput inequality bounds $|\sum z_n|$ via autocorrelations:

\begin{theorem}[\lean{EM/LargeSieveAnalytic.lean\#L2005}{vanDerCorputBound}]
\label{thm:vdc}
$\left\|\sum_{n \leq N} z_n\right\|^2
\leq \frac{N + H}{H+1}\bigl(N + 2\sum_{h=1}^{H}
  |\mathrm{Re}\sum_{n \leq N-h} z_n \overline{z_{n+h}}|\bigr)$.
\end{theorem}

For the EM walk, lag-$h$ autocorrelations involve $h$-step multiplier
products.  At $h = 1$, the autocorrelation equals the multiplier
character sum (Theorem~\ref{thm:shift-one}), so VdC with a single
shift gives only $O(N)$.  Higher lags may decorrelate:

\begin{definition}[\defname{HigherOrderDecorrelation} (\abbr{HOD})]
For every missing prime~$q$, non-trivial~$\chi$, and $\varepsilon > 0$:
there exists $H_0$ such that for $H \geq H_0$, $N_0$ such that for
$N \geq N_0$ and all $1 \leq h \leq H$:
$\|R_h(N)\| \leq \varepsilon N$.
\end{definition}

\begin{theorem}[\lean{EM/LargeSieveAnalytic.lean\#L2428}{hod\_implies\_ccsb}]
\label{thm:hod-mc}
$\mathrm{HOD} \Longrightarrow \mathrm{CCSB} \Longrightarrow \MC$.
\end{theorem}

\paragraph{Conditional multiplier equidistribution.}

\begin{definition}[\defname{ConditionalMultiplierEquidist} (\abbr{CME})]
For every missing prime~$q$, non-trivial~$\chi$, $\varepsilon > 0$,
$N_0$ such that for $N \geq N_0$ and every
$c \in (\ZZ/q\ZZ)^\times$:
$\|\sum_{\substack{n < N \\ w(n) = c}} \chi(m(n))\| \leq \varepsilon N$.
\end{definition}

\begin{theorem}[\lean{EM/LargeSieveAnalytic.lean\#L2481}{cme\_implies\_dec}]
$\mathrm{CME} \Longrightarrow \mathrm{DecorrelationHypothesis}$.
\end{theorem}

\begin{theorem}[\lean{EM/LargeSieveAnalytic.lean\#L2567}{cme\_implies\_ccsb}]
\label{thm:cme-ccsb}
$\mathrm{CME} \Longrightarrow \mathrm{CCSB}$.
\end{theorem}

\begin{proof}[Proof sketch]
The walk telescoping identity gives
$\sum \chi(w(n)) = \sum \chi(w(n))\chi(m(n)) - (\chi(w(N)) - \chi(w(0)))$.
The product sum decomposes by fiber:
$\sum \chi(w(n))\chi(m(n)) = \sum_a \chi(a) \cdot \sum_{w(n)=a} \chi(m(n))$.
CME bounds each fiber sum by $\varepsilon' N$; the triangle inequality
sums over at most $|(\ZZ/q\ZZ)^\times|$ fibers; the boundary term
$\chi(w(N)) - \chi(w(0))$ has norm~$\leq 2$ and is absorbed for large~$N$.
\end{proof}

This is the key reduction that bypasses PED, BRE, and the $d \geq 3$
barrier.  The proof works for all character orders because it uses
only the fiber decomposition and telescoping---no block rotation
estimate is needed.

\begin{theorem}[\lean{EM/LargeSieveAnalytic.lean\#L2671}{cme\_implies\_mc}]
$\mathrm{CME} \Longrightarrow \MC$.
\end{theorem}

\begin{proof}
Compose \texttt{cme\_implies\_ccsb} with \texttt{complex\_csb\_mc'}.
\end{proof}

\begin{theorem}[\lean{EM/LargeSieveAnalytic.lean\#L2678}{cme\_chain\_mc}]
$\mathrm{CME} + \mathrm{PEDImpliesCSB} \Longrightarrow \MC$.
\end{theorem}

This older route through the Dec $\to$ PED $\to$ CCSB chain is
superseded by the direct CME $\to$ CCSB reduction above, which
requires no additional hypotheses.

\subsection{The Complete Hypothesis Hierarchy}

\[
\mathrm{PED} \;<\; \mathrm{Dec} \;<\; \mathrm{CME}
\;\xrightarrow{\text{proved}}\;
\mathrm{CCSB} \;\approx\; \mathrm{HOD}
\;\approx\; \mathrm{SVE},
\]
where ``$<$'' means strictly weaker (proved implication, known not
to reverse) and ``$\approx$'' means equivalent.
HOD~$\Leftrightarrow$~CCSB via van der Corput;
SVE~$\Leftrightarrow$~CCSB via Parseval;
CME~$\Rightarrow$~CCSB via telescoping + fiber decomposition
(Theorem~\ref{thm:cme-ccsb}).

Every hypothesis implies MC.  The PED route has an open BRE bridge
for $d \geq 3$ characters, but this is now bypassed: the direct
CME~$\to$~CCSB arrow is proved for all character orders.  CME is the
\emph{sharpest sufficient condition}---the weakest hypothesis known
to imply MC.

%% =========================================================================
\section{Methodology: Human--AI Collaboration}
\label{app:agents}
%% =========================================================================

This work was produced through a sustained collaboration between a
human author and an AI system (Claude, Anthropic) across 43+~sessions.
The human author directed the mathematical strategy---proof
architecture, dead-end identification, and editorial control---while
the AI system handled Lean~4 formalization, Mathlib API search,
literature scouting, and exploration of candidate proof strategies.

The interaction was organized at scale via an \emph{agent swarm}:
a multi-agent system built on the Claude Agent SDK\@.  The swarm comprises
seven specialized agents, each with its own system prompt, tool access,
and model:

\begin{itemize}[nosep]
\item A \emph{coordinator} that reads the current proof state,
  selects the most promising action, dispatches specialists, and updates
  shared state files.
\item A \emph{formalizer} that writes and compiles Lean code
  in rapid iteration cycles.
\item A \emph{literature scout} that searches papers and Mathlib
  for relevant results.
\item Four \emph{attack vector specialists} focused on analytic,
  algebraic, combinatorial, and information-theoretic approaches.
\item A \emph{paper writer} that maintains this document.
\end{itemize}

Agent prompts are \emph{self-evolving}: after each session the
coordinator updates them to record dead ends, new Mathlib discoveries,
and shifted priorities.  This prevents agents from rediscovering settled
territory.  All agent state (progress, strategy log, findings) is stored
as git-tracked markdown, making the exploration history fully
reproducible.

The division of labor between human and AI was sharp:

\begin{itemize}[nosep]
\item \textbf{Human:} mathematical direction, proof strategy,
  identification of dead ends, evaluation of intermediate results,
  architectural decisions on the reduction hierarchy, and editorial
  control over the final formalization and paper.
\item \textbf{AI (Claude):} Lean~4 formalization using Mathlib,
  Mathlib API search, literature scouting, exploration of candidate proof
  strategies, and drafting of this paper.
\end{itemize}

The human author guided the proof effort across 43+~sessions, suggesting
attack vectors (algebraic, analytic, combinatorial, sieve-theoretic),
identifying when an approach had reached a dead end, and pushing toward
the sharpest possible reductions.  The AI agents wrote all Lean code,
searched Mathlib for relevant lemmas, explored dozens of proof strategies
to completion or refutation, and maintained the evolving paper.

The swarm is optimized for formalization and reduction, not mathematical
discovery.  The next breakthrough, if it comes, will probably be a human
insight about the structure of $\minFac$ on EM products---not something
an agent finds by systematic search.

%% =========================================================================
\section{Glossary of Definitions and Hypotheses}
\label{sec:glossary}
%% =========================================================================

The table below collects every named definition, hypothesis, and key
theorem introduced in this paper, with abbreviations and the section
where each is defined.

\medskip
\renewcommand{\arraystretch}{1.25}
\footnotesize
\begin{longtable}{@{}l l p{7.2cm} l@{}}
\toprule
\textbf{Abbr.} & \textbf{Name} & \textbf{Meaning} & \textbf{Ref.} \\
\midrule
\endfirsthead
\toprule
\textbf{Abbr.} & \textbf{Name} & \textbf{Meaning} & \textbf{Ref.} \\
\midrule
\endhead
\midrule
\multicolumn{4}{r}{\emph{continued on next page}} \\
\endfoot
\bottomrule
\endlastfoot

\multicolumn{4}{l}{\emph{Core sequence and walk}} \\
--- & Walk / Multiplier
    & $\walkZ(q,n) = \Prod(n) \bmod q$;\; $\multZ(q,n) = \seq(n\!+\!1) \bmod q$
    & Def.~\ref{def:walk-mult} \\
\abbr{MC} & MullinConjecture
    & Every prime appears in the Euclid--Mullin sequence
    & Conj.~\ref{conj:mullin} \\
\midrule

\multicolumn{4}{l}{\emph{Algebraic hypotheses (\S\ref{sec:walk}--\S\ref{sec:bootstrap})}} \\
\abbr{SE} & SubgroupEscape
    & No proper subgroup of $(\ZZ/q\ZZ)^\times$ contains all multipliers
    & Def.~\ref{def:se} \\
\abbr{HH} & HittingHypothesis
    & The walk reaches $-1$ cofinally: $\forall N,\,\exists n \geq N,\; q \mid \Prod(n)+1$
    & Def.~\ref{def:hh} \\
\abbr{DH} & DynamicalHitting
    & $\SE(q) \Rightarrow \HH(q)$ for every missing prime~$q$
    & Def.~\ref{def:dh} \\
\abbr{PRE} & PrimeResidueEscape
    & Every proper subgroup of $(\ZZ/p\ZZ)^\times$ is escaped by some odd prime $< p$
    & Thm.~\ref{thm:pre} \\
$\abbr{PRE}_\ell$ & PowerResidueEscape
    & Multipliers escape the index-$\ell$ subgroup of $(\ZZ/q\ZZ)^\times$
    & \S\ref{sec:bootstrap} \\
--- & ThresholdHitting
    & DH restricted to primes $q \geq B$
    & \S\ref{sec:bootstrap} \\
\midrule

\multicolumn{4}{l}{\emph{Character-analytic hypotheses (\S\ref{sec:character})}} \\
\abbr{CCSB} & ComplexCharSumBound
    & Walk char sums $S_\chi(N) = o(N)$ for all non-trivial $\chi$
    & \S\ref{sec:character} \\
\abbr{MMCSB} & MultiModularCSB
    & CCSB simultaneously for all primes $q$ in a range
    & \S\ref{sec:character} \\
\abbr{ALS} & AnalyticLargeSieve
    & Large sieve inequality adapted to EM walk
    & \S\ref{sec:character} \\
\abbr{PED} & PositiveEscapeDensity
    & Positive density of $n$ with $\multZ(q,n) \notin H$, for every proper $H$
    & \S\ref{sec:character} \\
--- & DecorrelationHypothesis
    & $\multZ(q,n)$ and $\multZ(q,n{+}1)$ are asymptotically independent
    & \S\ref{sec:character} \\
\abbr{BRE} & BlockRotationEstimate
    & Cancellation in block sums of characters applied to walk
    & \S\ref{sec:character} \\
\abbr{SVE} & SubquadraticVisitEnergy
    & $\sum_{a} |\{n \leq N : \walkZ(q,n)=a\}|^2 = o(N^2/(q{-}1))$
    & \S\ref{sec:character} \\
\abbr{HOD} & HigherOrderDecorrelation
    & Higher-order correlation bounds for walk increments
    & \S\ref{sec:character} \\
\abbr{CME} & ConditionalMultiplierEquidist
    & Conditional equidist.\ of multipliers given walk state; implies CCSB (proved)
    & \S\ref{sec:character} \\
\midrule

\multicolumn{4}{l}{\emph{Named theorems}} \\
--- & Confinement
    & If SE fails, the walk is confined to a proper coset
    & Thm.~\ref{thm:confinement} \\
--- & Walk--Divisibility Bridge
    & $\walkZ(q,n) = -1 \Leftrightarrow q \mid \Prod(n)+1$
    & Thm.~\ref{thm:bridge} \\
--- & One-prime gap
    & $\MC({<}\,q)$ + cofinal hit $\Rightarrow$ $\MC(q)$
    & Thm.~\ref{thm:one-prime-gap} \\
--- & QR Obstruction
    & SE fails for at most $1.6\%$ of primes (index-2 subgroup)
    & \S\ref{sec:bootstrap} \\
--- & Gauss sum inversion
    & Character sums $\leftrightarrow$ exponential sums via Gauss sums
    & \S\ref{sec:character} \\
\end{longtable}
\renewcommand{\arraystretch}{1.0}
\normalsize

%% =========================================================================

\begin{thebibliography}{99}

\bibitem{Mullin1963}
A.\,A.~Mullin.
\newblock Recursive function theory (a modern look at a Euclidean idea).
\newblock \emph{Bull.\ Amer.\ Math.\ Soc.}, 69:737, 1963.

\bibitem{BoIr2016}
A.\,R.~Booker and S.\,A.~Irvine.
\newblock The {E}uclid--{M}ullin graph.
\newblock \emph{J.\ Number Theory}, 165:30--57, 2016.

\bibitem{BoSa2012}
A.\,R.~Booker.
\newblock A variant of the {E}uclid--{M}ullin sequence containing every prime.
\newblock \emph{J.\ Integer Sequences}, 19:Article 16.6.4, 2016.

\bibitem{Gordon1961}
B.~Gordon.
\newblock Sequences in groups with distinct partial products.
\newblock \emph{Pacific J.\ Math.}, 11(4):1309--1313, 1961.

\bibitem{PollackTrevino2014}
P.~Pollack and E.~Trevi{\~n}o.
\newblock The primes that {E}uclid forgot.
\newblock \emph{Amer.\ Math.\ Monthly}, 121(5):433--437, 2014.

\bibitem{Hardy2009}
M.~Hardy and C.~Woodgold.
\newblock Prime simplicity.
\newblock \emph{Math.\ Intelligencer}, 31:44--52, 2009.

\bibitem{Hooley1967}
C.~Hooley.
\newblock On {A}rtin's conjecture.
\newblock \emph{J.\ Reine Angew.\ Math.}, 225:209--220, 1967.

\bibitem{LagariasOdlyzko1977}
J.\,C.~Lagarias and A.\,M.~Odlyzko.
\newblock Effective versions of the {C}hebotarev density theorem.
\newblock In \emph{Algebraic Number Fields ({D}urham Symposium)},
  pages 409--464. Academic Press, 1977.

\bibitem{Alladi1977}
K.~Alladi.
\newblock On the distribution of the largest prime factor.
\newblock \emph{Stud.\ Sci.\ Math.\ Hungar.}, 12:1--9, 1977.

\bibitem{Hildebrand1986}
A.~Hildebrand.
\newblock On the number of positive integers $\leq x$ and free of
  prime factors $> y$.
\newblock \emph{J.\ Number Theory}, 22(3):289--307, 1986.

\bibitem{CoxVdP1968}
C.\,D.~Cox and A.\,J.~van~der~Poorten.
\newblock On a sequence of prime numbers.
\newblock \emph{J.\ Austral.\ Math.\ Soc.}, 8:571--574, 1968.

\bibitem{IwaniecKowalski2004}
H.~Iwaniec and E.~Kowalski.
\newblock \emph{Analytic Number Theory}.
\newblock Amer.\ Math.\ Soc.\ Colloq.\ Publ., vol.~53, 2004.

\bibitem{DiaconisShahshahani1981}
P.~Diaconis and M.~Shahshahani.
\newblock Generating a random permutation with random transpositions.
\newblock \emph{Z.\ Wahrsch.\ Verw.\ Gebiete}, 57:159--179, 1981.

\bibitem{ChungDiaconisGraham1987}
F.\,R.\,K.~Chung, P.~Diaconis, and R.\,L.~Graham.
\newblock Random walks arising in random number generation.
\newblock \emph{Ann.\ Probab.}, 15(3):1148--1165, 1987.

\bibitem{Sarnak2010}
P.~Sarnak.
\newblock Three lectures on the {M}\"obius function, randomness,
  and dynamics.
\newblock Lecture notes, IAS, 2010.
\newblock Available at \url{https://publications.ias.edu/sarnak/paper/512}.

\bibitem{Artin1927}
E.~Artin.
\newblock Beweis des allgemeinen {R}eziprozit\"atsgesetzes.
\newblock \emph{Abh.\ Math.\ Semin.\ Univ.\ Hambg.}, 5:353--363, 1927.

\end{thebibliography}

\end{document}
